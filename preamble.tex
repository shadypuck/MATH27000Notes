\usepackage[margin=1in]{geometry}
\usepackage{csquotes}
\usepackage{fancyhdr}
\usepackage{marginnote}
\usepackage[style=apa]{biblatex}
\usepackage{enumitem}
\usepackage{scrextend}
\usepackage[bottom]{footmisc}
\usepackage{xr}
\usepackage{siunitx}
\usepackage{tikz,graphicx}
\usepackage{float,subcaption}
\usepackage{amsmath,amssymb,amsbsy,amsthm}
\usepackage{bm,physics,mathtools}
\usepackage[colorlinks,allcolors=black,urlcolor=cyan]{hyperref}
\usepackage{subfiles}

\MakeOuterQuote{"}

\fancypagestyle{main}{
    \fancyhf{}
    \fancyhead[L]{\leftmark}
    \fancyhead[R]{MATH 27000}
    \fancyfoot[R]{Labalme\ \thepage}
}
\fancypagestyle{plain}{
    \fancyhead{}
    \renewcommand{\headrulewidth}{0pt}
}

\reversemarginpar

\addbibresource{\subfix{../main.bib}}
\DefineBibliographyStrings{english}{bibliography={References}}

\setitemize[3]{label={\scriptsize$\blacksquare$}}
\setitemize[4]{label={\tikz[scale=0.06,baseline={(0,-0.14)}]{
    \draw [line width=0.3pt] (0,1) -- (1.2,0) -- (0,-1) -- (3.5,0) -- cycle;
    \fill (1.2,0) -- (0,-1) -- (3.5,0);
}}}

\deffootnotemark{\textsuperscript{\textup{[}\thefootnotemark\textup{]}}}
\deffootnote[1.8em]{0em}{0em}{\textsuperscript{\thefootnote}}

\usetikzlibrary{decorations.markings,calc,intersections,backgrounds}
\colorlet{rex}{red!80!black!90!orange!80}
\colorlet{rey}{red!80!black!90!orange!50}
\colorlet{blx}{blue!90!green!80}
\colorlet{bly}{blue!90!green!50}
\colorlet{yex}{yellow!50!orange}
\colorlet{orx}{orange!80!black!90!yellow!80}
\colorlet{orz}{orange!80!black!90!yellow!20}
\colorlet{grx}{green!50!black}

\graphicspath{{../ExtFiles/}}

\DeclareMathOperator{\re}{Re}
\DeclareMathOperator{\im}{Im}
\DeclareMathOperator{\Hom}{Hom}
\DeclareMathOperator{\sgn}{sgn}
\DeclareMathOperator{\len}{len}
\DeclareMathOperator{\diam}{diam}

\newcommand{\N}{\mathbb{N}}
\newcommand{\Z}{\mathbb{Z}}
\newcommand{\Q}{\mathbb{Q}}
\newcommand{\R}{\mathbb{R}}
\newcommand{\C}{\mathbb{C}}
\newcommand{\D}{\mathbb{D}}
\newcommand{\Ss}{\mathbb{S}}
\newcommand{\Hh}{\mathbb{H}}
\newcommand{\mO}{\mathcal{O}}

\newcommand{\e}[1][]{\text{e}^{#1}}
\newcommand{\inp}[1]{\left\langle{#1}\right\rangle}