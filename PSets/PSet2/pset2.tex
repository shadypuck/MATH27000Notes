\documentclass[../psets.tex]{subfiles}

\pagestyle{main}
\renewcommand{\leftmark}{Problem Set \thesection}
\stepcounter{section}

\begin{document}




\section{Power Series and Cauchy's Theorem}
\subsection*{Set A: Graded for Completion}
\begin{enumerate}[ref={A.\arabic*}]
    \item \marginnote{4/5:}\textcite{bib:FischerLieb}, QI.3.2. Prove the Cauchy-Hadamard formula, which states that the radius of convergence $r$ is equal to
    \begin{equation*}
        r = \frac{1}{\limsup_{k\to\infty}\sqrt[k]{|a_k|}}
    \end{equation*}
    \item \textcite{bib:FischerLieb}, QI.4.4. Show that the function $\tan z$ never takes on the values $\pm i$ and that therefore,
    \begin{equation*}
        \dv{z}(\tan z) \neq 0
    \end{equation*}
    everywhere. Show that the tangent function maps the strip $S_0=\{z:-\pi/2<\re z<\pi/2\}$ biholomorphically onto $\C\setminus\{it:t\in\R,\ t\geq 1\}$.\par
    Also use level sets to illustrate the conformal mapping.
    \item Fix $a,b,c\in\C$ so that $c$ is not a negative integer or 0. Show that the \textbf{hypergeometric} function
    \begin{equation*}
        F(a,b,c;z) := \sum_{k=0}^\infty\frac{a(a+1)\cdots(a+k-1)b(b+1)\cdots(b+k-1)}{c(c+1)\cdots(c+k-1)}\frac{z^k}{k!}
    \end{equation*}
    converges on the unit disk and satisfies the differential equation
    \begin{equation*}
        z(1-z)F''(z)+[c-(a+b+1)z]F'(z)-abF(z) = 0
    \end{equation*}
\end{enumerate}


\subsection*{Set B: Graded for Content}
\begin{enumerate}[label={\textbf{\arabic*.}}]
    \item \textcite{bib:FischerLieb}, QII.2.3. Compute the \textbf{Fresnel integrals}
    \begin{equation*}
        \int_0^\infty\cos(x^2)\dd{x} = \sqrt{\frac{\pi}{8}}
        = \int_0^\infty\sin(x^2)\dd{x}
    \end{equation*}
    \emph{Hint}: Apply the Cauchy integral theorem to sectors with center 0 and corners given by $R$ and $\e[i\pi/4]R$, where $R\to\infty$.
    \item These problems illustrate the geometric intuition for the radius of convergence of a power series. Do the parts in order.
    \begin{enumerate}
        \item For each nonzero natural number $n\in\N$, compute the power series expansion for the function $1/z$ around the point $1/n$. What are their radii of convergence?
        \item Describe the set of points $w\in\C$ such that the power series expansion for $1/z$ about $w$ has radius of convergence equal to 1.
        \item Suppose that 
        \begin{equation*}
            f(z) = \frac{1}{z(z-1)(z-i)(z-1-i)}
        \end{equation*}
        Find the unique point $w$ in the unit square $\{\re(z),\im(z)\in[0,1]\}$ such that the radius of convergence of the power series for $w$ is maximal. Justify your answer.
    \end{enumerate}
    \item This problem is to hint at the general formulation of the Cauchy integral theorem. Please solve this problem only using things we have seen in class to this point.
    \begin{enumerate}
        \item Show that the "L-shaped" domain
        \begin{equation*}
            L = \{z:\re(z),\im(z)\in(0,2)\text{ and not both }\re(z),\im(z)\in(0,1]\}
        \end{equation*}
        is star-shaped (hence the Cauchy integral theorem applies).
        \item Show that the "double L-shaped" domain
        \begin{equation*}
            U = \{z:|\re(z)|,\im(z)\in(0,2)\text{ and not both }|\re(z)|,\im(z)\in(0,1]\}
        \end{equation*}
        is not star-shaped.
        \item Nevertheless, by breaking up $U$ into two copies of $L$ and using the Cauchy integral theorem for the resultant star-shaped domains, show that for any closed curve $\gamma$ in $U$ and any $f\in\mathcal{O}(U)$, we have that $\int_\gamma f\dd{z}=0$.
        \item Show that $\C^*:=\C\setminus\{0\}$ can be written as the union of two star-shaped domains.
        \item Why doesn't your proof for part (c) show that $\int_\gamma f\dd{z}=0$ for any $f\in\mathcal{O}(\C^*)$ and any closed curve $\gamma$ in $\C^*$?
    \end{enumerate}
\end{enumerate}




\end{document}