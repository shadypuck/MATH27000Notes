\documentclass[../psets.tex]{subfiles}

\pagestyle{main}
\renewcommand{\leftmark}{Problem Set \thesection}
\setcounter{section}{3}

\begin{document}




\section{Modulus Principles, Meromorphicity, and M\"{o}bius Transforms}
\subsection*{Set A: Graded for Completion}
\begin{enumerate}[ref={A.\arabic*}]
    \item \marginnote{5/3:}\textcite{bib:FischerLieb}, QIII.1.1.
    \begin{enumerate}
        \item Determine the order of the zero of $\sum_{n=1}^kb_n(z-z_0)^{-n}$ at $\infty$.
        \begin{proof}
            Define
            \begin{equation*}
                f^*(z) = f(1/z)
                = \sum_{n=1}^k\frac{b_n}{(\tfrac{1}{z}-z_0)^n}
                = \sum_{n=1}^k\frac{b_nz^n}{(1-z_0z)^n}
                = z^1\underbrace{\sum_{n=1}^k\frac{b_nz^{n-1}}{(1-z_0z)^n}}_{h(z)}
            \end{equation*}
            Therefore, since $h(0)\neq 0$, $f$ has a zero of order \fbox{1} at $\infty$.
        \end{proof}
        \item For the following functions, determine the value $w_0=f(\infty)$ and its multiplicity.
        \begin{align*}
            f(z) &= \frac{2z^4-2z^3-z^2-z+1}{z^4-z^3-z+1}&
            f(z) &= \frac{z^4+iz^3+z^2+1}{z^4+iz^3+z^2-iz}
        \end{align*}
        \begin{proof}
            Use consecutive applications of L'H\^{o}pital's rule:
            \begin{equation*}
                \lim_{z\to\infty}f(z) = \frac{2\cdot 4!}{4!}
                = 2
            \end{equation*}
            Therefore, for the left function above,
            \begin{equation*}
                \boxed{w_0 = f(\infty) = 2}
            \end{equation*}
            Define
            \begin{equation*}
                g(z) = f(z)-w_0
                = \frac{2z^4-2z^3-z^2-z+1}{z^4-z^3-z+1}-\frac{2z^4-2z^3-2z+2}{z^4-z^3-z+1}
                = \frac{-z^2+z-1}{z^4-z^3-z+1}
            \end{equation*}
            Then
            \begin{equation*}
                g^*(z) = \frac{-\tfrac{1}{z^2}+\tfrac{1}{z}-1}{\tfrac{1}{z^4}-\tfrac{1}{z^3}-\tfrac{1}{z}+1}
                = \frac{-z^2+z^3-z^4}{1-z-z^3+z^4}
                = z^2\cdot\frac{-1+z-z^2}{1-z-z^3+z^4}
            \end{equation*}
            Thus $w_0$ has multiplicity \fbox{2.}
        \end{proof}
    \end{enumerate}
    \item \textcite{bib:FischerLieb}, QIII.3.1. Let $f$ and $g$ be entire functions such that $|f|\leq|g|$. Show that $f=cg$ for some constant $c$.
    \begin{proof}
        If $|f|\leq|g|$, then
        \begin{equation*}
            \left| \frac{f}{g} \right| \leq 1
        \end{equation*}
        Thus, $f/g$ is bounded and entire, so it must be constant by Liouville's theorem. But if $f/g=c$, then $f=cg$, as desired.\par
        We do need a proof that $f/g$ is entire, which means considering the behavior of $f/g$ at any point that $g(z)=0$.
    \end{proof}
    \item \textcite{bib:FischerLieb}, QIII.4.1.
    \begin{enumerate}
        \item Let $S,T\in\Mob$. Show that a point $z_1\in\hat{\C}$ is a fixed point of $T$ if and only if $Sz_1$ is a fixed point of $STS^{-1}$.
        \begin{proof}
            Suppose $z_1$ is a fixed point of $T$. Then $Tz_1=z_1$. Thus,
            \begin{equation*}
                STS^{-1}(Sz_1) = STz_1 = Sz_1
            \end{equation*}
            as desired.\par
            Suppose $Sz_1$ is a fixed point of $STS^{-1}$. Then
            \begin{align*}
                STS^{-1}(Sz_1) &= Sz_1\\
                STz_1 &= Sz_1
            \end{align*}
            Applying $S^{-1}$ to both sides of the above yields the desired result.
        \end{proof}
        \item Suppose $T$ has exactly one fixed point $z_1$. Show that there is an $S\in\Mob$ such that $STS^{-1}$ is a translation. Moreover, show that for every $z\in\hat{\C}$, we have
        \begin{equation*}
            \lim_{n\to\infty}T^nz = z_1
        \end{equation*}
        where $T^n=T\circ\cdots\circ T$ denotes the $n$-fold composition of $T$ with itself.
        \item Suppose $T$ has exactly two fixed points $z_1$ and $z_2$. Show that there is an $S\in\Mob$ such that $STS^{-1}$ is of the form $z\mapsto az$, where $a\in\C^*$, and that the pair $\{a,a^{-1}\}$ is uniquely determined by $T$.
        \item Show that if we have $|a|\neq 1$ in part (c), then after a possible renumbering of our fixed points, we have
        \begin{equation*}
            \lim_{n\to\infty}T^nz = z_1
        \end{equation*}
        for all $z\in\hat{\C}\setminus\{z_2\}$. In the case that $|a|=1$, show that every point in $\hat{\C}\setminus\{z_1,z_2\}$ lies on a $T$-invariant M\"{o}bius circle.
    \end{enumerate}
    \item \textcite{bib:FischerLieb}, QIII.5.1. Let $f$ be the branch of the logarithm on $\C\setminus\R_{\geq 0}$ that takes the value $-i\pi/2$ at $-i$. Determine\dots
    \begin{align*}
        f(i)&&
        f(-\e)&&
        f(-1-i\sqrt{3})&&
        f((-1-i\sqrt{3})^2)
    \end{align*}
    \begin{proof}
        \begin{align*}
            f(i) &= -\frac{3\pi i}{2}&
            f(-\e) &= 1-i\pi
        \end{align*}
    \end{proof}
    \item If $z_1$ and $z_2$ are related by inversion in a circle $C$, and $z_3$ and $z_4$ are arbitrary (distinct) points of $C$, show that the cross ratio of the four points has modulus 1.
\end{enumerate}


\subsection*{Set B: Graded for Content}
\begin{enumerate}[label={\textbf{\arabic*.}}]
    \item \textcite{bib:FischerLieb}, QIII.3.4. Consider the function $f(z)=z+\e[z]$. Show that for all $t\in[0,2\pi]$,
    \begin{equation*}
        \lim_{r\to\infty}f(r\e[it]) = \infty
    \end{equation*}
    and that the convergence is uniform with respect to $t$ on the sets $\{t:|t-\pi|\leq\tfrac{\pi}{2}\}$ and $\{t:|t|\leq\alpha\}$ for every $\alpha<\pi/2$. How does this agree with Proposition 3.4?
    \item \textcite{bib:FischerLieb}, QIII.5.2.
    \begin{enumerate}
        \item Find a maximal domain on which holomorphic functions $\log(1-z)^2$ and $\sqrt{z+\sqrt{z}}$, respectively, can be defined.
        \item Show that a logarithm of the tangent function exists on the set
        \begin{equation*}
            G = \C\setminus\bigcup_{k\in\Z}[k\pi-\tfrac{\pi}{2},k\pi]
        \end{equation*}
    \end{enumerate}
    \item Show that a fractional linear transformation
    \begin{equation*}
        z \mapsto \frac{az+b}{cz+d}
    \end{equation*}
    maps the upper half plane to itself if and only if $a,b,c,d\in\R$ and $ad-bc>0$.
    \item Suppose that $U$ is a domain, $f\in\mO(U)$ is never zero, and suppose that a holomorphic branch of the logarithm exists on $f(U)$; then the function $\log[f(z)]$ is holomorphic. By considering the real part of $\log f$, show that the maximum modulus principles for harmonic functions and for holomorphic functions are equivalent.
\end{enumerate}




\end{document}