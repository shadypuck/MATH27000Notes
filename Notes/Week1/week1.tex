\documentclass[../notes.tex]{subfiles}

\pagestyle{main}
\renewcommand{\chaptermark}[1]{\markboth{\chaptername\ \thechapter\ (#1)}{}}

\begin{document}




\chapter{???}
\section{Holomorphic Functions}
\begin{itemize}
    \item \marginnote{3/19:}We begin by reviewing some properties of the \textbf{complex numbers}.
    \item \textbf{Complex numbers}: The field of elements $z=x+iy$ where $x,y\in\R$ and $i^2=-1$. \emph{Denoted by} $\pmb{\C}$.
    \begin{figure}[h!]
        \centering
        \begin{tikzpicture}[
            every node/.style=black
        ]
            \small
            \draw [-stealth] (-1.5,0) -- (1.5,0) node[right]{$\R$};
            \draw [-stealth] (0,-1.5) -- (0,1.5) node[above]{$i\R$};

            \footnotesize
            \draw (1,0.1)    -- ++(0,-0.2) node[below]{$x$};
            \draw (0.1,0.8)  -- ++(-0.2,0) node[left] {$y$};
            \draw (0.1,-0.8) -- ++(-0.2,0) node[left] {$-y$};

            \fill [rex] (1,0.8)  coordinate (z)    circle (1.5pt) node[above right]{$z$};
            \fill [rex] (1,-0.8) coordinate (zbar) circle (1.5pt) node[below right]{$\bar{z}$};
            \draw [rex,semithick,<->,shorten <=3pt,shorten >=3pt] (z) to[bend left=20] (zbar);
        \end{tikzpicture}
        \caption{The complex plane.}
        \label{fig:complexPlane}
    \end{figure}
    \vspace{-0.4em}
    \begin{itemize}
        \item Can be visualized as a two-dimensional plane with the number $z$ corresponding to the point $(x,y)$.
    \end{itemize}
    \item \textbf{Real part}: The number $x$. \emph{Denoted by} $\bm{\re(z)}$.
    \item \textbf{Imaginary part}: The number $y$. \emph{Denoted by} $\bm{\im(z)}$.
    \item \textbf{Complex conjugate} (of $z$): The complex number defined as follows. \emph{Denoted by} $\bm{\bar{z}}$. \emph{Given by}
    \begin{equation*}
        \bar{z} := x-iy
    \end{equation*}
    \item Now recall the definition of a \emph{real} function that is \textbf{differentiable} at a point $x_0\in\R$.
    \begin{itemize}
        \item $f'(x_0)(x-x_0)$ is the "best linear approximation" of $f$ near $x_0$, where $\bm{f'(x_0)}$ is also defined below.
    \end{itemize}
    \item \textbf{Differentiable} ($f$ at $x_0$): A function $f:\R\to\R$ for which the following limit exists. \emph{Constraint}
    \begin{equation*}
        \lim_{x\to x_0}\frac{f(x)-f(x_0)}{x-x_0} =: f'(x_0)
    \end{equation*}
    \item We now build up to defining a notion of complex differentiability.
    \begin{itemize}
        \item Observe that the constraint above is equivalent to the constraint
        \begin{equation*}
            f(x) = f(x_0)+[\underbrace{f'(x_0)+e(x)}_{\Delta(x)}](x-x_0)
        \end{equation*}
        where $e(x)\to 0$ as $x\to x_0$.
        \item Note that we are defining a new function $\Delta(x)$ above, with the property that $\Delta(x_0)=f'(x_0)$.
    \end{itemize}
    \pagebreak
    \item \textbf{Holomorphic} ($f$ at $z_0$): A function $f:\C\to\C$ for which the following limit exists. \emph{Also known as} \textbf{$\pmb{\C}$-differentiable}. \emph{Constraints}
    \begin{equation*}
        \lim_{z\to z_0}\frac{f(z)-f(z_0)}{z-z_0} =: f'(z_0)
        \qquad\Longleftrightarrow\qquad
        f(z) = f(z_0)+\Delta(z)(z-z_0)
    \end{equation*}
    where $\Delta$ is continuous at $z_0$ and $\Delta(z_0)=f'(z_0)$.
    \begin{itemize}
        \item Is this the true definition of "holomorphic" / "$\C$-differentiable" function, or is this just a naive first pass??
    \end{itemize}
    \item Properties of holomorphic functions: Let $U\subset\C$ be open.
    \begin{enumerate}
        \item The holomorphic functions on $U$ form a ring $\bm{\mathcal{O}(U)}$.
        \begin{itemize}
            \item Equivalently, the $\C$-differentiation operator is $\C$-linear.
            \item Equivalently, if $f,g$ are holomorphic, then $f+g$ and $fg$ are holomorphic, too.
            \item Equivalently (and most simply), we have the sum rule and the product rule (and the quotient rule if the function in the denominator is nonzero).
        \end{itemize}
        \item We have the chain rule.
        \item Holomorphic implies continuous.
    \end{enumerate}
    \item Examples: Polynomials, rational functions $p(z)/q(z)$ (away from their \textbf{poles}).
    \item Noney\footnote{What does "Noney" mean??}: Consider the function $f:\C\to\C$ defined by
    \begin{equation*}
        z \mapsto \bar{z}
    \end{equation*}
    \begin{itemize}
        \item TPS\footnote{What does "TPS" mean??}: Why?
        \item Notice that
        \begin{align*}
            f(0) &= 0&
            f(t) &= t&
            f(it) &= -it
        \end{align*}
        \item Thus,
        \begin{align*}
            \Delta(t) &= 1&
            \Delta(it) &= -1
        \end{align*}
        for all $t$.
        \item But this means that $\Delta$ can't be continuous!
        \item Yet $f$ is clearly $\R$-differentiable! What gives?!
        \item Note that --- viewing $f$ as a mapping of $\R^2\to\R^2$ --- we have
        \begin{equation*}
            Df =
            \begin{pmatrix}
                1 & 0\\
                0 & -1\\
            \end{pmatrix}
        \end{equation*}
    \end{itemize}
\end{itemize}



\section{Chapter I: Analysis in the Complex Plane}
\emph{From \textcite{bib:FischerLieb}.}
\begin{itemize}
    \item The preface only contains comments and instructions for an instructor planning to use this textbook for a course.
    \item The chapter begins with two paragraphs.
    \begin{itemize}
        \item The first discusses topic covered in the chapter.
        \item The second gives some historical background on these topics.
    \end{itemize}
\end{itemize}


\subsection*{Section I.0: Notations and Basic Concepts}
\begin{itemize}
    \item Goal: Reiew the fundamental topological and analytical concepts of real analysis.
    \item Defines the \textbf{complex numbers}, \textbf{complex plane}, and \textbf{complex conjugate}.
    \item \textbf{Absolute value} (of $z$): The Euclidean distance of $z$ from zero. \emph{Also known as} \textbf{modulus}. \emph{Denoted by} $\bm{|z|}$. \emph{Given by}
    \begin{equation*}
        |z| := \sqrt{x^2+y^2}
    \end{equation*}
    \item \textbf{Imaginary unit}. \emph{Denoted by} $\bm{i}$.
    \item Relating the modulus and complex conjugate.
    \begin{equation*}
        |z| = \sqrt{z\bar{z}}
    \end{equation*}
    \item \textbf{Open disk} (of radius $\varepsilon$ and center $z_0$): The set defined as follows. \emph{Also known as} \textbf{$\bm{\varepsilon}$-neighborhood} (of $z_0$). \emph{Denoted by} $\bm{D_\varepsilon(z_0)}$, $\bm{U_\varepsilon(z_0)}$. \emph{Given by}
    \begin{equation*}
        D_\varepsilon(z_0) = U_\varepsilon(z_0) := \{z\in\C:|z-z_0|<\varepsilon\}
    \end{equation*}
    \item \textbf{Unit disk}: The set defined as follows. \emph{Denoted by} $\pmb{\D}$. \emph{Given by}
    \begin{equation*}
        \D := D_1(0)
    \end{equation*}
    \item \textbf{Unit circle}: The set defined as follows. \emph{Denoted by} $\pmb{\Ss}$. \emph{Given by}
    \begin{equation*}
        \Ss := \{z\in\C:|z|=\varepsilon\}
    \end{equation*}
    \item \textbf{Upper half plane}: The set defined as follows. \emph{Denoted by} $\pmb{\Hh}$. \emph{Given by}
    \begin{equation*}
        \Hh := \{z\in\C:\im z>0\}
    \end{equation*}
    \item $\bm{\pmb{\C}^*}$: The set defined as follows. \emph{Given by}
    \begin{equation*}
        \C^* := \C\setminus\{0\}
    \end{equation*}
    \item \marginnote{3/21:}\textbf{Neighborhood} (of $z_0$): A set $U$ which contains an $\varepsilon$-neighborhood.
    \item \textbf{Open} (set): A set that is a neighborhood of each of its points.
    \item \textbf{Closed} (set): A complement of an open set.
    \item \textbf{Interior} (of $M$): The largest open set contained in $M$. \emph{Denoted by} $\bm{\mathring{M}}$.
    \item \textbf{Closure} (of $M$): The smallest closed set containing $M$. \emph{Denoted by} $\bm{\overline{M}}$.
    \item \textbf{Topological boundary} (of $M$): The set defined as follows. \emph{Also known as} \textbf{boundary}. \emph{Denoted by} $\bm{\partial M}$. \emph{Given by}
    \begin{equation*}
        \partial M := \overline{M}\setminus\mathring{M}
    \end{equation*}
    \item \textbf{Relatively open} (set in $M$): The intersection of an open set $U$ with an arbitrary set $M$. \emph{Also known as} \textbf{open} (set in $M$).
    \item \textbf{Relatively closed} (set in $M$): The intersection of a closed set $U$ with an arbitrary set $M$. \emph{Also known as} \textbf{open} (set in $M$).
\end{itemize}




\end{document}