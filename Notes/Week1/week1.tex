\documentclass[../notes.tex]{subfiles}

\pagestyle{main}
\renewcommand{\chaptermark}[1]{\markboth{\chaptername\ \thechapter\ (#1)}{}}

\begin{document}




\chapter{???}
\section{Holomorphic Functions}
\begin{itemize}
    \item \marginnote{3/19:}We begin by reviewing some properties of the \textbf{complex numbers}.
    \item \textbf{Complex numbers}: The field of elements $z=x+iy$ where $x,y\in\R$ and $i^2=-1$. \emph{Denoted by} $\pmb{\C}$.
    \begin{figure}[h!]
        \centering
        \begin{tikzpicture}[
            every node/.style=black
        ]
            \small
            \draw [-stealth] (-1.5,0) -- (1.5,0) node[right]{$\R$};
            \draw [-stealth] (0,-1.5) -- (0,1.5) node[above]{$i\R$};

            \footnotesize
            \draw (1,0.1)    -- ++(0,-0.2) node[below]{$x$};
            \draw (0.1,0.8)  -- ++(-0.2,0) node[left] {$y$};
            \draw (0.1,-0.8) -- ++(-0.2,0) node[left] {$-y$};

            \fill [rex] (1,0.8)  coordinate (z)    circle (1.5pt) node[above right]{$z$};
            \fill [rex] (1,-0.8) coordinate (zbar) circle (1.5pt) node[below right]{$\bar{z}$};
            \draw [rex,semithick,<->,shorten <=3pt,shorten >=3pt] (z) to[bend left=20] (zbar);
        \end{tikzpicture}
        \caption{The complex plane.}
        \label{fig:complexPlane}
    \end{figure}
    \vspace{-0.4em}
    \begin{itemize}
        \item Can be visualized as a two-dimensional plane with the number $z$ corresponding to the point $(x,y)$.
    \end{itemize}
    \item \textbf{Real part}: The number $x$. \emph{Denoted by} $\bm{\re z}$.
    \item \textbf{Imaginary part}: The number $y$. \emph{Denoted by} $\bm{\im z}$.
    \item \textbf{Complex conjugate} (of $z$): The complex number defined as follows. \emph{Denoted by} $\bm{\bar{z}}$. \emph{Given by}
    \begin{equation*}
        \bar{z} := x-iy
    \end{equation*}
    \item Now recall the definition of a \emph{real} function that is \textbf{differentiable} at a point $x_0\in\R$.
    \begin{itemize}
        \item $f'(x_0)(x-x_0)$ is the "best linear approximation" of $f$ near $x_0$, where $\bm{f'(x_0)}$ is also defined below.
    \end{itemize}
    \item \textbf{Differentiable} ($f:\R\to\R$ at $x_0$): A function $f$ for which the following limit exists. \emph{Constraint}
    \begin{equation*}
        \lim_{x\to x_0}\frac{f(x)-f(x_0)}{x-x_0} =: f'(x_0)
    \end{equation*}
    \item We now build up to defining a notion of complex differentiability.
    \begin{itemize}
        \item Observe that the constraint above is equivalent to the constraint
        \begin{equation*}
            f(x) = f(x_0)+[\underbrace{f'(x_0)+e(x)}_{\Delta(x)}](x-x_0)
        \end{equation*}
        where $e(x)\to 0$ as $x\to x_0$.
        \item Note that we are defining a new function $\Delta(x)$ above, with the property that $\Delta(x_0)=f'(x_0)$.
    \end{itemize}
    \pagebreak
    \item \textbf{Holomorphic} ($f$ at $z_0$): A function $f:\C\to\C$ for which the following limit exists. \emph{Also known as} \textbf{$\pmb{\C}$-differentiable}. \emph{Constraints}
    \begin{equation*}
        \lim_{z\to z_0}\frac{f(z)-f(z_0)}{z-z_0} =: f'(z_0)
        \qquad\Longleftrightarrow\qquad
        f(z) = f(z_0)+\Delta(z)(z-z_0)
    \end{equation*}
    where $\Delta$ is continuous at $z_0$ and $\Delta(z_0)=f'(z_0)$.
    \begin{itemize}
        \item Is this the true definition of "holomorphic" / "$\C$-differentiable" function, or is this just a na\"{i}ve first pass??
    \end{itemize}
    \item Properties of holomorphic functions: Let $U\subset\C$ be open.
    \begin{enumerate}
        \item The holomorphic functions on $U$ form a ring $\bm{\mathcal{O}(U)}$.
        \begin{itemize}
            \item Equivalently, the $\C$-differentiation operator is $\C$-linear.
            \item Equivalently, if $f,g$ are holomorphic, then $f+g$ and $fg$ are holomorphic, too.
            \item Equivalently (and most simply), we have the sum rule and the product rule (and the quotient rule if the function in the denominator is nonzero).
        \end{itemize}
        \item We have the chain rule.
        \item Holomorphic implies continuous.
    \end{enumerate}
    \item Examples: Polynomials, rational functions $p(z)/q(z)$ (away from their \textbf{poles}).
    \item Noney\footnote{What does "Noney" mean??}: Consider the function $f:\C\to\C$ defined by
    \begin{equation*}
        z \mapsto \bar{z}
    \end{equation*}
    \begin{itemize}
        \item TPS\footnote{What does "TPS" mean??}: Why?
        \item Notice that
        \begin{align*}
            f(0) &= 0&
            f(t) &= t&
            f(it) &= -it
        \end{align*}
        \item Thus,
        \begin{align*}
            \Delta(t) &= 1&
            \Delta(it) &= -1
        \end{align*}
        for all $t$.
        \item But this means that $\Delta$ can't be continuous!
        \item Yet $f$ is clearly $\R$-differentiable! What gives?!
        \item Note that --- viewing $f$ as a mapping of $\R^2\to\R^2$ --- we have
        \begin{equation*}
            Df =
            \begin{pmatrix}
                1 & 0\\
                0 & -1\\
            \end{pmatrix}
        \end{equation*}
    \end{itemize}
    \item The above example suggests that our definition of complex differentiability may have been to na\"{i}ve, so we'll do some further investigations now.
    \item Observe that $\C\cong\R^2$ as $\R$-vector spaces.
    \item \textbf{Differentiable} ($f:\R^2\to\R^2$ at $x_0$): A function $f$ for which there exists an $\R$-linear map $A:\R^2\to\R^2$ satisfying the following constraint. \emph{Constraint}
    \begin{equation*}
        \lim_{h\to 0}\frac{\norm{f(x_0+h)-f(x_0)-Ah}}{\norm{h}} = 0
    \end{equation*}
    \begin{itemize}
        \item We also denote $A$ by $Df$.
    \end{itemize}
    \pagebreak
    \item Example: Consider the function $f:\C\to\R$ defined by
    \begin{equation*}
        x+iy \mapsto x
    \end{equation*}
    \begin{itemize}
        \item Differentiable with total derivative
        \begin{equation*}
            Df =
            \begin{pmatrix}
                1 & 0\\
            \end{pmatrix}
        \end{equation*}
    \end{itemize}
    \item Observation: While $\C\cong\R^2$ as $\R$-vector spaces, as a $\C$-vector space, there is \emph{additional} structure.
    \begin{itemize}
        \item In particular, all "vectors" should commute with the "multiplication by $i$" map $J:\C\to\C$ defined by any one of the following three maps.
        \begin{align*}
            z &\mapsto z&
            x+iy &\mapsto xi-y&
            \begin{pmatrix}
                0 & -1\\
                1 & 0\\
            \end{pmatrix}
        \end{align*}
    \end{itemize}
    \item Exercise: In $(\re,\im)$ coordinates, write down the matrix for "multiply by $w$" for any $w\in\C$.
    \begin{itemize}
        \item Let $w=a+bi$ and let $v=x+iy$. Then
        \begin{align*}
            wv &= (a+bi)(x+iy)
                = ax-by+i(bx+ay)\\
            &=
            \begin{pmatrix}
                ax-by\\
                bx+ay\\
            \end{pmatrix}
                = \underbrace{
                    \begin{pmatrix}
                        a & -b\\
                        b & a\\
                    \end{pmatrix}
                }_W
                \begin{pmatrix}
                    x\\
                    y\\
                \end{pmatrix}
        \end{align*}
        \item The matrix $W$ above is the desired result.
    \end{itemize}
    \item TPS: Is $f:\C\to\C$ defined as follows a complex linear map? Why not?
    \begin{equation*}
        x+iy \mapsto
        \begin{pmatrix}
            1 & 1\\
            0 & 1\\
        \end{pmatrix}
        \begin{pmatrix}
            x\\
            y\\
        \end{pmatrix}
        = (x+y)+iy
    \end{equation*}
    \begin{itemize}
        \item Among other properties, a complex linear map should satisfy
        \begin{equation*}
            if(x+iy) = f[i(x+iy)]
        \end{equation*}
        for the scalar $i\in\C$.
        \item However, we have that
        \begin{equation*}
            if(x+iy) = i[(x+y)+iy]
            = -y+i(x+y)
            \neq (x-y)+ix
            = f(-y+ix)
            = f[i(x+iy)]
        \end{equation*}
    \end{itemize}
    \item What about the following map?
    \begin{equation*}
        A =
        \begin{pmatrix}
            1 & 2\\
            -2 & 1\\
        \end{pmatrix}
    \end{equation*}
    \begin{itemize}
        \item A complex linear map should satisfy
        \begin{align*}
            A(v+w) &= Av+Aw&
            \lambda Av &= A(\lambda v)
        \end{align*}
        for all $v,w,\lambda\in\C$.
        \item Let $v,w\in\C$ be arbitrary. Then
        \begin{align*}
            A(v+w) &=
            \begin{pmatrix}
                1 & 2\\
                -2 & 1\\
            \end{pmatrix}
            \left[
                \begin{pmatrix}
                    v_1\\
                    v_2\\
                \end{pmatrix}
                +
                \begin{pmatrix}
                    w_1\\
                    w_2\\
                \end{pmatrix}
            \right]
            =
            \begin{pmatrix}
                (v_1+w_1)+2(v_2+w_2)\\
                -2(v_1+w_1)+(v_2+w_2)\\
            \end{pmatrix}\\
            &=
            \begin{pmatrix}
                1 & 2\\
                -2 & 1\\
            \end{pmatrix}
            \begin{pmatrix}
                v_1\\
                v_2\\
            \end{pmatrix}
            +
            \begin{pmatrix}
                1 & 2\\
                -2 & 1\\
            \end{pmatrix}
            \begin{pmatrix}
                w_1\\
                w_2\\
            \end{pmatrix}
            = Av+Aw
        \end{align*}
        \item Let $v,\lambda\in\C$. Then
        \begin{align*}
            \lambda Av &= (\lambda_1+i\lambda_2)\cdot[(v_1+2v_2)+i(-2v_1+v_2)]\\
            &= [\lambda_1(v_1+2v_2)-\lambda_2(-2v_1+v_2)]+i[\lambda_2(v_1+2v_2)+\lambda_1(-2v_1+v_2)]\\
            &= [(\lambda_1v_1-\lambda_2v_2)+2(\lambda_2v_1+\lambda_1v_2)]+i[-2(\lambda_1v_1-\lambda_2v_2)+(\lambda_2v_1+\lambda_1v_2)]\\
            &= A[(\lambda_1v_1-\lambda_2v_2)+i(\lambda_2v_1+\lambda_1v_2)]\\
            &= A(\lambda v)
        \end{align*}
        \item Therefore, since $A$ satisfies the two properties, it is complex linear.
    \end{itemize}
    \item Conclusion: To reiterate from the above, $A$ must commute with $J$ to be complex linear.
    \item Implication: Every $\C$-linear map of $\C$ is just multiplication by a complex number.
    \begin{itemize}
        \item This is a special case of the following more general result, which holds for any field $K$.
        \begin{equation*}
            \Hom_K(K,K) \cong K
        \end{equation*}
    \end{itemize}
    \item Now let's revisit differentiability.
    \item It turns out that a condition for $\C$-differentiability \emph{equivalent} to the definition of "holomorphic" given above is that there exists a $\C$-linear map $A:\C\to\C$ such that
    \begin{equation*}
        \lim_{h\to 0}\frac{\norm{f(x_0+h)-f(x_0)-Ah}}{\norm{h}} = 0
    \end{equation*}
    \begin{itemize}
        \item From the above discussion, we know that this $A$ is just multiplication by some $w\in\C$.
        \item All of the values in the above norms are complex numbers, so \emph{another} equivalent condition is
        \begin{equation*}
            \lim_{z\to z_0}\frac{|f(z)-f(z_0)-w\cdot(z-z_0)|}{|z-z_0|} = 0
        \end{equation*}
        \item This condition is wholly mathematically equivalent to our holomorphic definition,
        \begin{equation*}
            `\lim_{z\to z_0}\frac{f(z)-f(z_0)}{z-z_0} = w
        \end{equation*}
    \end{itemize}
    \item So when is an $\R$-differentiable function actually holomorphic?
    \begin{itemize}
        \item Let $f:\R^2\to\R^2$ map $(x,y)\mapsto(g,h)$.
        \item Let
        \begin{equation*}
            A = Df =
            \begin{pmatrix}
                g_x & g_y\\
                h_x & h_y\\
            \end{pmatrix}
        \end{equation*}
        where the subscript notation views $g$, for instance, as $g(x,y)$ and denotes the partial derivative of $g$ with respect to $x$.
        \item Let $J$ (the "multiply by $i$") function be defined as above.
        \item Then the "commute with $i$" condition is equivalent to
        \begin{equation*}
            J^{-1}AJ = A
        \end{equation*}
        \item Expanding the product on the left above in terms of $g_x,g_y,h_x,h_y$, we obtain
        \begin{equation*}
            \begin{pmatrix}
                h_y & -h_x\\
                -g_y & g_x\\
            \end{pmatrix}
            =
            \begin{pmatrix}
                0 & 1\\
                -1 & 0\\
            \end{pmatrix}
            \begin{pmatrix}
                g_x & g_y\\
                h_x & h_y\\
            \end{pmatrix}
            \begin{pmatrix}
                0 & -1\\
                1 & 0\\
            \end{pmatrix}
            =
            \begin{pmatrix}
                g_x & g_y\\
                h_x & h_y\\
            \end{pmatrix}
        \end{equation*}
        \item This condition is equivalent to $A$ satisfying the \textbf{Cauchy-Riemann equations}.
    \end{itemize}
    \item \textbf{Cauchy-Riemann equations}: The following two equations, which identify when a complex function is holomorphic. \emph{Also known as} \textbf{CR equations}. \emph{Given by}
    \begin{align*}
        g_x &= h_y\\
        g_y &= -h_x
    \end{align*}
    \item These equations are satisfied when $A$ is of the form
    \begin{equation*}
        A =
        \begin{pmatrix}
            a & -b\\
            b & a\\
        \end{pmatrix}
    \end{equation*}
    \item So at this point, we can differentiate $f$ with respect to $z$. But what if we want to differentiate it with respect to $x$ and $y$ (of $z=x+iy$)?
    \begin{itemize}
        \item We will need the following change of basis.
        \begin{itemize}
            \item Since $z=x+iy$ and $\bar{z}=x-iy$, we have
            \begin{align*}
                2x &= z+\bar{z}&
                    2iy &= z-\bar{z}\\
                x &= \frac{1}{2}(z+\bar{z})&
                    y &= -\frac{i}{2}(z-\bar{z})
            \end{align*}
            \item This tells us that
            \begin{align*}
                \pdv{x}{z} &= \frac{1}{2}&
                \pdv{y}{z} &= -\frac{i}{2}
            \end{align*}
        \end{itemize}
        \item We can now invoke the multivariable chain rule and simplify the resultant expression.
        \begin{equation*}
            \pdv{f}{z} = \pdv{f}{x}\pdv{x}{z}+\pdv{f}{y}\pdv{y}{z}
            = \frac{1}{2}(f_x-if_y)
        \end{equation*}
        \begin{itemize}
            \item Note that once again, the subscript notation "$f_x$" means $\pdv*{f}{x}$.
        \end{itemize}
        \item Note that we can also similarly work out that
        \begin{equation*}
            \pdv{f}{\bar{z}} = \frac{1}{2}(f_x+if_y)
        \end{equation*}
        \begin{itemize}
            \item Observe in particular that
            \begin{align*}
                f_x &= g_x+ih_x&
                f_y &= g_y+ih_y
            \end{align*}
            \item Thus, the CR equations ($g_x=h_y$ and $g_y=-h_x$) being satisfied is equivalent to
            \begin{equation*}
                \pdv{f}{\bar{z}} = \frac{1}{2}(f_x+if_y)
                = \frac{1}{2}[(g_x+ih_x)+i(g_y+ih_y)]
                = 0
            \end{equation*}
            \item Note that $\pdv*{f}{\bar{z}}$ is not actually a derivative since $f$ depends on $z$, not $\bar{z}$. Rather, we use "$\pdv*{f}{\bar{z}}$" to denote the following operator applied to $f$.
            \begin{equation*}
                \pdv{\bar{z}} := \frac{1}{2}\left( \pdv{x}+i\pdv{y} \right)
            \end{equation*}
        \end{itemize}
    \end{itemize}
    \item Theorem: The $\R$-differentiable function $f:U\to\C$ is holomorphic iff $\pdv*{f}{\bar{z}}=0$. Moreover, if it is, then
    \begin{equation*}
        f'(z_0) = \eval{\pdv{f}{z}}_{z_0}
    \end{equation*}
\end{itemize}



\section{Chapter I: Analysis in the Complex Plane}
\emph{From \textcite{bib:FischerLieb}.}
\begin{itemize}
    \item The preface only contains comments and instructions for an instructor planning to use this textbook for a course.
    \item The chapter begins with two paragraphs.
    \begin{itemize}
        \item The first discusses topic covered in the chapter.
        \item The second gives some historical background on these topics.
    \end{itemize}
\end{itemize}


\subsection*{Section I.0: Notations and Basic Concepts}
\begin{itemize}
    \item Goal: Reiew the fundamental topological and analytical concepts of real analysis.
    \item Defines the \textbf{complex numbers}, \textbf{complex plane}, and \textbf{complex conjugate}.
    \item \textbf{Absolute value} (of $z$): The Euclidean distance of $z$ from zero. \emph{Also known as} \textbf{modulus}. \emph{Denoted by} $\bm{|z|}$. \emph{Given by}
    \begin{equation*}
        |z| := \sqrt{x^2+y^2}
    \end{equation*}
    \item \textbf{Imaginary unit}. \emph{Denoted by} $\bm{i}$.
    \item Relating the modulus and complex conjugate.
    \begin{equation*}
        |z| = \sqrt{z\bar{z}}
    \end{equation*}
    \item \textbf{Open disk} (of radius $\varepsilon$ and center $z_0$): The set defined as follows. \emph{Also known as} \textbf{$\bm{\varepsilon}$-neighborhood} (of $z_0$). \emph{Denoted by} $\bm{D_\varepsilon(z_0)}$, $\bm{U_\varepsilon(z_0)}$. \emph{Given by}
    \begin{equation*}
        D_\varepsilon(z_0) = U_\varepsilon(z_0) := \{z\in\C:|z-z_0|<\varepsilon\}
    \end{equation*}
    \item \textbf{Unit disk}: The set defined as follows. \emph{Denoted by} $\pmb{\D}$. \emph{Given by}
    \begin{equation*}
        \D := D_1(0)
    \end{equation*}
    \item \textbf{Unit circle}: The set defined as follows. \emph{Denoted by} $\pmb{\Ss}$. \emph{Given by}
    \begin{equation*}
        \Ss := \{z\in\C:|z|=\varepsilon\}
    \end{equation*}
    \item \textbf{Upper half plane}: The set defined as follows. \emph{Denoted by} $\pmb{\Hh}$. \emph{Given by}
    \begin{equation*}
        \Hh := \{z\in\C:\im z>0\}
    \end{equation*}
    \item $\bm{\pmb{\C}^*}$: The set defined as follows. \emph{Given by}
    \begin{equation*}
        \C^* := \C\setminus\{0\}
    \end{equation*}
    \item \marginnote{3/21:}\textbf{Neighborhood} (of $z_0$): A set $U$ which contains an $\varepsilon$-neighborhood.
    \item \textbf{Open} (set): A set that is a neighborhood of each of its points.
    \item \textbf{Closed} (set): A complement of an open set.
    \item \textbf{Interior} (of $M$): The largest open set contained in $M$. \emph{Denoted by} $\bm{\mathring{M}}$.
    \item \textbf{Closure} (of $M$): The smallest closed set containing $M$. \emph{Denoted by} $\bm{\overline{M}}$.
    \item \textbf{Topological boundary} (of $M$): The set defined as follows. \emph{Also known as} \textbf{boundary}. \emph{Denoted by} $\bm{\partial M}$. \emph{Given by}
    \begin{equation*}
        \partial M := \overline{M}\setminus\mathring{M}
    \end{equation*}
    \item \textbf{Relatively open} (set in $M$): The intersection of an open set $U$ with an arbitrary set $M$. \emph{Also known as} \textbf{open} (set in $M$).
    \item \textbf{Relatively closed} (set in $M$): The intersection of a closed set $U$ with an arbitrary set $M$. \emph{Also known as} \textbf{open} (set in $M$).
\end{itemize}




\end{document}