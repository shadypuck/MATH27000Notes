\documentclass[../notes.tex]{subfiles}

\pagestyle{main}
\renewcommand{\chaptermark}[1]{\markboth{\chaptername\ \thechapter\ (#1)}{}}
\setcounter{chapter}{3}

\begin{document}




\chapter{Extrema}
\section{Poles and Maximum Moduli}
\begin{itemize}
    \item \marginnote{4/9:}Announcement.
    \begin{itemize}
        \item Midterm next week in class.
        \item Material up through today, though probably not much on today's content.
    \end{itemize}
    \item Last time.
    \begin{itemize}
        \item Cauchy integral formula: If $U$ is a domain, $D\subset\subset U$, and $z\in D$, then
        \begin{equation*}
            f(z) = \frac{1}{2\pi i}\int_\gamma\frac{f(\zeta)}{\zeta-z}\dd{\zeta}
        \end{equation*}
        \item This implies Riemann's removable singularity theorem, which states that if $f\in\mO(U\setminus\{z_0\})$ and $f$ is bounded near $z_0$, then there exists a $\hat{f}\in\mO(U)$ which continues $f$ at $z_0$.
        \begin{itemize}
            \item Example: $\sin(z)/z\in\mO(\C^*)$ has a continuation to $\C$.
            \item In particular, take the Taylor series at zero and evaluate:
            \begin{equation*}
                \widehat{\frac{\sin(z)}{z}}(0) = 1-\frac{0^3}{3!}+\frac{0^5}{5!}-\cdots = 1
            \end{equation*}
        \end{itemize}
        \item Alternatively, if $f\in\mO(U\setminus\{z_0\})$ and $|f(z)|\to\infty$ as $z\to z_0$, then $z_0$ is a \textbf{pole} of $f$.
    \end{itemize}
    \item Today.
    \begin{itemize}
        \item Finish up what we couldn't last time.
        \item Say something about harmonic functions.
    \end{itemize}
    \item \textbf{Meromorphic} (function): A function $f:U\to\C$ such that $f\in\mO(U\setminus P)$ and each $p\in P$ is a pole, where $P\subset U$ is a finite set of points.
    \item Example: Consider $1/z\in\mO(\C^*)$.
    \begin{itemize}
        \item This has a pole at zero.
        \item Thus, $1/z$ is \emph{holomorphic} on the punctured plane $\C^*$, but \emph{meromorphic} on the whole complex plane $\C$.
    \end{itemize}
    \item Example: The same argument applies to $1/z^k$ ($k\in\N$).
    \item Example: The function from PSet 2, Q2c:
    \begin{equation*}
        f(z) = \frac{1}{z(z-1)(z-i)(z-1-i)}
    \end{equation*}
    \item It follows that
    \begin{equation*}
        \{f:f\text{ is holomorphic}\} \subset \{f:f\text{ is meromorphic}\}
    \end{equation*}
    \item Fact: All of the examples kind of look the same.
    \begin{itemize}
        \item More generally, suppose $f$ has a pole at $p$ and is holomorphic on $U\setminus\{p\}$. Pick a disk $D\ni p$ such that $f\neq 0$ on $D$. Then $g=1/f\in\mO(D\setminus\{p\})$ and as $z\to p$, $g(z)\to 0$.
        \item Thus, we've got a function that's holomorphic and bounded near a point, so by Riemann's removable singularity theorem, it has a unique holomorphic extension $\hat{g}\in\mO(D)$.
        \begin{itemize}
            \item In particular, $g(p)=0$.
        \end{itemize}
        \item Note: We do \emph{not} need to choose $D$ small enough such that it contains only one point in $P$. However, we will for the time being just to simplify things. The reason we can do this is because singularities --- as points of a finite set --- are isolated.
    \end{itemize}
    \item There exists a power series for $g$ about $p$ such that
    \begin{equation*}
        g(z) = \sum_{k=0}^\infty a_k(z-p)^k
    \end{equation*}
    \begin{itemize}
        \item We know that $a_0=0$ because $g(p)=0$.
        \item It can also happen such that some (or [potentially infinitely] many) of the remaining $a_i$ are zero.
        \begin{itemize}
            \item Example: if $f=1/z^3$, then $g=z^3$ and $a_i=0$ ($i>3$).
        \end{itemize}
        \item Now let $L$ be the largest natural number such that $a_i=0$ for all $0\leq i<L$.
        \begin{itemize}
            \item Because $a_0=0$, $L\geq 1$.
            \item Additionally, $a_L\neq 0$.
        \end{itemize}
        \item Then we can rewrite the power series as
        \begin{equation*}
            g(z) = (z-p)^Lh(z)
        \end{equation*}
        where\dots
        \begin{enumerate}
            \item $h(z)=\sum_{k=L}^\infty a_k(z-p)^{k-L}$;
            \item $h(p)\neq 0$ (and $h$ is nonzero near $p$).
        \end{enumerate}
        \item We say that $g$ has a \textbf{zero} (of order $L$ at $p$).
        \begin{itemize}
            \item Similarly, we say that $f$ has a \textbf{pole} (of order $L$ at $p$).
        \end{itemize}
        \item Thus,
        \begin{equation*}
            f(z) = \frac{1}{(z-p)^L}\frac{1}{h(z)}
        \end{equation*}
        where, moreover, $1/h\in\mO(D')$ for some smaller disk $D'$.
        \item Example: $1/(z^2+z)$ goes to $z(z+1)$.
        \item Takeaway: Near any pole $p$, $f$ must look like
        \begin{equation*}
            \frac{1}{(z-p)^L}\cdot\phi(z)
        \end{equation*}
        where $\phi$ is holomorphic around $p$.
        \begin{itemize}
            \item This implies that there exists a \textbf{Laurent series} expansion around any pole.
            \item In particular, near $p$,
            \begin{equation*}
                f(z) = \sum_{k=-L}^\infty a_k(z-p)^k
            \end{equation*}
        \end{itemize}
    \end{itemize}
    \item \textbf{Zero} (of order $L$ at $p$): A point $p$ of a holomorphic complex function $g$ such that $g(p)=0$ and $g(z)=(z-p)^Lh(z)$ where $h(p)\neq 0$.
    \item \textbf{Pole} (of order $L$ at $p$): A point $p$ of a holomorphic complex function $f$ such that $1/f(p)=0$ and $f(z)=1/(z-p)^Lh(z)$ where $h(p)\neq 0$.
    \item \textbf{Laurent series}: A power series including a finite number of negative coefficients. \emph{Given by}
    \begin{equation*}
        \sum_{k=-L}^\infty a_k(z-p)^k
    \end{equation*}
    \item TPS: Consider $\cot(z)=\cos(z)/\sin(z)$, which has a pole at zero. What is the order of the pole? What is the Laurent series?
    \begin{itemize}
        \item The pole is order 1.
        \begin{itemize}
            \item One way to see this is to observe how $\tan z$ has a nonzero tangent at 0, so $\tan z=z+\cdots$. Thus, we can only divide one $z$ out of its power series.
            \item Alternatively, we have
            \begin{equation*}
                \cot(z) = \frac{1}{z}\cdot\frac{z}{\sin(z)}\cdot\cos(z)
            \end{equation*}
            from which we can observe that $\cos(z)\in\mO(\C)$, and $\sin(z)/z\in\mO(\C)$ (at zero, the extension gives 1) so $z/\sin(z)$ is holomorphic near zero. Thus, we can define $\phi(z)=z\cos(z)/\sin(z)$.
            \begin{itemize}
                \item What if we tried $\tilde{\phi}(z)=z^2\cos(z)/\sin(z)$? What's different? Well, $\tilde{\phi}$ is still holomorphic, but $\tilde{\phi}(0)=0$, which is a problem. Notice that $\phi(0)=1$!
            \end{itemize}
            \item As a last way, we could investigate the power series of $\cot(z)^{-1}=\tan(z)$ directly:
            \begin{equation*}
                \tan z = z+\frac{z^3}{3}+\frac{2z^5}{15}
            \end{equation*}
        \end{itemize}
        \item The Laurent series was not discussed in class, but here's some comments.
        \begin{itemize}
            \item It would begin from $k=-1$.
            \item We could construct it from the power series for cosine and sine using Calderon's formula above.
            \item Figuring out the formula for the power series of an inverted power series is a good exercise!!
        \end{itemize}
    \end{itemize}
    \item What if $|f(z)|\to\infty$ as $|z|\to\infty$? Then we say that $f$ has a \textbf{pole} (at $\infty$).
    \begin{itemize}
        \item Otherwise, there exist sequences $z_n\to\infty$ and $w_n\to\infty$ such that $f(z_n)\to\infty$ and $f(w_n)$ stays bounded. This is an \textbf{essential singularity} (at $\infty$).
        \item We can mull over this until Thursday when we introduce the solution, the \textbf{Riemann sphere}.
        \item If $f(z)$ stays bounded, then $f$ has a \textbf{removable singularity} (at $\infty$).
    \end{itemize}
    \item \textbf{Pole} (at $\infty$): A function $f$ such that $|f(z)|\to\infty$ as $|z|\to\infty$.
    \item \textbf{Essential singularity} (at $\infty$): A function $f$ for which there exist sequences $z_n\to\infty$ and $w_n\to\infty$ such that $f(z_n)\to\infty$ and $f(w_n)$ stays bounded.
    \item \textbf{Removable singularity} (at $\infty$): A function $f$ that stays bounded as $|z|\to\infty$.
    \item We're now going to switch to a completely different topic.
    \item Suppose $f\in\mO(U)$. When does $|f(z)|$ get the biggest? Equivalently, where does $|f(z)|$ take a local max? \emph{Hint}: Look at the Cauchy integral formula!
    \begin{itemize}
        \item There are no such points, at least on the interior of $U$!
    \end{itemize}
    \item Theorem (maximum modulus principle): Let $f\in\mO(U)$. If $|f(z)|$ has a local maximum on $U$, then $f$ is constant.
    \begin{proof}
        Let $z_0$ be a local maximum of $|f(z)|$. Pick $D\ni z_0$ small enough such that $|f(z)|\leq|f(z_0)|$ for all $z\in D$. Let $r$ be the radius of $D$. Now invoking the CIF,
        \begin{align*}
            |f(z_0)| &= \left| \frac{1}{2\pi i}\int_{\partial D}\frac{f(\zeta)}{\zeta-z_0}\dd\zeta \right|\\
            &= \frac{1}{2\pi}\left| \int_{\partial D}\frac{f(\zeta)}{\zeta-z_0}\dd\zeta \right|\\
            &\leq \frac{1}{2\pi}\int_{\partial D}\left| \frac{f(\zeta)}{\zeta-z_0} \right|\dd\zeta\\
            &= \frac{1}{2\pi}\int_0^{2\pi}\left| \frac{f(z_0+r\e[i\theta])}{r\e[i\theta]}\cdot ir\e[i\theta] \right|\dd\theta\\
            &= \frac{1}{2\pi}\int_0^{2\pi}|f(z_0+r\e[i\theta])|\dd\theta\\
            &\leq \frac{1}{2\pi}\cdot 2\pi\cdot\max_{\partial D}|f(\zeta)|\\
            &= \max_{\partial D}|f(\zeta)|\\
            &\leq |f(z_0)|
        \end{align*}
        But since the above inequality begins and ends with the same value, all $\leq$'s must be $=$'s. Thus, in particular,
        \begin{align*}
            \frac{1}{2\pi}\int_0^{2\pi}|f(z_0+r\e[i\theta])|\dd\theta &= |f(z_0)|\\
            \frac{1}{2\pi}\int_0^{2\pi}\big( |f(z_0+r\e[i\theta])|-|f(z_0)| \big)\dd\theta &= 0
        \end{align*}
        Combining this with the fact that the above integrand is always $\leq 0$ because $f(z_0)$ is a local maximum, we have that
        \begin{align*}
            |f(z_0+r\e[i\theta])|-|f(z_0)| &= 0\\
            |f(\zeta)| &= |f(z_0)|
        \end{align*}
        on $\partial D$. Note that this is true for all small $\partial D$'s centered at $z_0$.\par
        Now since $|f|$ is constant on $\partial D$, we must have that $|f|^2=f\cdot\bar{f}$ is constant on $\partial D$. Taking the Wirtinger derivative and using its product rule gets us
        \begin{equation*}
            0 = \pdv{z}(f\cdot\bar{f}) = f_z\cdot\bar{f}+f\cdot\bar{f}_z
        \end{equation*}
        Since $f$ is holomorphic (hence satisfies the CR equations) and $f_{\bar{z}}=\bar{f}_z$, we have that
        \begin{equation*}
            \bar{f}_z = f_{\bar{z}} = 0
        \end{equation*}
        Thus,
        \begin{equation*}
            0 = f_z\cdot\bar{f}+f\cdot 0 = f_z\cdot\bar{f}
        \end{equation*}
        By the zero-product property, either $f_z=0$ and $\bar{f}=0$. In the first case, this means that $f$ is constant, as desired. In the second case, this means that $f$ is zero (and hence constant), as desired.\par
        At this point, we have shown that $f$ is constant on a small disk. Therefore, we need only invoke the identity theorem, which tells us that since the function is constant for a little bit somewhere, it is constant everywhere.
    \end{proof}
    \item Another way to prove this is by considering the derivative of the Cauchy integral formula and where it's equal to zero.
    \item Corollary (minimum modulus principle): If $f\in\mO(U)$, $f\neq 0$ on $U$ (hence $1/f\in\mO(U)$), and $|f(z)|$ takes a minimum in $U$, then $f$ is constant.
    \item Application of the maximum modulus principle (the fundamental theorem of algebra): If $p$ is a polynomial of degree $d$ in $\C$, then $p$ has $d$ roots in $\C$ (counted with multiplicity).
    \begin{proof}
        Suppose inductively that $d\geq 1$.\par
        Step 1 (show that there exists one root): Suppose for the sake of contradiction that $p$ has no zeros. Since $p$ is a polynomial, we know that $|p(z)|\to\infty$ as $|z|\to\infty$. Thus, there exists $R>0$ such that for all $z$ with $|z|>R$, $|p(z)|\geq|p(0)|$. Then $|p(z)|$ must take a minimum on $\overline{D_R}$. But to keep $p$ from being constant by the minimum modulus principle, the minimum has to be on $\partial D_R$. Now take a slightly bigger disk; our global minimum is now in the interior, so $p$ is constant, a contradiction. It follows that $p$ must have a zero in $D_R$.\par
        Step 2: Suppose $p$ has a root at $z_0$. Then power series for $p$ at $z_0$ is $p(z)=(z-z_0)p_1(z)$. $p_1$ is a polynomial of degree $d-1$.\par
        Step 3: Now iterate to find that $p$ is a product of monomials.
    \end{proof}
    \item Algebraists love to prove this with only algebra, but in reality, the proof is complex analysis.\footnote{How did this proof work??}
    \item We did not get to say something about harmonic functions today, but Calderon will leave the content in his notes in case we want to look at it.
    \begin{itemize}
        \item The statement: Harmonic functions follow a version of the CIF.
        \item There's a related PSet problem.
    \end{itemize}
\end{itemize}



\section{Modulus Principles and Harmonic Functions}
\begin{itemize}
    \item \marginnote{4/11:}Last time.
    \begin{itemize}
        \item Maximum modulus principle: If $f\in\mO(U)$, $f(z)\neq 0$ for all $z\in U$, and $|f|$ takes a max inside $U$, then $f$ is constant.
        \begin{itemize}
            \item Analogous result: The minimum modulus principle.
        \end{itemize}
        \item This result implies the fundamental theorem of algebra.
        \begin{itemize}
            \item Proof idea: $|f(z)|<|f(\zeta)|$ for $\zeta\in\partial D$, so $f$ must have a zero.
        \end{itemize}
    \end{itemize}
    \item Another corollary: We have a better understanding of the mapping properties of holomorphic functions.
    \item Recall that conformal (angle-preserving diffeomorphism) iff biholomorphic (bijective, $f,f^{-1}$ holomorphic).
    \item In real analysis, we have the \textbf{inverse function theorem}.
    \item \textbf{Inverse function theorem}: If $f:\R^2\to\R^2$ is $C^1$ and $Df(x)\neq 0$, then $f$ is locally a diffeomorphism from $x\in U$ to $V\ni f(x)$.
    \item So for $f\in\mO(U)$\dots
    \begin{itemize}
        \item If $f'$ is never 0 on $U$, then $f(U)$ is open;
        \item If $f'$ is never 0 on $U$ and $f:U\to f(U)$ is a bijection, then $f$ is biholomorphic.
    \end{itemize}
    \item Claim: The "if $f'$ is never 0 on $U$" condition is actually unnecessary!
    \item Theorem: Let $f\in\mO(U)$.
    \begin{enumerate}
        \item Open mapping theorem: If $f$ is nonconstant, then $f(U)$ is open.
        \begin{proof}
            To prove that $f(U)$ is open, it will suffice to show that every $w_0\in f(U)$ is contained in some neighborhood that's a subset of $f(U)$. Let $w_0=f(z_0)$. Pick a disk $D\subset\subset U$ such that $f(z)-w_0\neq 0$ on $\partial D$; this is possible because the zeroes of a nonconstant holomorphic function (like $f-w_0$) must be isolated, or otherwise $f$ would be constant. Thus, we may define the positive number
            \begin{equation*}
                \delta := \inf_{z\in\partial D}|f(z)-w_0|
            \end{equation*}
            Now pick $w$ such that $|w-w_0|<\delta/2$. Then by the triangle inequality, we have that for all $z\in\partial D$,
            \begin{equation*}
                |f(z)-w| \geq |f(z)-w_0|-|w-w_0|
                \geq \delta-|w-w_0|
                > \delta-\frac{\delta}{2}
                = \frac{\delta}{2}
            \end{equation*}
            This means that $|f-w|$ is always greater than the number $\delta/2$ on the boundary of $D$, but since
            \begin{equation*}
                |f(z_0)-w| = |w-w_0|
                < \frac{\delta}{2}
            \end{equation*}
            $|f-w|$ does not obtain its minimum on the boundary of $D$. Thus, since all other hypotheses of the minimum modulus principle are satisfied, there must be a zero of $|f-w|$ on $U$. This means that there exists a $z\in U$ such that $f(z)=w$, and hence $w\in f(U)$. Therefore, since the choice of $w\in D_{\delta/2}(w_0)$ was arbitrary, we know that $D_{\delta/w}(w_0)\subset f(U)$, as desired.
        \end{proof}
        \item Complex inverse function theorem: If $f$ is bijective, it's biholomorphic.
        \begin{proof}
            % {\color{white}hi}
            % \begin{itemize}
            %     \item $Z:=\{z\in U\mid f'(z)=0\}$.
            %     \item $f:U\setminus Z\to f(U)\setminus f(Z)$ is biholomorphic.
            %     \begin{itemize}
            %         \item $f:\C\to\C$ is holomorphic: $(f:\R^2\to\R^2)\in C^\infty\subset C^1$.
            %         \item $Df(x)\neq 0$ at all $x\in U\setminus Z$ + IFT: $f$ is a diffeomorphism at all $x\in U\setminus Z$.
            %         \item Thus, $f^{-1}:f(U)\setminus f(Z)\to U\setminus Z$ is differentiable, and hence holomorphic.
            %         \item $f$ is bijective, $f$ is holomorphic, and $f^{-1}$ is holomorphic: $f$ is biholomorphic.
            %     \end{itemize}
            %     \item $f(U)$ is a domain.
            %     \begin{itemize}
            %         \item $f(U)$ is connected.
            %         \begin{itemize}
            %             \item $f$ is holomorphic: $f$ is continuous.
            %             \item $U$ is a domain: $U$ is connected.
            %             \item Theorem 9.11: $f(U)$ is connected.
            %         \end{itemize}
            %         \item $f(U)$ is open.
            %         \begin{itemize}
            %             \item $U$ is a domain.
            %             \item $f$ is bijective: $f$ is nonconstant.
            %             \item Open mapping theorem: $f(U)$ is open.
            %         \end{itemize}
            %     \end{itemize}
            %     \item $f(Z)$ is discrete in $f(U)$.
            %     \begin{itemize}
            %         \item $f$ is nonconstant: $f'$ is nonzero.
            %         \item If $Z$ were not discrete, then since $f'$ is holomorphic, $f'$ would be zero: $Z$ is discrete.
            %         \item Open mapping theorem: Each $z\in Z$ is surrounded by an open neighborhood containing no other elements of $Z$, that gets mapped to an open neighborhood. These open neighborhood images cannot coincide; otherwise $f$ would not be bijective. Thus, $f(Z)$ is discrete.
            %     \end{itemize}
            %     \item Thus, $f(U)\setminus f(Z)$ is a domain.
            %     \begin{itemize}
            %         \item Just punctured now, if it was not already.
            %     \end{itemize}
            %     \item $f^{-1}:f(U)\to U$ is continuous.
            %     \begin{itemize}
            %         \item Let $X\subset U$ be open.
            %         \item Open mapping theorem: $(f^{-1})^{-1}(X)=f(X)$ is open.
            %         \item Therefore, $f^{-1}$ is continuous.
            %     \end{itemize}
            %     \item $f^{-1}$ is bounded near $z\in f(Z)$.
            %     \begin{itemize}
            %         \item $f^{-1}$ is continuous: $f^{-1}$ maps bounded sets to bounded sets.
            %     \end{itemize}
            %     \item Riemann's removable singularity theorem: There exists an analytic continuation of $f^{-1}$ to $f(U)$.
            %     \item Therefore, $f$ is bijective and holomorphic and $f^{-1}$ is holomorphic, so $f$ is biholomorphic.
            % \end{itemize}


            Define the set
            \begin{equation*}
                Z := \{z\in U\mid f'(z)=0\}
            \end{equation*}
            of zeroes of $f'$. To prove that $f$ is biholomorphic, we will quickly show that $f:U\setminus Z\to f(U)\setminus f(Z)$ is biholomorphic and then build up to the point where we can use Riemann's removable singularity theorem to analytically continue this restriction. Let's begin.\par
            Since $f\in\mO(U)$ by hypothesis, $f\in C^\infty\subset C^1$. Additionally, by the definition of $z$, $Df(x)\neq 0$ at all $x\in U\setminus Z$. Thus, by the real inverse function theorem, $f$ is a diffeomorphism at all $x\in U\setminus Z$. Consequently, $f^{-1}:f(U)\setminus f(Z)\to U\setminus Z$ is differentiable, and hence holomorphic. This combined with the hypothesis that $f:U\setminus Z\to f(U)\setminus f(Z)$ is bijective and holomorphic implies that $f:U\setminus Z\to f(U)\setminus f(Z)$ is biholomorphic.\par
            Now the first part of the plan is complete. The next step involves building up to the point that we can apply Riemann's removable singularity theorem to $f^{-1}:f(U)\setminus f(Z)\to U\setminus Z$. To do so, we need only verify that $f(U)\setminus f(Z)$ is a domain and $f^{-1}$ is bounded near any $f(z)\in f(Z)$, since $f(z)\in f(Z)\subset f(U)$ by definition and we have just shown that $f^{-1}\in\mO(f(U)\setminus f(Z))$.\par
            First, we verify that $f(U)\setminus f(Z)$ is a domain. To do so, we begin by checking that $f(U)$ is a domain. Since $U$ is a domain (hence connected) and $f$ is holomorphic (hence continuous), Theorem 9.11\footnote{See MATH 16210 Honors Calculus II notes.} tells us that $f(U)$ is connected. Additionally, since $U$ is a domain (hence open) and $f$ is bijective (hence nonconstant), the open mapping theorem implies that $f(U)$ is open. But since $f(U)$ is connected and open, it must be a domain, as desired. Next, we check that $f(Z)$ is discrete in $f(U)$. Since $f$ is nonconstant (per the above), $f'$ is nonzero. It follows since $f'$ is holomorphic that $Z$ must be discrete (otherwise, $f'$ holomorphic would be zero on a nondiscrete set, and hence would be zero everywhere, a contradiction). Thus, every $z\in Z$ is contained in an open neighborhood $N_z\subset U$ disjoint from all other $N_{z'}$. It follows by the open mapping theorem that each $f(N_z)$ is an \emph{open} neighborhood of $f(z)$, and by the fact that $f$ is bijective that the set of $f(N_z)$ is pairwise disjoint. Thus, $f(Z)$ is discrete in $f(U)$, as desired. Therefore, $f(U)\setminus f(Z)$ is a (punctured) domain, as desired.\par
            Second, we verify that $f^{-1}$ is bounded near any $f(z)\in f(Z)$. To do so, we begin by checking that $f^{-1}:f(U)\to U$ is continuous. Let $X\subset U$ be open. Since $f$ is bijective, $(f^{-1})^{-1}(X)=f(X)$. By the open mapping theorem, $f(X)$ is open. Thus, by the open-set definition of continuity, $f^{-1}$ is continuous, as desired. But then since $f^{-1}$ is continuous, it maps compact sets to compact sets. Therefore, a closed and bounded neighborhood of $f(z)$ will maps to a closed and bounded neighborhood of $z$, as desired.\par
            At this point, we may invoke Riemann's removable singularity theorem to analytically continue $f^{-1}:f(U)\setminus f(Z)\to U\setminus Z$ to $f(U)$. Therefore, since $f:U\to f(U)$ is bijective and holomorphic by hypothesis and $f^{-1}:f(U)\to U$ is holomorphic, $f$ is biholomorphic by definition, as desired.
        \end{proof}
    \end{enumerate}
    \item Preview: There is also a geometric reason why $f\in\mO(U)$ with zeros can't be injective.
    \item So the maximum modulus principle gets us a lot, and in fact, these kinds of arguments can be used to say even more!
    \item Example: Where do $\re(f)$ and $\im(f)$ take their max?
    \item Recall that $h:U\to\R$ is \emph{harmonic} if $\Delta h=0$, where
    \begin{equation*}
        \Delta := \pdv[2]{x}+\pdv[2]{y}
        = 4\cdot\pdv{}{z}{\bar{z}}
    \end{equation*}
    \begin{itemize}
        \item Examples of harmonic functions: $f\in\mO(U)$, $\re(f)$, $\im(f)$.
        \item Nonexample: $|f|$ is not! Take $f(z)=z$; then $\Delta|f|=1/|z|$.
    \end{itemize}
    \item Where do harmonic functions take their maxima?
    \begin{itemize}
        \item This is essentially equivalent to asking about $\re(f)$ for the following reason.
    \end{itemize}
    \item A characterization: If $u:U\to\R$ is $C^2$ and harmonic where $U$ is convex, then there exists $f\in\mO(U)$ such that $u=\re(f)$.
    \begin{proof}
        Since $u$ is harmonic,
        \begin{equation*}
            0 = \Delta u
            = \pdv{\bar{z}}(\pdv{u}{z})
        \end{equation*}
        This means that $u_z$ is holomorphic! This combined with the fact that $U$ is convex (hence star-shaped) implies by the CIT that $\int_\gamma u_z\dd{z}=0$ for any closed loop $\gamma\subset U$. Thus, by the proposition associated with Figure \ref{fig:contLoopInteg}, there exists a primitive $g$ for $u_z$ on $U$. From here, it follows by the rules of complex differentiation that
        \begin{equation*}
            \pdv{z}(\re g) = \pdv{z}\left[ \frac{1}{2}(g+\bar{g}) \right]
            = \frac{1}{2}\pdv{g}{z}
            = \frac{1}{2}u_z
        \end{equation*}
        and
        \begin{equation*}
            \pdv{\bar{z}}(\re g) = \frac{1}{2}\overline{g_z}
            = \frac{1}{2}u_{\bar{z}}
        \end{equation*}
        Therefore, $u=\re(2g)+C$, as desired.
    \end{proof}
    \item Harmonic functions also satisfy a version of the Cauchy Integral Formula!
    \begin{itemize}
        \item Let $D$ be a disk centered at $z$ of radius $R$.
        \item Then
        \begin{align*}
            u(z) &= \re f(z)\\
            &= \re\left[ \frac{1}{2\pi i}\int_{\partial D}\frac{f(\zeta)}{\zeta-z}\dd\zeta \right]\\
            &= \frac{1}{2\pi}\im\left[ \int_0^{2\pi}i\cdot f(z+R\e[i\theta])\dd\theta \right]\\
            &= \frac{1}{2\pi}\int_0^{2\pi}\re f(z+R\e[i\theta])\dd\theta\\
            &= \frac{1}{2\pi}\int_0^{2\pi}u(z+R\e[i\theta])\dd\theta
        \end{align*}
        \item This is called the "mean value property for harmonic functions."
        \item On the PSet, we'll prove a version for any disk containing $z$ of radius $R$, namely
        \begin{equation*}
            u(z) = \int_0^{2\pi}u(\zeta)P_R(\zeta,z)\dd\theta
        \end{equation*}
        \begin{itemize}
            \item $P_R$ is the Poisson kernel defined by
            \begin{equation*}
                P_R(\zeta,z) = \frac{1}{2\pi}\re\left( \frac{\zeta+z}{\zeta-z} \right)
            \end{equation*}
            \item Note that by definition, the Poisson kernel is harmonic!
        \end{itemize}
    \end{itemize}
    \item Theorem (Maximum modulus principle for harmonic functions): Suppose $h:U\to\R$ is harmonic. If $h$ takes a local maximum (or minimum) at $z_0\in U$, then $h$ must be \emph{locally constant}, that is, constant in a neighborhood of $z_0$.
    \begin{proof}
        We use the same strategy as we did for the holomorphic version.\par
        In particular, suppose $z_0$ is a local maximum. Pick a disk $D_R(z_0)$ about $z_0$ such that $h(z_0)\geq h(z)$ for all $z\in D_R(z_0)$. Using our new CIF, we have that for all $r<R$,
        \begin{equation*}
            h(z_0) = \frac{1}{2\pi}\int_0^{2\pi}h(z_0+r\e[i\theta])\dd\theta
        \end{equation*}
        By a similar integrand argument to before (noting that $h$ is real, so we don't need absolute values), we can conclude that
        \begin{equation*}
            h(z_0) = h(z_0+r\e[i\theta])
        \end{equation*}
        for all $r<R$. Therefore, $h$ is constant on $D_R(z_0)$, as desired.
    \end{proof}
    \item Corollary: Suppose $f\in\mO(U)$. If $\re(f)$ or $\im(f)$ take a maximum in $U$, then $f$ must be everywhere constant.
    \item Corollary: If $U$ is bounded, then $h$ is either constant or takes its maximum and minimum on $\partial U$.
    \item Application: Dirichlet problem (on a disk).
    \begin{itemize}
        \item Let $U$ be a convex domain, and let $g$ be a function on $\partial U$. Does there exist a function $u$ such that $u=g$ on $\partial D$ and $\Delta u=0$ (i.e., $u$ is harmonic)?
        \item This is like finding a steady state for the heat equation.
        \item If $U$ is a disk, the answer is yes, and the function is unique!
        \item Existence.
        \begin{itemize}
            \item Set
            \begin{equation*}
                u(z) := \int_0^{2\pi}g(\zeta)P_R(\zeta,z)\dd\theta
            \end{equation*}
            \item Then
            \begin{equation*}
                \Delta_zu = \Delta_z\int_0^{2\pi}g(\zeta)P_R(\zeta,z)\dd\theta
                = \int_0^{2\pi}g(\zeta)\underbrace{\Delta_zP_R(\zeta,z)}_0\dd\theta
                = 0
            \end{equation*}
            \item Note that $\Delta_zP_R(\zeta,z)=0$ because $P_R$ is harmonic, as mentioned earlier.
            \item The only hard part here is showing that $u$ has a continuous extension to $\partial D_R$.
        \end{itemize}
        \item Uniqueness.
        \begin{itemize}
            \item Suppose that there exist two solutions $g_1,g_2$. Then $g_1-g_2$ is harmonic and $g_1-g_2=0$ on $\partial D$. But then by the maximum (and minimum) modulus principles, $g_1-g_2=0$ on $U$. Therefore, $g_1=g_2$ on $U$, as desired.
        \end{itemize}
    \end{itemize}
\end{itemize}




\end{document}