\documentclass[../notes.tex]{subfiles}

\pagestyle{main}
\renewcommand{\chaptermark}[1]{\markboth{\chaptername\ \thechapter\ (#1)}{}}
\stepcounter{chapter}

\begin{document}




\chapter{???}
\section{Office Hours}
\begin{itemize}
    \item \marginnote{3/25:}What exactly are the Wirtinger derivatives?
    \begin{itemize}
        \item The $\pdv*{z}$ and $\pdv*{\bar{z}}$ operators.
    \end{itemize}
    \item The initial definition of holomorphic is accurate. It's na\"{i}ve, but it works out.
    \item Noney: Non example.
    \begin{itemize}
        \item As in, we have some examples of holomorphic functions and then we have an example of a function that is \emph{not} holomorphic.
    \end{itemize}
    \item TPS: Think Pair Share.
    \item Met Panteleymon and helped him with partial fractions!
    \item The $\Delta$ notation does mean the same Laplacian as $\vec{\nabla}^2$ from Quantum Mechanics.
    \item Calderon is not related to Calder\'{o}n; he was Argentinian, Calderon is half-Filipino and has no accent on his name. Both Spanish colonies but that's it.
    \item We can do all of the problems except Problem 1 at this point.
    \begin{itemize}
        \item For this, though, we can just look up the definition of the complex sine function.
        \item We basically just need to know what $\sin(i)$ is and what sine looks like along the imaginary axis.
    \end{itemize}
\end{itemize}




\end{document}