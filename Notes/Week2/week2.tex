\documentclass[../notes.tex]{subfiles}

\pagestyle{main}
\renewcommand{\chaptermark}[1]{\markboth{\chaptername\ \thechapter\ (#1)}{}}
\stepcounter{chapter}

\begin{document}




\chapter{Consequences of Power Series}
\section{Office Hours}
\begin{itemize}
    \item \marginnote{3/25:}What exactly are the Wirtinger derivatives?
    \begin{itemize}
        \item The $\pdv*{z}$ and $\pdv*{\bar{z}}$ operators.
    \end{itemize}
    \item The initial definition of holomorphic is accurate. It's na\"{i}ve, but it works out.
    \item Noney: Non example.
    \begin{itemize}
        \item As in, we have some examples of holomorphic functions and then we have an example of a function that is \emph{not} holomorphic.
    \end{itemize}
    \item TPS: Think Pair Share.
    \item Met Panteleymon and helped him with partial fractions!
    \item The $\Delta$ notation does mean the same Laplacian as $\vec{\nabla}^2$ from Quantum Mechanics.
    \item Calderon is not related to Calder\'{o}n; he was Argentinian, Calderon is half-Filipino and has no accent on his name. Both Spanish colonies but that's it.
    \item We can do all of the problems except Problem 1 at this point.
    \begin{itemize}
        \item For this, though, we can just look up the definition of the complex sine function.
        \item We basically just need to know what $\sin(i)$ is and what sine looks like along the imaginary axis.
    \end{itemize}
\end{itemize}



\section{Power Series}
\begin{itemize}
    \item \marginnote{3/26:}Recall: We already know that\dots
    \begin{itemize}
        \item Polynomials are elements of $\mathcal{O}(\C)$;
        \item Rational functions $P(z)/Q(z)$ are elements of $\mathcal{O}(\C\setminus V(Q))$.
    \end{itemize}
    \item \textbf{Affine algebraic set}: The set of solutions in an algebraically closed field $K$ of a system of polynomial equations with coefficients in $K$. \emph{Also known as} \textbf{variety}. \emph{Denoted by} $\bm{V(f_1,\ldots,f_n)}$.
    \item Today, we want to determine how the other elementary functions behave over the complex numbers.
    \begin{itemize}
        \item Other functions we want: $\exp$, $\log$, $\sin$, $\cos$.
        \item We will do $\log$ later, but all the others today.
    \end{itemize}
    \item \textbf{Exponential function}: The complex function defined as follows. \emph{Denoted by} $\textbf{e}^{\bm{z}}$, $\textbf{exp}\bm{(z)}$. \emph{Given by}
    \begin{equation*}
        \e[z] = \exp(z) := \sum_{k=0}^\infty\frac{z^k}{k!}
    \end{equation*}
    \item Na\"{i}vely, this power series is just be a polynomial $P(z)\in\mathcal{O}(\C)$.
    \item More rigorously, however, we must specify which kind of convergence we mean for the power series.
    \begin{itemize}
        \item As one example, we could say that for all $z$,
        \begin{equation*}
            \e[z] = P(z) = \lim_{N\to\infty}\sum_{k=0}^N\frac{z^k}{k!}
        \end{equation*}
        \begin{itemize}
            \item This would be \textbf{pointwise convergence}.
        \end{itemize}
        \item But there's an issue: Pointwise convergence of functions doesn't preserve anything, e.g., continuity.
    \end{itemize}
    \item \textbf{Pointwise} (convergent $\{f_n\}$): A sequence of functions $f_n:\C\to\C$ such that for all $z\in\C$, we have $f_n(z)\to f(z)$.
    \item TPS: Come up with an example of a sequence of continuous functions $\{f_n\}$ that converges pointwise to $f$, such that the $f_n$ are all\dots
    \begin{enumerate}
        \item Continuous but $f$ is not;
        \begin{itemize}
            \item $f_n(x)=\arctan(nx)$.
            \item Converges to the sign function $f(x)=\sgn(x)$.
        \end{itemize}
        \item Odd but $f$ is not;
        \item Differentiable but $f$ is not.
        \begin{itemize}
            \item These last two cases were not discussed in class.
        \end{itemize}
    \end{enumerate}
    \item We now recall a few definitions and lemmas from real analysis.
    \item \textbf{Locally uniformly} (convergent $\{f_n\}$): A sequence of functions $f_n:U\to\C$ and a function $f:U\to\C$ such that for all compact $K\subset U$,
    \begin{equation*}
        \sup_{z\in K}|f_n(z)-f(z)| \to 0
    \end{equation*}
    \item Lemma: If $f_n\to f$ locally uniformly and the $f_n$ are continuous (or integrable), then so is $f$.
    \begin{itemize}
        \item This lemma is \emph{not} true if we sub in "differentiable!"
        \item See the Stone-Weierstrass theorem for suitable constraint.
    \end{itemize}
    \item Thus, to resolve the original question, we mean that $P_N(z)\to\exp(z)$ locally uniformly.
    \item Aside: Which functions have power series?
    \begin{itemize}
        \item Remember Taylor polynomials from Calc II? \textbf{Taylor's theorem} tells us which ones converge.
    \end{itemize}
    \item \textbf{Taylor's theorem}: If $f:\R\to\R$ is $C^{k+1}$ and $P_\alpha^k(x)$ is the $k^\text{th}$ Taylor polynomial about $\alpha\in\R$, then for all $\beta\in\R$, there exists some $x\in(\alpha,\beta)$ such that
    \begin{equation*}
        f(\beta)-P_\alpha^k(\beta) = \frac{(\beta-\alpha)^{k+1}}{(k+1)!}f^{(k+1)}(x)
    \end{equation*}
    \begin{itemize}
        \item Essentially a version of the mean value theorem (MVT) for higher-order derivatives.
        \item We can use the term of the right side of the equals sign above to get a bound on the error of the Taylor polynomial.
    \end{itemize}
    \item \textbf{Analytic} (function): A function $f:\R\to\R$ for which the Taylor polynomials converge (locally uniformly) to $f$.
    \item Non example: The $C^\infty$ function $f:\R\to\R$
    \begin{equation*}
        f(x) =
        \begin{cases}
            \e[-1/x^2] & x\neq 0\\
            0 & x=0
        \end{cases}
    \end{equation*}
    \begin{itemize}
        \item An excellent exercise in real analysis is to check that for all $k$, the Taylor polynomial about 0 is 0.
        \item If we take the Taylor polynomial at some point farther from zero, the polynomial will approximate $f$ well up until zero, but then it will "hit a wall."
        \begin{itemize}
            \item The point is that $f$ is decaying more rapidly toward 0 than any polynomial possibly could, so the polynomial just thinks it's seeing 0.
        \end{itemize}
    \end{itemize}
    \item \textbf{Absolutely} (locally uniformly convergent power series): A power series $P(z)=\sum_{k=0}^\infty a_kz^k$ for which $A_N:\C\to\R$ locally uniformly converges, where
    \begin{equation*}
        A_N(z) := \sum_{k=0}^N|a_kz^k|
    \end{equation*}
    \item Absolute local uniform convergence allows you to reorder the terms in the polynomial.
    \begin{itemize}
        \item It also explains why you cannot reorder the terms in the series $S=1+1-1+1-1+\cdots$, i.e., why manipulating the order allows you to get any number: This series $S$ does not converge absolutely!
        \item Formally, if $\sigma:\N\to\N$ is a permutation and $\sum^\infty a_k$ converges absolutely, then $\sum^\infty a_{\sigma(k)}$ converges.
    \end{itemize}
    \item Exercise: Show that
    \begin{equation*}
        \sum_{k=0}^\infty z^k \to \frac{1}{1-z}
    \end{equation*}
    converges absolutely locally uniformly on $\D=\{|z|<1\}$.
    \begin{proof}
        To prove this, we just have to show that $\sum^\infty|z|^k$ converges on $|z|<1$. But it does so converge because this latter series is just a standard real geometric series.
    \end{proof}
    \item This example generalizes somewhat into the following lemma.
    \item Lemma: Let $P(z)$ be a power series about 0. If there exists $z_1\neq 0$ such that $|a_kz_1^k|\leq M$ for all $k$, then $P(z)=\sum a_kz^k$ converges on the disk $|z|<|z_1|$.
    \begin{proof}
        Uses standard series convergence results from real analysis. May be in \textcite{bib:FischerLieb}??
    \end{proof}
    \item \textbf{Disk of convergence}: The largest disk centered at zero on which you converge.
    \item \textbf{Radius of convergence}: The radius of the disk of convergence.
    \item \textbf{Cauchy-Hadamard formula}: The radius of convergence is given by
    \begin{equation*}
        \text{rad} = (\limsup|a_k|^{1/k})^{-1}
    \end{equation*}
    \begin{itemize}
        \item We will be using this result on PSet 2.
        \item We will also be proving it there!
    \end{itemize}
    \item What are power series representations good for? Here's an example of how they can be applied to help with PSet 1, QA.4.
    \begin{itemize}
        \item Question: For $|a|<1$ and $\gamma(t)=\e[it]$ a parameterization of a closed loop oriented counterclockwise, compute
        \begin{equation*}
            \int_\gamma\frac{1}{z-a}\dd{z}
        \end{equation*}
        \item Answer:
        \begin{itemize}
            \item Since $|a|<1$, we know that on $\gamma$, $|a/\gamma(t)|<1$.
            \item Thus, we have that
            \begin{align*}
                \int_\gamma\frac{1}{z-a}\dd{z} &= \int_\gamma\frac{1}{z}\frac{1}{1-a/z}\dd{z}\\
                &= \int_\gamma\frac{1}{z}\sum_{k=0}^\infty\left( \frac{a}{z} \right)^k\dd{z}\\
                &= \int_\gamma\sum_{k=0}^\infty\frac{a^k}{z^{k+1}}\dd{z}\\
                &= \sum_{k=0}^\infty\int_\gamma\frac{a^k}{z^{k+1}}\dd{z}\\
                &= \cdots\\
                &= \int_\gamma\frac{1}{z}\dd{z}
            \end{align*}
            \item We have the second equality because the power series converges.
            \item We have the fourth equality because of the lemma about integrable $f_n$ and the fact that the power series converges.
            \item The dots indicate some more steps that we will need to work out for ourselves on PSet 1.
        \end{itemize}
    \end{itemize}
    \item Lemma (from real analysis): If $f_n\to f$ locally uniformly and $f_n'\to g$ locally uniformly, then $f$ is differentiable and $f'=g$.
    \begin{itemize}
        \item This is true for both differentiable and holomorphic functions.
    \end{itemize}
    \item Claim: This lemma implies that convergent power series are holomorphic.
    \begin{proof}
        If
        \begin{equation*}
            f_N = \sum_{k=0}^Na_kz^k
        \end{equation*}
        then
        \begin{equation*}
            f_N' = \sum_{k=0}^Nk\cdot a_kz^{k-1}
        \end{equation*}
        We want to show that $\{f_N'\}$ converges (locally absolutely uniformly). \textcite{bib:FischerLieb} do this by hand. We can also use the Cauchy-Hadamard formula, which we will do presently.\par
        Let's look at $\limsup(k\cdot a_k)^{1/k}$. But this is just equal to
        \begin{equation*}
            \limsup|k\cdot a_k|^{1/k} %= \limsup(k^{1/k}\cdot|a_k|^{1/k})
            \leq \limsup(|k|^{1/k})\cdot\limsup(|a_k|^{1/k})
            = 1\cdot\limsup(|a_k|^{1/k})
            = \limsup|a_k|^{1/k}
        \end{equation*}
        Moreover, equality holds because that $k^{1/k}$ factor just decays toward 1; think about how $k$ increases linearly and the $k^\text{th}$ root decays faster.
    \end{proof}
    \item Proposition: Any convergent power series is holomorphic (on its disk) and its derivative is also a power series with the same radius of convergence. It follows that power series are analytic functions and are $C^\infty$.
    \item Spoiler: Every holomorphic function is analytic.
    \item Corollary: Power series representations are unique.
    \begin{enumerate}
        \item If $P(z)=\sum a_kz^k$ is convergent, then
        \begin{equation*}
            a_k = \frac{1}{k!}P^{(k)}(0)
        \end{equation*}
        \item If $P(z)=0$ in a neighborhood of zero, then $a_k=0$ for all $k$.
        \item If $P(z)=Q(z)$ (where $Q(z)=\sum b_kz^k$) in a neighborhood of 0, then $a_k=b_k$ for all $k$.
    \end{enumerate}
    \item Let's now return to the exponential function, which got this whole discussion started.
    \item We now know that the definition
    \begin{equation*}
        \exp(z) = \sum_{k=0}^\infty\frac{z^k}{k!}
    \end{equation*}
    makes sense.
    \item By manipulating this power series, we can get lots of fun properties.
    \begin{enumerate}
        \item $\exp(z)=[\exp(z)]'$.
        \begin{itemize}
            \item We obtain this via term-by-term differentiability.
            \item This is just our favorite formula $\dv*{t}(\e[t])=\e[t]$ from calculus.
        \end{itemize}
        \item $\overline{\exp(z)}=\exp(\bar{z})$.
        \item $\exp(a+b)=\exp(a)\cdot\exp(b)$.
        \item $|\exp(z)|=\exp[\re(z)]$.
    \end{enumerate}
    \item \textbf{Complex cosine}: The complex function defined as follows. \emph{Denoted by} $\textbf{cos}\bm{(z)}$. \emph{Given by}
    \begin{equation*}
        \cos(z) := \frac{1}{2}(\e[iz]+\e[-iz])
    \end{equation*}
    \item \textbf{Complex sine}: The complex function defined as follows. \emph{Denoted by} $\textbf{sin}\bm{(z)}$. \emph{Given by}
    \begin{equation*}
        \sin(z) := \frac{1}{2i}(\e[iz]-\e[-iz])
    \end{equation*}
    \item \textbf{Complex hyperbolic cosine}: The complex function defined as follows. \emph{Denoted by} $\textbf{cosh}\bm{(z)}$. \emph{Given by}
    \begin{equation*}
        \cosh(z) := \cos(iz)
    \end{equation*}
    \item \textbf{Complex hyperbolic sine}: The complex function defined as follows. \emph{Denoted by} $\textbf{sinh}\bm{(z)}$. \emph{Given by}
    \begin{equation*}
        \sinh(z) := i\sin(iz)
    \end{equation*}
    \item We also have
    \begin{equation*}
        \e[iz] = \cos(z)+i\sin(z)
    \end{equation*}
    \begin{itemize}
        \item If $z$ is real and in $[0,2\pi]$, then this simplifies to Euler's formula
        \begin{equation*}
            \e[i\theta] = \cos(\theta)+i\sin(\theta)
        \end{equation*}
    \end{itemize}
    \item Calderon draws some mappings of the exponential function but doesn't linger on what's going on.
    \item These are the preliminaries; now, we'll dive into the meat of the course.
\end{itemize}



\section{Cauchy's Theorem}
\begin{itemize}
    \item \marginnote{3/28:}The last three classes have been real analysis with complex numbers; now we get into \emph{complex} analysis.
    \item \textbf{Domain}: A connected, open set $U\subset\C$.
    \item Recall.
    \begin{itemize}
        \item $\gamma:[a,b]\to\C$ is a piecewise $C^1$ curve.
        \item $f:\C\to\C$ is continuous.
        \item We define
        \begin{equation*}
            \int_\gamma f\dd{z} := \int_a^bf(\gamma(t))\cdot\gamma'(t)\dd{t}
        \end{equation*}
        \item FTC: If $f=F'$ (i.e., $F$ is a \textbf{primitive} of $f$) on a domain $U\subset\C$, then for all paths $\gamma$ in $U$,
        \begin{equation*}
            \int_\gamma f\dd{z} = F(\gamma(b))-F(\gamma(a))
        \end{equation*}
    \end{itemize}
    \item \textbf{Primitive} (of $f$): A differentiable function whose derivative is equal to the original function $f$. \emph{Also known as} \textbf{antiderivative}, \textbf{indefinite integral}. \emph{Denoted by} $\bm{F}$.
    \item Corollary to the FTC: If $f=F'$, then for any closed curve $\gamma$ in $U$,
    \begin{equation*}
        \int_\gamma f\dd{z} = 0
    \end{equation*}
    \begin{itemize}
        \item To see why this is true intuitively, look at an example such as $f(z)=1/z\in\mathcal{O}(\C^*)$, which doesn't have a primitive and
        \begin{equation*}
            \int_\gamma\frac{1}{z}\dd{z} \neq 0
        \end{equation*}
    \end{itemize}
    \item Example: Find a primitive of the convergent power series
    \begin{equation*}
        P(z) = \sum_{k=1}^\infty a_kz^k
    \end{equation*}
    \begin{itemize}
        \item Via term-by-term integration, we obtain
        \begin{equation*}
            \sum_{k=0}^\infty\frac{a_k}{k+1}z^{k+1}
        \end{equation*}
    \end{itemize}
    \item If $\gamma$ is any closed loop in the disk of convergence,
    \begin{equation*}
        \int_\gamma P(z)\dd{z} = 0
    \end{equation*}
    \begin{itemize}
        \item It follows since they are defined in terms of convergent power series that for all closed loops $\gamma$,
        \begin{equation*}
            \int_\gamma\e[z]\dd{z} = \int_\gamma\sin(z)\dd{z}
            = \int_\gamma\cos(z)\dd{z}
            = 0
        \end{equation*}
    \end{itemize}
    \item Question: When is there a primitive?
    \begin{itemize}
        \item $f:\R\to\R$ continuous always has a primitive by the FTC, specifically that defined by
        \begin{equation*}
            F(x) := \int_a^xf(t)\dd{t}
        \end{equation*}
        which is differentiable with $F'=f$.
    \end{itemize}
    \item Proposition: If $f:U\to\C$ is continuous and $\int_\gamma f\dd{z}=0$ for every closed loop in $U$, then $f$ has a primitive on $U$.
    \begin{proof}
        Let's try the most na\"{i}ve thing: The FTC. Consider a domain.
        \begin{figure}[H]
            \centering
            \begin{tikzpicture}[
                every node/.style=black,
                scale=1.3
            ]
                \footnotesize
                \draw [xscale=2,name path=U] plot[smooth cycle] coordinates {
                    ($(0:1)+(0.15*rand,0.3*rand)$)
                    ($(45:1.2)+(0.15*rand,0.3*rand)$)
                    % ($(60:1)+(0.15*rand,0.3*rand)$)
                    ($(90:1.2)+(0.15*rand,0.3*rand)$)
                    ($(135:1.2)+(0.15*rand,0.3*rand)$)
                    % ($(150:1)+(0.15*rand,0.3*rand)$)
                    ($(180:1)+(0.15*rand,0.3*rand)$)
                    ($(225:1.2)+(0.15*rand,0.3*rand)$)
                    % ($(240:1)+(0.15*rand,0.3*rand)$)
                    ($(270:1.3)+(0.15*rand,0.3*rand)$)
                    ($(315:1)+(0.15*rand,0.3*rand)$)
                    % ($(330:1)+(0.15*rand,0.3*rand)$)
                };
                \path [name path=Utrace] (0,0) -- (45:1.8);
                \path [name intersections={of=U and Utrace}] (intersection-1) node[above right=-1pt]{$U$};
                \draw [scale=0.3,xscale=2,xshift=1cm,yshift=0.5cm] plot[smooth cycle] coordinates {
                    ($(0:1)+(0.15*rand,0.3*rand)$)
                    ($(45:1)+(0.15*rand,0.3*rand)$)
                    % ($(60:1)+(0.15*rand,0.3*rand)$)
                    ($(90:1)+(0.15*rand,0.3*rand)$)
                    ($(135:1)+(0.15*rand,0.3*rand)$)
                    % ($(150:1)+(0.15*rand,0.3*rand)$)
                    ($(180:0.8)+(0.15*rand,0.3*rand)$)
                    ($(225:0.8)+(0.15*rand,0.3*rand)$)
                    % ($(240:1)+(0.15*rand,0.3*rand)$)
                    ($(270:1)+(0.15*rand,0.3*rand)$)
                    ($(315:1)+(0.15*rand,0.3*rand)$)
                    % ($(330:1)+(0.15*rand,0.3*rand)$)
                };
                \path [name path=slit] (-0.3,-0.3) -- ++(-1,-1);
                \draw [name intersections={of=U and slit}] (intersection-1) -- (-0.3,-0.3);
                \node at (-0.9,-0.3) {*};
        
                \fill [rex] (0,-0.6) coordinate (a) circle (2pt) node[below=2pt]{$a$};
                \fill [rex] (-1.3,0.4) coordinate (z) circle (2pt) node[left=1pt]{$z$};
                \fill [rex] (-0.7,0.1) coordinate (z') circle (1.5pt) node[above right=-1pt,yshift=-1pt]{$z'$};
        
                \begin{scope}[on background layer]
                    \draw [yex,thick,decoration={
                        markings,
                        mark=at position 0.4 with \arrow{>},
                        mark=at position 0.7 with {\node[below left]{$\gamma_1^{}$};}
                    },postaction={decorate}] plot[smooth] coordinates {(a) (0.1,-0.3) (-0.1,-0.1) (-0.4,-0.1) (-0.7,0.1) (-1.4,-0.2) (z)};
                    \draw [bly,thick,decoration={
                        markings,
                        mark=at position 0.5 with \arrow{>},
                        mark=at position 0.4 with {\node[above right=-2pt]{$\gamma_2^{}$};}
                    },postaction={decorate}] plot[smooth] coordinates {(a) (0.2,-0.4) (0.8,-0.4) (1.5,0) (0.8,0.7) (-0.5,0.4) (z)};
                    \draw [orx,decoration={
                        markings,
                        mark=at position 0.5 with \arrow{>},
                        mark=at position 0.5 with {\node[below left=-1pt]{$\delta$};}
                    },postaction={decorate}] (z) -- (z');
                \end{scope}
            \end{tikzpicture}
            \caption{Continuous and zero closed-loop integrals implies integrable.}
            \label{fig:contLoopInteg}
        \end{figure}
        Doesn't have to be simply connected; it can have a \textbf{hole}, \textbf{slit}, and/or \textbf{puncture}. Essentially, to define $F(z)$, choose $a\in U$ and $\gamma$ connecting $a$ and $z$ and define
        \begin{equation*}
            F(z) = \int_\gamma f\dd{z}
        \end{equation*}
        Claim: This definition is well-defined regardless of the choice of $a$ and $\gamma$. In particular, the integral is independent of choice of $\gamma$ because any two $\gamma$ can be paired into a closed loop, and we have by hypothesis that the integral over any closed loop is zero.\par
        We now need to show that $F$ is differentiable with $F'=f$. Take $z,z'$ close enough that they can be connected by a straight line path $\delta$. Consider
        \begin{equation*}
            \lim_{z'\to z}\frac{F(z')-F(z)}{z'-z}
        \end{equation*}
        Now we know that
        \begin{equation*}
            F(z')-F(z) = \int_\delta f\dd{z}
        \end{equation*}
        Let $\gamma:[0,1]\to\C$ be defined by $t\mapsto tz'+(1-t)z$; a parameterization we can choose arbitrarily. Then
        \begin{equation*}
            F(z')-F(z) = \int_\delta f\dd{z}
            = \int_0^1f[tz+(1-t)z']\cdot(z'-z)\dd{t}
        \end{equation*}
        % Therefore,
        % \begin{equation*}
        %     \frac{F(z')-F(z)}{z'-z} = \int_\delta f\dd{z}
        %     = \int_0^1f[tz+(1-t)z']\dd{t}
        % \end{equation*}
        so dividing both sides by $z'-z$ and taking the limit yields
        \begin{align*}
            \lim_{z'\to z}\frac{F(z')-F(z)}{z'-z} &= \lim_{z'\to z}\int_0^1f[tz+(1-t)z']\dd{t}\\
            &= \int_0^1\lim_{z'\to z}f(tz+z'-tz')\dd{t}\\
            &= \int_0^1f(tz+z-tz)\dd{t}\\
            &= \int_0^1f(z)\dd{t}\\
            &= f(z)\int_0^1\dd{t}\\
            % &= f(z)\cdot 1\\
            &= f(z)
        \end{align*}
        and we have everything we wanted.
    \end{proof}
    \item What allows us to interchange the limit and the integral in the final set of equations?
    \begin{itemize}
        \item Roughly speaking, uniform convergence.
    \end{itemize}
    \item \textbf{Star-shaped} (domain): A domain $U\subset\C$ for which there exists $a\in U$ such that for all $z\in U$, the segment $a\to z$ is in $U$.
    \begin{figure}[h!]
        \centering
        \vspace{-1em}
        \begin{tikzpicture}
            \footnotesize
            \draw [semithick,name path=U] plot[smooth cycle] coordinates {
                ($(0  :1)+(0.2*rand,0.2*rand)$)
                ($(30 :1)+(0.2*rand,0.2*rand)$)
                ($(60 :1)+(0.2*rand,0.2*rand)$)
                ($(90 :1)+(0.2*rand,0.2*rand)$)
                ($(120:1.1)+(0.2*rand,0.2*rand)$)
                ($(150:0.7)+(0.2*rand,0.2*rand)$)
                ($(180:0.7)+(0.2*rand,0.2*rand)$)
                ($(210:0.7)+(0.2*rand,0.2*rand)$)
                ($(240:1.1)+(0.2*rand,0.2*rand)$)
                ($(270:1)+(0.2*rand,0.2*rand)$)
                ($(300:1)+(0.2*rand,0.2*rand)$)
                ($(330:1)+(0.2*rand,0.2*rand)$)
            };
    
            \fill [rex] (0.3,-0.3) coordinate (a) circle (2pt) node[black,below=2pt]{$a$};
    
            \begin{scope}[on background layer]
                \foreach \x in {0,15,...,345} {
                    \path [name path=\x] (a) -- ++(\x:2);
                    \draw [rey,name intersections={of=U and \x}] (a) -- (intersection-1);
                }
            \end{scope}
        \end{tikzpicture}
        \vspace{-3em}
        \caption{Star-shaped domain.}
        \label{fig:starShapedDomain}
    \end{figure}
    \vspace{-1em}
    \begin{itemize}
        \item There are star-shaped regions that are not \textbf{convex}, such as the one in Figure \ref{fig:starShapedDomain}!
        \begin{itemize}
            \item Convex implies star-shaped, but not vice versa.
        \end{itemize}
        \item Examples of domains that are \emph{not} star-shaped.
        \begin{enumerate}
            \item The annulus of two circles.
            \item Puncturing the unit disk.
        \end{enumerate}
        \item Star-shaped implies \textbf{simply connected}.
        \item Star-shaped is nice because we don't have to check every single curve; see the following lemma.
    \end{itemize}
    \item Lemma: If $U$ is star-shaped and for every triangle with one vertex at $a$, we have $\int_\triangle f\dd{z}=0$, then $F$ has a primitive in $U$.
    \begin{figure}[h!]
        \centering
        \vspace{-1em}
        \begin{tikzpicture}
            \footnotesize
            \draw [semithick,name path=U] plot[smooth cycle] coordinates {
                ($(0  :1)+(0.2*rand,0.2*rand)$)
                ($(30 :1)+(0.2*rand,0.2*rand)$)
                ($(60 :1)+(0.2*rand,0.2*rand)$)
                ($(90 :1)+(0.2*rand,0.2*rand)$)
                ($(120:1.1)+(0.2*rand,0.2*rand)$)
                ($(150:0.7)+(0.2*rand,0.2*rand)$)
                ($(180:0.5)+(0.2*rand,0.2*rand)$)
                ($(210:0.7)+(0.2*rand,0.2*rand)$)
                ($(240:1.1)+(0.2*rand,0.2*rand)$)
                ($(270:1)+(0.2*rand,0.2*rand)$)
                ($(300:1)+(0.2*rand,0.2*rand)$)
                ($(330:1)+(0.2*rand,0.2*rand)$)
            };
    
            \fill [orx] (0.3,-0.3) coordinate (a) circle (2pt) node[black,below=2pt]{$a$};
    
            \begin{scope}[on background layer]
                \foreach \x in {0,15,...,345} {
                    \path [name path=\x] (a) -- ++(\x:2);
                    \draw [help lines,name intersections={of=U and \x}] (a) -- (intersection-1);
                }
    
                \fill [orx] ($(a)+(120:1)$) coordinate (b) circle (2pt);
                \fill [orx] ($(a)+(105:0.7)$) coordinate (c) circle (2pt);
                \draw [orx,very thick] (a) -- (b) -- (c) -- cycle;
            \end{scope}
        \end{tikzpicture}
        \vspace{-3em}
        \caption{A triangle in a star-shaped domain.}
        \label{fig:triangleStar}
    \end{figure}
    \vspace{-1em}
    \begin{proof}
        What should be our candidate for $F(z)$? Define
        \begin{equation*}
            F(z) = \int_\gamma f\dd{z}
        \end{equation*}
        where $\gamma$ is the line segment from $a\to z$ that we know exists because $U$ is star-shaped.\par
        We now have to show that $F$ is holomorphic with $F'=f$, but we just do this as before by constructing a "closed loop," except our closed loop this time will just be a triangle as drawn in Figure \ref{fig:triangleStar}.
    \end{proof}
    \item With these definitions, we now state and prove one of the two main theorems in this class.
    \item \textbf{Cauchy Integral Theorem}: Suppose $U$ is a star-shaped domain and $f:U\to\C$ is holomorphic. Then $\int_\gamma f\dd{z}=0$ for any closed loop $\gamma$ in $U$.
    \begin{itemize}
        \item Whereas the FTC says if you have an \emph{integral}, then the integral around a closed loop is zero. This theorem says that if you have a \emph{derivative}, then the integral around a closed loop is zero.
        \item This is Round 1 of the theorem. In round 2, we'll swap the "star-shaped" hypothesis for "simply connected."
    \end{itemize}
    \item Today we're at least going to prove this, and possibly look at an application, too. If we don't get to the application today, we'll see it next Tuesday.
    \item Proof idea: Prove that $f$ has a primitive.
    \begin{proof}
        In order to prove this theorem, we'll use the preceding lemma. Thus, all we need to show is that for every triangle with one vertex on the center of the star, $\int_\triangle f\dd{z}=0$. Since we only have to check this for \emph{triangles}, we can use a really lovely result called \textbf{Goursat's lemma}.
    \end{proof}
    \item \textbf{Goursat's lemma}: If $f$ is holomorphic in a neighborhood of a triangle including the interior, then $\int_\triangle f\dd{z}=0$.
    \begin{proof}
        Idea: Estimate the integral.\par
        \begin{figure}[h!]
            \centering
            \begin{subfigure}[b]{0.3\linewidth}
                \centering
                \begin{tikzpicture}
                    \fill [rey!15,xshift=-0.866cm,yshift=-0.5cm] (90:1) -- (210:1) -- (330:1) -- cycle;
                    \draw [rex,thick] (30:1) -- (150:1) -- (270:1) -- cycle;
        
                    \draw [thick] (90:2) -- (210:2) -- (330:2) -- cycle;
        
                    \draw [orx,semithick,->,shorten >=3pt,yshift=1cm] (90:0.7) -- (210:0.7) -- (330:0.7) -- (90:0.7);
                    \draw [orx,semithick,->,shorten >=3pt] (30:0.7) -- (150:0.7) -- (270:0.7) -- (30:0.7);
                    \draw [orx,semithick,->,shorten >=3pt,xshift=0.866cm,yshift=-0.5cm] (90:0.7) -- (210:0.7) -- (330:0.7) -- (90:0.7);
                    \draw [orx,semithick,->,shorten >=3pt,xshift=-0.866cm,yshift=-0.5cm] (90:0.7) -- (210:0.7) -- (330:0.7) -- (90:0.7);
                \end{tikzpicture}
                \caption{Picking $\triangle_1$.}
                \label{fig:goursata}
            \end{subfigure}
            \begin{subfigure}[b]{0.3\linewidth}
                \centering
                \begin{tikzpicture}
                    \fill [rey!15,xshift=-0.866cm,yshift=-0.5cm] (90:1) -- (210:1) -- (330:1) -- cycle;
                    \draw [rex,thick] (30:1) -- (150:1) -- (270:1) -- cycle;
                    \fill [bly!30,xshift=-0.866cm,yshift=-0.5cm] (30:0.5) -- (150:0.5) -- (270:0.5) -- cycle;
                    \draw [blx,thick,xshift=-0.866cm,yshift=-0.5cm] (30:0.5) -- (150:0.5) -- (270:0.5) -- cycle;
        
                    \draw [thick] (90:2) -- (210:2) -- (330:2) -- cycle;
                \end{tikzpicture}
                \caption{Picking $\triangle_2$.}
                \label{fig:goursatb}
            \end{subfigure}
            \begin{subfigure}[b]{0.3\linewidth}
                \centering
                \begin{tikzpicture}
                    \fill [rey!15,xshift=-0.866cm,yshift=-0.5cm] (90:1) -- (210:1) -- (330:1) -- cycle;
                    \draw [rex,thick] (30:1) -- (150:1) -- (270:1) -- cycle;
                    \fill [bly!30,xshift=-0.866cm,yshift=-0.5cm] (30:0.5) -- (150:0.5) -- (270:0.5) -- cycle;
                    \draw [blx,thick,xshift=-0.866cm,yshift=-0.5cm] (30:0.5) -- (150:0.5) -- (270:0.5) -- cycle;
                    \fill [orx!30,xshift=-0.866cm,yshift=-0.5cm,xshift=0.217cm,yshift=0.125cm] (30:0.25) -- (150:0.25) -- (270:0.25) -- cycle;
                    \draw [orx,thick,xshift=-0.866cm,yshift=-0.5cm,xshift=0.217cm,yshift=0.125cm] (30:0.25) -- (150:0.25) -- (270:0.25) -- cycle;
        
                    \draw [thick] (90:2) -- (210:2) -- (330:2) -- cycle;
                \end{tikzpicture}
                \caption{Picking $\triangle_3$.}
                \label{fig:goursatc}
            \end{subfigure}
            \caption{Proving Goursat's lemma.}
            \label{fig:goursat}
        \end{figure}
        Fix some $z_0\in\blacktriangle$ (the exact value will be determined later). We know that $f$ is holomorphic at $z_0$, which implies that there exists a linear approximation
        \begin{equation*}
            f(z) = \underbrace{f(z_0)+f'(z_0)\cdot(z-z_0)}_\text{linear}+E(z)\cdot(z-z_0)
        \end{equation*}
        where our error function $E(z)\to 0$ as $z\to z_0$. Now the underlined linear portion above is a (stupid) power series, but since it technically \emph{is} a "convergent power series," our previous results imply that it has primitives. In particular, its integral around a closed loop (like a triangle) will be zero. This means that
        \begin{equation*}
            \int_\triangle f\dd{z} = \underbrace{\int_\triangle[f(z_0)+f'(z_0)\cdot(z-z_0)]\dd{z}}_0+\int_\triangle E(z)\cdot(z-z_0)\dd{z}
            = \int_\triangle E(z)\cdot(z-z_0)\dd{z}
        \end{equation*}
        Goursat's idea: Choose a good $z_0$. To do this, we'll subdivide the original black triangle (see Figure \ref{fig:goursata}) by choosing midpoints and breaking it into four triangles. Keep using the counterclockwise orientation in all cases. All of the red segments get cancelled out from integrating in both directions, so
        \begin{equation*}
            \int_{\triangle_0}f\dd{z} = \sum\int_{4\text{ sub }\triangle\text{'s}}f\dd{z}
        \end{equation*}
        Choose $\triangle_1$ in first stage such that $|\int_{\triangle_1}f\dd{z}|$ is the greatest among the first stage sub-triangles. Thus,
        \begin{equation*}
            \left| \int_\triangle f\dd{z} \right| \leq 4\cdot\left| \int_{\triangle_1}f\dd{z} \right|
        \end{equation*}
        Now subdivide $\triangle_1$ and choose $\triangle_2$ the same way (see Figure \ref{fig:goursatb}), so that
        \begin{equation*}
            \left| \int_\triangle f\dd{z} \right| \leq 4\cdot\left| \int_{\triangle_1}f\dd{z} \right|
            \leq 4\cdot 4\cdot\left| \int_{\triangle_2}f\dd{z} \right|
        \end{equation*}
        Iterating this process, we obtain
        \begin{equation*}
            \left| \int_\triangle f\dd{z} \right| \leq 4^n\cdot\left| \int_{\triangle_n}f\dd{z} \right|
        \end{equation*}
        First thing to observe:
        \begin{align*}
            \len(\triangle_n) &= 2^{-n}\cdot\len(\triangle_0)&
            \diam(\triangle_n) &= 2^{-n}\cdot\diam(\triangle_0)
        \end{align*}
        Now fix $\varepsilon>0$ and take $n$ big enough such that on all of $\triangle_n$,
        \begin{equation*}
            |E(z)| < \frac{\varepsilon}{\len(\triangle_0)\cdot\diam(\triangle_0)}
        \end{equation*}
        Choose $z_0\in\bigcap_{n=1}^\infty\blacktriangle_n$. Then
        \begin{align*}
            \left| \int_\triangle f\dd{z} \right| &\leq 4^n\cdot\left| \int_{\triangle_n}f\dd{z} \right|\\
            &= 4^n\cdot\left| \int_{\triangle_n}E(z)\cdot(z-z_0)\dd{z} \right|\\
            &\leq 4^n\cdot\len(\triangle_n)\cdot\max_{\triangle_n}|E(z)\cdot(z-z_0)|\\
            &= 4^n\cdot\len(\triangle_n)\cdot\max|E(z)|\cdot\max|z-z_0|\\
            &\leq 4^n\cdot\len(\triangle_n)\cdot\diam(\triangle_n)\cdot\max|E(z)|\\
            &= 4^n\cdot 2^{-n}\len(\triangle_0)\cdot 2^{-n}\diam(\triangle_0)\cdot\max|E(z)|\\
            &= \len(\triangle_0)\cdot\diam(\triangle_0)\cdot\max|E(z)|\\
            &< \varepsilon
        \end{align*}
        Since we can choose $\varepsilon$ arbitrarily small, we can thus send the original integral of $f$ over $\triangle$ to zero.
    \end{proof}
    \item We now end class with an example of how complex analysis can be useful, even in calculus!
    \item Example: Evaluate the following \textbf{Dirichlet integral} using complex analysis.
    \begin{equation*}
        \int_0^\infty\frac{\sin(x)}{x}\dd{x}
    \end{equation*}
    \begin{itemize}
        \item We will do so via a focused analysis of the function $f:\C\to\C$ defined by
        \begin{equation*}
            z \mapsto \frac{\e[iz]}{z}
        \end{equation*}
        \begin{itemize}
            \item This function is not holomorphic everywhere, but it is on the punctured plane $\mathcal{O}(\C^*)$.
            \item However, we only need the upper half $\mathcal{O}(\Hh)$ presently.
        \end{itemize}
        \item More specifically, define $U$ to be a domain containing $\gamma$ as defined as in Figure \ref{fig:dirichletIntegral}.
        \begin{figure}[H]
            \centering
            \begin{tikzpicture}[
                every node/.style=black,
                text height=1.5ex,text depth=0.25ex
            ]
                \small
                \draw (-2.5,0) -- (2.5,0) node[right]{$\R$};
                \draw (0,-0.5) -- (0,2.5) node[above]{$i\R$};
        
                \node at (-0.7,1.5) {$\gamma$};
                \node at (-2.2,0.3) {$U$};
        
                \footnotesize
                \draw [rex,thick,decoration={
                    markings,
                    mark=at position 0.09 with \arrow{>},
                    mark=at position 0.09 with {\node[above=2pt]{$\gamma_1^{}$};},
                    mark=at position 0.19 with \arrow{>},
                    mark=at position 0.19 with {\node[above left=1pt]{$\gamma_2^{}$};},
                    mark=at position 0.33 with \arrow{>},
                    mark=at position 0.33 with {\node[above=2pt]{$\gamma_3^{}$};},
                    mark=at position 0.65 with \arrow{>},
                    mark=at position 0.65 with {\node[below]{$\gamma_4^{}$};}
                },postaction={decorate}] (-2,0) node[below]{$-R$}
                    -- (-0.3,0) node[below]{$-r$}
                    arc[start angle=180,end angle=0,radius=3mm] node[below]{$r$}
                    -- (2,0) node[below]{$R$}
                    arc[start angle=0,end angle=180,radius=2cm] -- cycle
                ;
            \end{tikzpicture}
            \caption{Dirichlet integral.}
            \label{fig:dirichletIntegral}
        \end{figure}
        % \item Now take $\int_\gamma f(z)\dd{z}$.
        % \item Consider the region of a half annulus.
        \item By the Cauchy integral theorem and our decomposition of $\gamma$,
        \begin{equation*}
            0 = \int_\gamma f(z)\dd{z} = \sum_{i=1}^4\int_{\gamma_i}f\dd{z}
        \end{equation*}
        \item We now integrate the segments one at a time.
        \begin{itemize}
            \item $\gamma_1$ and $\gamma_3$: Recalling our definition of $\sin(z)$ from last class, we have that
            \begin{align*}
                \int_{\gamma_1\gamma_3}\frac{\e[ix]}{x}\dd{x} &= \int_{-R}^{-r}\frac{\e[ix]}{x}\dd{x}+\int_r^R\frac{\e[ix]}{x}\dd{x}\\
                &= \int_{-r}^{-R}-\frac{\e[ix]}{x}\dd{x}+\int_r^R\frac{\e[ix]}{x}\dd{x}\\
                &= \int_r^R-\frac{\e[-ix]}{x}\dd{x}+\int_r^R\frac{\e[ix]}{x}\dd{x}\\
                &= \int_r^R\frac{\e[ix]-\e[-ix]}{x}\dd{x}\\
                &= 2i\int_r^R\frac{\sin(x)}{x}\dd{x}
            \end{align*}
            \item $\gamma_2$: We can explicitly compute this integral as $r\to 0$, using the parameterization $\gamma_2:[0,\pi]\to\C$ defined by $\theta\mapsto r\e[i(\pi-\theta)]$.
            \begin{align*}
                \lim_{r\to 0}\int_{\gamma_2}\frac{\e[iz]}{z}\dd{z} &= \lim_{r\to 0}\int_0^\pi\frac{\e[{ir\e[i(\pi-\theta)]}]}{r\e[i(\pi-\theta)]}\cdot -ir\e[i(\pi-\theta)]\dd{\theta}\\
                &= -i\lim_{r\to 0}\int_0^\pi\e[{ir\e[i(\pi-\theta)]}]\dd{\theta}\\
                &= -i\int_0^\pi\e[0]\dd{\theta}\\
                &= -i\pi
            \end{align*}
            \item $\gamma_4$: We need to bound the $R\e[i\theta]$ term as $R\to\infty$; see his notes!
            \begin{equation*}
                \int_0^\pi\e[{iR\e[i\theta]}]i\dd\theta \to 0
            \end{equation*}
        \end{itemize}
        \item Therefore, by transitivity,
        \begin{align*}
            2i\int_0^\infty\frac{\sin(x)}{x}\dd{x}-i\pi &= 0\\
            \int_0^\infty\frac{\sin(x)}{x}\dd{x} &= \frac{\pi}{2}
        \end{align*}
    \end{itemize}
\end{itemize}



\section{Office Hours}
\begin{itemize}
    \item PSet 1, QA.4: Are $a,b$ real or complex?
    \begin{itemize}
        \item They can be complex.
        \item Hint for this problem: Think about QA.3.
    \end{itemize}
    \item PSet 1, QB.2: As in, only "takes on" real values, i.e., is a function of the form $f:U\to\R$?
    \begin{itemize}
        \item Yes.
    \end{itemize}
    \item We have to give him a heads up before the PSet due date that we want to use a PSet extension.
\end{itemize}




\end{document}