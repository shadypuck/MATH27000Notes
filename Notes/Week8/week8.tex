\documentclass[../notes.tex]{subfiles}

\pagestyle{main}
\renewcommand{\chaptermark}[1]{\markboth{\chaptername\ \thechapter\ (#1)}{}}
\setcounter{chapter}{7}

\begin{document}




\chapter{???}
\section{Counting Zeroes and Laurent Series}
\begin{itemize}
    \item \marginnote{5/7:}Announcements.
    \begin{itemize}
        \item PSet 5 posted, due next Friday.
        \begin{itemize}
            \item 10 problems (a few more than usual), but heavily computational so shouldn't be that bad.
            \item After today's class, we'll have everything we need to do it.
        \end{itemize}
        \item Draft of final project report due this Friday.
        \begin{itemize}
            \item It doesn't need to be a full draft, but it should be a pretty well-fleshed-out outline. The more information we can give him, the better feedback he can give us.
        \end{itemize}
        \item Make sure to check out the Canvas post on final presentation scheduling.
    \end{itemize}
    \item Last time.
    \begin{itemize}
        \item We deduced the residue theorem from the general CIT.
        \begin{itemize}
            \item It states: Suppose $U$ is a domain, $S\subset U$ is a discrete (though not necessarily finite) set of singularities, $f\in\mO(U\setminus S)$, and $\Gamma=\sum c_i\gamma_i$ is nulhomologous in $U$ (though not necessarily nulhomologous in $U\setminus S$). Then
            \begin{equation*}
                \frac{1}{2\pi i}\int_\Gamma f\dd{z} = \sum_{s\in S}\wn(\Gamma,s)\res_sf
            \end{equation*}
        \end{itemize}
        \item The residue of $f$ about $s$ is
        \begin{equation*}
            \res_sf = \frac{1}{2\pi i}\int_{\partial D}f\dd{z}
        \end{equation*}
        where $D$ is some little disk about $s$ (and only this $s\in S$).
        \begin{itemize}
            \item If $s$ is a pole, then $\res_sf$ is also equal to the $a_{-1}$ coefficient of the Laurent expansion.
        \end{itemize}
        \item What the residue theorem gets us.
        \begin{itemize}
            \item Lets us compute integrals over complicated paths.
            \begin{itemize}
                \item We get to work with Laurent series instead of integrals, which is easier.
            \end{itemize}
            \item Lets us compute sums, as in the Basel problem.
            \begin{itemize}
                \item We choose a function, introduce the residue, and express the sum in terms of the integral.
            \end{itemize}
        \end{itemize}
    \end{itemize}
    \item Today.
    \begin{itemize}
        \item \textbf{The argument principle}: A theoretical (not practical) ramification of the residue theorem.
        \item Talking a bit more about essential singularities.
    \end{itemize}
    \item Consider the function $f'/f$, where $f$ is either holomorphic or meromorphic on $U$.
    \begin{itemize}
        \item Another way to think about this function is as the derivative of $\log f$.
        \item We want to investigate the singularities of $f'/f$.
        \item Suppose $f$ has a zero of order $k$ at $s$.
        \begin{itemize}
            \item Then locally, $f(z)=(z-s)^kg(z)$ where $g(s)\neq 0$.
            \item If $f$ looks like this at $s$, then $f'(z)=k(z-s)^{k-1}g(z)+(z-s)^kg'(z)$.
            \item Thus,
            \begin{equation*}
                \frac{f'}{f}(z) = \frac{k}{z-s}+\frac{g'(z)}{g(z)}
            \end{equation*}
            \item Since $g(s)\neq 0$, $g'/g$ is a well-defined number in a disk about $s$.
            \item The other term gives a well-defined pole.
            \item Therefore, if $f$ has a zero of order $k$ at $s$, then $f'/f$ has a simple pole at $s$ with $\res_s(f'/f)=k$.
        \end{itemize}
        \item Exercise: If $f$ has a pole of order $k$ at $s$, then $f'/f$ has a simple pole with $\res_s(f'/f)=-k$.
        \begin{itemize}
            \item Prove the same way (or see the notes).
        \end{itemize}
    \end{itemize}
    \item Corollary (the argument principle): Suppose $U$ is a domain, $f$ is meromorphic on $U$, and $\gamma$ is an SCC oriented counterclockwise that doesn't hit any poles or zeroes of $f$. Then
    \begin{equation*}
        \frac{1}{2\pi i}\int_\gamma\frac{f'}{f}\dd{z} = \sum_{s\in S}\res_s(f'/f)
        = \#\text{ zeroes in }\gamma-\#\text{ poles in }\gamma
    \end{equation*}
    \item \textbf{Simple closed curve}: A curve $\gamma$ that separates the inside from the outside. \emph{Also known as} \textbf{SCC}.
    \begin{itemize}
        \item See Figure \ref{fig:nulhomExa} for two examples.
    \end{itemize}
    \item TPS: Use the argument principle to compute the number of zeroes of the following polynomial inside the unit disk $\D$.
    \begin{equation*}
        f(z) = 2z^4-5z+2
    \end{equation*}
    \begin{itemize}
        \item Since $f$ is a polynomial, it has no poles.
        \item Thus, by the argument principle, the number of zeroes in $\partial\D$ is
        \begin{equation*}
            \frac{1}{2\pi i}\int_{\partial\D}\frac{f'}{f}\dd{z} = \frac{1}{2\pi i}\int_{\partial\D}\frac{8z^3-5}{2z^4-5z+2}\dd{z}
        \end{equation*}
        \item The integral on the right is "a pain in the butt" to compute, but we definitely could.
        \begin{itemize}
            \item We'd just have to do a partial fraction decomposition, substitute in $z=\e[i\theta]$, and bash it out.
        \end{itemize}
        \item A better way to find the number of zeroes uses \textbf{Rouch\'{e}'s theorem}, which we'll introduce shortly.
    \end{itemize}
    \item Understanding the argument principle geometrically.
    \emph{picture}
    \begin{itemize}
        \item Suppose you have an SCC $\gamma$ enclosing a zero or order 1 and a pole of order 1.
        \item We'll now investigate $f$ as a mapping of $\C\to\C$ and $\C\to\hat{\C}$.
        \item As a mapping into the complex plane, $f$ maps $\gamma$ to $f(\gamma)$.
        \begin{itemize}
            \item Note that the curve $f(\gamma)$ is just another mapping of the circle into the complex plane.
        \end{itemize}
        \item As a mapping into the Riemann sphere, $f$ maps the zero to 0 and the pole to $\infty$.
        \item Now draw little counterclockwise-oriented curves around the zero and pole.
        \begin{itemize}
            \item Since the pole has order 1, its little loop maps to a little loop around $\infty\in\hat{\C}$ that goes around 1 time.
            \begin{itemize}
                \item If the pole had order 2, for example, then a little loop that goes around it 1 time would map to a loop that goes around $\infty$ 2 times.
            \end{itemize}
            \item Similarly, since the zero has order 1, its little loop maps to a little loop around $0\in\hat{\C}$ that goes around 1 time.
        \end{itemize}
        \item Now pull the two loops down the Riemann sphere to the equator.
        \begin{itemize}
            \item Observe that their orientations are now inverses, with the orientation of the curve around $\infty$ having flipped.
            \begin{itemize}
                \item This is like the Coriolis effect!
            \end{itemize}
            \item Projecting the pulled-down curves into $\C$, we can observe that the one around $\infty$ is oriented clockwise and encompasses $f(\gamma)$ while the one around 0 is oriented counterclockwise and situated within $f(\gamma)$.
        \end{itemize}
        \item ??
        \item This all results in the order of zeros minus the order of poles is equal to the winding number of $f(\gamma)$ about zero.
    \end{itemize}
    \item Here's a computation that justifies all of the handwavey stuff above:
    \begin{align*}
        \wn(f(\gamma),0) &= \frac{1}{2\pi i}\int_{f(\gamma)}\frac{1}{z}\dd{z}\\
        &= \frac{1}{2\pi i}\int_0^1\frac{1}{f(\gamma(t))}[f(\gamma(t))]'\dd{t}\\
        &= \frac{1}{2\pi i}\int_0^1\frac{f'(\gamma(t))}{f(\gamma(t))}\gamma'(t)\dd{t}\\
        &= \frac{1}{2\pi i}\int_\gamma\frac{f'}{f}\dd{z}
    \end{align*}
    \begin{itemize}
        \item This computation is another proof of the argument principle!
    \end{itemize}
    \item Rouch\'{e}'s theorem: Let $U$ be a domain, $\gamma\subset U$ an SCC, and $f,g\in\mO(U)$. Suppose also that $|g|<|f|$ on $\gamma$. Then $f$ and $f+g$ have the same number of zeroes.
    \begin{proof}
        Set $h_\lambda(z):=f+\lambda g$ where $\lambda\in[0,1]$. Thus, $h_0=f$ and $h_1=f+g$. It follows by the argument principle that the number of zeroes of $h_\lambda$ in $\gamma$ (which is a discrete set) is
        \begin{equation*}
            \frac{1}{2\pi i}\int_\gamma\frac{h_\lambda'}{h_\lambda}\dd{z} = \frac{1}{2\pi i}\int_\gamma\frac{f'+\lambda g'}{f+\lambda g}\dd{z}
        \end{equation*}
        which is a continuous map in $\lambda$. Essentially, we have shown that the "number of zeroes in $\gamma$" function is a continuous function from $[0,1]\to\N_0$. Hence, as a continuous function into a discrete set, it is constant.\par
        Note that we used the $|g|<|f|$ condition to ensure that $f+\lambda g$ in the denominator of the above integral is never zero, and hence the integral is always well-defined.
    \end{proof}
    \item Example: Solving the TPS from earlier.
    \begin{itemize}
        \item $g(z)=2z^4$ has four zeroes inside $\D$.
        \item $f(z)=-5z$ has one zero inside $\D$.
        \item On $\partial\D$, we have that $|z|=1$ and hence
        \begin{equation*}
            2 = |g| < |f| = 5
        \end{equation*}
        so $2z^4-5z$ has the same number of zeroes as $-5z$ (or 1) by Rouch\'{e}'s theorem.
        \item Redefine $f(z)=2z^4-5z$ and $g(z)=2$.
        \item On $\partial\D$, we similarly have that
        \begin{equation*}
            |f| = |2z^4-5z|
            \geq \big| |2z^4|-|5z| \big|
            = |2-5|
            = 3
            > 2
            = |g|
        \end{equation*}
        so $2z^4-5z+2$ has the same number of zeroes as $2z^4-5z$ (or 1) by Rouch\'{e}'s theorem.
    \end{itemize}
    \item Takeaway: Whenever you're asked to compute zeroes, Rouch\'{e}'s theorem is probably the way to go.
    \item Exercise: Prove the FTA using Rouch\'{e}'s theorem.
    \begin{itemize}
        \item Idea: On big enough circles, eventually the top-degree term dominates.
    \end{itemize}
    \item Let's now talk a bit more about essential singularities.
    \item Suppose $U$ is a small disk, $s\in U$, and $f\in\mO(U\setminus S)$ (so $s$ is an isolated singularity). Then one of three things can happen.
    \begin{enumerate}
        \item $s$ is removable.
        \begin{itemize}
            \item In this case, we can remove it using Riemann's removable singularity theorem and get an analytic continuation $\hat{f}\in\mO(U)$.
            \item This is really nice, because then we get a power series near $s$:
            \begin{equation*}
                f(z) = \sum_{k=0}^\infty a_k(z-s)^k
            \end{equation*}
        \end{itemize}
        \item $s$ is a pole.
        \begin{itemize}
            \item We get a similar series called a Laurent series with:
            \begin{equation*}
                f(z) = \sum_{k=-N}^\infty a_k(z-s)^k
            \end{equation*}
        \end{itemize}
        \item $s$ is essential.
        \begin{itemize}
            \item We get a Laurent series
            \begin{equation*}
                f(z) = \sum_{k=-\infty}^\infty a_k(z-s)^k
            \end{equation*}
        \end{itemize}
    \end{enumerate}
    \item A word on convergence.
    \begin{itemize}
        \item Before, we used to say the magic words "absolutely locally uniformly" and we'd get convergence. Now we can't do that, and we need the following.
        \item Recall that uniform convergence $f_k\to f$ means that
        \begin{equation*}
            \sup_{z\in K}|f_k(z)-f(z)| \to 0
        \end{equation*}
        \item In addition, recall that local uniform convergence $f_k\to f$ means that
        \begin{equation*}
            \sup_{K\subset U}\sup_{z\in K}|f_k(z)-f(z)| \to 0
        \end{equation*}
    \end{itemize}
    \item Theorem: Suppose $\{f_k\}\in\mO(U\setminus S)$ is such that the $f_k\to f$ locally uniformly. Then $f\in\mO(U\setminus S)$ and moreover $f_k^{(n)}\to f^{(n)}$ for all $n$.
    \begin{proof}
        Goursat plus Morera for the first statement. CIF for the second statement. More detail in the notes; Calderon is also happy to talk.
    \end{proof}
    \item We now prove that such Laurent expansions exist.
    \begin{itemize}
        \item Step 1: "Pull off" the singular part.
        \begin{itemize}
            \item This is a theorem called the \textbf{Laurent decomposition}.
        \end{itemize}
        \item Step 2: Express $f_\infty$ as
        \begin{equation*}
            \sum_{k=-\infty}^{-1}a_k(z-s)^k
        \end{equation*}
    \end{itemize}
    \item Theorem (Laurent decomposition): There exists a unique $f_\infty\in\mO(\C\setminus\{s\})$ with $f=f_0+f_\infty$ such that $f_0\in\mO(U)$ and $f_\infty\to 0$ as $z\to\infty$.
    \begin{proof}
        Uniqueness is not interesting; see the book.\par
        Existence: Let $z\in U$ be arbitrary. Let $D_2$ be a counterclockwise-oriented curve in $U$ containing 0 and $z$. Let $D_1$ be a counterclockwise-oriented curve in $U$ containing just 0. Then $z\in D_2\setminus D_1$. Altogether, this looks like Figure \ref{fig:laurentDecomp}.
        \begin{figure}[H]
            \centering
            \begin{tikzpicture}[
                every node/.style=black
            ]
                \footnotesize
                \node [yex,label={[yshift=2mm]below:$0$}] {\normalsize *};
        
                \draw [orx,thick,decoration={
                    markings,
                    mark=at position 0.26 with \arrow{>},
                    mark=at position 0.85 with {\node[below right=-2pt]{$D_1$};}
                },postaction=decorate] circle (5mm);
                \draw [blx,thick,decoration={
                    markings,
                    mark=at position 0.255 with \arrow{>},
                    mark=at position 0.3 with {\node[above left=-2pt]{$D_2$};}
                },postaction=decorate] circle (1.1cm);
                \draw [dashed] (135:0.1) circle (1.5cm) node[right=1.5cm]{$U$};
        
                \fill [rex] (215:0.8) circle (1.5pt) node[below right=-1pt]{$z$};
            \end{tikzpicture}
            \caption{Laurent decomposition theorem.}
            \label{fig:laurentDecomp}
        \end{figure}
        Additionally, $\partial D_2-\partial D_1$ is nulhomologous in $U\setminus 0$. Thus, by the general CIF,
        \begin{equation*}
            f(z) = \frac{1}{2\pi i}\int_{\partial D_2-\partial D_1}\frac{f(\zeta)}{\zeta-z}\dd\zeta
            = \underbrace{\frac{1}{2\pi i}\int_{\partial D_2}\frac{f(\zeta)}{\zeta-z}\dd\zeta}_{f_0}-\underbrace{\frac{1}{2\pi i}\int_{\partial D_1}\frac{f(\zeta)}{\zeta-z}\dd\zeta}_{-f_\infty}
        \end{equation*}
        Finally, $-f_\infty$ --- as defined above --- has an extension from $U$ to all of $\C\setminus\{0\}$ just like in the proof of the general CIF from the 4/30 lecture.
    \end{proof}
    \item Comments on the Laurent decomposition.
    \begin{itemize}
        \item Think of $f_0$ as the nonnegative terms of the Laurent expansion, and $f_\infty$ as everything else (all the negative terms).
    \end{itemize}
    \item Step 2 proof.
    \begin{proof}
        For poles, we could just multiply by $(z-s)^k$. Here, we have to do something very clever. WLOG, let $s=0$. By step 1, $f_\infty$ extends to $\infty$ and we may set $f_\infty(\infty)=0$. Thus, let's think of $f_\infty\in\mO(\hat{\C}\setminus\{0\})$. Set $g(z)=f_\infty(1/z)$. Now $g$ has a removable singularity at 0. Moreover, $g\in\mO(\C)$. Then
        \begin{equation*}
            g(z) = \sum_{k=0}^\infty b_k(z-0)^k
        \end{equation*}
        converges for all $z$. But therefore
        \begin{equation*}
            f_\infty(z) = g(1/z)
            = \sum_{k=1}^\infty b_k\left( \frac{1}{z} \right)^k
            = \sum_{k=-\infty}^{-1}a_kz^k
        \end{equation*}
        as desired.
    \end{proof}
\end{itemize}




\end{document}