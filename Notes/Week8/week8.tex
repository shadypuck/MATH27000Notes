\documentclass[../notes.tex]{subfiles}

\pagestyle{main}
\renewcommand{\chaptermark}[1]{\markboth{\chaptername\ \thechapter\ (#1)}{}}
\setcounter{chapter}{7}

\begin{document}




\chapter{Applying the Complex Logarithm}
\section{Counting Zeroes and Laurent Series}
\begin{itemize}
    \item \marginnote{5/7:}Announcements.
    \begin{itemize}
        \item PSet 5 posted, due next Friday.
        \begin{itemize}
            \item 10 problems (a few more than usual), but heavily computational so shouldn't be that bad.
            \item After today's class, we'll have everything we need to do it.
        \end{itemize}
        \item Draft of final project report due this Friday.
        \begin{itemize}
            \item It doesn't need to be a full draft, but it should be a pretty well-fleshed-out outline. The more information we can give him, the better feedback he can give us.
        \end{itemize}
        \item Make sure to check out the Canvas post on final presentation scheduling.
    \end{itemize}
    \item Last time.
    \begin{itemize}
        \item We deduced the residue theorem from the general CIT.
        \begin{itemize}
            \item It states: Suppose $U$ is a domain, $S\subset U$ is a discrete (though not necessarily finite) set of singularities, $f\in\mO(U\setminus S)$, and $\Gamma=\sum c_i\gamma_i$ is nulhomologous in $U$ (though not necessarily nulhomologous in $U\setminus S$). Then
            \begin{equation*}
                \frac{1}{2\pi i}\int_\Gamma f\dd{z} = \sum_{s\in S}\wn(\Gamma,s)\res_sf
            \end{equation*}
        \end{itemize}
        \item The residue of $f$ about $s$ is
        \begin{equation*}
            \res_sf = \frac{1}{2\pi i}\int_{\partial D}f\dd{z}
        \end{equation*}
        where $D$ is some little disk about $s$ (and only this $s\in S$).
        \begin{itemize}
            \item If $s$ is a pole, then $\res_sf$ is also equal to the $a_{-1}$ coefficient of the Laurent expansion.
        \end{itemize}
        \item What the residue theorem gets us.
        \begin{itemize}
            \item Lets us compute integrals over complicated paths.
            \begin{itemize}
                \item We get to work with Laurent series instead of integrals, which is easier.
            \end{itemize}
            \item Lets us compute sums, as in the Basel problem.
            \begin{itemize}
                \item We choose a function, introduce the residue, and express the sum in terms of the integral.
            \end{itemize}
        \end{itemize}
    \end{itemize}
    \item Today.
    \begin{itemize}
        \item \textbf{The argument principle}: A theoretical (not practical) ramification of the residue theorem.
        \item Talking a bit more about essential singularities.
    \end{itemize}
    \item Consider the function $f'/f$, where $f$ is either holomorphic or meromorphic on $U$.
    \begin{itemize}
        \item Another way to think about this function is as the derivative of $\log f$.
        \item We want to investigate the singularities of $f'/f$.
        \item Suppose $f$ has a zero of order $k$ at $s$.
        \begin{itemize}
            \item Then locally, $f(z)=(z-s)^kg(z)$ where $g(s)\neq 0$.
            \item If $f$ looks like this at $s$, then $f'(z)=k(z-s)^{k-1}g(z)+(z-s)^kg'(z)$.
            \item Thus,
            \begin{equation*}
                \frac{f'}{f}(z) = \frac{k}{z-s}+\frac{g'(z)}{g(z)}
            \end{equation*}
            \item Since $g(s)\neq 0$, $g'/g$ is a well-defined number in a disk about $s$.
            \item The other term gives a well-defined pole.
            \item Therefore, if $f$ has a zero of order $k$ at $s$, then $f'/f$ has a simple pole at $s$ with $\res_s(f'/f)=k$.
        \end{itemize}
        \item Exercise: If $f$ has a pole of order $k$ at $s$, then $f'/f$ has a simple pole with $\res_s(f'/f)=-k$.
        \begin{itemize}
            \item Prove the same way (or see the notes).
        \end{itemize}
    \end{itemize}
    \item Corollary (the argument principle): Suppose $U$ is a domain, $f$ is meromorphic on $U$, and $\gamma$ is an SCC oriented counterclockwise that doesn't hit any poles or zeroes of $f$. Then
    \begin{equation*}
        \frac{1}{2\pi i}\int_\gamma\frac{f'}{f}\dd{z} = \sum_{s\in S}\res_s(f'/f)
        = \#\text{ zeroes in }\gamma-\#\text{ poles in }\gamma
    \end{equation*}
    \item \textbf{Simple closed curve}: A curve $\gamma$ that separates the inside from the outside. \emph{Also known as} \textbf{SCC}.
    \begin{itemize}
        \item See Figure \ref{fig:nulhomExa} for two examples.
    \end{itemize}
    \item TPS: Use the argument principle to compute the number of zeroes of the following polynomial inside the unit disk $\D$.
    \begin{equation*}
        f(z) = 2z^4-5z+2
    \end{equation*}
    \begin{itemize}
        \item Since $f$ is a polynomial, it has no poles.
        \item Thus, by the argument principle, the number of zeroes in $\partial\D$ is
        \begin{equation*}
            \frac{1}{2\pi i}\int_{\partial\D}\frac{f'}{f}\dd{z} = \frac{1}{2\pi i}\int_{\partial\D}\frac{8z^3-5}{2z^4-5z+2}\dd{z}
        \end{equation*}
        \item The integral on the right is "a pain in the butt" to compute, but we definitely could.
        \begin{itemize}
            \item We'd just have to do a partial fraction decomposition, substitute in $z=\e[i\theta]$, and bash it out.
        \end{itemize}
        \item A better way to find the number of zeroes uses \textbf{Rouch\'{e}'s theorem}, which we'll introduce shortly.
    \end{itemize}
    \item Understanding the argument principle geometrically.
    \begin{itemize}
        \item See \verb|IMG_9306.JPG| and \verb|IMG_9364.JPG| for some context.
        \item Suppose you have an SCC $\gamma$ enclosing a zero or order 2 and a pole of order 1.
        \item We'll now investigate $f$ as a mapping of $\C\to\C$ and $\C\to\hat{\C}$.
        \item As a mapping into the complex plane, $f$ maps $\gamma$ to $f(\gamma)$.
        \begin{itemize}
            \item Note that the curve $f(\gamma)$ is just another mapping of the circle into the complex plane.
        \end{itemize}
        \item As a mapping into the Riemann sphere, $f$ maps the zero to 0 and the pole to $\infty$.
        \item Now draw little counterclockwise-oriented curves around the zero and pole in the domain.
        \begin{itemize}
            \item Since the pole has order 1, its little loop maps to a little loop around $\infty\in\hat{\C}$ that goes around 1 time.
            \item Since the zero has order 2, its little loop maps to a little loop around $0\in\hat{\C}$ that goes around 2 times.
            \begin{itemize}
                \item This is because locally, the area near an order 2 zero looks like $z^2$ and the complex function $z^2$ rotates the complex plane around on top of itself twice. Thus, a single loop will get spun around twice, a double loop will get spun around 4 times, and so on.
            \end{itemize}
        \end{itemize}
        \item Now pull the two loops down the Riemann sphere to the equator.
        \begin{itemize}
            \item Observe that their orientations are now inverses, with the orientation of the curve around $\infty$ having flipped.
            \begin{itemize}
                \item This is like the Coriolis effect!
            \end{itemize}
            \item Projecting the pulled-down curves into $\C$, we can observe that the one around $\infty$ is oriented clockwise and encompasses $f(\gamma)$ while the one around 0 is oriented counterclockwise and situated within $f(\gamma)$.
        \end{itemize}
        \item This all results in the order of zeros minus the order of poles is equal to the winding number of $f(\gamma)$ about zero.
        \item See Week 9 office hours for more.
    \end{itemize}
    \item Here's a computation that justifies all of the handwavey stuff above:
    \begin{align*}
        \wn(f(\gamma),0) &= \frac{1}{2\pi i}\int_{f(\gamma)}\frac{1}{z}\dd{z}\\
        &= \frac{1}{2\pi i}\int_0^1\frac{1}{f(\gamma(t))}[f(\gamma(t))]'\dd{t}\\
        &= \frac{1}{2\pi i}\int_0^1\frac{f'(\gamma(t))}{f(\gamma(t))}\gamma'(t)\dd{t}\\
        &= \frac{1}{2\pi i}\int_\gamma\frac{f'}{f}\dd{z}
    \end{align*}
    \begin{itemize}
        \item This computation is another proof of the argument principle!
    \end{itemize}
    \item Rouch\'{e}'s theorem: Let $U$ be a domain, $\gamma\subset U$ an SCC, and $f,g\in\mO(U)$. Suppose also that $|g|<|f|$ on $\gamma$. Then $f$ and $f+g$ have the same number of zeroes.
    \begin{proof}
        Set $h_\lambda(z):=f+\lambda g$ where $\lambda\in[0,1]$. Thus, $h_0=f$ and $h_1=f+g$. It follows by the argument principle that the number of zeroes of $h_\lambda$ in $\gamma$ (which is a discrete set) is
        \begin{equation*}
            \frac{1}{2\pi i}\int_\gamma\frac{h_\lambda'}{h_\lambda}\dd{z} = \frac{1}{2\pi i}\int_\gamma\frac{f'+\lambda g'}{f+\lambda g}\dd{z}
        \end{equation*}
        which is a continuous map in $\lambda$. Essentially, we have shown that the "number of zeroes in $\gamma$" function is a continuous function from $[0,1]\to\N_0$. Hence, as a continuous function into a discrete set, it is constant.\par
        Note that we used the $|g|<|f|$ condition to ensure that $f+\lambda g$ in the denominator of the above integral is never zero, and hence the integral is always well-defined.
    \end{proof}
    \item Example: Solving the TPS from earlier.
    \begin{itemize}
        \item $g(z)=2z^4$ has four zeroes inside $\D$.
        \item $f(z)=-5z$ has one zero inside $\D$.
        \item On $\partial\D$, we have that $|z|=1$ and hence
        \begin{equation*}
            2 = |g| < |f| = 5
        \end{equation*}
        so $2z^4-5z$ has the same number of zeroes as $-5z$ (or 1) by Rouch\'{e}'s theorem.
        \item Redefine $f(z)=2z^4-5z$ and $g(z)=2$.
        \item On $\partial\D$, we similarly have that
        \begin{equation*}
            |f| = |2z^4-5z|
            \geq \big| |2z^4|-|5z| \big|
            = |2-5|
            = 3
            > 2
            = |g|
        \end{equation*}
        so $2z^4-5z+2$ has the same number of zeroes as $2z^4-5z$ (or 1) by Rouch\'{e}'s theorem.
    \end{itemize}
    \item Takeaway: Whenever you're asked to compute zeroes, Rouch\'{e}'s theorem is probably the way to go.
    \item Exercise: Prove the FTA using Rouch\'{e}'s theorem.
    \begin{itemize}
        \item Idea: On big enough circles, eventually the top-degree term dominates.
    \end{itemize}
    \item Let's now talk a bit more about essential singularities.
    \item Suppose $U$ is a small disk, $s\in U$, and $f\in\mO(U\setminus S)$ (so $s$ is an isolated singularity). Then one of three things can happen.
    \begin{enumerate}
        \item $s$ is removable.
        \begin{itemize}
            \item In this case, we can remove it using Riemann's removable singularity theorem and get an analytic continuation $\hat{f}\in\mO(U)$.
            \item This is really nice, because then we get a power series near $s$:
            \begin{equation*}
                f(z) = \sum_{k=0}^\infty a_k(z-s)^k
            \end{equation*}
        \end{itemize}
        \item $s$ is a pole.
        \begin{itemize}
            \item We get a similar series called a Laurent series with:
            \begin{equation*}
                f(z) = \sum_{k=-N}^\infty a_k(z-s)^k
            \end{equation*}
        \end{itemize}
        \item $s$ is essential.
        \begin{itemize}
            \item We get a Laurent series
            \begin{equation*}
                f(z) = \sum_{k=-\infty}^\infty a_k(z-s)^k
            \end{equation*}
        \end{itemize}
    \end{enumerate}
    \item A word on convergence.
    \begin{itemize}
        \item Before, we used to say the magic words "absolutely locally uniformly" and we'd get convergence. Now we can't do that, and we need the following.
        \item Recall that uniform convergence $f_k\to f$ means that
        \begin{equation*}
            \sup_{z\in K}|f_k(z)-f(z)| \to 0
        \end{equation*}
        \item In addition, recall that local uniform convergence $f_k\to f$ means that
        \begin{equation*}
            \sup_{K\subset U}\sup_{z\in K}|f_k(z)-f(z)| \to 0
        \end{equation*}
    \end{itemize}
    \item Theorem: Suppose $\{f_k\}\in\mO(U\setminus S)$ is such that the $f_k\to f$ locally uniformly. Then $f\in\mO(U\setminus S)$ and moreover $f_k^{(n)}\to f^{(n)}$ for all $n$.
    \begin{proof}
        Goursat plus Morera for the first statement. CIF for the second statement. More detail in the notes; Calderon is also happy to talk.
    \end{proof}
    \item We now prove that such Laurent expansions exist.
    \begin{itemize}
        \item Step 1: "Pull off" the singular part.
        \begin{itemize}
            \item This is a theorem called the \textbf{Laurent decomposition}.
        \end{itemize}
        \item Step 2: Express $f_\infty$ as
        \begin{equation*}
            \sum_{k=-\infty}^{-1}a_k(z-s)^k
        \end{equation*}
    \end{itemize}
    \item Theorem (Laurent decomposition): There exists a unique $f_\infty\in\mO(\C\setminus\{s\})$ with $f=f_0+f_\infty$ such that $f_0\in\mO(U)$ and $f_\infty\to 0$ as $z\to\infty$.
    \begin{proof}
        Uniqueness is not interesting; see the book.\par
        Existence: Let $z\in U$ be arbitrary. Let $D_2$ be a counterclockwise-oriented curve in $U$ containing 0 and $z$. Let $D_1$ be a counterclockwise-oriented curve in $U$ containing just 0. Then $z\in D_2\setminus D_1$. Altogether, this looks like Figure \ref{fig:laurentDecomp}.
        \begin{figure}[H]
            \centering
            \begin{tikzpicture}[
                every node/.style=black
            ]
                \footnotesize
                \node [yex,label={[yshift=2mm]below:$0$}] {\normalsize *};
        
                \draw [orx,thick,decoration={
                    markings,
                    mark=at position 0.26 with \arrow{>},
                    mark=at position 0.85 with {\node[below right=-2pt]{$D_1$};}
                },postaction=decorate] circle (5mm);
                \draw [blx,thick,decoration={
                    markings,
                    mark=at position 0.255 with \arrow{>},
                    mark=at position 0.3 with {\node[above left=-2pt]{$D_2$};}
                },postaction=decorate] circle (1.1cm);
                \draw [dashed] (135:0.1) circle (1.5cm) node[right=1.5cm]{$U$};
        
                \fill [rex] (215:0.8) circle (1.5pt) node[below right=-1pt]{$z$};
            \end{tikzpicture}
            \caption{Laurent decomposition theorem.}
            \label{fig:laurentDecomp}
        \end{figure}
        Additionally, $\partial D_2-\partial D_1$ is nulhomologous in $U\setminus 0$. Thus, by the general CIF,
        \begin{equation*}
            f(z) = \frac{1}{2\pi i}\int_{\partial D_2-\partial D_1}\frac{f(\zeta)}{\zeta-z}\dd\zeta
            = \underbrace{\frac{1}{2\pi i}\int_{\partial D_2}\frac{f(\zeta)}{\zeta-z}\dd\zeta}_{f_0}-\underbrace{\frac{1}{2\pi i}\int_{\partial D_1}\frac{f(\zeta)}{\zeta-z}\dd\zeta}_{-f_\infty}
        \end{equation*}
        Finally, $-f_\infty$ --- as defined above --- has an extension from $U$ to all of $\C\setminus\{0\}$ just like in the proof of the general CIF from the 4/30 lecture.
    \end{proof}
    \item Comments on the Laurent decomposition.
    \begin{itemize}
        \item Think of $f_0$ as the nonnegative terms of the Laurent expansion, and $f_\infty$ as everything else (all the negative terms).
    \end{itemize}
    \item Step 2 proof.
    \begin{proof}
        For poles, we could just multiply by $(z-s)^k$. Here, we have to do something very clever. WLOG, let $s=0$. By step 1, $f_\infty$ extends to $\infty$ and we may set $f_\infty(\infty)=0$. Thus, let's think of $f_\infty\in\mO(\hat{\C}\setminus\{0\})$. Set $g(z)=f_\infty(1/z)$. Now $g$ has a removable singularity at 0. Moreover, $g\in\mO(\C)$. Then
        \begin{equation*}
            g(z) = \sum_{k=0}^\infty b_k(z-0)^k
        \end{equation*}
        converges for all $z$. But therefore
        \begin{equation*}
            f_\infty(z) = g(1/z)
            = \sum_{k=1}^\infty b_k\left( \frac{1}{z} \right)^k
            = \sum_{k=-\infty}^{-1}a_kz^k
        \end{equation*}
        as desired.
    \end{proof}
\end{itemize}



\section{Riemann Mapping Theorem}
\begin{itemize}
    \item \marginnote{5/9:}Announcements.
    \begin{itemize}
        \item Sign up for a presentation time.
        \item Draft due tomorrow.
        \item Feedback forms due next week.
        \item Next Monday's OH's are TBD.
        \item Substitute teacher (postdoc) next Tuesday because Calderon will be out of town.
    \end{itemize}
    \item Last time: Using the logarithm to find some things out about zeroes.
    \item Today: Using the logarithm to find some things out about conformal maps.
    \item Recall.
    \begin{itemize}
        \item $f:U\to\C$ conformal means angle-preserving diffeomorphism.
        \item $f:U\to\C$ biholomorphic means bijective, holomorphic, and $f^{-1}$ holomorphic.
        \item Conformal iff biholomorphic.
        \item If $f\in\mO(U)$ is bijective and $f'$ is nonzero on $U$, then $f^{-1}$ is holomorphic and hence $f\in\Bihol(U)$.
        \begin{itemize}
            \item In fact, using the $f'(z)\neq 0$ for all $z\in U$ and the open mapping theorem to imply that $f^{-1}$ is holomorphic was unnecessary! The complex inverse function theorem proved that all we need for biholomorphism is a bijectivity and a holomorphicity condition.
        \end{itemize}
        \item Important note: Above, we used the bijectivity condition to imply a "nonzeroness" to the derivative. Suppose $f'(z_0)=0$ and $f(z_0)=0$. Then $f$ can't be bijective for the following reason.
        \begin{itemize}
            \item This combined with the hypothesis that $f\in\mO(U)$ and hence $f$ has a Taylor series expansion means that
            \begin{equation*}
                f(z) = \underbrace{f(z_0)}_0+\underbrace{f'(z_0)}_0(z-z_0)+\sum_{k=2}^\infty\frac{f^{(k)}(z_0)}{k!}(z-z_0)^k
                = \sum_{k=2}^\infty\frac{f^{(k)}(z_0)}{k!}(z-z_0)^k
            \end{equation*}
            \item Hence, we can factor $(z-z_0)^2$ out of $f$, so $f$ locally looks like $z^2$ and therefore is not injective. This is branching behavior like in Figure \ref{fig:exponentialSingularity}; $z^2$ cuts a slit in the complex plane at the positive real axis and spins it around and then back to where it started.
        \end{itemize}
    \end{itemize}
    \item We wish to investigate conformal, bounded maps on bounded, simply connected domains. Before we begin in earnest, we will show that we can simplify any function in this class to an analogous function in a more restricted class that will be easier to work with.
    \item Let $U$ be a bounded, simply connected domain. Suppose $f\in\mO(U)$ is conformal and bounded.
    \begin{figure}[H]
        \centering
        \begin{tikzpicture}[
            every node/.style=black
        ]
            \footnotesize
            \begin{scope}[
                xshift=-4cm,yshift=-2cm,
                scale=0.7
            ]
                \draw [blx,thick] circle (1cm) node[above left=4.3mm]{$\D$};
                \draw [thick] (0.1,0.9)
                    to[out=180,in=90] (-0.4,0.6)
                    to[out=-90,in=90] (-0.35,0.4)
                    to[out=-90,in=90] (-0.6,-0.4)
                    to[out=-90,in=180] (-0.2,-0.7)
                    to[out=0,in=-170] (0.4,-0.4)
                    to[out=10,in=-90] (0.7,-0.1)
                    to[out=90,in=-90] (0.5,0.4)
                    to[out=90,in=0] cycle
                ;
                \node [rex,label={[xshift=-7pt,yshift=7pt]below right:$z_0$}] at (-0.1,0.5) {\small *};
            \end{scope}
            \begin{scope}[
                xshift=-2cm,yshift=0.5cm,
                scale=1.1
            ]
                \draw [blx,thick] circle (1cm);
                \draw [thick] (0.1,0.9)
                    to[out=180,in=90] (-0.4,0.6) node[below left=-2pt]{$U$}
                    to[out=-90,in=90] (-0.35,0.4)
                    to[out=-90,in=90] (-0.6,-0.4)
                    to[out=-90,in=180] (-0.2,-0.7)
                    to[out=0,in=-170] (0.4,-0.4)
                    to[out=10,in=-90] (0.7,-0.1)
                    to[out=90,in=-90] (0.5,0.4)
                    to[out=90,in=0] cycle
                ;
                \fill [blx] circle (1.5pt) node[below]{$0$};
                \draw [blx,semithick] (0,0) -- node[pos=0.2,above]{$S$} (170:1);
            \end{scope}
            \begin{scope}[
                xshift=2cm,yshift=0.5cm,
                scale=1.5
            ]
                \draw [yex,thick] circle (1cm);
                \draw [thick] (0.6,0.65)
                    to[out=180,in=10] (-0.1,0.4) node[above,xshift=-1mm]{$f(U)$}
                    to[out=-170,in=90,in looseness=0.5] (-0.8,0.1)
                    to[out=-90,in=90] (-0.4,-0.1)
                    to[out=-90,in=90] (-0.45,-0.4)
                    to[out=-90,in=180] (-0.1,-0.8)
                    to[out=0,in=180] (0.3,-0.2)
                    to[out=0,in=-90] (0.7,0)
                    to[out=90,in=-90] (0.6,0.3)
                    to[out=90,in=0] cycle
                ;
                \fill [yex] circle (1.5pt) node[below]{$0$};
                \draw [yex,semithick] (0,0) -- node[pos=0.8,below]{$R$} (5:1);
            \end{scope}
            \begin{scope}[
                xshift=4cm,yshift=-2cm,
                scale=0.7
            ]
                \draw [yex,thick] circle (1cm) node[above right=4.3mm]{$\D$};
                \draw [thick] (0.6,0.65)
                    to[out=180,in=10] (-0.1,0.4)
                    to[out=-170,in=90,in looseness=0.5] (-0.8,0.1)
                    to[out=-90,in=90] (-0.4,-0.1)
                    to[out=-90,in=90] (-0.45,-0.4)
                    to[out=-90,in=180] (-0.1,-0.8)
                    to[out=0,in=180] (0.3,-0.2)
                    to[out=0,in=-90] (0.7,0)
                    to[out=90,in=-90] (0.6,0.3)
                    to[out=90,in=0] cycle
                ;
                \node [rex] at (-0.2,-0.6) {\small *};
            \end{scope}
            \begin{scope}[
                xshift=6.5cm,yshift=-2cm,
                scale=0.7
            ]
                \draw [yex,thick] circle (1cm);
                \draw [thick] (0.6,0.65)
                    to[out=180,in=10] (-0.1,0.5)
                    to[out=-170,in=90,in looseness=0.5] (-0.8,0.2)
                    to[out=-90,in=90] (-0.4,0)
                    to[out=-90,in=90] (-0.45,-0.2)
                    to[out=-90,in=180] (-0.1,-0.5)
                    to[out=0,in=180] (0.3,-0.1)
                    to[out=0,in=-90] (0.7,0.1)
                    to[out=90,in=-90] (0.6,0.4)
                    to[out=90,in=0] cycle
                ;
                \node [rex,label={[xshift=6pt,yshift=1pt]left:$0$}] at (0,0) {\small *};
            \end{scope}
    
            \draw [->] ($(-4,-2)!0.25!(-2,0.5)$) to[bend left=15] node[above left=-2pt]{$\times S$} ($(-4,-2)!0.63!(-2,0.5)$);
            \draw [->] ($(-2,0.5)!0.3!(2,0.5)$) to[bend left=15] node[above]{$f$} ($(-2,0.5)!0.6!(2,0.5)$);
            \draw [->] ($(2,0.5)!0.5!(4,-2)$) to[bend left=15] node[above right=-2pt]{$/R$} ($(2,0.5)!0.75!(4,-2)$);
            \draw [->] ($(-4,-2)!0.1!(4,-2)$) to[bend left=15] ($(-4,-2)!0.9!(4,-2)$);
            \draw [->] ($(4,-2)!0.33!(6.5,-2)$) to[bend left=15] node[above]{M\"{o}bius} ($(4,-2)!0.67!(6.5,-2)$);
        \end{tikzpicture}
        \caption{Constructing an analogous, simpler class of bounded conformal maps.}
        \label{fig:analogousConformal}
    \end{figure}
    \begin{itemize}
        \item Intuitively, these constraints imply that $f$ maps a blob to a blob.
        \begin{itemize}
            \item Conformal maps send simply connected domains to simply connected domains.
            \item Bounded maps send bounded domains to bounded domains, i.e., the image blob must lie in some disk.
        \end{itemize}
        \item By the maximum modulus principle, if $|f(z)|=R$, then $z\in\partial U$.
        \begin{itemize}
            \item This is valid because a conformal map must be nonconstant.
        \end{itemize}
        \item Now here comes the critical step: We can rescale the problem of a map from a blob of radius $S$ to a blob of radius $R$ to the problem of a map from a blob in $\D$ to another blob in $\D$.
        \begin{itemize}
            \item We do this by prescaling the initial $\D$ up by $S$ and postscaling the image disk down by $R$.
        \end{itemize}
        \item Therefore, via rescaling, we can reduce this problem to understanding maps from $U\subset\D\to\D$.
        \item Now fix $z_0\in U$. Recall from PSet 4 that we can use a M\"{o}bius transformation to take $f(z_0)\mapsto 0$.
        \begin{itemize}
            \item Fact/Exercise: A M\"{o}bius transformation $\D\to\D$ of the form $z\mapsto(az+b)/(cz+d)$ for $a,b,c,d\in\R$ acts transitively on points in $\D$ (i.e., any point can be taken to zero).
            \item It's kind of obvious if you do this with the upper half plane instead of the disk and then bring it to the disk.
        \end{itemize}
        \item Therefore, via rescaling \emph{and} M\"{o}bius transformations, we can reduce this problem to understanding maps from $U\subset\D\to\D$ that take $z_0\in U\mapsto 0$.
    \end{itemize}
    \item Let's begin investigating these maps.
    \item The first (and easiest) case we'll tackle is when $U=\D$. We'll tackle it with the \textbf{Schwarz Lemma}.
    \item Schwarz Lemma: Suppose $f:\D\to\D$ (not necessarily surjective) is conformal and $f(0)=0$. Then\dots
    \begin{enumerate}
        \item $|f(z)|\leq|z|$;
        \begin{proof}
            Consider the function defined by the following ratio.
            \begin{equation*}
                \frac{f(z)}{z}
            \end{equation*}
            This function is holomorphic for all $z\neq 0$. But what happens at zero? We know that $f(0)=0$ by hypothesis. Thus, $f$ has a power series expansion at zero. In particular, since $a_0=f(0)=0$,
            \begin{equation*}
                f(z) = a_1z+a_2z^2+\cdots
            \end{equation*}
            so
            \begin{equation*}
                \frac{f(z)}{z} = a_1+a_2z+\cdots
            \end{equation*}
            This means that $f(z)/z$ has a removable singularity which we can fill in via Riemann's removable singularity theorem to get
            \begin{equation*}
                \frac{f(0)}{0} = f'(0)
            \end{equation*}
            Now pick an $r\in(0,1)$ and let $z\in\partial D_r(0)$ be arbitrary. Then
            \begin{equation*}
                \left| \frac{f(z)}{z} \right| = \frac{|f(z)|}{|z|}
                = \frac{|f(z)|}{r}
                \leq \frac{1}{r}
            \end{equation*}
            Essentially, we have bounded the value of $f(z)/z$ on the boundary of a disk. Moreover, it follows by the maximum modulus principle that this bound must hold for all $z\in D_r(0)$. (Otherwise, we would get a local maximum inside $D_r(0)$; this would imply by the MMP that $f$ is constant, which we can't have because the function is conformal [hence nonconstant] by hypothesis.) Thus, we have shown that
            \begin{equation*}
                |f(z)| \leq \frac{1}{r}\cdot|z|
            \end{equation*}
            for all $z\in D_r(0)$. To finish it off, we can send $r\to 1$, thereby including all $z\in\D$ in our argument and transforming the above inequality into the desired one.
        \end{proof}
        \item If $|f(z)|\leq|z|$ is an equality for any $z\in\D$, then $f$ is a rotation;
        \begin{proof}
            Let $z_0\in\D$ be the point at which $|f(z_0)|=|z_0|$. It follows that
            \begin{equation*}
                \left| \frac{f(z_0)}{z_0} \right| = 1
            \end{equation*}
            Additionally, we have by (1) that for all $z\in\D$,
            \begin{equation*}
                \left| \frac{f(z)}{z} \right| \leq 1
            \end{equation*}
            Thus, $|f(z)/z|$ takes a maximum inside $\D$ (in particular, it takes its max at $z_0$). It follows by the maximum modulus principle that $f(z)/z$ is constant on $\D$.\par
            Let $c$ be this constant value. Naturally, it follows that in particular, $c=f(z_0)/z_0$ and hence
            \begin{equation*}
                |c| = \left| \frac{f(z_0)}{z_0} \right| = 1
            \end{equation*}
            But via algebraic rearrangement this means that $f(z)=cz$ where $c\in\C$ has $|c|=1$. Therefore, $f$ is a rotation, as desired.
        \end{proof}
        \item If $|f'(0)|=1$, then $f$ is a rotation.
        \begin{proof}
            % By (1), $|f(z)/z|\leq 1$ for all $z\in\D$. $|f(z)|=|z|$ implies $|f(z)/z|=1$ for some $z\in\D$, so if $f(z)/z$ takes a maximum inside $\D$, then it must be constant by the maximum modulus principle. But a constant function with $|f(z)/z|=1$ is just a rotation.

            In the proof of part (1), we showed that the power series expansion of $f(z)/z$ is
            \begin{equation*}
                \frac{f(z)}{z} = a_1+a_2z+\cdots
                = f'(0)+\frac{f''(0)}{2!}z+\cdots
            \end{equation*}
            Thus, since $|f'(0)|=1$, we have that
            \begin{equation*}
                \left| \widehat{\frac{f(0)}{0}} \right| = \left| f'(0)+\frac{f''(0)}{2!}\cdot 0+\cdots \right|
                = |f'(0)|
                = 1
            \end{equation*}
            Thus, as in the proof of part (2), we have shown that there exists a point in $\D$ at which $|f(z)/z|=1$. Therefore, $f$ is a rotation by a symmetric argument, as desired.
        \end{proof}
    \end{enumerate}
    \item Note that we did not use conformality anywhere in our proof of the Schwarz Lemma!
    \begin{itemize}
        \item All we actually needed was a holomorphicity condition.
        \item We'll need full conformality, however, to prove the following sort-of converse.
    \end{itemize}
    \item Lemma (a "converse" to the Schwarz Lemma): Suppose $f:\D\to\D$ is conformal and not onto with $f(0)=0$. Then there exists $F:\D\to\D$ (still conformal) such that $F(0)=0$ and $|F'(0)|>|f'(0)|$.
    \begin{figure}[H]
        \centering
        \begin{tikzpicture}[
            every node/.style=black
        ]
            \footnotesize
            \begin{scope}[xshift=-3.3cm]
                \filldraw [thick,draw=rex,fill=rez] circle (1cm) node[above left=7mm]{$\D$};
    
                \fill [yex] circle (2pt) node[above=1pt]{$0$};
            \end{scope}
            \begin{scope}
                \draw [semithick,dashed] circle (1cm) node[above left=7mm]{$\D$};
                \filldraw [thick,draw=rex,fill=rez] plot[domain=0:360,samples=100] (\x:{0.6+0.1*cos(6*\x)});
                \node at (-145:0.7) {$U$};
    
                \fill [yex] circle (2pt) node[above=1pt]{$0$};
                \fill [blx] (30:0.7) circle (2pt) node[below right]{$w$};
            \end{scope}
            \begin{scope}[yshift=3.3cm]
                \draw [semithick,dashed] circle (1cm);
                \filldraw [thick,draw=rex,fill=rez] (-45:0.8)
                    to[out=45  ,in=-45] (-135:0.2)
                    to[out=135 ,in=45 ] (135:0.8)
                    to[out=-135,in=63 ] (153:0.6)
                    to[out=-117,in=81 ] (171:0.8)
                    to[out=-99 ,in=99 ] (189:0.6)
                    to[out=-81 ,in=117] (207:0.8)
                    to[out=-63 ,in=135] (225:0.6)
                    to[out=-45 ,in=153] (243:0.8)
                    to[out=-27 ,in=171] (261:0.6)
                    to[out=-9  ,in=189] (279:0.8)
                    to[out=9   ,in=207] (297:0.6)
                    to[out=27  ,in=225] cycle
                ;
                \node at (-15:0.6) {$T(U)$};
    
                \fill [yex] (-135:0.4) circle (2pt);
                \fill [blx] circle (2pt) node[above=1pt]{$0$};
            \end{scope}
            \begin{scope}[xshift=3.3cm,yshift=3.3cm]
                \draw [semithick,dashed] circle (1cm);
                \filldraw [thick,draw=rex,fill=rez] (157.5:0.89)
                    to[out=247.5,in=22.5 ] (112.5:0.45)
                    to[out=22.5 ,in=-22.5] ( 67.5:0.89)
                    to[out=157.5,in=-13.5] ( 76.5:0.77)
                    to[out=166.5,in=-4.5 ] ( 85.5:0.89)
                    to[out=175.5,in=13.5 ] ( 94.5:0.77)
                    to[out=184.5,in=22.5 ] (103.5:0.89)
                    to[out=193.5,in=31.5 ] (112.5:0.77)
                    to[out=202.5,in=40.5 ] (121.5:0.89)
                    to[out=211.5,in=49.5 ] (130.5:0.77)
                    to[out=220.5,in=58.5 ] (139.5:0.89)
                    to[out=229.5,in=67.5 ] (148.5:0.77)
                    to[out=238.5,in=56.5 ] cycle
                ;
    
                \fill [yex] (112.5:0.63) circle (2pt);
                \node at (-0.2,-0.5) {$\sqrt{T(0)}$}
                    edge [out=130,in=-160,->,shorten >=3pt] (112.5:0.63)
                ;
                \fill [blx] circle (2pt);
            \end{scope}
            \begin{scope}[xshift=3.3cm]
                \draw [semithick,dashed] circle (1cm);
                \filldraw [thick,draw=rex,fill=rez] (157.5:0.75)
                    to[out=247.5,in=202.5] (-67.5:0.45)
                    to[out=22.5 ,in=-22.5] ( 67.5:0.75)
                    to[out=157.5,in=-13.5] ( 76.5:0.65)
                    to[out=166.5,in=-4.5 ] ( 85.5:0.75)
                    to[out=175.5,in=13.5 ] ( 94.5:0.65)
                    to[out=184.5,in=22.5 ] (103.5:0.75)
                    to[out=193.5,in=31.5 ] (112.5:0.65)
                    to[out=202.5,in=40.5 ] (121.5:0.75)
                    to[out=211.5,in=49.5 ] (130.5:0.65)
                    to[out=220.5,in=58.5 ] (139.5:0.75)
                    to[out=229.5,in=67.5 ] (148.5:0.65)
                    to[out=238.5,in=56.5] cycle
                ;
    
                \fill [yex] circle (2pt) node[above=1pt]{$0$};
                \fill [blx] (-67.5:0.63) circle (2pt);
            \end{scope}
    
            \draw [->] (-2.2,0) to[bend left=15] node[above]{$f$} (-1.1,0);
            \draw [->] (-0.1,1.1) to[bend left=15] node[left]{$T$} (-0.1,2.2);
            \draw [->] (0.1,2.2) to[bend right=15] node[right]{$T^{-1}$} (0.1,1.1);
            \draw [->] (1.1,3.4) to[bend left=15] node[above]{$\sqrt{}$} (2.2,3.4);
            \draw [->] (2.2,3.2) to[bend right=15] node[below]{$z\mapsto z^2$} (1.1,3.2);
            \draw [->] (3.4,2.2) to[bend left=15] node[right]{$S$} (3.4,1.1);
            \draw [->] (3.2,1.1) to[bend right=15] node[left]{$S^{-1}$} (3.2,2.2);
            \draw [->] (1.1,0.1) to[bend left=15] node[above]{$h$} (2.2,0.1);
            \draw [->] (2.2,-0.1) to[bend right=15] node[below]{$h^{-1}$} (1.1,-0.1);
    
            \draw [->] (-2.5,-0.8) to[bend right=15] node[pos=0.18,above]{$F$} (2.5,-0.8);
        \end{tikzpicture}
        \caption{A "converse" to the Schwarz Lemma.}
        \label{fig:convSchwarz}
    \end{figure}
    \begin{proof}
        % Since $f$ is conformal and its domain $\D$ is simply connected, $\im(f)$ is simply connected. Some argument about preservation of exterior winding number being zero??\par
        % Claim: For all $z_0\in U\subsetneq\D$ simply connected, there exists a map $h:U\to\D$ with $h(z_0)=0$ and $|h'(z_0)|>1$.
        % \begin{proof}
        %     See the \textcite{bib:FischerLieb} for more details.\par
        %     WLOG, let $z_0=0$ (else, use a M\"{o}bius transformation). We know that $0\in U$ and there exists points $w\in\D$ such that $w\notin U$. Use a M\"{o}bius transformation $T:w\mapsto 0$. Now $0\notin T(U)$. Since we have conformality, we may take the square root, which is not defined on all of $\D$ but is defined on $T(U)$. (This is because $z^{1/2}=\e[\log(z)/2]$ and we don't have any logarithms defined on zero, and because $T(U)$ is simply connected.) Lastly, take a M\"{o}bius transformation $S:\sqrt{T(0)}\mapsto 0$. Define $h=S\circ\sqrt{}\circ T$. Explicitly construct an inverse $g=h^{-1}=T^{-1}\circ z^2\circ S^{-1}$. $g$ is not a rotation, so by the Schwarz lemma, $|g'(0)|<1$, so this must mean that $|h'(0)|>1$.
        % \end{proof}
        % Claim implies lemma: Set $F=h\circ f$ and then use the chain rule because since $h'(z_0)>1$.

        % {\color{white}hi}
        % \begin{itemize}
        %     \item Summary of argument.
        %     \begin{itemize}
        %         \item This is a \emph{constructive} proof, in that we will actually construct $F$ as opposed to just proving its existence in the abstract. In particular\dots
        %         \item We will construct a function $h$, and prove that it has some special properties.
        %         \item We will define $F=h\circ f$.
        %         \item We will use the special properties of $h$ to prove that $F$, as defined, has the desired properties.
        %     \end{itemize}
        %     \item Let $U:=f(\D)$.
        %     \item $f$ is conformal and $\D$ is simply connected: $f(\D)$ is simply connected.
        %     \begin{itemize}
        %         \item Some argument about preservation of exterior winding number being zero?? Footnote.
        %     \end{itemize}
        %     \item There exists $h:U\to\D$ with $h(0)=0$ and $|h'(0)|>1$.\footnote{We can also prove a more general version of this claim in which we let $z_0\in U$ be arbitrary and construct $h$ with $h(z_0)=0$ and $|h'(z_0)|>1$ by adding in an additional M\"{o}bius transformation in the beginning sending $z_0\mapsto 0$. This is what we did in class, but it it not actually necessary for the proof, so I have left out the argument here. My raw, unedited notes from class are still available in the .tex file, for future reference.}
        %     \item Construct $h$.
        %     \begin{itemize}
        %         % \item Let $R$ be a M\"{o}bius transformation sending $z_0\mapsto 0$.
        %         \item $f$ is not onto: There exists $w\in\D$ such that $w\notin U$.
        %         \item Let $T$ be a M\"{o}bius transformation sending $w\mapsto 0$.
        %         \item $T$ is a M\"{o}bius transformation $\Longrightarrow$ $T$ is biholomorphic $\Longrightarrow$ $T$ is conformal and $U$ is simply connected: $T(U)$ is simply connected.
        %         \item $T(U)$ is simply connected: There exists a logarithm on $T(U)$.
        %         \item There exists a logarithm on $T(U)$: There exists a square root function $\sqrt{}:z\mapsto z^{1/2}=\e[0.5\log z]$ on $U$.
        %         \item Let $S$ be a M\"{o}bius transformation sending $\sqrt{T(0)}\mapsto 0$.
        %         \item Define $h:U\to\D$ by
        %         \begin{equation*}
        %             h(z) := (S\circ\sqrt{}\circ T)(z)
        %         \end{equation*}
        %     \end{itemize}
        %     \item $h(0)=0$.
        %     \begin{itemize}
        %         \item By definition.
        %     \end{itemize}
        %     \item $|h'(0)|>1$.
        %     \begin{itemize}
        %         \item Define $h^{-1}:h(U)\to U$ by
        %         \begin{equation*}
        %             h^{-1}(z) := (T^{-1}\circ[z\mapsto z^2]\circ S^{-1})(z)
        %         \end{equation*}
        %         \item $h^{-1}$ is a composition of maps that are holomorphic and bijective on their domains: $h^{-1}$ is conformal.
        %         \item Contrapositive of the Schwarz Lemma ($h^{-1}$ is conformal, $h^{-1}(0)=0$, and $h^{-1}$ is not a rotation): $|(h^{-1})'(0)|<1$.
        %         \item Inverse function theorem: $|h'(0)|>1$.
        %     \end{itemize}
        %     \item Let $F=h\circ f$.
        %     \item $f:\D\to U$ and $h:U\to\D$: $F:\D\to\D$.
        %     \item $f(0)=0$ and $h(0)=0$: $F(0)=h(f(0))=h(0)=0$.
        %     \item $|h'(0)|>1$:
        %     \begin{equation*}
        %         |F'(0)| = |h'(f(0))\cdot f'(0)|
        %         = |h'(0)|\cdot|f'(0)|
        %         > 1\cdot|f'(0)|
        %         = |f'(0)|
        %     \end{equation*}
        % \end{itemize}


        We will prove this claim via \emph{constructive} means, actually building a function $F$ as opposed to just proving its existence in the abstract. In particular\dots
        \begin{enumerate}
            \item We will construct a function $h$, and prove that it has some special properties;\footnote{We can also prove a more general version of this claim in which we let $z_0\in U$ be arbitrary and construct $h$ with $h(z_0)=0$ and $|h'(z_0)|>1$ by adding in an additional M\"{o}bius transformation in the beginning sending $z_0\mapsto 0$. This is what we did in class, but it it not actually necessary for the proof, so I have left out the argument here. My raw, unedited notes from class are still available in the .tex file, for future reference.}
            \item We will define $F=h\circ f$;
            \item We will use the special properties of $h$ to prove that $F$, as defined, has the desired properties.
        \end{enumerate}
        Let's begin.\par
        Let $U:=f(\D)$. Since $f$ is conformal and $\D$ is simply connected, $f(\D)$ is simply connected.\footnote{Some argument about how conformal maps preserve the winding number of $\gamma\subset U$ about an exterior point being zero??} We now construct the aforementioned function $h$.\par
        To construct $h$, we will build three other functions --- $T$, $\sqrt{}$, and $S$ --- and let $h$ be their composition.
        \begin{enumerate}
            \item Since $f$ is not onto by hypothesis, there exists $w\in\D$ such that $w\notin U$. Let $T$ be a M\"{o}bius transformation sending $w\mapsto 0$.
            \item Since $T$ is a M\"{o}bius transformation (hence biholomorphic and hence conformal) and $U$ is simply connected, $T(U)$ is simply connected. Thus, there exists a logarithm function on $T(U)$. It follows that there exists a square root function $\sqrt{}:z\mapsto z^{1/2}=\e[0.5\log z]$ on $U$.
            \item Lastly, let $S$ be a M\"{o}bius transformation sending $\sqrt{T(0)}\mapsto 0$.
        \end{enumerate}
        At this point, we can finally define $h:U\to\D$ by
        \begin{equation*}
            h(z) := (S\circ\sqrt{}\circ T)(z)
        \end{equation*}
        We will now prove two special properties of $h$: $h(0)=0$ and $|h'(0)|>1$. By the construction of $T,\sqrt{},S$, we immediately obtain $h(0)=0$. As to the other property, begin with defining $h^{-1}:h(U)\to U$ by
        \begin{equation*}
            h^{-1}(z) := (T^{-1}\circ[z\mapsto z^2]\circ S^{-1})(z)
        \end{equation*}
        Since $h^{-1}$ is a composition of holomorphic and bijective maps, $h^{-1}$ is conformal. This combined with the facts that $h^{-1}(0)=0$ and $h^{-1}$ is not a rotation (clearly) implies by the contrapositive of the Schwarz Lemma that $|(h^{-1})'(0)|<1$. Therefore, by the inverse function theorem, $|h'(0)|>1$.\par
        At this point, we may define a function $F$ by
        \begin{equation*}
            F := h\circ f
        \end{equation*}
        We now check $F$'s four properties. Since $f:\D\to U$, $h:U\to\D$, and $F=h\circ f$, we have that $F:\D\to\D$ via function composition, as desired. Since $T$, $\sqrt{}$, and $S$ are conformal, we have that $F$ is conformal, as desired. Since $f(0)=0$ by hypothesis, $h(0)=0$ by the above, and $F=h\circ f$, we have that
        \begin{equation*}
            F(0) = h(f(0))
            = h(0)
            = 0
        \end{equation*}
        as desired. And since $|h'(0)|>1$, we conclude by observing the following.
        \begin{equation*}
            |F'(0)| = |h'(f(0))\cdot f'(0)|
            = |h'(0)|\cdot|f'(0)|
            > 1\cdot|f'(0)|
            = |f'(0)|
        \end{equation*}
    \end{proof}
    \item Takeaway: The Schwarz Lemma tells us that conformal maps from $\D\to\D$ have derivative $\leq 1$, and this lemma tells us that maps from $U\subsetneq\D\to\D$ can be made to have a larger derivative.
    \item Note that the behavior in the Schwarz and this lemma is analogous to that of \emph{entire} biholomorphims, where $f$ conformal implies $f$ biholomorphic implies $f(z)=az+b$, this plus $f(0)=0$ implies $f(z)=az$, and $f:\D\to\D$ implies $|a|\leq 1$.
    \pagebreak
    \item We now prove our main result for the day.
    \item Riemann mapping theorem: Let $U$ be any simply connected domain in $\C$ (but not $\C$ itself). Then there exists a conformal map $f:U\to\D$.
    \item Comments on the RMT.
    \begin{itemize}
        \item So every simply connected domain is either $\C$ or the unit disk.
        \item Exercise: If we stipulate also that some $z_0\in U\mapsto 0$ and $f'(z_0)\in\R_{>0}$, then $f$ is unique.
        \item Why we delayed proving this masterpiece theorem until the end of the class: We just developed the tool we need.
    \end{itemize}
    \item Three-step proof outline.
    \begin{enumerate}
        \item Map $U$ into $\D$.
        \begin{proof}[Proof Sketch]
            Suppose $U$ misses some open set in $\hat{\C}$. (If it misses some open set, it misses a ball.) Now, just use a M\"{o}bius transformation, specifically the one that takes this ball to the entire northern hemisphere, thereby making $U$ something else that lives in the southern hemisphere. Make sure that this map also sends some point $z_0\in U$ to $0\in\hat{\C}$! Then remember that the southern hemisphere is just the unit disk (see Figure \ref{fig:stereographicProjection}.) Great pictures on the board!\par
            If $U$ doesn't miss an open set (e.g., $U=\C\setminus\R_{\leq 0}$), then use a log, which we would be able to do because the set is simply connected.
        \end{proof}
        \item "Maximize" among all such maps.
        \begin{proof}[Proof Sketch]
            If $f:U\to\D$ is conformal and not onto, then use the Converse Lemma to find $F$ with a bigger derivative at $z_0$.\par
            Thus, "just" take an $f$ maximizing $|f'(z_0)|$. We can't find something with a bigger derivative at this point, so the function must be onto.\par
            We now pull a rabbit out of the hat in the form of the real analytical/metric space form of the \textbf{Arzel\`{a}-Ascoli theorem}.
        \end{proof}
        \item Take a limit of this maximizing sequence (this is where the real magic happens, using the argument principle and such).
        \begin{proof}[Proof Sketch]
            Set $\mathcal{F}=\{f:U\to\D\mid\text{conformal, }f(z_0)=0\}$. This family is uniformly bounded (e.g., by the unit disk). Let
            \begin{equation*}
                \alpha := \sup_{f\in\mathcal{F}}|f'(z_0)|
            \end{equation*}
            The set of all $f_k$ such that $|f_k'(z_0)|\to\alpha$ has a uniformly convergent subsequence. If $f_k\to f$ locally uniformly, then by the theorem from last class, $f$ is holomorphic. Additionally, $f$ is also conformal because it is holomorphic and bijective (see the following proposition), hence biholomorphic.
        \end{proof}
    \end{enumerate}
    \item \textbf{Arzel\`{a}-Ascoli theorem}: If $\{f_k\}:X\to Y$ (where $X,Y$ are metric spaces) is a family of (locally) uniformly bounded and (locally) equicontinuous functions, then there exists a (locally) uniformly convergent subsequence.
    \item \textbf{Equicontinuity}: Same $\varepsilon$'s and $\delta$'s work for every $f_k$.
    \item Proposition: If $f_k$ are all conformal, then $f$ is either conformal or constant.
    \begin{proof}
        Suppose that $f$ is not bijective. Hence, $f$ is not injective. Thus, there are two points $z_1\neq z_2$ such that $f(z_1)=w=f(z_2)$. Pick disks $D_i$ about $z_i$ such that on each $D_i$, only $z_i\mapsto w$.\par
        Thus, $f(z)=w$ (i.e., $f(z)-w=0)$ twice on $D_1\cup D_2$. Thus, by the argument principle,
        \begin{equation*}
            2 = \int_{\partial D_1+\partial D_2}\frac{f'(z)}{f(z)-w}\dd{z}
            = \lim_{k\to\infty}\int_{\partial D_1+\partial D_2}\frac{f_k'(z)}{f_k(z)-w}\dd{z}
            \leq 1
        \end{equation*}
        which is a contradiction since the $f_k$ are injective, so $f$ is bijective.
    \end{proof}
    \item With just a bit of work, we can get from here to Montel's theorem.
    \item Montel's theorem: If $\{f_k\}$ is a family of (locally) uniformly bounded holomorphic functions, then there exists a (locally) uniformly convergent subsequence.
    \begin{proof}
        (Locally) uniformly bounded plus holomorphic implies (locally) equicontinuous.
    \end{proof}
    \item Montel's theorem allows us to extract limits.
\end{itemize}




\end{document}