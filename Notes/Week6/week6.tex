\documentclass[../notes.tex]{subfiles}

\pagestyle{main}
\renewcommand{\chaptermark}[1]{\markboth{\chaptername\ \thechapter\ (#1)}{}}
\setcounter{chapter}{5}

\begin{document}




\chapter{Special Meromorphic Functions}
\section{Meromorphic Functions and M\"{o}bius Transformations}
\begin{itemize}
    \item \marginnote{4/23:}Last time.
    \begin{itemize}
        \item We defined the Riemann sphere
        \begin{equation*}
            \hat{\C} := \C\cup\{\infty\}
        \end{equation*}
        \item When is a function $f:\hat{\C}\to\C$ "holomorphic at $\infty$?"
        \begin{itemize}
            \item Liouville: $f\in\mO(\hat{\C})$ is constant.
        \end{itemize}
        \item $f$ is a meromorphic function on $U\subset\hat{\C}$ if and only if $f$ is a holomorphic map to $\hat{\C}$.
        \item $f^{-1}(\infty)$ gives the set of poles.
    \end{itemize}
    \item Note for people doing residues on their final project.
    \begin{itemize}
        \item Recall that if $f:U\to\hat{\C}$ is meromorphic, then it has a Laurent series around each pole $p$.
        \item Essentially,
        \begin{equation*}
            f(z) = \frac{1}{(z-p)^k}h(z)
        \end{equation*}
        \begin{itemize}
            \item $h$ is holomorphic.
            \item $h(p)\neq 0$.
            \item $k$ is the order of the pole.
        \end{itemize}
        \item Since $h$ is holomorphic, it has a power series expansion
        \begin{equation*}
            h(z) = \sum_{i=0}^\infty a_i(z-p)^i
        \end{equation*}
        with $a_0\neq 0$.
        \item Thus,
        \begin{align*}
            f(z) &= \sum_{i=0}^\infty a_i(z-p)^{i-k}\\
            &= \sum_{i=-k}^\infty a_i(z-p)^i
        \end{align*}
        \item This allows us to define the \textbf{principal part}.
    \end{itemize}
    \item \textbf{Principal part} (of a Laurent series): The sum of the terms with a negative exponent.
    \item TPS: Suppose you have a pole $p$ in a disk $D$ within the radius of convergence of the Laurent series. Compute
    \begin{equation*}
        \int_{\partial D}f\dd{z}
    \end{equation*}
    \begin{itemize}
        \item Because we are in the radius of convergence of the Laurent series, the series converges locally absolutely uniformly, so we can switch the sum and the integral in the following and evaluate.
        \begin{align*}
            \int_{\partial D}f\dd{z} &= \int_{\partial D}\sum_{j=-k}^\infty a_j(z-p)^j\dd{z}
                = \sum_{j=-k}^\infty\int_{\partial D}a_j(z-p)^j\dd{z}\\
            &= \sum_{\substack{j=-k\\j\neq -1}}^\infty 0+\int_{\partial D}\frac{a_{-1}}{z-p}\dd{z}
                = a_{-1}\cdot 2\pi i
        \end{align*}
        \item This $a_{-1}$ coefficient is clearly special, so it gets a special name.
    \end{itemize}
    \item \textbf{Residue} (of $f$ at $p$): The $a_{-1}$ coefficient in the Laurent expansion of $f$ about a pole $p$. \emph{Denoted by} $\bm{\res_p(f)}$. \emph{Given by}
    \begin{equation*}
        \res_p(f) := a_{-1}
    \end{equation*}
    \item Corollary: $f$ has a primitive on $D$ (containing a single pole $p$) iff
    \begin{equation*}
        \res_p(f) = 0
    \end{equation*}
    \begin{proof}
        This goes back to the proposition from the 3/28 class. If the residue is zero, then the closed loop integral is zero, so by homotopy, $f$ has a primitive on the disk?? And if it has a primitive, then it's holomorphic on $D$ and therefore the residue is zero.
    \end{proof}
    \item Theorem: Every meromorphic function $f:\hat{\C}\to\hat{\C}$ (by which we mean a meromorphic function on $\C$ that is either bounded or with a pole at $\infty$) is rational, i.e., $f=p/q$.
    \begin{proof}
        % Claim 1: The set of poles of $f$ (including $\infty$) is finite.
        % \begin{proof}
        %     The set of poles is discrete. We know this for the following reason. $1/f$ has zeroes at the poles, and we know that zeroes must be discrete (otherwise $f$ would be constant and zero).
            
        %     It is also compact (because it lives in $\hat{\C}$). Discrete + compact implies finite.
        % \end{proof}
        % Claim 2: Write $f$ minus the sum of the principal parts of all the poles of $f$, and call this $g$.
        % \begin{proof}
        %     The principal part is just what the pole looks like locally. But it is a function in its own right! Thus, we can subtract it off just like any other function. Now $g$ is not defined everywhere, but it has an analytic continuation to everywhere.
            
        %     The principal part about $p$ only has a pole at $p$, so we \emph{can} subtract it off without introducing any new poles.

        %     How do we subtract a pole at $\infty$? If $f$ has a pole at $\infty$, then it is a polynomial. But $f$ having a pole at $\infty$ is equivalent to $f(1/z)$ having a pole at 0. So just take the principal part of $f(1/z)$ and then switch out all the $z$'s for $1/z$'s again and subtract that.
        % \end{proof}
        % Claim 3: $g$ is holomorphic on $\hat{\C}$, and therefore is constant.

        We will prove this theorem by dividing it into three consecutive claims.
        \begin{enumerate}
            \item The set of poles of $f$ (including $\infty$) is finite.
            \item Write $f$ minus the sum of the principal parts of all the poles of $f$, and call this $g$.
            \item $g$ is holomorphic on $\hat{\C}$, and therefore is constant.
        \end{enumerate}
        Once we have Claim 3, we will know that $f=c+\sum\text{principal parts}$, where all principal parts are rational, meaning that $f$ is rational! Let's begin.\par
        For claim 1, we will argue that the set of poles of $f$ is discrete and compact because it is a result of set theory that discrete compact sets are finite. To confirm that the set of poles is discrete, observe that $1/f$ has zeroes where $f$ has poles, and we know that these zeroes must be discrete because otherwise $1/f$ would be constant and zero (think power series). To confirm that the set of poles is also compact, observe that it lives in $\hat{\C}$ (which is compact) and is a closed set (since its discrete, it cannot be open).\par
        For claim 2, it may seem unintuitive that we can subtract off a bunch of partial Laurent series about different points from a function defined more broadly than a Laurent series around a certain pole. However, observe that the principal part of a Laurent expansion around a certain pole is just is a function in its own right! Thus, we can subtract it off just like any other function. After doing this, $g$ is not defined everywhere, but it has an analytic continuation (claim 3) to everywhere, as desired. Note that the principal part of a Laurent expansion about $p$ only has a pole at $p$, so we \emph{can} subtract it off without introducing any new poles. Additionally, note that we can subtract off a pole at $\infty$ as follows. Essentially, if $f$ has a pole at $\infty$, then it is a polynomial. But $f$ having a pole at $\infty$ is equivalent to $f(1/z)$ having a pole at 0. So just take the principal part of the Laurent expansion of $f(1/z)$ at zero and then switch out all the $z$'s for $1/z$'s again and subtract that.
    \end{proof}
    \item To get a global picture of the poles of $f$, we need a global picture of the zeroes of $1/f$.
    \item TPS.
    \begin{enumerate}
        \item For $k\in\N$, draw $x\mapsto x^k$ on $\R$. For $y$ near 0, how many preimages does $y$ have?
        \begin{itemize}
            \item If $k$ is odd, $y$ always has one preimage.
            \item If $k$ is even, then $y>0$ has two preimages, $y=0$ has one preimage, and $y<0$ has no preimages.
        \end{itemize}
        \item Draw what happens to the sector $\{z\mid z\in r\e[i\theta],\ \theta\in(0,2\pi/k]\}$ under $z\mapsto z^k$.
        \begin{itemize}
            \item Maps to the circle of radius $r^k$ bijectively.
            \item Additionally, the interior of the sector goes to the disk minus the slit along the positive real axis.
        \end{itemize}
        \item Do 1, but now for $z\mapsto z^k$ in $\C$.
        \begin{itemize}
            \item For $y\neq 0$, there are $k$ distinct roots of $y$.
        \end{itemize}
        \item Draw a "global" picture of $z\mapsto z^k$.
        \begin{itemize}
            \item For $z\mapsto z^2$, for instance, we cut $\C$ into an upper half plane and lower half plane and wrap them both around into circles.
            \item \emph{See class pictures}.
        \end{itemize}
    \end{enumerate}
    \item Last time.
    \begin{itemize}
        \item Proposition: If $X$ is a Riemann surface and $f:X\to\hat{\C}$ is meromorphic, then $f$ is onto.
    \end{itemize}
    \item Now we'll do a bit on M\"{o}bius transformations.
    \begin{itemize}
        \item One of Calderon's favorite topics; there will be a bit on the next problem set about these.
    \end{itemize}
    \item What we just saw is that if $p(z)=z^n$, then $\# p^{-1}(w)=n$\footnote{Recall that $\#$ denotes cardinality; in this case, cardinality of the preimage.} for $w$ near zero but nonzero.
    \begin{itemize}
        \item If $w=0$, then you still get $n$ preimages if you count with multiplicity.
    \end{itemize}
    \item If $p(z)$ is a general polynomial of degree $n$, we also know (by the FTA) that $\# p^{-1}(0)=n$.
    \begin{itemize}
        \item Moreover, for all $w\in\hat{\C}$, $\# p^{-1}(w)=n$ with multiplicity because $p(z)-w$ has $n$ roots.
    \end{itemize}
    \item Let $f(z)=p(z)/q(z)$ be a rational function, such as $z/(z-3)^2$.
    \begin{itemize}
        \item Here, $\# f^{-1}(\infty)=2$.
        \item Here as well, $f^{-1}(0)=1$ on $\C$ and $=2$ on $\hat{\C}$.
    \end{itemize}
    \item \textbf{Degree} (of a rational function): The natural number defined as follows, where $f=p/q$ is a rational function representated with $p,q$ coprime. \emph{Denoted by} $\bm{\deg(f)}$. \emph{Given by}
    \begin{equation*}
        \deg(f) := \max(\deg(p),\deg(q))
    \end{equation*}
    \item Theorem: If $f=p/q$, then for all $w\in\hat{\C}$, $\# f^{-1}(w)=\deg(f)$ when counted with multiplicity.
    \begin{proof}
        Follows from the FTA. Proof in the notes.
    \end{proof}
    \item Example:
    \begin{itemize}
        \item If $f(z)=z/(z-3)^2$, then we count 3 twice with multiplicity (giving a pole at $\infty$). We can also count $0,\infty$ as two distinct poles that give us zero.
        \item This just reiterates the previous example.
    \end{itemize}
    \item Holomorphic symmetries.
    \begin{itemize}
        \item What are all biholomorphisms?
        \item They are kind of like change of coordinate maps.
        \item Can we have entire biholomorphisms? Can we have biholomorphisms $f:\hat{\C}\to\hat{\C}$.
        \item $\Aut(\C)=\Bihol(\C)$. On Thursday, we'll prove that
        \begin{equation*}
            \Bihol(\C) = \{z\mapsto az+b\}
        \end{equation*}
        \item For the other one,
        \begin{equation*}
            \Bihol(\hat{\C}) = \left\{ z\mapsto\frac{az+b}{cz+d}\mid a,b,c,d\in\C,\ ad-bc\neq 0 \right\}
        \end{equation*}
        \begin{itemize}
            \item The condition $ad-bc\neq 0$ is the same as saying that the map is nonconstant.
            \item This is the set of fractional linear transformations.
            \item It is isomorphic to
            \begin{equation*}
                \PGL_2(\C) = \left\{
                    \begin{pmatrix}
                        a & b\\
                        c & d\\
                    \end{pmatrix}
                    \mid \det\neq 0
                \right\}
            \end{equation*}
            \begin{itemize}
                \item This is the projective linear group.
                \item This group acts on the projective linear space $\mathbb{P}^1(\C)=\hat{\C}$ (ask Calderon later??).
            \end{itemize}
        \end{itemize}
        \begin{proof}
            Let $f:\hat{\C}\to\hat{\C}$ be a biholomorphism. Thus, it is bijective and hence $\deg(f)=1$. Additionally, it is meromorphic and hence rational map. But a rational map with degree 1 is just a map of the given kind!
        \end{proof}
    \end{itemize}
    \item A different geometric way of thinking about these \textbf{M\"{o}bius transformations}: Circle inversion.
    \begin{itemize}
        \item This is the map $r\e[i\theta]\mapsto r^{-1}\e[i\theta]$. In terms of $z$, this map is $z\mapsto 1/\bar{z}$.
        \item This map is not holomorphic. In fact, it is \textbf{antiholomorphic} and \textbf{anticonformal}.
        \item This is the projecting up and down map.
        \item It can also be thought of as reflecting over the equator.
        \item $\Mob^\pm$ (the \textbf{extended M\"{o}bius group}) is the group generated by circle inversions.
        \item $\Mob$ is the orientation-preserving subgroup.
    \end{itemize}
    \item Four examples.
    \begin{itemize}
        \item Scaling $z\mapsto rz$ where $r\in\R$.
        \item Translation.
        \item Rotation.
        \item One last one.
    \end{itemize}
    \item Theorem: $\Mob=\Bihol(\hat{\C})\cong\PGL_2(\C)$.
    \begin{proof}
        $\subset$: Obvious, once you've digested the definitions. This is just because circle inversions are anti-conformal. Two inversions implies conformal, which is equivalent to biholomorphic.\par
        $\supset$: We have algebraically that
        \begin{equation*}
            \frac{az+b}{cz+d} = \frac{bc-ad}{c^2}\left( z+\frac{d}{c} \right)^{-1}+\frac{a}{c}
        \end{equation*}
        where we have two translations, an inversion, and a scaling/rotation.
    \end{proof}
\end{itemize}



\section{The Complex Logarithm}
\begin{itemize}
    \item \marginnote{4/25:}Review: What do certain complex analysis objects look like?
    \begin{itemize}
        \item Holomorphic functions look like conformal maps and branching.
        \item Meromorphic functions look like conformal maps to $\hat{\C}$ and branching.
        \item But what do essential singularities look like?
        \begin{itemize}
            \item Recall that an essential singularity is a singularity $z$ for which there exist $z_n\to z$ and $w_n\to z$ such that $f(z_n)$ stays bounded and $|f(w_n)|\to\infty$.
        \end{itemize}
    \end{itemize}
    \item Theorem (Casorati-Weierstrass): Suppose $f\in\mO(U\setminus\{z_0\})$ where $z_0$ is an essential singularity. Then for all $w_0\in\C$, there exist $z_n\to z_0$ such that $f(z_n)\to w_0$.
    \begin{proof}
        Suppose for the sake of contradiction that there exists a $w_0\in\C$ for which every $z_n\to z_0$ is such that $|f(z_n)-w_0|>\varepsilon$. Pictorially, this means that no matter how close the points of the sequence $\{z_n\}$ get to $z_0$, $f(z_n)$ will always stay outside of a disk (of radius $\varepsilon$) around $w_0$. Define
        \begin{equation*}
            g(z) := \frac{1}{f(z)-w_0}
        \end{equation*}
        Since $f\in\mO(U\setminus\{z_0\})$ and $g$ is a rational function of $f$, we know that $g\in\mO(U\setminus\{z_0\})$. Additionally, let $\{z_n\}$ be an arbitrary sequence converging to $z_0$. Then
        \begin{align*}
            |f(z_n)-w_0| &> \varepsilon\\
            \frac{1}{\varepsilon} &> \frac{1}{|f(z_n)-w_0|}\\
            \frac{1}{\varepsilon} &> \left| \frac{1}{f(z_n)-w_0} \right|\\
            \frac{1}{\varepsilon} &> |g(z_n)|
        \end{align*}
        Thus, we have shown that $g$ is bounded near $z_0$. This is the last condition we need to invoke Riemann's removable singularity theorem and learn that $g$ can be analytically continued to $z_0$. But if $\hat{g}\in\mO(U)$, then
        \begin{equation*}
            f = \frac{1}{g}+w_0
        \end{equation*}
        has either a removable singularity or a pole at $z_0$. This contradicts the assumption that $f$ has an essential singularity at $z_0$.
    \end{proof}
    \item Implication: $f(U\setminus\{z_0\})$ is dense in $\C$ and hence $f$ is \emph{very} not injective.
    \item Example: $z\mapsto\e[z]$.
    \begin{figure}[H]
        \centering
        \begin{tikzpicture}
            \begin{scope}[
                xshift=-6cm,
                yscale=0.333
            ]
                \draw [-stealth] (-2,0) -- (2,0) node[right]{$\R$};
                \draw [-stealth] (0,-6) -- (0,6) node[above]{$i\R$};
    
                \draw [thick,pux]
                    (0,{pi/16}) -- ++(1.8,0)
                    (0,{pi/8}) -- ++(1.8,0)
                    (0,{3*pi/16}) -- ++(1.8,0)
                    (0,{pi/4}) -- ++(1.8,0)
                ;
                \draw [thick,blx]
                    (0,{pi/16}) -- ++(-1.8,0)
                    (0,{pi/8}) -- ++(-1.8,0)
                    (0,{3*pi/16}) -- ++(-1.8,0)
                    (0,{pi/4}) -- ++(-1.8,0)
                ;
                \draw [ultra thick,orx] (0,0) -- (0,{pi/4});
                \draw [thick,grx]
                    (-1.8,{-pi/16}) -- ++(3.6,0)
                    (-1.8,{-pi/8}) -- ++(3.6,0)
                    (-1.8,{-3*pi/16}) -- ++(3.6,0)
                    (-1.8,{-pi/4}) -- ++(3.6,0)
                ;
    
                \draw [yex,very thick] (1.1,{-pi}) -- (1.1,{2*pi});
    
                \filldraw [draw=orx,fill=orx!30] (1.3,-2) -- (1.5,-2) -- (1.5,-2.6) -- (1.3,-2.6) -- cycle;
                \filldraw [draw=orx,fill=orx!30,yshift=6.28cm] (1.3,-2) -- (1.5,-2) -- (1.5,-2.6) -- (1.3,-2.6) -- cycle;
            \end{scope}
            \draw [semithick,|->] (-3.3,0) -- (-2.7,0);
            \begin{scope}[z={(-0.2,-3)},yscale=0.2]
                \draw [-stealth] (0,0,0) -- (2,0,0) node[right]{$\R^+$};
                \draw [-stealth] (0,-10,0) -- (0,10,0) node[above]{$i\R$};
    
                \draw [pux,thick]
                    ({0.333*cos(11.25)},{pi/16},{-0.333*sin(11.25)}) -- ({2.017*cos(11.25)},{pi/16},{-2.017*sin(11.25)})
                    ({0.333*cos(22.5)},{pi/8},{-0.333*sin(22.5)}) -- ({2.017*cos(22.5)},{pi/8},{-2.017*sin(22.5)})
                    ({0.333*cos(33.75)},{3*pi/16},{-0.333*sin(33.75)}) -- ({2.017*cos(33.75)},{3*pi/16},{-2.017*sin(33.75)})
                    ({0.333*cos(45)},{pi/4},{-0.333*sin(45)}) -- ({2.017*cos(45)},{pi/4},{-2.017*sin(45)})
                ;
                \draw [blx,thick]
                    (0,{pi/16},0) -- ({0.333*cos(11.25)},{pi/16},{-0.333*sin(11.25)})
                    (0,{pi/8},0) -- ({0.333*cos(22.5)},{pi/8},{-0.333*sin(22.5)})
                    (0,{3*pi/16},0) -- ({0.333*cos(33.75)},{3*pi/16},{-0.333*sin(33.75)})
                    (0,{pi/4},0) -- ({0.333*cos(45)},{pi/4},{-0.333*sin(45)})
                ;
                \draw [grx,thick]
                    (0,{-pi/16},0) -- ({2.017*cos(-11.25)},{-pi/16},{-2.017*sin(-11.25)})
                    (0,{-pi/8},0) -- ({2.017*cos(-22.5)},{-pi/8},{-2.017*sin(-22.5)})
                    (0,{-3*pi/16},0) -- ({2.017*cos(-33.75)},{-3*pi/16},{-2.017*sin(-33.75)})
                    (0,{-pi/4},0) -- ({2.017*cos(-45)},{-pi/4},{-2.017*sin(-45)})
                ;
                \draw [orx,ultra thick] plot[domain=0:45] ({0.333*cos(\x)},{\x*pi/180},{-0.333*sin(\x)});
    
                % \draw [yex,very thick,dashed] plot[domain=-360:-180] ({cos(\x)},{\x*pi/180},{-sin(\x)});
                \draw [yex,very thick] plot[domain=-180:-9] ({cos(\x)},{\x*pi/180},{-sin(\x)});
                \draw [yex,very thick,dashed] plot[domain=-9:102] ({cos(\x)},{\x*pi/180},{-sin(\x)});
                \draw [yex,very thick] plot[domain=102:360] ({cos(\x)},{\x*pi/180},{-sin(\x)});
    
                \filldraw [draw=orx,fill=orx!30]
                    plot[domain=-149:-115] ({1.223*cos(\x)},{\x*pi/180},{-1.223*sin(\x)})
                    -- ({1.494*cos(-115)},{-115*pi/180},{-1.494*sin(-115)}) --
                    plot[domain=-115:-149] ({1.494*cos(\x)},{\x*pi/180},{-1.494*sin(\x)})
                    -- cycle
                ;
                \filldraw [draw=orx,fill=orx!30]
                    plot[domain=211:245] ({1.223*cos(\x)},{\x*pi/180},{-1.223*sin(\x)})
                    -- ({1.494*cos(245)},{245*pi/180},{-1.494*sin(245)}) --
                    plot[domain=245:211] ({1.494*cos(\x)},{\x*pi/180},{-1.494*sin(\x)})
                    -- cycle
                ;
    
                \draw (0,{-pi},0)
                    -- (-2,{-pi},0)
                    to[bend right=1.333] (-2,{-3*pi/4},2)
                    to[bend right=20] (0,{-pi/2},2)
                    to[bend left=20] (2,{-pi/4},2)
                    to[bend right=1.333] (2,0,0)
                    to[bend left=1.333] (2,{pi/4},-2)
                    to[bend left=20] (0,{pi/2},-2)
                    to[bend right=20] (-2,{3*pi/4},-2)
                    to[bend left=1.333] (-2,{pi},0)
                    to[bend right=1.333] (-2,{5*pi/4},2)
                    to[bend right=20] (0,{3*pi/2},2)
                    to[bend left=20] (2,{7*pi/4},2)
                    to[bend right=1.333] (2,{2*pi},0)
                    -- (0,{2*pi},0)
                ;
            \end{scope}
            \draw [semithick,|->] (3,0) -- (3.6,0);
            \begin{scope}[xshift=6cm]
                \draw [-stealth] (-2,0) -- (2,0) node[right]{$\R$};
                \draw [-stealth] (0,-2) -- (0,2) node[above]{$i\R$};
                
                \draw [thick,pux]
                    (11.25:0.333) -- (11.25:2.017)
                    (22.5:0.333) -- (22.5:2.017)
                    (33.75:0.333) -- (33.75:2.017)
                    (45:0.333) -- (45:2.017)
                ;
                \draw [thick,blx]
                    (11.25:0.333) -- (0,0)
                    (22.5:0.333) -- (0,0)
                    (33.75:0.333) -- (0,0)
                    (45:0.333) -- (0,0)
                ;
                \draw [ultra thick,orx] (0.333,0) arc[start angle=0,end angle=45,radius=0.333cm];
                \draw [thick,grx]
                    (0,0) -- (-11.25:2.017)
                    (0,0) -- (-22.5:2.017)
                    (0,0) -- (-33.75:2.017)
                    (0,0) -- (-45:2.017)
                ;
    
                \draw [yex,very thick] circle (1cm);
    
                \filldraw [draw=orx,fill=orx!30]
                    plot[domain=-149:-115] ({1.223*cos(\x)},{1.223*sin(\x)})
                    -- ({1.494*cos(-115)},{1.494*sin(-115)}) --
                    plot[domain=-115:-149] ({1.494*cos(\x)},{1.494*sin(\x)})
                    -- cycle
                ;
            \end{scope}
        \end{tikzpicture}
        \caption{The essential singularity of $\e[z]$.}
        \label{fig:exponentialSingularity}
    \end{figure}
    \begin{itemize}
        \item This function has an essential singularity at $\infty$.
        \item Say we want to approach the positive real number $\e[a]$ ($a\in\R$). Then we can take the sequence of points $z_n:=a+2\pi ni$. As $n\to\infty$, $|a+2\pi ni|\to\infty$. Additionally,
        \begin{equation*}
            \e[a+2\pi ni] = \e[a]\cdot(\e[2\pi i])^n
            = \e[a]\cdot 1^n
            = \e[a]
        \end{equation*}
        \item For the negative real number $-\e[a]$, choose $z_n=a+\pi i+2\pi ni$. Then
        \begin{equation*}
            \e[a+\pi i+2\pi ni] = \e[a]\cdot\e[i\pi]\cdot(\e[2\pi i])^n
            = \e[a]\cdot -1\cdot 1^n
            = -\e[a]
        \end{equation*}
        \item For 0, choose $z_n=-n$.
        \item How about the complex number $r\e[i\theta]$? Pick $a\in\R$ such that $r=\e[a]$. Then choose $z_n=a+i\theta+2\pi ni$.
        \item What is shown in Figure \ref{fig:exponentialSingularity} is an \textbf{$\bm{\infty}$ branched cover}.
    \end{itemize}
    \item Theorem (Little Picard): If $f$ is entire and nonconstant, then $f(\C)$ equals $\C$, except maybe one point.
    \item Theorem (Big Picard): For any neighborhood $U$ of an essential singularity $z_0$, $f(U\setminus\{z_0\})$ is all of $\C$ except maybe one point.
    \item TPS: Verify that the big Picard theorem holds for $\exp$ and $\infty$.
    \begin{itemize}
        \item Let $U$ be a neighborhood of $\infty$, as defined when we discussed the Riemann sphere.
        \item This is very much related to what was discussed above!
        \item Indeed, let $r\e[i\theta]\in\C$ be arbitrary. Choose $n$ big enough so that $a+i\theta+2\pi ni\in U$. Then $\exp(a+i\theta+2\pi ni)=r\e[i\theta]$, as desired.
        \item Here, 0 is our single point exception.
    \end{itemize}
    \item Corollary (of Casorati-Weierstrass):
    \begin{equation*}
        \Bihol(\C) = \{z\mapsto az+b\}
    \end{equation*}
    \begin{proof}
        For the backwards inclusion, let $f(z)=az+b$ be an arbitrary linear map. $f$ is a polynomial (hence holomorphic). Its inverse
        \begin{equation*}
            f^{-1}(z) = a^{-1}z-a^{-1}b
        \end{equation*}
        is a polynomial (hence holomorphic). And
        \begin{equation*}
            f\circ f^{-1} = f^{-1}\circ f = \id
        \end{equation*}
        so $f$ is bijective. Therefore, $f$ is biholomorphic by definition.\par
        For the forwards inclusion, suppose first that $f\in\Bihol(\C)$ has a pole at infinity. Then $f\in\Bihol(\hat{\C})$. It follows by the result from last time that there exist $a,b,c,d\in\C$ with $ad-bc\neq 0$ such that
        \begin{equation*}
            f(z) = \frac{az+b}{cz+d}
        \end{equation*}
        Additionally, since we know that $f(\infty)=\infty$, we must have $c=0$ (otherwise, $f(\infty)=a/c$). But if $c=0$, then
        \begin{equation*}
            f(z) = \frac{a}{d}z+\frac{b}{d} \in \{z\mapsto az+b\}
        \end{equation*}
        as desired.\par
        Now suppose that $f\in\Bihol(\C)$ does not have a pole at infinity. Since $f\notin\Bihol(\hat{\C})$, this also means that $f$ does not have a removable singularity at infinity. Thus, $f$ has an essential singularity at infinity. Consequently, if we let $D$ be an open neighborhood of $\infty$, then the Casorati-Weierstrass theorem implies that $f(D)$ is dense in $\C$. Additionally, by the open mapping theorem, $f(D)$ is open. These last two results combined imply that $f(D)=\C$. But since $\C\setminus D$ is also in the domain of $f$ (i.e., $f$ must map some points in $\C\setminus D$ to points that it's already covered in $D$), $f$ is not injective, hence not bijective, hence not biholomorphic. This is a contradiction, and therefore biholomorphic functions can only have a \emph{pole} at infinity, a case in which we have shown they are linear.
    \end{proof}
    \item We now finally define a (\emph{not} the) logarithm.
    \item \textbf{Logarithm} (of $z$): A point $w\in\C$ such that $\e[w]=z$, where $z\in\C$.
    \begin{itemize}
        \item Naturally, $w$ is only well-defined up to $\pm 2\pi in$ ($n\in\Z$).
    \end{itemize}
    \item Log identities don't always hold.
    \item TPS: Which are always true and which are only true for the appropriate choice of log?
    \begin{itemize}
        \item $\exp(\log(z))=z$ is always true.
        \item $\log(\exp(z))=z$ is only true if we happen to choose the correct complex number raised to the exponential.
        \item $\log(zw)=\log(z)+\log(w)$ is similarly only true for the appropriate choices of log.
    \end{itemize}
    \item Setting aside well-definedness for now, let's define what it means to raise one complex number to the power of another.
    \item $\bm{z^w}$: The complex number defined as follows. \emph{Given by}
    \begin{equation*}
        z^w := \e[w\log z]
    \end{equation*}
    \begin{itemize}
        \item Such a definition requires a choice of the logarithm.
        \item Such a choice is called a \textbf{branch} of the logarithm.
    \end{itemize}
    \item TPS: What do $z^n$ and $z^{1/n}$ give, where $n\in\N$?
    \begin{itemize}
        \item $z^n$ gives exactly one value.
        \begin{equation*}
            z^n = \e[n\log z]
            = \e[n(a+bi+2\pi ik)]
            = \e[na]\cdot\e[bni]\cdot(\e[2\pi in])^k
            = \e[na]\e[bni]
        \end{equation*}
        \begin{itemize}
            \item In the above, $k\in\Z$.
            \item Additionally, it doesn't matter what exact $b$ we choose because if we increase or decrease it by $2\pi ik$, we will just multiply the value by $1^k=1$. Essentially, $\e[na]$ gives a unique magnitude for the result, and $bn$ gives a unique argument for the result.
        \end{itemize}
        \item $z^{1/n}$ gives $n$ values, each of which differs by $\e[2\pi i/n]$ (i.e., by an $n^\text{th}$ root of 1).
        \begin{equation*}
            z^{1/n} = (r\e[i\theta+2\pi ik])^{1/n}
            = r^{1/n}\e[i\theta/n]\zeta_k
        \end{equation*}
        \begin{itemize}
            \item In the above, the $\zeta_k=\e[k\cdot 2\pi i/n]$ are the $n^\text{th}$ roots of unity.
        \end{itemize}
    \end{itemize}
    \item Example: We can now calculate $i^i$!
    \begin{equation*}
        i^i = \e[i\log i]
        = \e[i(\pi i/2+2\pi ik)]
        = \e[-\pi/2]\cdot\e[-2\pi k]
    \end{equation*}
    \item \textbf{Principal branch} (of log): The branch that is real for real numbers.
    \item \textbf{Logarithm function} (on $U$): A continuous inverse to $\exp$, i.e., $\e[\log z]=z$.
    \item Think of a logarithm function as lifting $U$ to the infinity spiral in the middle of Figure \ref{fig:exponentialSingularity}.
    \begin{itemize}
        \item Imagine $U$ is the orange domain in the top-down view on the right side of Figure \ref{fig:exponentialSingularity}.
        \item Project $U$ up, down, or both onto a connected open subset of the infinity spiral.
        \begin{itemize}
            \item In the middle of Figure \ref{fig:exponentialSingularity}, we see $U$ projected up infinitely, but with a logarithm, we would have to make a \emph{choice} here.
        \end{itemize}
        \item Then invert $U$ back onto the original complex plane on the left side of Figure \ref{fig:exponentialSingularity}.
    \end{itemize}
    \pagebreak
    \item Observe that\dots
    \begin{itemize}
        \item If $U\ni 0$, then no logarithm function exists on $U$;
        \item If $U$ contains a loop $\gamma$ winding around $U$, then no logarithm function exists on $U$.
    \end{itemize}
    \item Proposition: If the logarithm makes sense on $U$, then it is holomorphic and
    \begin{equation*}
        \dv{z}\log(z) = \frac{1}{z}
    \end{equation*}
    \begin{proof}
        See \textcite{bib:FischerLieb}.
    \end{proof}
    \item Observe that since the derivative of the logarithm is never zero, the logarithm must be injective. Therefore, it's biholomorphic!
    \item Corollary: No log exists on $\C^*$.
    \begin{proof}
        Suppose for the sake of contradiction that a logarithm exists on $\C^*$. Then $1/z$ has a primitive on $\C^*$, so by the 3/28 lecture,
        \begin{equation*}
            \int_\gamma\frac{1}{z}\dd{z} = 0
        \end{equation*}
        for all loops $\gamma$. But since we know that
        \begin{equation*}
            \int_\gamma\frac{1}{z}\dd{z} = 2\pi i \neq 0
        \end{equation*}
        we have a contradiction.
    \end{proof}
    \item But, there are logs on domains covering $\C^*$.
    \begin{figure}[h!]
        \centering
        \begin{tikzpicture}
            \begin{scope}[z={(-0.2,-3)},yscale=0.2]
                \draw [densely dashed] (0,0,0) -- (1.65,0,0);
                \draw [-stealth] (1.65,0,0) -- (2,0,0) node[right]{$\R^+$};
    
                % \draw [densely dashed]
                %     (0,-6,0) -- (0,{-pi/2},0)
                %     (0,{3*pi/2},0) -- (0,-1,0)
                % ;
                % \draw (0,{-pi/2},0) -- (0,-1,0);
                \draw [-stealth] (0,-6,0) -- (0,16,0) node[above]{$i\R$};
    
                % \draw (0,{-pi/2},0)
                %     -- (0,{-pi/2},2)
                %     to[bend left=20] (2,{-pi/4},2)
                %     to[bend right=1.333] (2,0,0)
                %     to[bend left=1.333] (2,{pi/4},-2)
                %     to[bend left=20] (0,{pi/2},-2)
                %     to[bend right=20] (-2,{3*pi/4},-2)
                %     to[bend left=1.333] (-2,{pi},0)
                %     to[bend right=1.333] (-2,{5*pi/4},2)
                %     to[bend right=20] (0,{3*pi/2},2)
                %     to[bend left=20] (2,{7*pi/4},2)
                %     to[bend right=1.333] (2,{2*pi},0)
                %     to[bend left=1.333] (2,{9*pi/4},-2)
                %     to[bend left=20] (0,{5*pi/2},-2)
                %     to[bend right=20] (-2,{11*pi/4},-2)
                %     to[bend left=1.333] (-2,{12*pi/4},0)
                %     to[bend right=1.333] (-2,{13*pi/4},2)
                %     to[bend right=20] (0,{14*pi/4},2)
                %     -- (0,{14*pi/4},0)
                % ;
    
                \draw (0,{-pi/2},0)
                    -- (0,{-pi/2},2)
                    to[bend left=20] (2,{-pi/4},2)
                    to[bend right=1.333] (2,0,0)
                    to[bend left=1.333] (2,{pi/4},-2)
                    to (1.62,0.81,-2)
                ;
                \draw [densely dashed] (1.62,0.81,-2)
                    to[bend left=20] (0,{pi/2},-2)
                    to[bend right=20] (-2,{3*pi/4},-2)
                    to[bend left=1] (-2,2.68,-0.47)
                ;
                \draw (-2,2.68,-0.47)
                    to[bend left=1.333] (-2,{pi},0)
                    to[bend right=1.333] (-2,{5*pi/4},2)
                    to[bend right=20] (0,{3*pi/2},2)
                    to[bend left=20] (2,{7*pi/4},2)
                    to[bend right=1.333] (2,{2*pi},0)
                    to[bend left=1.333] (2,{9*pi/4},-2)
                    to[bend left=20] (0,{5*pi/2},-2)
                    to[bend right=20] (-2,{11*pi/4},-2)
                    to[bend left=1.333] (-2,{12*pi/4},0)
                    to[bend right=1.333] (-2,{13*pi/4},2)
                    to[bend right=20] (0,{14*pi/4},2)
                    -- (0,{14*pi/4},0)
                ;
    
                \draw [orange,opacity=0.1,line width=3mm] plot[domain=-90:-45] ({cos(\x)*(cos(90+\x/2)/2+1)},{\x*pi/180},{-sin(\x)*(cos(90+\x/2)/2+1)});
                \draw [red,opacity=0.1,line width=3mm] plot[domain=-45:0] ({cos(\x)*(cos(90+\x/2)/2+1)},{\x*pi/180},{-sin(\x)*(cos(90+\x/2)/2+1)});
                \draw [blue,opacity=0.1,line width=3mm] plot[domain=0:215] ({cos(\x)*(cos(90+\x/2)/2+1)},{\x*pi/180},{-sin(\x)*(cos(90+\x/2)/2+1)});
                \draw [green!70!blue,opacity=0.1,line width=3mm] plot[domain=215:450] ({cos(\x)*(cos(90+\x/2)/2+1)},{\x*pi/180},{-sin(\x)*(cos(90+\x/2)/2+1)});
                \draw [purple,opacity=0.1,line width=3mm] plot[domain=450:630] ({cos(\x)*(cos(90+\x/2)/2+1)},{\x*pi/180},{-sin(\x)*(cos(90+\x/2)/2+1)});
    
                \draw [yex,very thick] plot[domain=-90:-9] ({cos(\x)*(cos(90+\x/2)/2+1)},{\x*pi/180},{-sin(\x)*(cos(90+\x/2)/2+1)});
                \draw [yex,very thick,densely dashed] plot[domain=-9:102] ({cos(\x)*(cos(90+\x/2)/2+1)},{\x*pi/180},{-sin(\x)*(cos(90+\x/2)/2+1)});
                \draw [yex,very thick] plot[domain=102:630,samples=50,smooth] ({cos(\x)*(cos(90+\x/2)/2+1)},{\x*pi/180},{-sin(\x)*(cos(90+\x/2)/2+1)});
            \end{scope}
            \draw [semithick,|->] (3,0) -- (3.6,0);
            \begin{scope}[xshift=6cm]
                \draw [-stealth] (-2,0) -- (2,0) node[right]{$\R$};
                \draw [-stealth] (0,-2) -- (0,2) node[above]{$i\R$};
    
                \draw [orange,opacity=0.1,line width=3mm] plot[domain=-90:-45] (\x:{cos(90+\x/2)/2+1});
                \draw [red,opacity=0.1,line width=3mm] plot[domain=-45:0] (\x:{cos(90+\x/2)/2+1});
                \draw [blue,opacity=0.1,line width=3mm] plot[domain=0:215] (\x:{cos(90+\x/2)/2+1});
                \draw [green!70!blue,opacity=0.1,line width=3mm] plot[domain=215:450] (\x:{cos(90+\x/2)/2+1});
                \draw [purple,opacity=0.1,line width=3mm] plot[domain=450:630] (\x:{cos(90+\x/2)/2+1});
    
                \draw [yex,very thick,decoration={
                    markings,
                    mark=at position 0.2 with \arrow{>}
                },postaction={decorate}] plot[domain=-90:630,samples=50,smooth] (\x:{cos(90+\x/2)/2+1});
            \end{scope}
        \end{tikzpicture}
        \caption{Motivating the winding number.}
        \label{fig:windingNumberMotivation}
    \end{figure}
    \begin{itemize}
        \item If a path $\gamma$ wraps around 0 more than once, we can break it into segments that individually have logarithms.
        \item Then, we find that
        \begin{equation*}
            \int_\gamma\frac{\dd{z}}{z} = \sum\int_{\gamma_i}\frac{\dd{z}}{z}
            = \text{total change of angle}
        \end{equation*}
        \item This leads into the following definition.
    \end{itemize}
    \item \textbf{Winding number} (of $\gamma$ about $z_0$): The number of times the curve $\gamma$ wraps around $z_0$ counterclockwise. \emph{Denoted by} $\bm{\wn(\gamma,z_0)}$. \emph{Given by}
    \begin{equation*}
        \wn(\gamma,z_0) := \frac{1}{2\pi i}\int_\gamma\frac{\dd{z}}{z-z_0}
    \end{equation*}
    \item Winding number examples.
    \begin{figure}[H]
        \centering
        \begin{subfigure}[b]{0.25\linewidth}
            \centering
            \begin{tikzpicture}
                \fill circle (2pt);
    
                \draw [yex,thick,xscale=1.3,decoration={
                    markings,
                    mark=at position 0.2 with \arrow{>}
                },postaction={decorate}] plot[domain=-90:630,samples=50,smooth] (\x:{(cos(45+\x/2)/3+1)});
            \end{tikzpicture}
            \caption{$\wn=2$.}
            \label{fig:windingNumberExa}
        \end{subfigure}
        \begin{subfigure}[b]{0.25\linewidth}
            \centering
            \begin{tikzpicture}
                \fill circle (2pt);
    
                \path (0,0) -- (0,{-4/3});
                \draw [yex,thick,decoration={
                    markings,
                    mark=at position 0.6 with \arrow{<}
                },postaction={decorate}] circle (1cm);
            \end{tikzpicture}
            \caption{$\wn=-1$.}
            \label{fig:windingNumberExb}
        \end{subfigure}
        \begin{subfigure}[b]{0.25\linewidth}
            \centering
            \begin{tikzpicture}
                \fill circle (2pt);
    
                \draw [
                    yex,thick,xscale=1.2,
                    decoration={
                        markings,
                        mark=at position 0.97 with \arrow{>}
                    },postaction={decorate}
                ]
                    (0,0.6)
                    to[out=170,in=180,looseness=2] (0,-0.7)
                    to[out=0,in=0,looseness=2] (0,0.2)
                    to[out=180,in=110,looseness=1.1] (-0.5,-0.5)
                    to[out=-70,in=0] (-0.7,-0.7)
                    to[out=180,in=180,in looseness=2] (0,0.4)
                    to[out=0,in=0,looseness=2] (0,{-4/3})
                    to[out=180,in=180,looseness=2] (0,1)
                    to[out=0,in=90,in looseness=0.5] (1,0.5)
                    to[out=-90,in=-10,out looseness=0.7] cycle
                ;
            \end{tikzpicture}
            \caption{$\wn=0$.}
            \label{fig:windingNumberExc}
        \end{subfigure}
        \caption{Winding number examples.}
        \label{fig:windingNumberEx}
    \end{figure}
    \item Reformulation: A branch of the logarithm exists on $U$ if and only if for all $\gamma\subset U$, the winding number of $\gamma$ about the origin is 0.
\end{itemize}




\end{document}