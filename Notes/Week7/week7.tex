\documentclass[../notes.tex]{subfiles}

\pagestyle{main}
\renewcommand{\chaptermark}[1]{\markboth{\chaptername\ \thechapter\ (#1)}{}}
\setcounter{chapter}{6}

\begin{document}




\chapter{???}
\section{Generalized Cauchy Theorems}
\begin{itemize}
    \item \marginnote{4/30:}Questions.
    \begin{itemize}
        \item PSet 4, QA.4: III.5.1 instead of II.5.1?
        \begin{itemize}
            \item Yep, should be Chapter 3, not Chapter 2.
        \end{itemize}
        \item PSet 4, QB.4: "a holomorphic branch of the logarithm exists on $U$" or on $f(U)$?
        \begin{itemize}
            \item Yep, should be $f(U)$.
            \item "Which one works, Steven?"
        \end{itemize}
    \end{itemize}
    \item Recall.
    \begin{itemize}
        \item The winding number of a curve $\gamma$ about a point $z_0\in\C$ is
        \begin{equation*}
            \wn(\gamma,z_0) := \frac{1}{2\pi i}\int_\gamma\frac{1}{z-z_0}\dd{z}
        \end{equation*}
        \item We can also compute the winding number geometrically (see Figure \ref{fig:windingNumberEx}).
    \end{itemize}
    \item Additional properties of the winding number.
    \begin{itemize}
        \item The winding number is invariant under homotopies of $\gamma$.
        \item Compute by counting how many times you pass a ray from $z_0$ going counterclockwise!
        \begin{itemize}
            \item Example: "I'm pointing in this direction, then I rotate, and eventually I point in this direction again, then I rotate, and eventually I'm back where I started so it's winding number 2."
        \end{itemize}
        \item We can also think of jumping to a higher plane on the infinity spiral every time we pass the ray.
    \end{itemize}
    \item TPS: Compute the winding number of $\gamma$ about the points in Figure \ref{fig:windingNumberTPS}.
    \begin{figure}[H]
        \centering
        \begin{tikzpicture}[
            every node/.style={black}
        ]
            \footnotesize
            \draw [yex,thick,decoration={
                markings,
                mark=at position 0.005 with \arrow{>},
                mark=at position 0.2 with \arrow{>},
                mark=at position 0.6 with \arrow{>},
                mark=at position 0.77 with \arrow{>},
                mark=at position 0.91 with \arrow{>}
            },postaction={decorate}] (0,0.7)
                to[out=180,in=180,in looseness=2] (0.5,-1)
                to[out=0,in=0,looseness=1.3] (0,1.2)
                to[out=180,in=180,looseness=1.4] (-0.3,-2)
                to[out=0,in=-20,looseness=2] (0.4,1.5)
                to[out=160,in=0] (-1,1.8)
                to[out=180,in=0] (-3.5,-1.7)
                to[out=180,in=180,looseness=1.5] (-3.6,0)
                to[out=0,in=180] (-1.2,-1)
                to[out=0,in=0,out looseness=3] cycle
            ;
            \fill [rex] (-3.5,-0.8) circle (2pt) node[left=1pt]{$a$};
            \fill [rex] (0,0.2) circle (2pt) node[above=1pt]{$b$};
            \fill [rex] (-0.9,-0.1) circle (2pt) node[below=1pt]{$c$};
            \fill [rex] (0.9,-0.3) circle (2pt) node[above=1pt]{$d$};
            \fill [rex] (-0.8,0.5) circle (2pt) node[below=1pt]{$e$};
            \fill [rex] (0.6,0.7) circle (2pt) node[below right=-1pt]{$f$};
            \fill [rex] (0.1,-0.2) circle (2pt) node[below=1pt]{$g$};
        \end{tikzpicture}
        \caption{Winding number regions.}
        \label{fig:windingNumberTPS}
    \end{figure}
    \begin{itemize}
        \item We get
        \begin{align*}
            \wn(\gamma,a) &= -1&
            \wn(\gamma,b) &= 3&
            \wn(\gamma,c) &= 2
        \end{align*}
        \begin{align*}
            \wn(\gamma,d) &= 2&
            \wn(\gamma,e) &= 2&
            \wn(\gamma,f) &= 2&
            \wn(\gamma,g) &= 3
        \end{align*}
        \item Do we notice any patterns?
        \begin{itemize}
            \item Connected regions of the plane appear to yield the same winding number!
            \item We formalize this notion via the following lemma.
        \end{itemize}
    \end{itemize}
    \item Lemma: $\wn(\gamma,z_0)$ is constant on components of $\C\setminus\im(\gamma)$. It is also 0 on the unbounded component.
    \begin{proof}
        We address the two claims sequentially.\par
        Claim 1: Treating $z_0$ as an argument, $\wn(\gamma,z_0)$ is a function from $\C\setminus\im(\gamma)$ to $\Z$ defined by
        \begin{equation*}
            z_0 \mapsto \frac{1}{2\pi i}\int_\gamma\frac{\dd{z}}{z-z_0}
        \end{equation*}
        This is a continuous function into a discrete space and therefore is constant.\par
        Claim 2: Let $z_0$ get very big. Then we can make
        \begin{equation*}
            \left| \frac{1}{2\pi i}\int_\gamma\frac{\dd{z}}{z-z_0} \right|
        \end{equation*}
        arbitrarily small. But an integer that can be made arbitrarily small is just zero.
    \end{proof}
    \begin{itemize}
        \item This is a complex analytic proof of a topological claim.
        \item Justifying that the codomain of the winding number function is the integers: We've done this heuristically using homotopy, but we could formalize it, too.
    \end{itemize}
    \item We now move onto today's main topic: The proof of the (very) general Cauchy Integral Theorem.
    \item First, we need a definition.
    \item \textbf{Simply connected} (domain): A domain $U\subset\C$ such that $\wn(\gamma,z_0)=0$ for all $\gamma\subset U$ and $z\notin U$.
    \begin{itemize}
        \item There are many other definitions, too.
        \begin{itemize}
            \item Topology: The \textbf{fundamental group} of $U$ is zero.
            \item Removing any \textbf{arc} (line segment across the domain) from $U$ turns it into a disconnected set.
            \item For all arcs $\delta_1,\delta_2$ with the same endpoints, $\delta_1$ and $\delta_2$ are homotopic.
        \end{itemize}
        \item The last definition above will be particularly useful for our purposes, as we'll see shortly.
        \item But these are all formal definitions; what can we think about intuitively?
        \begin{itemize}
            \item A good first thing to think about is a blob in the plane.
            \item But the interior of a fractal domain would also count.
            \item A square minus a slit at 1, $1/2$, $1/3$, \dots is also simply connected (though not path connected).
        \end{itemize}
    \end{itemize}
    \item Jordan curve theorem: Suppose $\gamma:S^1\to\C$ is a continuous injection. Then $\gamma$ bounds a disk.
    \begin{itemize}
        \item Consequence: A domain that is simply connected is homeomorphic to a disk.
        \item This appears stupidly obvious, but it was only rigorously proved in the early 1910s.
        \begin{itemize}
            \item The issue is that we don't really know what \emph{continuous} means.
            \item If $\gamma$ is $C^1$, this is easy.
        \end{itemize}
    \end{itemize}
    \item The two generalizations and their proofs.
    \begin{itemize}
        \item The proof of generalization 1 is very simple, straightforward, and clever.
        \item The proof of generalization 2 is much more general and uses almost everything we've done.
    \end{itemize}
    \item We are now ready to state and prove a first generalization of the CIT.
    \item Cauchy Integral Theorem: Suppose that $U$ is simply connected and $f\in\mO(U)$. Then $\int_\gamma f\dd{z}=0$ for any closed loop $\gamma$ in $U$.
    \begin{proof}
        Let $\gamma$ be an arbitrary closed loop in $U$. Because any two arcs with the same endpoints are homotopic, $\gamma$ is homotopic to the constant path $\tilde{\gamma}:[0,1]\to\{\gamma(0)\}$. This constant path has the property that
        \begin{equation*}
            \int_{\tilde{\gamma}}f\dd{z} = \int_0^1f(\tilde{\gamma}(t))\tilde{\gamma}'(t)\dd{t}
            = \int_0^1f(\tilde{\gamma}(t))\cdot 0\dd{t}
            = 0
        \end{equation*}
        Since integrals are the same for homotopic paths, it follows that
        \begin{equation*}
            \int_\gamma f\dd{z} = \int_{\tilde{\gamma}}f\dd{z}
            = 0
        \end{equation*}
        as desired.
    \end{proof}
    \item We now build up to an even more general version of the CIT.
    \begin{figure}[h!]
        \centering
        \begin{subfigure}[b]{0.3\linewidth}
            \centering
            \begin{tikzpicture}
                \filldraw [thick,draw=blx,fill=blz,decoration={
                    markings,
                    mark=at position 0.25 with \arrow{>}
                },postaction=decorate] (-0.5,-2)
                    to[out=0,in=-90] (0,-1.5)
                    to[out=90,in=180] (0.2,-1.2)
                    to[out=0,in=-90] (1,-0.3)
                    to[out=90,in=-60] (0.7,0.5)
                    to[out=120,in=-120] (0.7,0.8)
                    to[out=60,in=0] (0.2,1.5)
                    to[out=180,in=20] (-0.5,0.3)
                    to[out=-160,in=180,out looseness=1.5] cycle
                ;
                \filldraw [thick,draw=blx,fill=blz,rotate=-10,decoration={
                    markings,
                    mark=at position 0 with \arrow{>}
                },postaction=decorate] (-1.5,0.5) ellipse (4mm and 5mm);
            \end{tikzpicture}
            \caption{Separate oriented domains.}
            \label{fig:nulhomExa}
        \end{subfigure}
        \begin{subfigure}[b]{0.3\linewidth}
            \centering
            \begin{tikzpicture}
                \footnotesize
                \filldraw [thick,draw=yex,fill=yez,rotate=-5,decoration={
                    markings,
                    mark=at position 0 with \arrow{>}
                },postaction=decorate] ellipse (1.3cm and 1.4cm);
                \filldraw [thick,draw=yex,fill=white,rotate=-5,decoration={
                    markings,
                    mark=at position 0 with \arrow{<}
                },postaction=decorate] (0,-0.1) ellipse (0.7cm and 0.8cm);
    
                \fill [rex] (0,-0.15) circle (2pt) node[black,above right=-1pt]{$d$};
            \end{tikzpicture}
            \caption{Nested curves.}
            \label{fig:nulhomExb}
        \end{subfigure}
        \begin{subfigure}[b]{0.3\linewidth}
            \centering
            \begin{tikzpicture}[
                every node/.style=black
            ]
                \footnotesize
                \filldraw [thick,draw=orx,fill=orz,decoration={
                        markings,
                        mark=at position 0.45 with \arrow{>}
                    },postaction=decorate] (-2,-0.2)
                    to[out=-90,in=173,out looseness=1.4] (-0.3,-1.3)
                    to[out=-7,in=180,out looseness=1.5] (1.5,-1.7)
                    to[out=0,in=-90] (2.7,-0.5)
                    to[out=90,in=0] (0,1.2)
                    to[out=180,in=90] cycle
                ;
                \filldraw [thick,draw=orx,fill=white,rotate around={-40:(-1,0.2)},decoration={
                    markings,
                    mark=at position 0.9 with \arrow{<}
                },postaction=decorate] (-1,0.2) ellipse (4mm and 5mm);
                \filldraw [thick,draw=orx,fill=white,rotate around={-10:(0.15,-0.5)},decoration={
                    markings,
                    mark=at position 0.55 with \arrow{<}
                },postaction=decorate] (0.15,-0.5) ellipse (5mm and 6mm);
                \filldraw [thick,draw=orx,fill=white,rotate around={5:(1.7,-0.1)},decoration={
                    markings,
                    mark=at position 0.65 with \arrow{<}
                },postaction=decorate] (1.7,-0.1) ellipse (5mm and 5.5mm);
    
                \fill [rex] (-1.1,0.1) circle (2pt) node[below right=-1pt]{$a$};
                \node at (-0.8,0.3) {$\gamma_4^{}$};
                \fill [rex] (0,-0.6) circle (2pt) node[below right=-1pt]{$b$};
                \node at (0.4,-0.4) {$\gamma_3^{}$};
                \fill [rex] (1.6,-0.2) circle (2pt) node[below right=-1pt]{$c$};
                \node at (1.9,0.1) {$\gamma_2^{}$};
                \node at (2.6,-1.5) {$\gamma_1^{}$};
            \end{tikzpicture}
            \caption{Multiple nested curves.}
            \label{fig:nulhomExc}
        \end{subfigure}
        \caption{Nulhomologous multicurve examples.}
        \label{fig:nulhomEx}
    \end{figure}
    \begin{itemize}
        \item Suppose $D\subset\C$ is a bounded domain, and $\partial D$ is a union of disjoint simple closed curves (SCCs).
        \item Let $\partial\vec{D}$ be the union of the boundaries, oriented so that $D$ is on the left.
        \begin{itemize}
            \item This is similar to how we orient curves when we're applying Stokes' Theorem.
            \item Here as well, the outer one goes counterclockwise and the inner one(s) goes clockwise.
        \end{itemize}
        \item More generally, we define a the concept of a \textbf{multicurve}.
        \item Using this definition, we define the \textbf{integral} of $f$ over a multicurve.
        \begin{itemize}
            \item This definition allows us to compute the winding number of $\Gamma$ about $z_0$.
        \end{itemize}
        \item Lastly, we define a special kind of multicurve called a \textbf{nulhomologous} multicurve.
        \begin{itemize}
            \item In Figure \ref{fig:nulhomExc}, $\Gamma=\gamma_1+\gamma_2+\gamma_3+\gamma_4$ is nulhomologous.
        \end{itemize}
    \end{itemize}
    \item \textbf{Multicurve}: A formal sum of SCCs $\gamma_i$ multiplied by coefficients $c_i\in\C$. \emph{Denoted by} $\bm{\Gamma}$. \emph{Given by}
    \begin{equation*}
        \Gamma = \sum c_i\gamma_i
    \end{equation*}
    \item \textbf{Integral} (of $f$ over $\Gamma$): The path integral defined as follows. \emph{Denoted by} $\bm{\int_\Gamma f\,\textbf{d}z}$. \emph{Given by}
    \begin{equation*}
        \int_\Gamma f\dd{z} := \sum_{i=0}^nc_i\int_{\gamma_i}f\dd{z}
    \end{equation*}
    \item \textbf{Nulhomologous} ($\Gamma$ in $U$): A multicurve $\Gamma$ in a domain $U$ for which $\Gamma=\partial\vec{D}$ for $D$ as in Figure \ref{fig:nulhomEx}. \emph{Also known as} \textbf{homologous} ($\Gamma$ in $U$ to 0).
    \item TPS: Compute $\wn(\partial\vec{D},z_0)$ for all $z_0\notin D$ for each of the domains $D$ in Figure \ref{fig:nulhomEx}.
    \begin{itemize}
        \item $\wn(\partial\vec{D},z_0)=0$ because we always get either nothing or a $+1$ and $-1$ and some zeroes.
    \end{itemize}
    \item Lemma: If $\Gamma$ is nulhomologous in $U$, then for all $z\notin U$, $\wn(\Gamma,z)=0$.
    \begin{itemize}
        \item The converse is not true!
        \begin{itemize}
            \item Example: If $U=\C^*$ and $\gamma_1,\gamma_2$ are intersecting closed curves (e.g., the unit circle and the unit circle translated half a unit to the right), then $\gamma_1+\gamma_2$ is still nulhomologous even though it doesn't bound a domain.
        \end{itemize}
        \item The condition "for all $z\notin U$, $\wn(\Gamma,z)=0$" is our general definition of nulhomologous in $U$; what we said earlier was just a precursor definition.
    \end{itemize}
    \item Example.
    \begin{figure}[h!]
        \centering
        \begin{tikzpicture}[scale=1.3]
            \filldraw [thick,draw=orx,fill=orz,decoration={
                markings,
                mark=at position 0.25 with \arrow{>}
            },postaction=decorate] (0.5,-0.4) circle (1.3cm);
            \filldraw [thick,draw=orx,fill=white,decoration={
                markings,
                mark=at position 0.25 with \arrow{<}
            },postaction=decorate] (0,0) circle (3mm);
            \filldraw [thick,draw=orx,fill=white,decoration={
                markings,
                mark=at position 0.25 with \arrow{<}
            },postaction=decorate] (1,0) circle (3mm);
            \filldraw [thick,draw=orx,fill=white,decoration={
                markings,
                mark=at position 0.63 with \arrow{<}
            },postaction=decorate] (-0.3,-0.8)
                to[out=-90,in=-90] (1.3,-0.8)
                to[out=90,in=0,in looseness=2] (0.5,-1)
                to[out=180,in=90,out looseness=2] cycle
            ;
    
            \node [rex,label={[yshift=2mm]below:\footnotesize 0}] at (0,0) {$*$};
            \node [rex,label={[yshift=2mm]below:\footnotesize 1}] at (1,0) {$*$};
        \end{tikzpicture}
        \caption{Nulhomologous multicurve in a punctured domain.}
        \label{fig:nulhomPuncture}
    \end{figure}
    \begin{itemize}
        \item Let
        \begin{equation*}
            f(z) = \frac{\sin(1/z)}{z-1}
        \end{equation*}
        \item Then $f\in\mO(\C\setminus\{0,1\})$.
        \item An example of a nulhomologous multicurve over which we could integrate $f$ is as follows.
    \end{itemize}
    \item We are now ready for the statement and proof of the most general version of the CIT and CIF we'll see in this course.
    \item Suppose $U$ is any domain, $\Gamma\subset U$ is nulhomologous, and $f\in\mO(U)$. Then:
    \begin{enumerate}
        \item General CIT: We have that
        \begin{equation*}
            \int_\Gamma f\dd{z} = 0
        \end{equation*}
        \item General CIF: For all $z\in U$ and not in $\im(\Gamma)$,
        \begin{equation*}
            \wn(\Gamma,z)\cdot f(z) = \frac{1}{2\pi i}\int_\Gamma\frac{f(\zeta)}{\zeta-z}\dd\zeta
        \end{equation*}
    \end{enumerate}
    \pagebreak
    \item Discussion of the proof.
    \begin{itemize}
        \item We'll sketch the proof today.
        \item Think back to the proof for star-shaped domains.
        \item We proved the CIT by saying, "if it's true for triangles, then we win."
        \begin{itemize}
            \item Using triangles, we built a primitive and then invoked Goursat's Lemma.
        \end{itemize}
        \item We proved the CIF by first defining the helper function
        \begin{equation*}
            g(\zeta) =
            \begin{cases}
                \frac{f(\zeta)-f(z)}{\zeta-z} & \zeta\neq z\\
                f'(z) & \zeta=z
            \end{cases}
        \end{equation*}
        \begin{itemize}
            \item Then we invoked the CIT to say
            \begin{equation*}
                \int_{\partial D}g\dd{z} = 0
            \end{equation*}
            \item The CIF then followed from this and the fact that
            \begin{equation*}
                \int_{\partial D}g\dd\zeta = \int_{\partial D}\frac{f(\zeta)}{\zeta-z}\dd\zeta-f(z)\underbrace{\int_{\partial D}\frac{1}{\zeta-z}\dd\zeta}_{2\pi i}
            \end{equation*}
        \end{itemize}
        \item So can't we just replace all the $\partial D$'s with $\Gamma$'s in the above lines and call it a day?
        \begin{itemize}
            \item No, because there's no analogy for the CIT. In other words, there may not be a primitive.
            \item Thus, we need to fix $\int_{\partial D}g\dd{z}=0$.
        \end{itemize}
        \item In sum, the idea of this proof is to prove the CIF and then simply get the CIT.
        \item We'll have time to prove the CIF today, but probably will not to get to the CIT.
    \end{itemize}
    \item We are now ready to sketch the full proof in broad strokes.
    \begin{proof}
        Define $h:U\to\C$ by
        \begin{equation*}
            h(z) := \int_\Gamma g(\zeta,z)\dd\zeta
        \end{equation*}
        We want to show that $h(z)=0$. We can't do anything as nice as showing that it's a continuous map into a discrete space, but there is still a clever idea. First off, we can see that $h(z)\to 0$ as $z\to\infty$ in $U$. Essentially, as before, the denominator $\zeta-z$ gets really big so the first term gets really small and the second term has that $\wn(\Gamma,z)$ term which goes to 0. What we now need to show is that $h$ extends to an entire function so that we can make the denominator \emph{arbitrarily} large. This is where we use the assumption that $\Gamma$ is nulhomologous.\par
        First, we will show that $h$ is continuous. We know that $g$ is continuous in $(\zeta,z)$ together. We have holomorphic in $\zeta$ for a fixed $z$.\footnote{There's a bit more detail in the notes, but not much.}\par
        Next, we need to show that $h$ is holomorphic on $U$. We know that $h$ is holomorphic as long as $z\neq\zeta$. On the other hand, what if $\zeta=z$? We will invoke Morera's theorem.\footnote{There's a bit about the triangle integral condition in the notes.}\par
        Last, we show that $h$ can be analytically continued outside of $U$. We know that on $U$,
        \begin{equation*}
            h(z) = \int\frac{f(\zeta)}{\zeta-z}\dd\zeta-f(z)\cdot 2\pi i\wn(\Gamma,z)
        \end{equation*}
        Outside of $U$, the second term disappears because $\Gamma$ is nulhomologous. Define
        \begin{equation*}
            h(z) := \int\frac{f(\zeta)}{\zeta-z}\dd\zeta
        \end{equation*}
        outside of $U$. Thus, we have two functions that agree on a patch, so we get analytic continuation.\par
        From here, we have an entire function that converges to $0$ at $\infty$ (hence is bounded), so is constant by Liouville's theorem with value that converges to zero (hence is zero).\footnote{There is a bit on the CIT in the notes.}
    \end{proof}
\end{itemize}




\end{document}