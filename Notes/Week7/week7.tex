\documentclass[../notes.tex]{subfiles}

\pagestyle{main}
\renewcommand{\chaptermark}[1]{\markboth{\chaptername\ \thechapter\ (#1)}{}}
\setcounter{chapter}{6}

\begin{document}




\chapter{The Winding Number}
\section{Generalized Cauchy Theorems}
\begin{itemize}
    \item \marginnote{4/30:}Questions.
    \begin{itemize}
        \item PSet 4, QA.4: III.5.1 instead of II.5.1?
        \begin{itemize}
            \item Yep, should be Chapter 3, not Chapter 2.
        \end{itemize}
        \item PSet 4, QB.4: "a holomorphic branch of the logarithm exists on $U$" or on $f(U)$?
        \begin{itemize}
            \item Yep, should be $f(U)$.
            \item "Which one works, Steven?"
        \end{itemize}
    \end{itemize}
    \item Recall.
    \begin{itemize}
        \item The winding number of a curve $\gamma$ about a point $z_0\in\C$ is
        \begin{equation*}
            \wn(\gamma,z_0) := \frac{1}{2\pi i}\int_\gamma\frac{1}{z-z_0}\dd{z}
        \end{equation*}
        \item We can also compute the winding number geometrically (see Figure \ref{fig:windingNumberEx}).
    \end{itemize}
    \item Additional properties of the winding number.
    \begin{itemize}
        \item The winding number is invariant under homotopies of $\gamma$.
        \item Compute by counting how many times you pass a ray from $z_0$ going counterclockwise!
        \begin{itemize}
            \item Example: "I'm pointing in this direction, then I rotate, and eventually I point in this direction again, then I rotate, and eventually I'm back where I started so it's winding number 2."
        \end{itemize}
        \item We can also think of jumping to a higher plane on the infinity spiral every time we pass the ray.
    \end{itemize}
    \item TPS: Compute the winding number of $\gamma$ about the points in Figure \ref{fig:windingNumberTPS}.
    \begin{figure}[H]
        \centering
        \begin{tikzpicture}[
            every node/.style={black}
        ]
            \footnotesize
            \draw [yex,thick,decoration={
                markings,
                mark=at position 0.005 with \arrow{>},
                mark=at position 0.2 with \arrow{>},
                mark=at position 0.6 with \arrow{>},
                mark=at position 0.77 with \arrow{>},
                mark=at position 0.91 with \arrow{>}
            },postaction={decorate}] (0,0.7)
                to[out=180,in=180,in looseness=2] (0.5,-1)
                to[out=0,in=0,looseness=1.3] (0,1.2)
                to[out=180,in=180,looseness=1.4] (-0.3,-2)
                to[out=0,in=-20,looseness=2] (0.4,1.5)
                to[out=160,in=0] (-1,1.8)
                to[out=180,in=0] (-3.5,-1.7)
                to[out=180,in=180,looseness=1.5] (-3.6,0)
                to[out=0,in=180] (-1.2,-1)
                to[out=0,in=0,out looseness=3] cycle
            ;
            \fill [rex] (-3.5,-0.8) circle (2pt) node[left=1pt]{$a$};
            \fill [rex] (0,0.2) circle (2pt) node[above=1pt]{$b$};
            \fill [rex] (-0.9,-0.1) circle (2pt) node[below=1pt]{$c$};
            \fill [rex] (0.9,-0.3) circle (2pt) node[above=1pt]{$d$};
            \fill [rex] (-0.8,0.5) circle (2pt) node[below=1pt]{$e$};
            \fill [rex] (0.6,0.7) circle (2pt) node[below right=-1pt]{$f$};
            \fill [rex] (0.1,-0.2) circle (2pt) node[below=1pt]{$g$};
        \end{tikzpicture}
        \caption{Winding number regions.}
        \label{fig:windingNumberTPS}
    \end{figure}
    \begin{itemize}
        \item We get
        \begin{align*}
            \wn(\gamma,a) &= -1&
            \wn(\gamma,b) &= 3&
            \wn(\gamma,c) &= 2
        \end{align*}
        \begin{align*}
            \wn(\gamma,d) &= 2&
            \wn(\gamma,e) &= 2&
            \wn(\gamma,f) &= 2&
            \wn(\gamma,g) &= 3
        \end{align*}
        \item Do we notice any patterns?
        \begin{itemize}
            \item Connected regions of the plane appear to yield the same winding number!
            \item We formalize this notion via the following lemma.
        \end{itemize}
    \end{itemize}
    \item Lemma: $\wn(\gamma,z_0)$ is constant on components of $\C\setminus\im(\gamma)$. It is also 0 on the unbounded component.
    \begin{proof}
        We address the two claims sequentially.\par
        Claim 1: Treating $z_0$ as an argument, $\wn(\gamma,z_0)$ is a function from $\C\setminus\im(\gamma)$ to $\Z$ defined by
        \begin{equation*}
            z_0 \mapsto \frac{1}{2\pi i}\int_\gamma\frac{\dd{z}}{z-z_0}
        \end{equation*}
        This is a continuous function into a discrete space and therefore is constant.\par
        Claim 2: Let $z_0$ get very big. Then we can make
        \begin{equation*}
            \left| \frac{1}{2\pi i}\int_\gamma\frac{\dd{z}}{z-z_0} \right|
        \end{equation*}
        arbitrarily small. But an integer that can be made arbitrarily small is just zero.
    \end{proof}
    \begin{itemize}
        \item This is a complex analytic proof of a topological claim.
        \item Justifying that the codomain of the winding number function is the integers: We've done this heuristically using homotopy, but we could formalize it, too.
    \end{itemize}
    \item We now move onto today's main topic: The proof of the (very) general Cauchy Integral Theorem.
    \item First, we need a definition.
    \item \textbf{Simply connected} (domain): A domain $U\subset\C$ such that $\wn(\gamma,z_0)=0$ for all $\gamma\subset U$ and $z\notin U$.
    \begin{itemize}
        \item There are many other definitions, too.
        \begin{itemize}
            \item Topology: The \textbf{fundamental group} of $U$ is zero.
            \item Removing any \textbf{arc} (line segment across the domain) from $U$ turns it into a disconnected set.
            \item For all arcs $\delta_1,\delta_2$ with the same endpoints, $\delta_1$ and $\delta_2$ are homotopic.
        \end{itemize}
        \item The last definition above will be particularly useful for our purposes, as we'll see shortly.
        \item But these are all formal definitions; what can we think about intuitively?
        \begin{itemize}
            \item A good first thing to think about is a blob in the plane.
            \item But the interior of a fractal domain would also count.
            \item A square minus a slit at 1, $1/2$, $1/3$, \dots is also simply connected (though not path connected).
        \end{itemize}
    \end{itemize}
    \item Jordan curve theorem: Suppose $\gamma:S^1\to\C$ is a continuous injection. Then $\gamma$ bounds a disk.
    \begin{itemize}
        \item Consequence: A domain that is simply connected is homeomorphic to a disk.
        \item This appears stupidly obvious, but it was only rigorously proved in the early 1910s.
        \begin{itemize}
            \item The issue is that we don't really know what \emph{continuous} means.
            \item If $\gamma$ is $C^1$, this is easy.
        \end{itemize}
    \end{itemize}
    \item The two generalizations and their proofs.
    \begin{itemize}
        \item The proof of generalization 1 is very simple, straightforward, and clever.
        \item The proof of generalization 2 is much more general and uses almost everything we've done.
    \end{itemize}
    \item We are now ready to state and prove a first generalization of the CIT.
    \item Cauchy Integral Theorem: Suppose that $U$ is simply connected and $f\in\mO(U)$. Then $\int_\gamma f\dd{z}=0$ for any closed loop $\gamma$ in $U$.
    \begin{proof}
        Let $\gamma$ be an arbitrary closed loop in $U$. Because any two arcs with the same endpoints are homotopic, $\gamma$ is homotopic to the constant path $\tilde{\gamma}:[0,1]\to\{\gamma(0)\}$. This constant path has the property that
        \begin{equation*}
            \int_{\tilde{\gamma}}f\dd{z} = \int_0^1f(\tilde{\gamma}(t))\tilde{\gamma}'(t)\dd{t}
            = \int_0^1f(\tilde{\gamma}(t))\cdot 0\dd{t}
            = 0
        \end{equation*}
        Since integrals are the same for homotopic paths, it follows that
        \begin{equation*}
            \int_\gamma f\dd{z} = \int_{\tilde{\gamma}}f\dd{z}
            = 0
        \end{equation*}
        as desired.
    \end{proof}
    \item We now build up to an even more general version of the CIT.
    \begin{figure}[h!]
        \centering
        \begin{subfigure}[b]{0.3\linewidth}
            \centering
            \begin{tikzpicture}
                \filldraw [thick,draw=blx,fill=blz,decoration={
                    markings,
                    mark=at position 0.25 with \arrow{>}
                },postaction=decorate] (-0.5,-2)
                    to[out=0,in=-90] (0,-1.5)
                    to[out=90,in=180] (0.2,-1.2)
                    to[out=0,in=-90] (1,-0.3)
                    to[out=90,in=-60] (0.7,0.5)
                    to[out=120,in=-120] (0.7,0.8)
                    to[out=60,in=0] (0.2,1.5)
                    to[out=180,in=20] (-0.5,0.3)
                    to[out=-160,in=180,out looseness=1.5] cycle
                ;
                \filldraw [thick,draw=blx,fill=blz,rotate=-10,decoration={
                    markings,
                    mark=at position 0 with \arrow{>}
                },postaction=decorate] (-1.5,0.5) ellipse (4mm and 5mm);
            \end{tikzpicture}
            \caption{Separate oriented domains.}
            \label{fig:nulhomExa}
        \end{subfigure}
        \begin{subfigure}[b]{0.3\linewidth}
            \centering
            \begin{tikzpicture}
                \footnotesize
                \filldraw [thick,draw=yex,fill=yez,rotate=-5,decoration={
                    markings,
                    mark=at position 0 with \arrow{>}
                },postaction=decorate] ellipse (1.3cm and 1.4cm);
                \filldraw [thick,draw=yex,fill=white,rotate=-5,decoration={
                    markings,
                    mark=at position 0 with \arrow{<}
                },postaction=decorate] (0,-0.1) ellipse (0.7cm and 0.8cm);
    
                \fill [rex] (0,-0.15) circle (2pt) node[black,above right=-1pt]{$d$};
            \end{tikzpicture}
            \caption{Nested curves.}
            \label{fig:nulhomExb}
        \end{subfigure}
        \begin{subfigure}[b]{0.3\linewidth}
            \centering
            \begin{tikzpicture}[
                every node/.style=black
            ]
                \footnotesize
                \filldraw [thick,draw=orx,fill=orz,decoration={
                        markings,
                        mark=at position 0.45 with \arrow{>}
                    },postaction=decorate] (-2,-0.2)
                    to[out=-90,in=173,out looseness=1.4] (-0.3,-1.3)
                    to[out=-7,in=180,out looseness=1.5] (1.5,-1.7)
                    to[out=0,in=-90] (2.7,-0.5)
                    to[out=90,in=0] (0,1.2)
                    to[out=180,in=90] cycle
                ;
                \filldraw [thick,draw=orx,fill=white,rotate around={-40:(-1,0.2)},decoration={
                    markings,
                    mark=at position 0.9 with \arrow{<}
                },postaction=decorate] (-1,0.2) ellipse (4mm and 5mm);
                \filldraw [thick,draw=orx,fill=white,rotate around={-10:(0.15,-0.5)},decoration={
                    markings,
                    mark=at position 0.55 with \arrow{<}
                },postaction=decorate] (0.15,-0.5) ellipse (5mm and 6mm);
                \filldraw [thick,draw=orx,fill=white,rotate around={5:(1.7,-0.1)},decoration={
                    markings,
                    mark=at position 0.65 with \arrow{<}
                },postaction=decorate] (1.7,-0.1) ellipse (5mm and 5.5mm);
    
                \fill [rex] (-1.1,0.1) circle (2pt) node[below right=-1pt]{$a$};
                \node at (-0.8,0.3) {$\gamma_4^{}$};
                \fill [rex] (0,-0.6) circle (2pt) node[below right=-1pt]{$b$};
                \node at (0.4,-0.4) {$\gamma_3^{}$};
                \fill [rex] (1.6,-0.2) circle (2pt) node[below right=-1pt]{$c$};
                \node at (1.9,0.1) {$\gamma_2^{}$};
                \node at (2.6,-1.5) {$\gamma_1^{}$};
            \end{tikzpicture}
            \caption{Multiple nested curves.}
            \label{fig:nulhomExc}
        \end{subfigure}
        \caption{Nulhomologous multicurve examples.}
        \label{fig:nulhomEx}
    \end{figure}
    \begin{itemize}
        \item Suppose $D\subset\C$ is a bounded domain, and $\partial D$ is a union of disjoint simple closed curves (SCCs).
        \item Let $\partial\vec{D}$ be the union of the boundaries, oriented so that $D$ is on the left.
        \begin{itemize}
            \item This is similar to how we orient curves when we're applying Stokes' Theorem.
            \item Here as well, the outer one goes counterclockwise and the inner one(s) goes clockwise.
        \end{itemize}
        \item More generally, we define a the concept of a \textbf{multicurve}.
        \item Using this definition, we define the \textbf{integral} of $f$ over a multicurve.
        \begin{itemize}
            \item This definition allows us to compute the winding number of $\Gamma$ about $z_0$.
        \end{itemize}
        \item Lastly, we define a special kind of multicurve called a \textbf{nulhomologous} multicurve.
        \begin{itemize}
            \item In Figure \ref{fig:nulhomExc}, $\Gamma=\gamma_1+\gamma_2+\gamma_3+\gamma_4$ is nulhomologous.
        \end{itemize}
    \end{itemize}
    \item \textbf{Multicurve}: A formal sum of SCCs $\gamma_i$ multiplied by coefficients $c_i\in\C$. \emph{Denoted by} $\bm{\Gamma}$. \emph{Given by}
    \begin{equation*}
        \Gamma = \sum c_i\gamma_i
    \end{equation*}
    \item \textbf{Integral} (of $f$ over $\Gamma$): The path integral defined as follows. \emph{Denoted by} $\bm{\int_\Gamma f\,\textbf{d}z}$. \emph{Given by}
    \begin{equation*}
        \int_\Gamma f\dd{z} := \sum_{i=0}^nc_i\int_{\gamma_i}f\dd{z}
    \end{equation*}
    \item \textbf{Nulhomologous} ($\Gamma$ in $U$): A multicurve $\Gamma$ in a domain $U$ for which $\Gamma=\partial\vec{D}$ for $D$ as in Figure \ref{fig:nulhomEx}. \emph{Also known as} \textbf{homologous} ($\Gamma$ in $U$ to 0).
    \item TPS: Compute $\wn(\partial\vec{D},z_0)$ for all $z_0\notin D$ for each of the domains $D$ in Figure \ref{fig:nulhomEx}.
    \begin{itemize}
        \item $\wn(\partial\vec{D},z_0)=0$ because we always get either nothing or a $+1$ and $-1$ and some zeroes.
    \end{itemize}
    \item Lemma: If $\Gamma$ is nulhomologous in $U$, then for all $z\notin U$, $\wn(\Gamma,z)=0$.
    \begin{itemize}
        \item The converse is not true!
        \begin{itemize}
            \item Example: If $U=\C^*$ and $\gamma_1,\gamma_2$ are intersecting closed curves (e.g., the unit circle and the unit circle translated half a unit to the right), then $\gamma_1+\gamma_2$ is still nulhomologous even though it doesn't bound a domain.
        \end{itemize}
        \item The condition "for all $z\notin U$, $\wn(\Gamma,z)=0$" is our general definition of nulhomologous in $U$; what we said earlier was just a precursor definition.
    \end{itemize}
    \item Example.
    \begin{figure}[h!]
        \centering
        \begin{tikzpicture}[scale=1.3]
            \filldraw [thick,draw=orx,fill=orz,decoration={
                markings,
                mark=at position 0.25 with \arrow{>}
            },postaction=decorate] (0.5,-0.4) circle (1.3cm);
            \filldraw [thick,draw=orx,fill=white,decoration={
                markings,
                mark=at position 0.25 with \arrow{<}
            },postaction=decorate] (0,0) circle (3mm);
            \filldraw [thick,draw=orx,fill=white,decoration={
                markings,
                mark=at position 0.25 with \arrow{<}
            },postaction=decorate] (1,0) circle (3mm);
            \filldraw [thick,draw=orx,fill=white,decoration={
                markings,
                mark=at position 0.63 with \arrow{<}
            },postaction=decorate] (-0.3,-0.8)
                to[out=-90,in=-90] (1.3,-0.8)
                to[out=90,in=0,in looseness=2] (0.5,-1)
                to[out=180,in=90,out looseness=2] cycle
            ;
    
            \node [rex,label={[yshift=2mm]below:\footnotesize 0}] at (0,0) {$*$};
            \node [rex,label={[yshift=2mm]below:\footnotesize 1}] at (1,0) {$*$};
        \end{tikzpicture}
        \caption{Nulhomologous multicurve in a punctured domain.}
        \label{fig:nulhomPuncture}
    \end{figure}
    \begin{itemize}
        \item Let
        \begin{equation*}
            f(z) = \frac{\sin(1/z)}{z-1}
        \end{equation*}
        \item Then $f\in\mO(\C\setminus\{0,1\})$.
        \item An example of a nulhomologous multicurve over which we could integrate $f$ is as follows.
    \end{itemize}
    \item We are now ready for the statement and proof of the most general version of the CIT and CIF we'll see in this course.
    \item Suppose $U$ is any domain, $\Gamma\subset U$ is nulhomologous, and $f\in\mO(U)$. Then:
    \begin{enumerate}
        \item General CIT: We have that
        \begin{equation*}
            \int_\Gamma f\dd{z} = 0
        \end{equation*}
        \item General CIF: For all $z\in U$ and not in $\im(\Gamma)$,
        \begin{equation*}
            \wn(\Gamma,z)\cdot f(z) = \frac{1}{2\pi i}\int_\Gamma\frac{f(\zeta)}{\zeta-z}\dd\zeta
        \end{equation*}
    \end{enumerate}
    \pagebreak
    \item Discussion of the proof.
    \begin{itemize}
        \item We'll sketch the proof today.
        \item Think back to the proof for star-shaped domains.
        \item We proved the CIT by saying, "if it's true for triangles, then we win."
        \begin{itemize}
            \item Using triangles, we built a primitive and then invoked Goursat's Lemma.
        \end{itemize}
        \item We proved the CIF by first defining the helper function
        \begin{equation*}
            g(\zeta) =
            \begin{cases}
                \frac{f(\zeta)-f(z)}{\zeta-z} & \zeta\neq z\\
                f'(z) & \zeta=z
            \end{cases}
        \end{equation*}
        \begin{itemize}
            \item Then we invoked the CIT to say
            \begin{equation*}
                \int_{\partial D}g\dd{z} = 0
            \end{equation*}
            \item The CIF then followed from this and the fact that
            \begin{equation*}
                \int_{\partial D}g\dd\zeta = \int_{\partial D}\frac{f(\zeta)}{\zeta-z}\dd\zeta-f(z)\underbrace{\int_{\partial D}\frac{1}{\zeta-z}\dd\zeta}_{2\pi i}
            \end{equation*}
        \end{itemize}
        \item So can't we just replace all the $\partial D$'s with $\Gamma$'s in the above lines and call it a day?
        \begin{itemize}
            \item No, because there's no analogy for the CIT. In other words, there may not be a primitive.
            \item Thus, we need to fix $\int_{\partial D}g\dd{z}=0$.
        \end{itemize}
        \item In sum, the idea of this proof is to prove the CIF and then simply get the CIT.
        \item We'll have time to prove the CIF today, but probably will not to get to the CIT.
    \end{itemize}
    \item We are now ready to sketch the full proof in broad strokes.
    \begin{proof}
        Define $h:U\to\C$ by
        \begin{equation*}
            h(z) := \int_\Gamma g(\zeta,z)\dd\zeta
        \end{equation*}
        We want to show that $h(z)=0$. We can't do anything as nice as showing that it's a continuous map into a discrete space, but there is still a clever idea. First off, we can see that $h(z)\to 0$ as $z\to\infty$ in $U$. Essentially, as before, the denominator $\zeta-z$ gets really big so the first term gets really small and the second term has that $\wn(\Gamma,z)$ term which goes to 0. What we now need to show is that $h$ extends to an entire function so that we can make the denominator \emph{arbitrarily} large. This is where we use the assumption that $\Gamma$ is nulhomologous.\par
        First, we will show that $h$ is continuous. We know that $g$ is continuous in $(\zeta,z)$ together. We have holomorphic in $\zeta$ for a fixed $z$.\footnote{There's a bit more detail in the notes, but not much.}\par
        Next, we need to show that $h$ is holomorphic on $U$. We know that $h$ is holomorphic as long as $z\neq\zeta$. On the other hand, what if $\zeta=z$? We will invoke Morera's theorem.\footnote{There's a bit about the triangle integral condition in the notes.}\par
        Last, we show that $h$ can be analytically continued outside of $U$. We know that on $U$,
        \begin{equation*}
            h(z) = \int\frac{f(\zeta)}{\zeta-z}\dd\zeta-f(z)\cdot 2\pi i\wn(\Gamma,z)
        \end{equation*}
        Outside of $U$, the second term disappears because $\Gamma$ is nulhomologous. Define
        \begin{equation*}
            h(z) := \int\frac{f(\zeta)}{\zeta-z}\dd\zeta
        \end{equation*}
        outside of $U$. Thus, we have two functions that agree on a patch, so we get analytic continuation.\par
        From here, we have an entire function that converges to $0$ at $\infty$ (hence is bounded), so is constant by Liouville's theorem with value that converges to zero (hence is zero).\footnote{There is a bit on the CIT in the notes.}
    \end{proof}
\end{itemize}



\section{The Residue Theorem}
\begin{itemize}
    \item \marginnote{5/2:}Reminder: PSet 4 due tomorrow.
    \item Recall.
    \begin{itemize}
        \item Cauchy integral theorem for star-shaped/simply connected domains: Given such a domain $U$ and $f\in\mO(U)$, $f$ has a primitive on $U$. It follows that for all $\gamma\subset U$, $\int_\gamma f\dd{z}=0$.
        \item A cycle $\Gamma=\sum c_i\gamma_i$ in a domain $U$ is \textbf{nulhomologous} if for all $z\notin U$,
        \begin{equation*}
            \wn(\Gamma,z) := \sum c_i\wn(\gamma_i,z)
            = 0
        \end{equation*}
        \item CIT/CIF in general: Let $U$ be a domain, $\Gamma\subset U$ nulhomologous in $U$, and $f\in\mO(U)$.
        \begin{enumerate}
            \item We have
            \begin{equation*}
                \int_\Gamma f\dd{z} = \sum c_i\int_{\gamma_i}f\dd{z}
                = 0
            \end{equation*}
            \item For all $z\notin\im(\Gamma)$,
            \begin{equation*}
                \wn(\Gamma,z)\cdot f(z) = \frac{1}{2\pi i}\int_\Gamma\frac{f(\zeta)}{\zeta-z}\dd\zeta
            \end{equation*}
        \end{enumerate}
        \item This is the most general version of the most important theorem(s) in this class.
        \item Comparing and contrasting the old and the new Cauchy theorems.
        \begin{itemize}
            \item In the first one, we put a restriction on our domain and no restriction on our curve. In the new one, we put a restriction on our curve and no restriction on our domain.
            \item In the old one, we worked to construct a primitive for $f$ on $U$. In the new one, $f$ need not have a primitive on $U$.
        \end{itemize}
    \end{itemize}
    \item Further multicurve examples.
    \begin{figure}[h!]
        \centering
        \begin{tikzpicture}[scale=1.5]
            \footnotesize
            \draw [thick,blx,decoration={
                markings,
                mark=at position 0.11 with {\node[black,right]{$\gamma_3^{}$};},
                mark=at position 0.12 with \arrow{>},
                mark=at position 0.26 with \arrow{>},
                mark=at position 0.355 with \arrow{>},
                mark=at position 0.59 with \arrow{>},
                mark=at position 0.693 with \arrow{>},
                mark=at position 0.96 with \arrow{>}
            },postaction=decorate] (3.1,0.6)
                to[out=180,in=180,out looseness=1.2,in looseness=1.7] (3.2,-0.7)
                to[out=0,in=0,out looseness=1.4,in looseness=1] (3.2,1)
                to[out=180,in=0,looseness=0.8] (0.9,-0.5)
                to[out=180,in=180,looseness=1.3] (1,0.4)
                to[out=0,in=0,looseness=1.3] (1,-0.45)
                to[out=180,in=180,looseness=1.6] (1.2,0.35)
                to[out=0,in=0,in looseness=1.8] (1,-0.8)
                to[out=180,in=-90] (0.5,-0.1)
                to[out=90,in=0,out looseness=2] (1,1)
                to[out=180,in=0,in looseness=0.7] (0.1,-0.5)
                to[out=180,in=180,looseness=1.5] (0,0.5)
                to[out=0,in=90,out looseness=1.1,in looseness=0.9] (0.3,0)
                to[out=-90,in=90] (0,-1)
                to[out=-90,in=-90,out looseness=0.7] (3.6,0)
                to[out=90,in=0] cycle
            ;
    
            \draw [thick,orx,decoration={
                markings,
                mark=at position 0.25 with \arrow{>},
                mark=at position 0.25 with {\node[black,below]{$\gamma_0^{}$};}
            },postaction=decorate] (0,0) ellipse (2mm and 3mm);
            \draw [thick,orx,rotate around={-20:(1,-0.1)},decoration={
                markings,
                mark=at position 0.25 with \arrow{>},
                mark=at position 0.77 with {\node[black,above=-1pt]{$\gamma_1^{}$};}
            },postaction=decorate] (1,-0.1) ellipse (2mm and 3mm);
            \draw [thick,orx,decoration={
                markings,
                mark=at position 0.25 with \arrow{>},
                mark=at position 0.7 with {\node[black,below]{$\gamma_\pi^{}$};}
            },postaction=decorate] ({pi},0) ellipse (3mm and 4mm);
    
            \node [rex,label={[yshift=5pt]below:0}] at (0,0) {\normalsize$*$};
            \node [rex,label={[yshift=5pt]below:1}] at (1,0) {\normalsize$*$};
            \node [rex,label={[yshift=5pt]below:$\pi$}] at ({pi},0) {\normalsize$*$};
        \end{tikzpicture}
        \vspace{-2em}
        \caption{Using coefficients to make a multicurve nulhomologous.}
        \label{fig:nulhomCoeff}
    \end{figure}
    \begin{enumerate}
        \item Let $U=\C\setminus\{p_1,\dots,p_k\}$. Enclose all points $p_i$ in one big loop, and then enclose each individual point in a smaller loop (see Figures \ref{fig:nulhomExc} and \ref{fig:nulhomPuncture}).
        \begin{itemize}
            \item Then the (unweighted) sum of all $k+1$ curves is nulhomologous.
        \end{itemize}
        \item Let $U=\C\setminus\{0,1,\pi\}$. $\gamma_3$, as drawn in Figure \ref{fig:nulhomCoeff}, is not nulhomologous.
        \begin{itemize}
            \item This is because the winding number of $\gamma_3$ about $\pi$ is 2, about $1$ is $-2$, and about 0 is $-1$.
            \item However, we can make this curve nulhomologous by introducing a counterclockwise-oriented curve about each point and taking
            \begin{equation*}
                \Gamma = \gamma_3+\gamma_0+2\gamma_1-2\gamma_\pi
            \end{equation*}
        \end{itemize}
    \end{enumerate}
    \item The points in the above examples should be thought of as singularities.
    \item \textbf{Homologous} (cycles in $U$): Two cycles $\Gamma_1,\Gamma_2\subset U$ for which $\Gamma_1-\Gamma_2$ is nulhomologous.
    \begin{itemize}
        \item Helpful analogy: Think of $\Gamma_1,\Gamma_2$ as elements of a vector space and this is us saying they're equivalent/homologous if $\Gamma_1-\Gamma_2=0$ (i.e., $\Gamma_1=\Gamma_2$).
        \item Example: $\gamma_3$ and $2\gamma_\pi-2\gamma_1-\gamma_0$ are homologous in $\C\setminus\{0,1,\pi\}$.
    \end{itemize}
    \item Corollary: If $\Gamma_1$ and $\Gamma_2$ are homologous, then for all $f\in\mO(U)$,
    \begin{equation*}
        \int_{\Gamma_1}f\dd{z} = \int_{\Gamma_2}f\dd{z}
    \end{equation*}
    \begin{proof}
        We have that
        \begin{align*}
            \int_{\Gamma_1}f\dd{z}-\int_{\Gamma_2}f\dd{z} &= \int_{\Gamma_1-\Gamma_2}f\dd{z}\\
            &= 0\tag*{CIT}
        \end{align*}
        Thus, adding $\int_{\Gamma_2}f\dd{z}$ to both sides of the above equation, we obtain
        \begin{equation*}
            \int_{\Gamma_1}f\dd{z} = \int_{\Gamma_2}f\dd{z}
        \end{equation*}
        as desired.
    \end{proof}
    \item This is the most important thing we get from the new CIT/CIF.
    \begin{itemize}
        \item It means that to compute integrals over complicated paths, we can just replace the contour with a homologous ones that is easier to compute!
    \end{itemize}
    \item TPS: Integrate the following function over $\gamma_3$ from Figure \ref{fig:nulhomCoeff}.
    \begin{equation*}
        f(z) = \frac{1}{z(z-1)(z-\pi)}
    \end{equation*}
    \begin{itemize}
        \item By the corollary,
        \begin{equation*}
            \int_{\gamma_3}f\dd{z} = 2\int_{\gamma_\pi}f\dd{z}-2\int_{\gamma_1}f\dd{z}-\int_{\gamma_0}f\dd{z}
        \end{equation*}
        \item One way to evaluate each of these integrals is with a partial fraction decomposition.
        \item Another way is by observing the following.
        \begin{itemize}
            \item Take the loop around a pole, say $\pi$, to be very small.
            \item Then on this loop, $z$ is close to $\pi$, $z-1$ is close to $\pi-1$, and only $z-\pi$ is meaningfully changing.
            \item Thus, as the radius of the loop approaches zero, the $1/z$ and $1/(z-1)$ terms approach $1/\pi$ and $1/(\pi-1)$, respectively. These "constants" can then be factored out, and the remaining integral of $1/(z-\pi)$ evaluated to $2\pi i$.
            \item This suggests that
            \begin{equation*}
                \int_{\gamma_\pi}f\dd{z} = \frac{1}{\pi(\pi-1)}\cdot 2\pi i
            \end{equation*}
        \end{itemize}
        \item We now give a more rigorous justification for the above heuristic.
        \begin{itemize}
            \item Observe that $\int_{\gamma_\pi}f\dd{z}$ is equal to $2\pi i$ times the \textbf{residue} of $f$ at $\pi$, which we may recall is just the $a_{-1}$ coefficient in the Laurent expansion at $\pi$.
            \item Let's compute this Laurent expansion!
            \item We have that
            \begin{equation*}
                f(z) = \frac{1}{z-\pi}\cdot\frac{1}{z(z-1)}
            \end{equation*}
            where the rightmost term must be holomorphic at $\pi$ by our previous inquiry.
            \item Thus, let
            \begin{equation*}
                g(z) := \frac{1}{z(z-1)}
            \end{equation*}
            \item The Taylor expansion of $g$ about $\pi$ is
            \begin{equation*}
                g(z) = g(\pi)+g'(\pi)(z-\pi)+\frac{g''(z)}{2}(z-\pi)^2+\cdots
            \end{equation*}
            \item Now recall how we motivated the residue:
            \begin{equation*}
                \int\sum_{k=-n}^\infty a_k(z-\pi)^k\dd{z} = \int a_{-1}(z-\pi)^{-1}\dd{z} = 2\pi i\cdot a_{-1}
            \end{equation*}
            \item Thus, around $\pi$,
            \begin{equation*}
                f(z) = \frac{g(\pi)}{z-\pi}+g'(\pi)+\frac{g''(\pi)}{2}(z-\pi)+\cdots
            \end{equation*}
            \item This means that $g(\pi)$ is the $a_{-1}$ coefficient! Thus,
            \begin{equation*}
                \res_\pi(f) = a_{-1} = g(\pi) = \frac{1}{\pi(\pi-1)}
            \end{equation*}
        \end{itemize}
        \item Computing the other two integrals similarly, we obtain in total that
        \begin{align*}
            \int_{\gamma_\pi}f\dd{z} &= \frac{2i}{\pi-1}&
            \int_{\gamma_1}f\dd{z} &= \frac{2\pi i}{1-\pi}&
            \int_{\gamma_0}f\dd{z} &= 2i
        \end{align*}
        \item Thus, in total,
        \begin{equation*}
            \int_{\gamma_3}f\dd{z} = \frac{2i(\pi+3)}{\pi-1}
        \end{equation*}
        \item The point is not that we get a nice final answer. The point is that we can compute complicated integrals in a much simpler way, e.g., just by fiddling with power series.
    \end{itemize}
    \item \textbf{Residue} (of $f$ at $p$): If $p$ is an isolated singularity and $D$ is a small disk whose only singularity is $p$, then the residue is defined as follows. \emph{Denoted by} $\bm{\res_pf}$. \emph{Given by}
    \begin{equation*}
        \res_pf := \frac{1}{2\pi i}\int_{\partial D}f\dd{z}
    \end{equation*}
    \item \textbf{Residue theorem}: Suppose $U$ is a domain, $S$ (think singularities) is a discrete set in $U$, $f\in\mO(U\setminus S)$, and $\Gamma$ is nulhomologous in $U$. Then
    \begin{equation*}
        \frac{1}{2\pi i}\int_\Gamma f\dd{z} = \sum_{s\in S}\wn(\Gamma,s)\cdot\res_s(f)
    \end{equation*}
    \begin{proof}
        For all $s\in S$, let $\gamma_s$ be a small loop about $s$ oriented counterclockwise. Define
        \begin{equation*}
            \Gamma' := \Gamma-\sum_{s\in S}\wn(\Gamma,s)\gamma_s
        \end{equation*}
        $\Gamma'$ is nulhomologous. Thus, by the CIT,
        \begin{equation*}
            \int_{\Gamma'}f\dd{z} = 0
        \end{equation*}
        Now for all $s\in S$,
        \begin{equation*}
            \wn(\Gamma',s) = \wn(\Gamma,s)-\wn(\Gamma,s)\cdot\underbrace{\wn(\gamma_s,s)}_1 = 0
        \end{equation*}
        Therefore, we have that
        \begin{align*}
            \int_\Gamma f\dd{z} &= \int_{\Gamma'}f\dd{z}+\int_{\sum_{s\in S}\wn(\Gamma,s)\gamma_s}f\dd{z}\\
            &= 0+\sum_{s\in S}\wn(\Gamma,s)\int_{\gamma_s}f\dd{z}\\
            &= 2\pi i\sum\wn(\Gamma,s)\res_sf
        \end{align*}
        as desired.
    \end{proof}
    \item Comments on the residue theorem.
    \begin{itemize}
        \item Example: $U=\C$, $S=\{0,1,\pi\}$, and $\Gamma=\gamma_3$.
        \item The residue theorem will be very important, hint hint.
        \item The proof is just exactly what we did in the example!
        \item Calderon's definition of math: "You do an example, you see something interesting, and then you make a theorem that says, `This happens always.'"
    \end{itemize}
    \item We now list some properties of the residue.
    \begin{itemize}
        \item These properties are true in general, but we'll prove them in the specific case that the functions are meromorphic because this allows us to use Laurent expansions.
    \end{itemize}
    \item Properties: Let $a\in\C$ and $f,g\in\mO(U\setminus S)$.
    \begin{enumerate}
        \item $\res_s(af+g)=a\res_sf+\res_sg$.
        \begin{proof}
            Using the linearity of the integral in the definition, we have
            \begin{align*}
                \res_s(af+g) &= \frac{1}{2\pi i}\int_{\partial D}(af+g)\dd{z}\\
                &= a\cdot\frac{1}{2\pi i}\int_{\partial D}f\dd{z}+\frac{1}{2\pi i}\int_{\partial D}g\dd{z}\\
                &= a\res_sf+\res_sg
            \end{align*}
            as desired.
        \end{proof}
        \item If $f$ has a \textbf{simple pole} at $s$, then
        \begin{equation*}
            \res_sf = \lim_{z\to s}f(z)(z-s)
        \end{equation*}
        \begin{proof}
            Taking the Laurent expansion, let
            \begin{equation*}
                f(z) = \sum_{k=-1}^\infty a_k(z-s)^k
            \end{equation*}
            Then
            \begin{align*}
                \lim_{z\to s}f(z)(z-s) &= \lim_{z\to s}\sum_{k=-1}^\infty a_k(z-s)^{k+1}\\
                &= a_{-1}\cdot 0^0+\sum_{k=0}^\infty a_k\cdot 0^{k+1}\\
                &= a_{-1}\\
                &= \res_sf
            \end{align*}
            as desired.
        \end{proof}
        \item If $f$ has a \textbf{simple zero} at $s$ and $g$ is holomorphic at $s$, then
        \begin{equation*}
            \res_s(g\cdot f) = g(s)\res_sf
        \end{equation*}
        \begin{proof}
            Taking the Laurent expansions, let
            \begin{align*}
                f(z) &= \sum_{k=-1}^\infty a_k(z-s)^k&
                g(z) &= \sum_{k=0}^\infty\frac{g^{(k)}(s)}{k!}(z-s)^k
            \end{align*}
            Then taking the Cauchy product, we obtain
            \begin{equation*}
                (f\cdot g)(z) = a_{-1}b_0(z-s)^{-1}+\cdots
            \end{equation*}
            Therefore,
            \begin{equation*}
                \res_s(g\cdot f) = b_0a_{-1}
                = g(s)\res_sf
            \end{equation*}
            as desired.
        \end{proof}
    \end{enumerate}
    \item \textbf{Simple} (pole): A pole of order 1.
    \item \textbf{Simple} (zero): A zero of order 1.
    \item Properties 2 and 3 imply that if $f$ has a simple zero at $s$ and $g$ is holomorphic, then
    \begin{equation*}
        \res_s(g/f) = \frac{g(s)}{f'(s)}
    \end{equation*}
    % \begin{proof}
    %     Let
    %     \begin{equation*}
    %         f(z) = \sum_{k=0}^\infty\frac{f^{(k)}(s)}{k!}(z-s)^k
    %     \end{equation*}
    %     Then
    %     \begin{equation*}
    %         \frac{1}{f(z)} = \frac{1}{f(s)}-\frac{f'(s)}{f(s)^2}(z-s)+\cdots
    %     \end{equation*}
    %     If $f$ has a simple zero at $s$, then $1/f$ has a simple pole at $s$. Thus,
    %     \begin{equation*}
    %         \frac{1}{f(z)} = \sum_{k=-1}^\infty
    %     \end{equation*}
        
    %     Therefore,
    %     \begin{align*}
    %         \res_s(g/f) &= \res_s(g\cdot 1/f)\\
    %         &= g(s)\cdot\res_s(1/f)\tag*{Property 3}\\
    %         &= g(s)\cdot\lim_{z\to s}\frac{1}{f(z)}(z-s)
    %     \end{align*}
    % \end{proof}
    \item Using residues to compute the real integral
    \begin{equation*}
        \int_{-\infty}^\infty\frac{1}{(x^2+1)^n}\dd{x}
    \end{equation*}
    which gets complicated as $n\in\N$ gets big.
    \begin{figure}[h!]
        \centering
        \begin{tikzpicture}[
            every node/.style=black
        ]
            \node [label={[xshift=4pt]left:$i$}]  at (0,0.4)  {*};
            \node [label={[xshift=4pt]left:$-i$}] at (0,-0.4) {*};
    
            \small
            \draw
                (-2.5,0) -- (2.5,0) node[right]{$\R$}
                (0,-0.5) -- (0,2.5) node[above]{$i\R$}
            ;
    
            \footnotesize
            \draw [rex,thick,decoration={
                markings,
                mark=at position 0.34 with \arrow{>},
                mark=at position 0.5 with {\node[above right=-2pt]{$\gamma_R^{}$};}
            },postaction=decorate] (-2,0) node[below]{$-R$}
                -- (2,0) node[below]{$R$}
                arc[start angle=0,end angle=180,radius=2cm]
            ;
        \end{tikzpicture}
        \caption{Using residues to compute contour integrals.}
        \label{fig:residueContour}
    \end{figure}
    \begin{itemize}
        \item We will use contour integration over $\gamma_R$, as shown in Figure \ref{fig:residueContour}.
        \item First, analytically continue the integrand to
        \begin{equation*}
            f(z) := \frac{1}{(z^2+1)^n}
        \end{equation*}
        \begin{itemize}
            \item Observe that $f$ is meromorphic on $\C$ and, specifically, $f\in\mO(\C\setminus\{\pm i\})$.
        \end{itemize}
        \item Since integrals over homotopic paths are the same, the integral over the boundary of a small disk around $i$ would be the same as the integral over $\gamma_R$. Thus,
        \begin{equation*}
            \res_if = \int_{\gamma_R}f\dd{z}
        \end{equation*}
        \item Additionally, we can split
        \begin{equation*}
            \int_{\gamma_R}f\dd{z} = \int_{-R}^Rf(x)\dd{x}+\int_{\gamma_R^{}\setminus[-R,R]}f\dd{z}
        \end{equation*}
        \item Now as $R\to\infty$, the magnitude of the denominator of $f$ will similarly diverge along $\gamma_R\setminus[-R,R]$. Consequently, the rightmost integral above goes to zero as $R\to\infty$, and we are left with
        \begin{equation*}
            \int_{-\infty}^\infty\frac{1}{(x^2+1)^n}\dd{x} = \lim_{R\to\infty}\int_{-R}^Rf(x)\dd{x}
            = \lim_{R\to\infty}\res_if
            = \res_if
        \end{equation*}
        \item Thus, all we need to solve this problem is to compute the Laurent expansion of $f$ about $i$.
        \item After more manipulations (see the notes), the final answer is
        \begin{equation*}
            \int_{-\infty}^\infty\frac{1}{(x^2+1)^n}\dd{x} = \frac{\pi(2n-2)!}{2^{2n-2}[(n-1)!]^2}
        \end{equation*}
    \end{itemize}
    \item Application of complex analysis to number theory: The Basel problem.
    \begin{itemize}
        \item This was an open problem for over 100 years.
        \item It asked for the value of the infinite sum
        \begin{equation*}
            1+\frac{1}{4}+\frac{1}{9}+\frac{1}{16}+\cdots = \sum_{n=1}^\infty\frac{1}{n^2} =: \zeta(2)
        \end{equation*}
        \item Euler --- who was from Basel --- proved (with some not-very-rigorous power series manipulations that were only rigorously justified later) that
        \begin{equation*}
            \zeta(2) = \frac{\pi^2}{6}
        \end{equation*}
        \begin{itemize}
            \item Calderon: "That's the good thing about being a pioneer: You don't have to do all the proofs."
        \end{itemize}
        \item But what about $\zeta(3)$, $\zeta(4)$, $\zeta(5)$, etc.?
        \item We know how to compute $\zeta(2n)$.
        \item Apery's theorem (1978): $\zeta(3)$ is proven to be irrational.
        \begin{itemize}
            \item The argument is supposedly quite complex, and Calderon has never studied it.
            \item Is $\zeta(3)$ transcendental like $\zeta(2n)$? Still an open question!
        \end{itemize}
        \item For $\zeta(5)$ and any other odd numbers, we're still out of luck.
    \end{itemize}
    \item A computation of $\zeta(2)$ using residues.
    \begin{figure}[h!]
        \centering
        \begin{tikzpicture}[
            scale=0.7,
            every node/.style=black
        ]
            \footnotesize
            \path (-6,0) -- (6,0);
            
            \foreach \x in {-3,...,3} {
                \node [rex,label={[yshift=4pt]below:$\x$}] at (\x,0) {\normalsize *};
            }
    
            \draw [yex,thick,decoration={
                markings,
                mark=at position 0.13 with \arrow{>},
                mark=at position 0.43 with \arrow{>},
                mark=at position 0.63 with \arrow{>},
                mark=at position 0.83 with \arrow{>}
            },postaction=decorate] (-2.5,-2.5)
                -- node[above]{$\im=-N-\frac{1}{2}$} (2.5,-2.5)
                -- node[pos=0.69,right=2pt]{$\re=N+\frac{1}{2}$} (2.5,2.5)
                -- node[below]{$\im=N+\frac{1}{2}$} (-2.5,2.5)
                -- node[pos=0.31,left=2pt]{$\re=-N-\frac{1}{2}$} cycle
            ;
        \end{tikzpicture}
        \caption{Basel problem solution using residues.}
        \label{fig:baselResidue}
    \end{figure}
    \begin{itemize}
        \item Let's investigate the helper function
        \begin{equation*}
            f(z) = \frac{\pi}{z^2\tan(\pi z)}
        \end{equation*}
        \item Observe that $f$ is meromorphic on $\C$. In particular\dots
        \begin{itemize}
            \item $f$ has a pole of order 3 at $z=0$;
            \item $f$ has a pole of order 1 at ever nonzero $n\in\Z$ (because tangent is periodic).
        \end{itemize}
        \item Thus, $f\in\mO(\C\setminus\Z)$.
        \item Let's compute the residue of $f$ about the nonzero poles.
        \begin{itemize}
            \item In these cases, the denominator has a simple zero and the numerator is holomorphic, so we can apply "Property 4" to learn that
            \begin{equation*}
                \res_nf = \frac{\eval{\pi}_n}{\eval{\dv*{z}[z^2\tan(\pi z)]}_n}
                = \frac{\pi}{2n\underbrace{\tan(\pi n)}_0+\pi n^2\underbrace{\sec^2(\pi n)}_1}
                = \frac{\pi}{\pi n^2}
                = \frac{1}{n^2}
            \end{equation*}
        \end{itemize}
        \item Define $\gamma_N$ to be the curve in Figure \ref{fig:baselResidue}. Then since $\gamma_N$ has a winding number of 1 around all of the poles it encloses, the residue theorem tells us that
        \begin{equation*}
            \frac{1}{2\pi i}\int_{\gamma_N}f\dd{z} = \sum_{n=-N}^N\res_nf
        \end{equation*}
        \begin{itemize}
            \item It follows from the above that
            \begin{equation*}
                \frac{1}{2\pi i}\int_{\gamma_N}f\dd{z} = \sum_{n=-N}^{-1}\res_nf+\res_0f+\sum_{n=1}^N\res_nf
                = \res_0f+2\sum_{n=1}^N\frac{1}{n^2}
            \end{equation*}
        \end{itemize}
        \item We now compute the above integral.
        \begin{itemize}
            \item We will do this by bounding it and showing that it converges to zero.
            \item To begin, let's bound the reciprocal of $\tan(\pi z)$, which is
            \begin{equation*}
                \cot(\pi z) = i\cdot\frac{\e[\pi iz]+\e[-\pi iz]}{\e[\pi iz]-\e[-\pi iz]}
            \end{equation*}
            \begin{itemize}
                \item When $\re(z)=N+1/2$ (for some $N\in\Z$) and $y=\im(z)\in\R$, we have that
                \begin{equation*}
                    \e[\pi iz] = \e[\pi i(N+1/2+yi)]
                    = \e[N\pi i]\cdot\e[\pi i/2]\cdot\e[-\pi y]
                    = \pm 1\cdot i\cdot\e[-\pi y]
                    = \pm i\e[-\pi y]
                \end{equation*}
                and
                \begin{equation*}
                    \e[-\pi iz] = \e[-\pi i(N+1/2+yi)]
                    = \e[-N\pi i]\cdot\e[-\pi i/2]\cdot\e[\pi y]
                    = \pm 1\cdot -i\cdot\e[\pi y]
                    = \mp i\e[\pi y]
                \end{equation*}
                so hence,
                \begin{equation*}
                    |\cot(\pi z)| = \left| \frac{\e[\pi iz]+\e[-\pi iz]}{\e[\pi iz]-\e[-\pi iz]} \right|
                    = \left| \frac{\pm i\e[-\pi y]+\mp i\e[\pi y]}{\pm i\e[-\pi y]-\mp i\e[\pi y]} \right|
                    = \left| \frac{\pm\e[-\pi y]-\pm\e[\pi y]}{\pm\e[-\pi y]+\pm\e[\pi y]} \right|
                    = \left| \frac{\e[-\pi y]-\e[\pi y]}{\e[-\pi y]+\e[\pi y]} \right|
                    \leq 1
                \end{equation*}
                % \item Similarly, when $x=\re(z)\in\R$ and $\im(z)=n+1/2$ (for some $n\in\Z$), we have that
                % \begin{align*}
                %     \e[\pi iz] &= \frac{\e[\pi ix]}{\e[\pi(n+1/2)]}&
                %     \e[-\pi iz] &= \frac{\e[\pi(n+1/2)]}{\e[\pi ix]}
                % \end{align*}
                % so hence,
                % \begin{equation*}
                %     |\cot(\pi z)| = \left| \frac{\e[\pi iz]+\e[-\pi iz]}{\e[\pi iz]-\e[-\pi iz]} \right|
                %     % = \left| \frac{\frac{\e[\pi ix]}{\e[\pi(n+1/2)]}+\frac{\e[\pi(n+1/2)]}{\e[\pi ix]}}{\frac{\e[\pi ix]}{\e[\pi(n+1/2)]}-\frac{\e[\pi(n+1/2)]}{\e[\pi ix]}} \right|
                %     % = \left| \frac{\e[2\pi ix]+\e[2\pi(n+1/2)]}{\e[2\pi ix]-\e[2\pi(n+1/2)]} \right|
                %     = \left| \frac{\e[2\pi(n+1/2)]+\e[2\pi ix]}{\e[2\pi(n+1/2)]-\e[2\pi ix]} \right|
                %     = \frac{|\e[2\pi(n+1/2)]+\e[2\pi ix]|}{|\e[2\pi(n+1/2)]-\e[2\pi ix]|}
                %     \leq \frac{\e[2\pi(n+1/2)]+1}{|\e[2\pi(n+1/2)]-\e[2\pi ix]|}
                % \end{equation*}
                \item On the other hand, notice that as $\im(z)\to\infty$, $\cot(z)\to 1$. Thus, if we want to keep $\cot(z)$ near 1 (and hence bounded in general), we need only require that $\im(z)=N+1/2$ is greater than some threshold. In fact, as we can see in the applet from the 3/21 lecture, if $\im(z)\geq 1/2$, then $|\cot(z)-1|$ is already less than $1/2$ and hence $|\cot(z)|\leq 2$.\footnote{Note that for the sake of bounding the $\cot(\pi z)$, we need not make the top and bottom of $\gamma_N$ diverge along with the right and left sides; they could stay at $\im(z)=\pm 1/2$ and we'd be totally fine on boundedness. However, we do have the top and bottom diverge so that the $z^2$ term in the denominator of $f(z)$ becomes large at \emph{all} points along $\gamma_N$ as $N\to\infty$; this fact will be used shortly when we compute $\int_{\gamma_N}f\dd{z}$.} An analogous argument holds based on the fact that as $\im(z)\to -\infty$, $\cot(z)\to -1$.
                \item Thus, $|\cot(z)|\leq 2$ for all $z\in\im(\gamma_N)$ and $N\in\N$.
            \end{itemize}
            \item Consequently, as $N\to\infty$, $f(z)\to 0$ for all $z\in\im(\gamma_N)$. Thus, the integral of $f$ over $\gamma_N$ goes to zero, too. In a statement,
            \begin{equation*}
                \lim_{N\to\infty}\int_{\gamma_N}f\dd{z} = 0
            \end{equation*}
        \end{itemize}
        \item Combining the above two results, we have that
        \begin{align*}
            \frac{1}{2\pi i}\lim_{N\to\infty}\int_{\gamma_N}f\dd{z} &= \lim_{N\to\infty}\left( \res_0f+2\sum_{n=1}^N\frac{1}{n^2} \right)\\
            \frac{1}{2\pi i}\cdot 0 &= \res_0f+2\sum_{n=1}^\infty\frac{1}{n^2}\\
            \sum_{n=1}^\infty\frac{1}{n^2} &= -\frac{1}{2}\res_0f
        \end{align*}
        \item Evidently, we must now compute $\res_0f$.
        \begin{itemize}
            \item The Laurent series for $\cot(z)$ about 0 is
            \begin{equation*}
                \cot(z) = \frac{1}{z}-\frac{z}{3}-\frac{z^3}{45}-\frac{2z^5}{945}-\cdots
            \end{equation*}
            \item Thus, near zero,\footnote{Notice that this Laurent series reflects the fact that $f$ has a zero of order 3 at $z=0$!}
            \begin{align*}
                f(z) &= \frac{\pi}{z^2}\cdot\left( \frac{1}{\pi z}-\frac{\pi z}{3}-\frac{\pi^3z^3}{45}-\frac{2\pi^5z^5}{945}-\cdots \right)\\
                &= \frac{1}{z^3}-\frac{\pi^2}{3z}-\frac{\pi^4}{45}z-\frac{2\pi^6}{945}z^4-\cdots
            \end{align*}
            \item Consequently,
            \begin{equation*}
                \res_0f = a_{-1} = -\frac{\pi^2}{3}
            \end{equation*}
        \end{itemize}
        \item Therefore,
        \begin{equation*}
            \sum_{n=1}^\infty\frac{1}{n^2} = -\frac{1}{2}\cdot -\frac{\pi^2}{3}
            = \frac{\pi^2}{6}
        \end{equation*}
        as desired.
    \end{itemize}
    \item This same argument easily extends to the even natural number values of the Riemann zeta function.
\end{itemize}




\end{document}