\documentclass[../notes.tex]{subfiles}

\pagestyle{main}
\renewcommand{\chaptermark}[1]{\markboth{\chaptername\ \thechapter\ (#1)}{}}
\setcounter{chapter}{8}

\begin{document}




\chapter{Advanced Topics and Applications}
\section{The Gelfond-Schneider Theorem}
\begin{itemize}
    \item \marginnote{5/14:}Today's lecture.
    \begin{itemize}
        \item By Ben, a postdoc.
        \item His choice of topic in complex analysis.
        \item Proof that $\e[\sqrt{2}]$ is irrational, pulled from the Math Library's one complex textbook.
        \item He chose this topic to illustrate how useful complex analysis is in other areas of math.
    \end{itemize}
    \item The main theorem we'll use here is the maximum modulus principle, in a slightly modified form.
    \item Maximum modulus principle (alternate statement): If $\Omega$ is a compact domain, $f\in\mO(\Omega)$, then
    \begin{equation*}
        |f(z)| \leq \max_{w\in\partial\Omega}|f(w)|
    \end{equation*}
    Moreover, if equality holds in any case, then $f$ is constant.
    \begin{proof}
        For $\Omega=B_p(r)$, this follows from the \textbf{mean-value property}.\footnote{Ben quickly explains how the mean-value property works.}
    \end{proof}
    \item Remark: An entire function with lots of zeroes must grow fast.
    \begin{proof}
        Let $f$ be the entire function, and suppose it has zeroes at $\{z_i\}$ with multiplicity $k_i$. Form the new function
        \begin{equation*}
            \frac{f(z)}{\prod_i(z-z_i)^{k_i}}
        \end{equation*}
        If we make $|z|$ large, then this function behaves like
        \begin{equation*}
            \frac{f(z)}{\prod_iz^{k_i}}
        \end{equation*}
        Since the above function is holomorphic, the MMP says it must obtain its maximum value on the boundary of an arbitrarily large ball around the compact set on which $f$ obtains all its zeroes. But that denominator is growing really fast, so $f$ must grow even faster to compensate.
    \end{proof}
    \item \textbf{Strictly ordered} ($f$ by $\rho$): An entire function $f$ for which there exists $C>1$ such that
    \begin{equation*}
        |f(z)| \leq C^{R^\rho}
    \end{equation*}
    where $R=|z|$.
    \begin{itemize}
        \item Alternatively, we say that "$f$ has \emph{strict order} $\leq\rho$."
        \item This gives a bound on the growth of the function.
        \item We will use $R$ to denote $|z|$ throughout lecture today.
    \end{itemize}
    \item \textbf{Algebraically independent} (functions): Two functions $f,g$ for which
    \begin{equation*}
        \sum_{i,j=1}^Na_{ij}f^ig^j = 0
    \end{equation*}
    where $a_{ij}\in\C$ implies that $a_{ij}=0$ for all $i,j$.
    \begin{itemize}
        \item We will apply this to $f(z)=z$ and $g(z)=\e[z]$.
    \end{itemize}
    \item Theorem (Gelfond-Schneider): Let $f_1,\dots,f_n$ be entire functions with strict order less than or equal to $\rho$ a positive number. Assume that at least two of these functions are algebraically independent. Assume $D:=\dv*{z}$ maps $\Q[f_1,\dots,f_n]$ into itself. Suppose $w_1,\dots,w_N$ are distinct complex numbers such that $f_i(w_j)\in\Q$ for all $1\leq i\leq n$ and $1\leq j\leq N$. Then $N\leq 4\rho$.
    \item Corollary: $\e[w]$ cannot be rational if $w\in\Q$.
    \begin{proof}
        Apply the Gelfond-Schneider theorem to $\Q[z,\e[z]]$. From here, note that if $\e[w]$ were rational, then $\e[w],\e[2w],\e[3w],\dots\in\Q$ which would eventually contradict the $N\leq 4\rho$ bound.
    \end{proof}
    \item If we prove the Gelfond-Schneider theorem under the hypothesis that $f_i(w_j)\in\overline{\Q}$, then our corollary may state that $\e[w]$ cannot be \textbf{algebraic}.
    \item \textbf{Algebraic number}: A number that is the zero of a one-variable polynomial.
    \item Lemma 1 (Siegel): Let
    \begin{align*}
        a_{11}x_1+\cdots+a_{1n}x_n &= 0\\
        &\ \ \vdots\\
        a_{r1}x_1+\cdots+a_{rn}x_n &= 0
    \end{align*}
    be such that (i) $a_{ij}\in\Z$, (ii) $n>r$, and (iii) $|a_{ij}|\leq A$. Then there exists an integral, nonzero solution $(x_1,\dots,x_n)$ to this system of equations with
    \begin{equation*}
        |x_j| \leq 2(2nA)^{\frac{r}{n-r}}
    \end{equation*}
    \begin{proof}
        We know that there has to be at least \emph{some} solution by condition (ii) and linear algebra, which confirms sufficient information and a nontrivial kernel.\par
        Let $T$ be the $r\times n$ matrix $(a_{ij})$. Then $T$ maps $\Z^n(B)$ into $\Z^r(nBA)$, where $\Z^m(s):=B_0(s)\cap\Z^m$.\footnote{Pronounced "the $m^\text{th}$-dimensional integer ball of radius $s$."} Find $x,y\in\Z^n(B)$ such that $T(x)=T(y)$ and hence $T(x-y)=0$. Via a pigeonhole principle argument, make $B$ big enough so that $\Z^r(nBA)$ (which is growing slower due to its smaller exponent of $r$) has cardinality smaller than $\Z^n(B)$; this will mean that two things have to map to the same thing. Then if we do the computation, we get the stated bound.\par
        Essentially, we're relying on the principle that integer balls in higher-dimensional Euclidean spaces have more points in the limit of large radius.
    \end{proof}
    \item \textbf{Size} (of a polynomial): The following number, where $P(x_1,\dots,x_n)=\sum_{I=(i_1,\dots,i_n)}a_Ix_1^{i_1}\cdots x_n^{i_n}$ is a polynomial. \emph{Denoted by} $\bm{\size(P)}$. \emph{Given by}
    \begin{equation*}
        \size(P) := \max_I|a_I|
    \end{equation*}
    \item \textbf{Denominator} (of $\{a_i\}\subset\Q$): A number $d$ such that $d\cdot a_i\in\Z$ for ever $a_i$ in the subset $\{a_i\}\subset\Q$. \emph{Denoted by} $\bm{\den(\{a_i\})}$.
    \item Lemma 2: Let $f_1,\dots,f_n$ be functions as in the Gelfond-Schneider theorem. Then there exists a constant $C_1$ such that if $\Q(T_1,\dots,T_n)$ is a polynomial with rational coefficients and degree less than or equal to $r$, then
    \begin{equation*}
        P^m(Q(f_1,\dots,f_n)) = Q_m(f_1,\dots,f_n)
    \end{equation*}
    where\dots
    \begin{enumerate}[label={\roman*.)}]
        \item $\deg(Q_m)\leq C_1(m+r)$;
        \item $\size(Q_m)\leq\size(Q)m!C_1^{m+r}$;
        \item There exists a denominator for the coefficients of $Q_m$ bounded by $\den(Q)C_1^{m+r}$.
    \end{enumerate}
    \item We are now ready to prove the Gelfond-Schneider theorem.
    \begin{proof}
        By hypothesis, we have common elements $w_1,\dots,w_N$ of $\C$ such that $f_i(w_j)\in\Q$ and $f_{ij}\in\{f_1,\dots,f_n\}$ algebraically independent. Let $L\in\Z^+$ be divisible by $2N$, $b_{ij}\in\Z$, and let $F=\sum_{i,j=1}^Lb_{ij}f^ig^j$ and let $L=2MN$ be such that
        \begin{equation*}
            D^mF(w_\ell) = 0\tag{$*$}
        \end{equation*}
        for $m=0,\dots,M-1$ and $\ell=1,\dots,N$; we will send both of these constants to infinity eventually.\par
        $(*)$ has $L^2$ unknowns and $MN$ equations. Multiply the equations in $(*)$ by a common denominator and using Lemma 2 and Siegel's Lemma, we can find $b_{ij}$ such that
        \begin{equation*}
            |b_{ij}| \leq M!C_2^{M+L} \leq M^MC_2^{M+L}\tag{$**$}
        \end{equation*}
        as $M\to\infty$. Note that in the second inequality, we used Stirling's approximation.\par
        The next observation is that $F\neq 0$ since $f$ and $g$ are algebraically independent. Let $s$ be the smallest integer such that $D^mf(w_i)=0$ for $m<s$ for all $i$ but $D^sF\neq 0$ at some $w_i$, which WLOG we will let be $w_1$.\par
        Let $\alpha:=D^sF(w_1)$. Then $\alpha\in\Q$ since $F(W_1)\in\Q$ so all its derivatives will, too. Additionally, $C:=\den(\alpha)\leq(C_1)^s$ as $s\to\infty$, this from (i) and (iii) of Lemma 2. Then $C\alpha\in\Z$, which implies that $|C\alpha|\geq 1$ and hence $|\alpha|\geq C^{-1}$. Thus, at this point, we have a lower bound on $|\alpha|$; the next step is to move toward an upper bound and then get what we want.\par
        We upper-bound $\alpha$ using the MMP. Compute
        \begin{equation*}
            D^sF(w_1) = \eval{s!\frac{F(w_1)}{(z-w_1)^s}}_{z=w_1}
        \end{equation*}
        Estimate
        \begin{equation*}
            H(z) := s!\frac{F(z)}{\prod_{i=1}^N(z-w_i)^s}\prod_{i>1}^N(w_1-w_i)^s
        \end{equation*}
        on the circle of radius $B=s^{1/2\rho}$. Then the MMP tells us that
        \begin{equation*}
            |D^sF(w_1)| = |H(w_1)|
            \leq \norm{H}_R
            \leq \frac{s^sC^{Ns}\norm{F}_R}{R^{Ns}}
        \end{equation*}
        Then after working this out, we get
        \begin{equation*}
            1 \leq |c\alpha|
            \leq \frac{s^{2s}C^{Ns}}{\e[Ns\log(s)/2\rho]}
        \end{equation*}
        which gets to $N\leq 4\rho$.
    \end{proof}
\end{itemize}



\section{Moduli Spaces of Elliptic Curves}
\begin{itemize}
    \item \marginnote{5/16:}Announcements.
    \begin{itemize}
        \item PSet 5 due tomorrow.
        \item Final Tuesday.
        \item Project due end of day Tuesday.
        \item Final presentations in my office (E313) unless you hear otherwise.
        \begin{itemize}
            \item We can show up in his office to watch other people's questions.
        \end{itemize}
        \item Stop rescheduling!
    \end{itemize}
    \item No notes will be posted for today; it's like three weeks worth of content.
    \begin{itemize}
        \item Nothing from Week 9 will be on the final exam!
    \end{itemize}
    \item Today: Moduli spaces of elliptic curves.
    \begin{itemize}
        \item A topic near and dear to Calderon's heart that uses complex analysis heavily.
        \item Calderon is first and foremost a topologist/geometer.
    \end{itemize}
    \item Theorem (Topological classification of surfaces): Consider a 2-manifold space locally homeomorphic to $\R^2$ that is compact without boundary (e.g., closed). All of the closed, orientable 2-manifolds are classified by their number of holes, i.e., homeomorphic go a genus $n$ surface.
    \item Let's equip our surface with a complex structure. Essentially, instead of charting pieces to $\R^2$, we'll chart them to $\C$!
    \begin{itemize}
        \item An elliptic curve $E$ is just a complex torus.
        \item What are holomorphic functions on Riemann surfaces?
        \begin{itemize}
            \item Recall that $f\in\mO(U)$ iff $f\circ\phi_U^{-1}$ is holomorphic on $\phi_U(U)$.
            \item They are constant on $\hat{\C}$! This is just Liouville's theorem again.
            \item They are also constant on $E$.
        \end{itemize}
        \item If $U\subset E$ is a nice open set, it maps to a domain.
    \end{itemize}
    \item There are many Riemann surface structures.
    \begin{itemize}
        \item For example, transition maps are translations.
        \item A projective plane curve is another.
    \end{itemize}
    \item \textbf{Complex projective space} (of dimension $n$): \emph{Denoted by} $\bm{\pmb{\C}\pmb{\Pp}^n}$. \emph{Given by}
    \begin{equation*}
        \C\Pp^n := \left( \C^{n+1}\setminus\{(0,\dots,0)\} \right)/\text{scaling}
    \end{equation*}
    \begin{itemize}
        \item This is the set of lines through the origin, e.g., $(1,1,1)=(2,2,2)$.
    \end{itemize}
    \item Example: $\R\Pp^1=\R^2\setminus 0/\text{scaling}$.
    \begin{itemize}
        \item We get rays and lines.
    \end{itemize}
    \item TPS: $\C\Pp^1\cong\hat{\C}$, where $\cong$ denotes a biholomorphic equivalence.
    \begin{itemize}
        \item Let $\{(x,y)\}/\text{scaling}\mapsto x/y$.
        \item This does preserve scaling: $(2x,2y)\mapsto x/y$, as well!
        \item It also is onto: $(x,1)\mapsto x\in\hat{\C}$, and any point $(x,0)\mapsto\infty$.
        \item Then to prove biholomorphic-ness, on charts to $\C$:
        \begin{itemize}
            \item If $y\neq 0$, send $x,y\mapsto x/y$.
            \item If $x\neq 0$, send $x,y\mapsto y/x$.
            \item The transition map between these two is $1/z$.
        \end{itemize}
    \end{itemize}
    \item We can homogenize...
    \item Uniformization: Every complex torus looks like $\C/\Lambda$, where
    \begin{equation*}
        \Lambda = \{n_1w_1+n_2w_2\mid n_i\in\Z,\ w_i\text{ being }\R\text{-linearly independent}\}
    \end{equation*}
    \begin{itemize}
        \item Every complex torus also has a representation as $\{y^2z=x^3+axz^2+bz^3\}$.
        \begin{itemize}
            \item Apply the Weierstrass $\wp$-function.
        \end{itemize}
    \end{itemize}
    \item Question: How many ways are there to do this and get different tori?
    \begin{itemize}
        \item If $c\in\C$, then $c\omega_2$ and $c\omega_1$ gives the same torus up to biholomorphism.
        \begin{itemize}
            \item You can scale the torus on the plane.
        \end{itemize}
    \end{itemize}
    \item Up to scaling, assume $\omega_1=1$.
    \begin{itemize}
        \item We'll now start calling $\omega_2$ by $\tau$.
    \end{itemize}
    \item Discussion of the Gaussian integers.
    \item The set of complex tori is equal to $\Hh$ with $SL_2\Z$ modded out. In particular, $SL_2\Z\acts\Z^2$ by changing basis.
    \item In fact, we're interested in $PSL_2\Z$, which lives inside $PSL_2\R$.
    \item Something with matrices and getting a tesselation of the upper half plane with circular arcs.
    \item In conclusion, the space of complex tori is called a modular curve.
\end{itemize}




\end{document}