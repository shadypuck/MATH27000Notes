\documentclass[../notes.tex]{subfiles}

\pagestyle{main}
\renewcommand{\chaptermark}[1]{\markboth{\chaptername\ \thechapter\ (#1)}{}}
\setcounter{chapter}{2}

\begin{document}




\chapter{Fundamental Theorems}
\section{Cauchy Integral Formula}
\begin{itemize}
    \item \marginnote{4/2:}Last time.
    \begin{itemize}
        \item Definition of star-shaped.
        \item Cauchy integral theorem: $U$ star-shaped, $f\in\mO(U)$ implies $\int_\gamma f\dd{z}=0$ for all closed (piecewise $C^1$) loops $\gamma$.
        \begin{enumerate}
            \item It suffices to prove the theorem for triangles.
            \item Apply Goursat's lemma to treat this triangle case.
        \end{enumerate}
        \item For Goursat's lemma, apply a clever estimate. Subdivide the big triangle into smaller ones, then
        \begin{align*}
            \left| \int_{\text{small }\triangle} f\dd{z} \right| = \left| \int_a^bf(\gamma(t))\cdot\gamma'(t)\dd{t} \right|
            \leq \int_a^b|f(\gamma(t))|\cdot|\gamma'(t)|\dd{t}
            % &\leq \max_{z\in\partial\triangle}|f|\int|\gamma'(t)|\dd{t}\\
            \leq \max_{z\in\partial\triangle}|f(z)|\cdot\len(\partial\triangle)
        \end{align*}
    \end{itemize}
    \item We'll now do a couple exercises to practice applying the concepts we've learned so far.
    \item TPS: Suppose $f\in\mO(\C)$. Let $A:=\int_0^1f(x)\dd{x}=F(1)-F(0)$, where to be clear we take the integral along the real axis. Let $\gamma$ be the piecewise $C^1$ path in yellow in Figure \ref{fig:CITexr1}. What is $\int_\gamma f\dd{z}$?
    \begin{figure}[h!]
        \centering
        \begin{tikzpicture}[
            every node/.style=black
        ]
            \footnotesize
            \draw
                (-0.5,0) -- (1.5,0)
                (0,-0.5) -- (0,1.5)
            ;
            \draw (0.1,1) -- ++(-0.2,0) node[left]{$i$};
    
            \draw [grx,thick,decoration={
                markings,
                mark=at position 0.55 with \arrow{>}
            },postaction=decorate] (0,0) -- node[above=1pt]{$\delta$} (1,0);
            \draw [yex,thick,decoration={
                markings,
                mark=at position 0.19 with \arrow{>},
                mark=at position 0.52 with \arrow{>},
                mark=at position 0.84 with \arrow{>}
            },postaction=decorate] (0,0) -- (0,1) -- node[above=1pt]{$\gamma$} (1,1) -- (1,0);
    
            \fill [rex]       circle (2pt) node[below left]{$0$};
            \fill [rex] (1,0) circle (2pt) node[below]     {$1$};
        \end{tikzpicture}
        \caption{Practicing with the Cauchy Integral Theorem (1).}
        \label{fig:CITexr1}
    \end{figure}
    \begin{itemize}
        \item Define $\delta$ such that $\int_\delta f\dd{z}=\int_0^1f(x)\dd{x}$.
        \item Then $\delta^{-1}\gamma$ is a closed loop, so
        \begin{equation*}
            0 = \int_{\delta^{-1}\gamma}f\dd{z}
        \end{equation*}
        \item Additionally, we have by definition that
        \begin{equation*}
            \int_{\delta^{-1}\gamma}f\dd{z} = \int_\gamma f\dd{z}-\int_\delta f\dd{z}
        \end{equation*}
        \item Thus, by transitivity and a bit of algebraic rearrangement,
        \begin{equation*}
            \int_\gamma f\dd{z} = \int_\delta f\dd{z} = A
        \end{equation*}
    \end{itemize}
    \pagebreak
    \item TPS: Now suppose $f\in\mO(\C^*)$, where we must note that $\C^*$ is \emph{not} star-shaped due to the hole at the origin. Suppose we know that $\int_\delta f\dd{z}=0$. What is $\int_\gamma f\dd{z}$? The paths $\gamma$ and $\delta$ are visualized in Figure \ref{fig:CITexr2a}. \emph{Hint}: It should be $-\int_\delta f\dd{z}$.
    \begin{figure}[h!]
        \centering
        \begin{subfigure}[b]{0.3\linewidth}
            \centering
            \begin{tikzpicture}[
                every node/.style=black
            ]
                \footnotesize
                \draw
                    (-1.5,0) -- (1.5,0)
                    (0,-1.5) -- (0,1.5)
                ;
                \filldraw [fill=white] circle (1.5pt);
        
                \draw [grx,thick,decoration={
                    markings,
                    mark=at position 0.35 with \arrow{>},
                    mark=at position 0.67 with \arrow{>},
                    mark=at position 0.3 with {\node{$\delta$};}
                },postaction=decorate] (-60:1) arc[start angle=-60,end angle=240,radius=1cm];
                \draw [yex,thick,decoration={
                    markings,
                    mark=at position 0.25 with \arrow{>},
                    mark=at position 0.79 with \arrow{>}
                },postaction=decorate] (240:1) -- (0,0.5) -- (-60:1);
                \node at (-0.35,0.25) {$\gamma_1^{}$};
                \node at (0.35,0.25) {$\gamma_2^{}$};
            \end{tikzpicture}
            \caption{Original setup.}
            \label{fig:CITexr2a}
        \end{subfigure}
        \begin{subfigure}[b]{0.3\linewidth}
            \centering
            \begin{tikzpicture}[
                every node/.style={black,text height=1.5ex,text depth=0.25ex}
            ]
                \footnotesize
                \path (0,1.5) -- (0,-1.5);
    
                \begin{scope}[xshift=-3mm]
                    \draw [grx,thick,decoration={
                        markings,
                        mark=at position 0.52 with \arrow{>},
                        mark=at position 0.1 with {\node[above left]{$\delta_2$};}
                    },postaction=decorate] (90:1) arc[start angle=90,end angle=240,radius=1cm];
                    \draw [yex,thick,decoration={
                        markings,
                        mark=at position 0.52 with \arrow{>},
                        mark=at position 0.3 with {\node[below right]{$\gamma_1^{}$};}
                    },postaction=decorate] (240:1) -- (0,0.5);
                    \draw [blx,thick,decoration={
                        markings,
                        mark=at position 0.6 with \arrow{>},
                        mark=at position 0.5 with {\node[below left,yshift=2pt]{$\alpha$};}
                    },postaction=decorate] (0,0.5) -- (0,1);
                    \node at (-0.5,0.25) {$L$};
    
                    \draw [help lines] plot[smooth cycle] coordinates {(0.2,0) (90:1.5) (135:1.5) (180:1.5) (223:1.5) (266:1.5)};
                    \node [below left] at (215:1.5) {$U$};
                \end{scope}
    
                \begin{scope}[xshift=3mm]
                    \draw [grx,thick,decoration={
                        markings,
                        mark=at position 0.52 with \arrow{>},
                        mark=at position 0.9 with {\node[above right]{$\delta_1$};}
                    },postaction=decorate] (-60:1) arc[start angle=-60,end angle=90,radius=1cm];
                    \draw [yex,thick,decoration={
                        markings,
                        mark=at position 0.52 with \arrow{>},
                        mark=at position 0.7 with {\node[below left]{$\gamma_2^{}$};}
                    },postaction=decorate] (0,0.5) -- (-60:1);
                    \draw [blx,thick,decoration={
                        markings,
                        mark=at position 0.6 with \arrow{>},
                        mark=at position 0.5 with {\node[below right,yshift=2pt]{$\alpha^{-1}$};}
                    },postaction=decorate] (0,1) -- (0,0.5);
                    \node at (0.5,0.25) {$R$};
    
                    \draw [help lines] plot[smooth cycle] coordinates {(-0.2,0) (90:1.5) (45:1.5) (0:1.5) (-43:1.5) (-86:1.5)};
                    \node [below right] at (-35:1.5) {$V$};
                \end{scope}
            \end{tikzpicture}
            \caption{Solution: Break in two.}
            \label{fig:CITexr2b}
        \end{subfigure}
        \begin{subfigure}[b]{0.3\linewidth}
            \centering
            \begin{tikzpicture}[
                every node/.style=black
            ]
                \footnotesize
                \path (0,1.5) -- (0,-1.5);
    
                \draw [grx,thick,decoration={
                    markings,
                    mark=at position 0.35 with \arrow{>},
                    mark=at position 0.67 with \arrow{>},
                    mark=at position 0.3 with {\node{$\delta$};}
                },postaction=decorate] (-60:1) arc[start angle=-60,end angle=240,radius=1cm];
                \draw [yex,thick,decoration={
                    markings,
                    mark=at position 0.25 with \arrow{>},
                    mark=at position 0.79 with \arrow{>}
                },postaction=decorate] (240:1) -- (0,0.5) -- (-60:1);
                \node at (-0.35,0.25) {$\gamma_1^{}$};
                \node at (0.35,0.25) {$\gamma_2^{}$};
    
                \draw [help lines] plot[smooth] coordinates {(0,0) (-70:1.5) (-35:1.5) (0:1.5) (45:1.5) (90:1.5) (135:1.5) (180:1.5) (215:1.5) (250:1.5) (0,0)};
                \node [below left] at (215:1.5) {$W$};
            \end{tikzpicture}
            \caption{Solution: Pizza pie.}
            \label{fig:CITexr2c}
        \end{subfigure}
        \caption{Practicing with the Cauchy Integral Theorem (2).}
        \label{fig:CITexr2}
    \end{figure}
    \begin{itemize}
        \item There are multiple ways to visualize why the domain does not see the hole/puncture. Here are some examples.
        \item Solution 1 (Figure \ref{fig:CITexr2b}): Cut the loop into two loops in star-shaped domains and add them.
        \begin{itemize}
            \item Draw a straight-line path $\alpha$ from $i/2$ up to $i$.
            \item Since $U$ and $V$ are both star-shaped domains, consecutive applications of the Cauchy Integral Theorem imply that
            \begin{align*}
                \int_{\delta_2\gamma_1\alpha}f\dd{z} = \int_Lf\dd{z} &= 0&
                \int_{\delta_1\alpha^{-1}\gamma_2}f\dd{z} = \int_Rf\dd{z} &= 0
            \end{align*}
            \item Additionally, we know that the sum of the two integrals above is equal to the integral along the entire path in Figure \ref{fig:CITexr2a} because the $\alpha$ and $\alpha^{-1}$ portions cancel. Mathematically,
            \begin{equation*}
                \int_\delta f\dd{z}+\int_\gamma f\dd{z} = \int_{\delta\gamma}f\dd{z}
                = \underbrace{\int_Lf\dd{z}}_0+\underbrace{\int_Rf\dd{z}}_0
                = 0
            \end{equation*}
            \item Therefore,
            \begin{equation*}
                \int_\gamma f\dd{z} = -\int_\delta f\dd{z} = 0
            \end{equation*}
        \end{itemize}
        \item Solution 2 (Figure \ref{fig:CITexr2c}): The pizza pie is star-shaped!
        \begin{itemize}
            \item We can actually draw a star-shaped domain $W$ encapsulating the entire path $\delta\gamma$.
            \item Thus, by the Cauchy Integral Theorem,
            \begin{equation*}
                \int_{\delta\gamma}f\dd{z} = 0
            \end{equation*}
            \item From here, we may proceed as before through
            \begin{align*}
                \int_\gamma f\dd{z}+\int_\delta f\dd{z} &= 0\\
                \int_\gamma f\dd{z} &= -\int_\delta f\dd{z} = 0
            \end{align*}
        \end{itemize}
    \end{itemize}
    \newpage
    \item We now investigate a more general principal than the Cauchy integral theorem called \textbf{homotopy}.
    \begin{itemize}
        \item Algebraic topologists would be insulted by the definition of this term that Calderon is about to give, but it will suffice for our purposes.
    \end{itemize}
    \item \textbf{Homotopic} (paths): Two paths $\gamma,\tilde{\gamma}\subset U$ a domain such that $\tilde{\gamma}$ is obtained from $\gamma$ by modifying $\gamma$ on a small disk $D\subset U$, keeping the endpoints fixed.
    \begin{figure}[h!]
        \centering
        \begin{tikzpicture}[
            every node/.style={black,opacity=1},
            scale=1.3
        ]
            \footnotesize
            \draw [xscale=2,name path=U] plot[smooth cycle] coordinates {
                ($(0:1)+(0.15*rand,0.3*rand)$)
                ($(45:1.2)+(0.15*rand,0.3*rand)$)
                ($(90:1.2)+(0.15*rand,0.3*rand)$)
                ($(135:1.2)+(0.15*rand,0.3*rand)$)
                ($(180:1)+(0.15*rand,0.3*rand)$)
                ($(225:1.2)+(0.15*rand,0.3*rand)$)
                ($(270:1.3)+(0.15*rand,0.3*rand)$)
                ($(315:1)+(0.15*rand,0.3*rand)$)
            };
            \path [name path=Utrace] (0,0) -- (45:1.8);
            \path [name intersections={of=U and Utrace}] (intersection-1) node[above right=-1pt]{$U$};
            \draw [scale=0.3,xscale=2,xshift=1cm,yshift=-0.5cm] plot[smooth cycle] coordinates {
                ($(0:1)+(0.15*rand,0.3*rand)$)
                ($(45:1)+(0.15*rand,0.3*rand)$)
                ($(90:1)+(0.15*rand,0.3*rand)$)
                ($(135:1)+(0.15*rand,0.3*rand)$)
                ($(180:0.8)+(0.15*rand,0.3*rand)$)
                ($(225:0.8)+(0.15*rand,0.3*rand)$)
                ($(270:1)+(0.15*rand,0.3*rand)$)
                ($(315:1)+(0.15*rand,0.3*rand)$)
            };
            \node at (-0.9,-0.3) {*};
            \node at (0.2,0.8) {*};
    
            \draw [rex,opacity=0.5,thick] (-1.5,-0.4)
                to[out=65,in=-130] (-1.2,0.1)
                to[out=50,in=-160] (-0.8,0.4) node[above]{$\gamma$}
                to[out=20,in=170] (-0.3,0.4)
                to[out=-10,in=-170] (0.3,0.37)
                to[out=10,in=-160] (1,0.6)
            ;
            \draw [blx,opacity=0.5,thick] (-1.5,-0.4)
                to[out=65,in=-130] (-1.2,0.1)
                to[out=50,in=-160] (-0.8,0) node[above]{$\tilde{\gamma}$}
                to[out=20,in=170] (-0.3,0.4)
                to[out=-10,in=-170] (0.3,0.37)
                to[out=10,in=-160] (1,0.6)
            ;
    
            \draw [yex] (-0.8,0.4) circle (5mm) node[below right=4mm]{$D$};
            \fill [yex] (-1.2,0.1) circle (1pt);
            \fill [yex] (-0.3,0.4) circle (1pt);
        \end{tikzpicture}
        \caption{Homotopic paths.}
        \label{fig:homotopicPaths}
    \end{figure}
    \item More generally, $\gamma$ and $\tilde{\gamma}$ are \textbf{homotopic} if there exists a finite sequence $\gamma=\gamma_0,\gamma_1,\dots,\gamma_n=\tilde{\gamma}$ such that $\gamma_i\to\gamma_{i+1}$ is obtained by modifying on a small ball.
    \begin{figure}[h!]
        \centering
        \begin{subfigure}[b]{0.33\linewidth}
            \centering
            \begin{tikzpicture}[
                every node/.style={black,opacity=1},
                scale=1.3
            ]
                \footnotesize
                \node at (-1.9,-0.8) {*};
                \node at (-0.6,-1) {*};
                \node at (-0.4,1) {*};
                \node at (1.3,0.8) {*};
        
                \draw [rex,opacity=0.5,thick] (-1.5,-0.4)
                    to[out=65,in=-130] (-1.2,0.1) node[above left]{$\gamma_0^{}$}
                    to[out=50,in=-160] (-0.8,0.4)
                    to[out=20,in=170] (-0.3,0.4)
                    to[out=-10,in=-170] (0.3,0.37)
                    to[out=10,in=-160] (1,0.6)
                ;
                \draw [blx,opacity=0.5,thick] (-1.5,-0.4)
                    to[out=5,in=175] (-0.8,-0.35) node[above]{$\gamma_1^{}$}
                    to[out=-5,in=170] (-0.3,0.4)
                    to[out=-10,in=-170] (0.3,0.37)
                    to[out=10,in=-160] (1,0.6)
                ;
                % \draw [thick,opacity=0.5] plot[smooth] coordinates {(-1.5,-0.4) (-0.8,-0.35) (0.7,-0.5) (1,0.6)};
        
                \fill [yex]
                    (-1.5,-0.4) circle (1pt)
                    (-0.3,0.4) circle (1pt)
                ;
                \draw [yex] (-1.04,0.1) circle (8mm);
            \end{tikzpicture}
            \caption{Stage 1.}
            \label{fig:homotopicGenerala}
        \end{subfigure}
        \begin{subfigure}[b]{0.32\linewidth}
            \centering
            \begin{tikzpicture}[
                every node/.style={black,opacity=1},
                scale=1.3
            ]
                \footnotesize
                \node at (-1.9,-0.8) {*};
                \node at (-0.6,-1) {*};
                \node at (-0.4,1) {*};
                \node at (1.3,0.8) {*};
        
                \draw [rex,opacity=0.5,thick] (-1.5,-0.4)
                    to[out=65,in=-130] (-1.2,0.1) node[above left]{$\gamma_0^{}$}
                    to[out=50,in=-160] (-0.8,0.4)
                    to[out=20,in=170] (-0.3,0.4)
                    to[out=-10,in=-170] (0.3,0.37)
                    to[out=10,in=-160] (1,0.6)
                ;
                \draw [blx,opacity=0.5,thick] (-1.5,-0.4)
                    to[out=5,in=175] (-0.8,-0.35) node[above]{$\gamma_2^{}$}
                    to[out=-5,in=170] (-0.3,0.4)
                    to[out=-10,in=-145] (0.7,-0.5)
                    to[out=35,in=-95,out looseness=0.6] (1,0.6)
                ;
                % \draw [thick,opacity=0.5] plot[smooth] coordinates {(-1.5,-0.4) (-0.8,-0.35) (0.7,-0.5) (1,0.6)};
        
                \fill [yex]
                    (-0.3,0.4) circle (1pt)
                    (1,0.6) circle (1pt)
                ;
                \draw [yex] (0.48,0.2) circle (8mm);
            \end{tikzpicture}
            \caption{Stage 2.}
            \label{fig:homotopicGeneralb}
        \end{subfigure}
        \begin{subfigure}[b]{0.33\linewidth}
            \centering
            \begin{tikzpicture}[
                every node/.style={black,opacity=1},
                scale=1.3
            ]
                \footnotesize
                \node at (-1.9,-0.8) {*};
                \node at (-0.6,-1) {*};
                \node at (-0.4,1) {*};
                \node at (1.3,0.8) {*};
        
                \draw [rex,opacity=0.5,thick] (-1.5,-0.4)
                    to[out=65,in=-130] (-1.2,0.1) node[above left]{$\gamma_0^{}$}
                    to[out=50,in=-160] (-0.8,0.4)
                    to[out=20,in=170] (-0.3,0.4)
                    to[out=-10,in=-170] (0.3,0.37)
                    to[out=10,in=-160] (1,0.6)
                ;
                \draw [blx,opacity=0.5,thick] (-1.5,-0.4)
                    to[out=5,in=175] (-0.8,-0.35) node[above right]{$\gamma_3^{}$}
                    to[out=-5,in=170] (-0.3,-0.43)
                    to[out=-10,in=-145,in looseness=0.5] (0.7,-0.5)
                    to[out=35,in=-95,out looseness=0.5] (1,0.6)
                ;
                % \draw [thick,opacity=0.5] plot[smooth] coordinates {(-1.5,-0.4) (-0.8,-0.35) (0.7,-0.5) (1,0.6)};
        
                \fill [yex]
                    (-0.8,-0.35) circle (1pt)
                    (0.7,-0.5) circle (1pt)
                ;
                \draw [yex] (-0.02,-0.15) circle (8mm);
            \end{tikzpicture}
            \caption{Stage 3.}
            \label{fig:homotopicGeneralc}
        \end{subfigure}
        \caption{A more general homotopy.}
        \label{fig:homotopicGeneral}
    \end{figure}
    \item Claim/TPS: This argument shows that if $\gamma$ and $\tilde{\gamma}$ are homotopic in $U$ and $f\in\mO(U)$, then
    \begin{equation*}
        \int_\gamma f\dd{z} = \int_{\tilde{\gamma}}f\dd{z}
    \end{equation*}
    \emph{Hint}: Just go one little bump at a time.
    \begin{proof}
        The start- and endpoints of the bump form a closed loop within a ball (a star-shaped domain), so the bump loop integrates to zero by the CIT. Thus, the integrals within the ball are the same. Additionally, the paths are literally the same outside of the bump, so the integrals there are the same, too. Therefore, the overall integrals are the same, too.
    \end{proof}
    \item Reality check: Let $f\in\mO(\C^*)$. As a particular example, consider $f(z)=1/z$. Now we know that
    \begin{equation*}
        \int_\circ\frac{1}{z}\dd{z} = 2\pi i \neq 0
    \end{equation*}
    even though we can break the unit circle into the sum of two paths. What's going on?
    \begin{itemize}
        \item The paths are not homotopic; we can't pull them through the hole in the plane.
        \item If we consider the upper hemi-circle and the lower hemi-circle, the two cannot be continuously deformed into each other because we always get stuck at the puncture.
    \end{itemize}
    \pagebreak
    \item We now prove a slightly stronger version of the Cauchy integral theorem.
    \item Corollary: Let $U$ be any domain, $D$ be a disk in $U$, and $z\in\mathring{D}$. Suppose $f\in\mO(U\setminus\{z\})$ and is bounded near $z$. Then
    \begin{equation*}
        \int_{\partial D}f\dd{z} = 0
    \end{equation*}
    \begin{proof}
        Step 1: Use homotopy.
        \begin{figure}[H]
            \centering
            \begin{tikzpicture}[
                every node/.style=black,
                scale=1.3
            ]
                \footnotesize
                \draw [xscale=2,name path=U] plot[smooth cycle] coordinates {
                    ($(0:1)+(0.15*rand,0.3*rand)$)
                    ($(45:1.2)+(0.15*rand,0.3*rand)$)
                    ($(90:1.2)+(0.15*rand,0.3*rand)$)
                    ($(135:1.2)+(0.15*rand,0.3*rand)$)
                    ($(180:1)+(0.15*rand,0.3*rand)$)
                    ($(225:1.2)+(0.15*rand,0.3*rand)$)
                    ($(270:1.3)+(0.15*rand,0.3*rand)$)
                    ($(315:1)+(0.15*rand,0.3*rand)$)
                };
                \path [name path=Utrace] (0,0) -- (45:1.8);
                \path [name intersections={of=U and Utrace}] (intersection-1) node[above right=-1pt]{$U$};
                \draw [scale=0.3,xscale=2,xshift=1cm,yshift=0.5cm] plot[smooth cycle] coordinates {
                    ($(0:1)+(0.15*rand,0.3*rand)$)
                    ($(45:1)+(0.15*rand,0.3*rand)$)
                    ($(90:1)+(0.15*rand,0.3*rand)$)
                    ($(135:1)+(0.15*rand,0.3*rand)$)
                    ($(180:0.8)+(0.15*rand,0.3*rand)$)
                    ($(225:0.8)+(0.15*rand,0.3*rand)$)
                    ($(270:1)+(0.15*rand,0.3*rand)$)
                    ($(315:1)+(0.15*rand,0.3*rand)$)
                };
                \path [name path=slit] (-0.3,-0.3) -- ++(-1,-1);
                \draw [name intersections={of=U and slit}] (intersection-1) -- (-0.3,-0.3);
                \node at (-0.1,-0.7) {*};
        
                \draw [blx,thick] (-1,0) circle (6mm) node[above right=5mm]{$D$};
                \draw [orx,thick] (-0.8,-0.2) circle (2mm) node[above left=1mm]{$\gamma_\varepsilon$};
                \fill [yex] (-0.8,-0.2) circle (1pt) node[below=-1pt]{$z$};
            \end{tikzpicture}
            \caption{Bounded holomorphic functions integrate to zero on disk boundaries.}
            \label{fig:holDiskBound}
        \end{figure}
        Via the above claim,
        \begin{equation*}
            \int_{\partial D}f\dd{z} = \int_{\gamma_\varepsilon}f\dd{z}
        \end{equation*}
        where $\gamma_\varepsilon$ is a circle around $z$ within the region where $f$ is bounded\footnote{We could also turn the plane into the sum of two star-shaped domains again.}.\par
        Step 2: We have that
        \begin{equation*}
            \left| \int_{\gamma_\varepsilon}f\dd{z} \right| \leq \max_{z\in\gamma_\varepsilon}|f(z)|\cdot\len(\gamma_\varepsilon)
        \end{equation*}
        Since $f$ is bounded near $z$, the maximum is finite. Additionally, the length term is just $2\pi\varepsilon$, so we can send $\varepsilon\to 0$ and thus send the integral to zero.
    \end{proof}
    \item We now look into the \textbf{Cauchy Integral Formula}.
    \item \textbf{Cauchy Integral Formula}: Suppose $U$ is any domain, $D\subset U$ is a disk (i.e., $D\subset\subset U$ or $\overline{D}\subset U$), $f\in\mO(U)$, and $z\in D$. Then
    \begin{equation*}
        f(z) = \frac{1}{2\pi i}\int_{\partial D}\frac{f(\zeta)}{\zeta-z}\dd\zeta
    \end{equation*}
    \begin{proof}
        We're going to try to use the corollary and define a function. In particular, define
        \begin{equation*}
            g(\zeta) =
            \begin{cases}
                \frac{f(\zeta)-f(z)}{\zeta-z} & \zeta\neq z\\
                f'(z) & \zeta=z
            \end{cases}
        \end{equation*}
        Because $f$ is holomorphic at $z$, $g$ is continuous at $z$ and hence bounded near $z$. We can also see that since $g$ is a rational function of holomorphic functions on $U\setminus\{z\}$, we have $g\in\mO(U\setminus\{z\})$.\par
        Now the corollary says that
        \begin{equation*}
            \int_{\partial D}g\dd{\zeta} = 0
        \end{equation*}
        Additionally, by the definition of $g$, we have that
        \begin{equation*}
            \int_{\partial D}g\dd{\zeta} = \int_{\partial D}\frac{f(\zeta)}{\zeta-z}\dd{\zeta}-\int_{\partial D}\frac{f(z)}{\zeta-z}\dd{\zeta}
        \end{equation*}
        $f(z)$ is just a complex number, so we can pull it out of the rightmost integral above. Additionally, under a change of variables and invoking PSet 1, QA.4, we have that
        \begin{equation*}
            \int_{\partial D}\frac{f(z)}{\zeta-z}\dd{\zeta} = f(z)\int_{\partial D}\frac{1}{\zeta-z}\dd{\zeta}
            = \int_\text{unit circle}\frac{1}{z-a}\dd{z}
            = 2\pi if(z)
        \end{equation*}
        Note: Another way to evaluate this integral is as follows. If $z$ is the center of the disk, then we win and can get $2\pi i$ using PSet 1, QA.4 directly. If $z$ isn't at the center of the disk, we are allowed to slide it. Here's why: Think about the integrand as a function of $z$, so
        \begin{equation*}
            \pdv{z}(\int_{\partial D}\frac{1}{\zeta-z}\dd{\zeta}) = \int_{\partial D}\pdv{z}(\frac{1}{\zeta-z})\dd{\zeta}
            = \int_{\partial D}\frac{1}{(\zeta-z)^2}\dd{\zeta}
            = 0
        \end{equation*}
        Since we're taking the integral and the limit with respect to different things, we can exchange them. Since the second integrand has a primitive, it equals zero. But this means that the integral does not change even as $z$ changes, which is equivalent to saying we can move $z$ around to wherever we want in the disk and the integral will still be $2\pi i$! In other words, if $z$ is somewhere where we can't evaluate the integral directly, we can move $z$ to somewhere where we \emph{can} evaluate the integral directly with no consequence.
    \end{proof}
    \item Implication of Cauchy's Integral Theorem: The values of the function are completely determined by the values on the boundary, i.e., holomorphic functions are determined by boundary values.
    \item Let's now prove another theorem.
    \item Theorem: Let $U$ be any domain, $f\in\mO(U)$. Then $f'\in\mO(U)$, $f''\in\mO(U)$, on and on.
    \begin{proof}
        Let's use the Cauchy integral formula. We have that
        \begin{equation*}
            f(z) = \frac{1}{2\pi i}\int_{\partial D}\frac{f(\zeta)}{\zeta-z}\dd\zeta
        \end{equation*}
        Now let's take the derivative, which we know exists because $f$ is holomorphic.
        \begin{equation*}
            \pdv{f}{z} = \frac{1}{2\pi i}\int_{\partial D}\pdv{z}(\frac{f(\zeta)}{\zeta-z})\dd\zeta
            = \frac{1}{2\pi i}\int_{\partial D}\frac{f(\zeta)}{(\zeta-z)^2}\dd\zeta
        \end{equation*}
        Thus, the derivative has a Cauchy integral formula. We can keep taking derivatives on the inside because the integrand is infinitely differentiable. Thus, we can keep taking derivatives on the outside. And that's the proof.
    \end{proof}
    \item Corollary: Holomorphic functions are $C^\infty$.
    \item Corollary: In general,
    \begin{equation*}
        f^{(n)}(z) = \frac{n!}{2\pi i}\int_{\partial D}\frac{f(\zeta)}{(\zeta-z)^{n+1}}\dd\zeta
    \end{equation*}
    \item This last result allows us to bound things really easily, giving us \textbf{Cauchy's inequalities}.
    \begin{itemize}
        \item Essentially, let $D$ have radius $R$ and let $z$ be the center of $D$. Then
        \begin{equation*}
            |f^{(n)}(z)| \leq \frac{n!}{2\pi i}\max_{\partial D}\left| \frac{f(\zeta)}{R^{n+1}} \right|\cdot 2\pi R
            = \frac{n!}{R^n}\max_{\partial D}|f(\zeta)|
        \end{equation*}
    \end{itemize}
    \item Liouville's Theorem: Suppose $f\in\mO(\C)$ (i.e., $f$ is \textbf{entire}) and $f$ is bounded. Then it's constant.
    \begin{proof}
        Take a point $z\in\C$. Take a huge ball with radius $R$. Cauchy's inequality says that if we take the derivative, then
        \begin{equation*}
            |f'(z)| \leq \frac{1}{R}\cdot\max_{\partial D}|f(\zeta)|
        \end{equation*}
        The maximum is bounded and $R$ is really big, so as $R\to\infty$, the derivative gets arbitrarily small. So if we've got an arbitrary function with zero derivative, then we've got a constant function.
    \end{proof}
    \item \textbf{Entire} (function): A complex-valued function that is holomorphic on the whole complex plane.
\end{itemize}



\section{Analytic Continuation and Removable Singularities}
\begin{itemize}
    \item \marginnote{4/4:}Last time.
    \begin{itemize}
        \item Cauchy integral formula: Let $U$ be any domain, $f\in\mO(U)$, $D\subset\subset U$, and $z\in D$. Then
        \begin{equation*}
            f(z) = \frac{1}{2\pi i}\int_{\partial D}\frac{f(\zeta)}{\zeta-z}\dd\zeta
        \end{equation*}
        \item Implies that holomorphic functions are $C^\infty$.
        \item Implies Cauchy's inequalities: If $D$ is a disk centered at $z_0$ of radius $R$, then
        \begin{equation*}
            |f^{(n)}(z_0)| \leq \frac{n!}{R^n}\sup_{\partial D}|f(\zeta)|
        \end{equation*}
        \item Implies Liouville's theorem: Any bounded entire function is constant.
    \end{itemize}
    \item Our focus today is on results we can get out of power series.
    \item Observe that if
    \begin{equation*}
        P(z) = \sum_{k=0}^\infty a_k(z-z_0)^k
    \end{equation*}
    is a convergent power series centered at $z_0$, then
    \begin{equation*}
        P^{(n)}(z_0) = n!a_n
    \end{equation*}
    \item Now let $f\in\mO(U)$.
    \begin{itemize}
        \item TPS: What should the power series for $f$ look like?
        \begin{itemize}
            \item Rearranging the above, we want
            \begin{equation*}
                a_k = \frac{f^{(k)}(z_0)}{k!}
            \end{equation*}
            \item The following power series formally has the right derivatives.
            \begin{equation*}
                P(z) = \sum_{k=0}^\infty\frac{f^{(k)}(z_0)}{k!}(z-z_0)^k
            \end{equation*}
        \end{itemize}
        \item Does this power series converge though, and if so, where?
        \begin{itemize}
            \item Recall that the Cauchy-Hadamard formula tells us that the radius of convergence $r$ satisfies
            \begin{equation*}
                r = \left( \limsup_{k\to\infty}|a_k|^{1/k} \right)^{-1}
            \end{equation*}
            \item Pick a $z_0\in U$ and a disk $D\subset\subset U$ of radius $R$. Then by the Cauchy inequalities,
            \begin{equation*}
                |a_k|^{1/k} = \left| \frac{f^{(k)}(z_0)}{k!} \right|^{1/k}
                \leq \frac{|\sup_{\partial D}f(\zeta)|^{1/k}}{R}
                \to \frac{1}{R}
            \end{equation*}
            \item Thus, returning to the Cauchy-Hadamard formula, the radius of convergence is $\geq R$.
        \end{itemize}
        \item So we've got a convergent power series, but why does this power series equal $f(z)$?
        \begin{itemize}
            \item We know that $P(z)$ and $f(z)$ have all the same derivatives.
            \item However, over $\R$, this is not enough! Recall the example of $\e[-1/x^2]$, which has the same derivatives as its power series at zero but is not equal to it.
            \item Over $\C$, however, we claim that having the same derivatives \emph{is} enough.
            \item Use CIF and expand $1/(\zeta-z)$.
            \item Note: To keep all of our $z$'s straight, recall that $z_0$ is a point, $\zeta$ lies on $\partial D$ where $D$ is centered at $z_0$, and $z$ is somewhere in $\mathring{D}$.
            \item Doing this, we obtain
            \begin{equation*}
                \frac{1}{\zeta-z} = \frac{1}{\zeta-z_0}\cdot\frac{1}{1-\left( \frac{z-z_0}{\zeta-z_0} \right)}
                = \frac{1}{\zeta-z_0}\sum_{k=0}^\infty\left( \frac{z-z_0}{\zeta-z_0} \right)^k
            \end{equation*}
            \item Thus,
            \begin{align*}
                f(z) &= \frac{1}{2\pi i}\int_{\partial D}\sum_{k=0}^\infty\frac{f(\zeta)}{(\zeta-z_0)^{n+1}}(z-z_0)^n\dd\zeta\\
                &= \sum_{k=0}^\infty\underbrace{\frac{1}{2\pi i}\int\frac{f(\zeta)}{(\zeta-z_0)^{n+1}}\dd\zeta}_{f^{(n)}(z_0)/n!}(z-z_0)^n\\
                &= P(z)
            \end{align*}
            \item Recall that we can bring the sum outside of the integral because of uniform convergence and our lemma about integrable functions from the 3/26 class.
        \end{itemize}
        \item All in all, we've shown that any holomorphic function has a power series representation on any disk that fits within the domain, and the power series representation is the one we think it should be.
    \end{itemize}
    \item We now discuss an important corollary to this result.
    \item The Identity Theorem: If two holomorphic functions $f,g\in\mO(U)$ agree on an open set in $U$, then $f=g$.
    \begin{proof}
        This is true for power series.\par
        For every point, there's a power series representation around that series so we can do something with a covering of open sets, though we do not need compactness for $U$.
    \end{proof}
    \begin{itemize}
        \item An analogous result does not hold on the reals. For example, there are plenty of functions that are zero for a while, then bump up to 1 for a while, so they're 0 and 1 on open sets without being either 0 or 1.
        \item Implication: "Holomorphic functions are very rigid."
    \end{itemize}
    \item In fact, more is true: If $z_n\to z_0$ where each $z_n$ is distinct and $f(z_n)=g(z_n)$ for all $n$, then $f=g$.
    \begin{itemize}
        \item So we don't even need an open set; all we need is an \textbf{accumulation point}.
    \end{itemize}
    \item \textbf{Analytic continuation} (of $f$): The function $g\in\mO(V)$ where $f\in\mO(U)$, $V\supset U$, and $f=g$ on $U$.
    \begin{itemize}
        \item Note that we get to say "\emph{the} function $g$\dots" because of the identity theorem.
        \item Formally, $g_1,g_2$ analytic continuations of $f$ and $g_1=g_2$ on $U$ open implies $g_1=g_2$.
    \end{itemize}
    \item Example: Consider $f(z)=z$ with $f\in\mO(\C^*)$. Then $g(z)=z$ with $g\in\mO(\C)$ is an analytic continuation of $f$.
    \item What we're essentially doing is taking the power series (which we get via "analytic") and extending them out into $V$.
    \item Example: The Riemann zeta function is defined by
    \begin{equation*}
        \zeta(s) = \sum_{n=1}^\infty n^{-s}
        = \sum_{n=1}^\infty\e[-s\log n]
    \end{equation*}
    Where does this make sense, i.e., where does the series converge?
    \begin{itemize}
        \item We have $|n^{-s}|=n^{-\re(s)}$.
        \item Thus, the series converges only when $\re(s)>1$.
        \item The Riemann hypothesis predicts where $\zeta(s)=0$. We know that it has some zeroes at the negative even integers, and the RH predicts that the rest of them fall on the line $\re(s)=1/2$.
        \item But $\zeta$ is only defined on a part of the complex plane not including these regions! Thus, to make sense of the RH, we need to analytically continue $\zeta$.
    \end{itemize}
    \item Given $f\in\mO(U)$, what is the "biggest" $V$ on which $f\in\mO(V)$? In layman's terms, where should $f$ live?
    \item Example: $1/z\in\mO(\C^*)$ and $1/z\notin\mO(\C)$.
    \begin{itemize}
        \item Note that we know the latter statement because if you're holomorphic, the integral around any closed loop in the domain is zero but the integral of this function on the unit circle is $2\pi i$, so it can't be holomorphic on $\C$. Contradiction.
    \end{itemize}
    \item Example: $\sin(1/z)\in\mO(\C^*)$, but is it in $\mO(\C)$?
    \begin{itemize}
        \item No; recall from PSet 1 that it's not even \emph{continuous} at 0.
    \end{itemize}
    \item Example: $\sin(z)/z\in\mO(\C^*)$, but is it in $\mO(\C)$?
    \item Recall Goursat's lemma: $f\in\mO(\text{nbhd}(\triangle))$ implies $\int_\triangle f\dd{z}=0$.
    \begin{itemize}
        \item If $U$ is star-shaped and $\int_\triangle f\dd{z}=0$ for all triangles, then $f$ has a primitive.
        \begin{itemize}
            \item Note that we do not need $f$ holomorphic for this result!
        \end{itemize}
        \item This latter result has a converse!
    \end{itemize}
    \item Morera's Theorem: If $U$ is any domain, $f:U\to\C$ is continuous, and $\int_\triangle f\dd{z}=0$ for all triangles, then $f$ is holomorphic.
    \begin{proof}
        Fix a disk $D\subset\subset U$. Disks are star-shaped! This combined with the fact that the integral over all triangles is zero implies that $f$ has a primitive $F\in\mO(U)$ by the result a couple lines up. But since $F$ is holomorphic, by our result from last class, $F'=f\in\mO(U)$, too.
    \end{proof}
    \item \textbf{Riemann's removable singularity theorem}: Suppose $U$ is a domain, $z\in U$, $f\in\mO(U\setminus\{z\})$, and $f$ is bounded near $z$. Then there exists a unique analytic continuation $\hat{f}\in\mO(U)$. \emph{Also known as} \textbf{Riemann extension theorem}.
    \begin{itemize}
        \item In this case, we call $z$ a \textbf{removable singularity}.
        \item Note: The contrapositive of this says that if there is not an analytic continuation (i.e., the function is honestly not holomorphic at a point and can't be extended to one, e.g., $1/z$), then $|f|$ has to blow up as you approach $z$ (in some direction).
    \end{itemize}
    \item \textbf{Singularity} (of $f$): A point $z_0$ such that $f\in\mO(U\setminus\{z_0\})$.
    \item \textbf{Removable} (singularity): A singularity of a function that that satisfies the hypotheses of Riemann's removable singularity theorem.
    \item If a singularity is not removable, then $f$ is not bounded near $z_0$. This leads to additional definitions.
    \item \textbf{Pole}: A non-removable singularity $z_0$ of a function $f$ for which $|f(z)|\to\infty$ as $z\to z_0$.
    \begin{itemize}
        \item So-named because of real analysis where a pole is an asymptote, and asymptotes kind of look like poles!
    \end{itemize}
    \item \textbf{Essential} (singularity): A non-removable singularity that is not a pole; equivalently, a singularity $z_0$ for which there exist sequences $z_n\to z_0$ and $w_n\to z_0$ such that $|f(z_n)|\to\infty$ and $|f(w_n)|$ stays bounded.
    \item Proving Riemann's removable singularity theorem.
    \begin{proof}
        Set
        \begin{equation*}
            F(\zeta) =
            \begin{cases}
                f(\zeta)(\zeta-z) & \zeta\neq z\\
                0 & \zeta=z
            \end{cases}
        \end{equation*}
        Then $F\in\mO(U\setminus\{z\})$ so $F$ is continuous at $z$.\par
        We want to show that $F$ is holomorphic (using Morera's theorem). To do this, we'll need to show that the integral over all triangles is zero. More specifically, all we need to do is show that $F$ is holomorphic in a little ball $D$ about $z$. Now we need to do some casework.\par
        Case 1: If $\triangle\not\ni z$, then we can draw a star-shaped domain surrounding the triangle on which $f$ will be holomorphic and invoke the CIT to imply that the integral is zero.\par
        Case 2: If $\triangle\ni z$, then $\int F\dd{z}$ is arbitrarily small. Recall that we get this by using homotopy to replace the integral over the triangle with the integral over some tiny $\gamma_\varepsilon$. Arbitrarily small because $f$ is bounded.\par
        Morera then tells us that $F\in\mO(U)$, so $F'=f\in\mO(U)$. Note that $F'=f$ because
        \begin{equation*}
            F'(z) = \lim_{\zeta\to z}\frac{F(\zeta)-F(z)}{\zeta-z}
            = \lim_{\zeta\to z}\frac{f(\zeta)(\zeta-z)-0}{\zeta-z}
            = \lim_{\zeta\to z}f(\zeta)
            = f(z)
        \end{equation*}
    \end{proof}
    \item Go back and add a $z_0$ everywhere and then it should all be ok.
    \item With the removable singularity theorem, we can now confirm that $\sin(z)/z$ has a removable singularity because although $1/z$ diverges, sine converges faster so this function is bounded near zero.
    \begin{itemize}
        \item We can prove boundedness with the Taylor series of $\sin(z)/z$.
    \end{itemize}
\end{itemize}



\section{Chapter II: The Fundamental Theorems of Complex Analysis}
\emph{From \textcite{bib:FischerLieb}.}
\subsection*{Section II.3: The Cauchy Integral Formula}
\begin{itemize}
    \item \marginnote{4/13:}Statement and proof of the CIF.
    \begin{itemize}
        \item \textcite{bib:FischerLieb} come at it in a slightly more complicated way than class.
    \end{itemize}
    \item Derivatives of the CIF.
    \item Holomorphic implies $C^\infty$.
    \item Morera's theorem and proof.
    \item The Riemann extension theorem and proof.
\end{itemize}


\subsection*{Section II.4: Power Series Expansions of Holomorphic Functions}
\begin{itemize}
    \item Proof of the typical power series formula.
    \item Power series examples.
    \begin{enumerate}
        \item $(z-a)^{-1}$.
        \begin{itemize}
            \item Has a standard power series.
            \item $(z-a)^{-n}$ is computed via term-by-term differentiation of this base power series and corresponding adjustments.
            \item More complicated rational functions are handled via partial fraction decomposition and then the above method.
        \end{itemize}
        \item Products of holomorphic functions.
        \begin{itemize}
            \item Use \textbf{Leibniz's rule}.
        \end{itemize}
        \item Inverse of a power series.
        \begin{itemize}
            \item Suppose $f(z)=\sum_{n=0}^\infty a_n(z-z_0)^n$ in a neighborhood of $z_0$, and assume that $f(z_0)=a_0\neq 0$.
            \item Then $1/f$ is holomorphic near $z_0$.
            \item Moreover, the coefficients of the expansion
            \begin{equation*}
                \frac{1}{f(z)} = \sum_{n=0}^\infty b_n(z-z_0)^n
            \end{equation*}
            can be determined.
            \item To do so, compare coefficients in
            \begin{equation*}
                1 = \frac{1}{f(z)}\cdot f(z)
                = \sum_{n=0}^\infty\left( \sum_{m=0}^nb_ma_{n-m} \right)(z-z_0)^n
            \end{equation*}
            \item This yields $b_0a_0=1$ and hence $b_0=a_0^{-1}$. Continuing, $b_0a_1+b_1a_0=0$, so $b_1=-b_0a_1a_0^{-1}=-a_1a_0^{-2}$. We can continue this computation as far as we like.
        \end{itemize}
        \item $\tan z$.
        \begin{itemize}
            \item Observe that $\tan z$ is odd, and hence the even-powered terms disappear.
            \item Substitute the power series for cosine, sine, and the undetermined one for tangent into the equation $\cos z\cdot\tan z=\sin z$ and compare coefficients.
            \item We could also do $\sin z/\tan z$ using the method of Examples 2-3 above.
        \end{itemize}
    \end{enumerate}
    \item \textbf{Leibniz's rule}: The derivative of the product $fg$ of two functions $f,g$ that are both holomorphic in a neighborhood of $z_0$ is given by
    \begin{equation*}
        \frac{(fg)^{(n)}(z_0)}{n!} = \sum_{m=0}^n\frac{f^{(m)}(z_0)}{m!}\cdot\frac{g^{(n-m)}(z_0)}{(n-m)!}
    \end{equation*}
    \begin{itemize}
        \item This is just the Cauchy product of the two formal power series!
    \end{itemize}
    \item "In order to determine the derivatives $f^{(n)}(z_0)$ of a function $f$ that is holomorphic at $z_0$, one only needs to know the values $f(z)$ on, say, a segment $(z_0-\delta,z_0+\delta)$ parallel to the real axis" \parencite[52]{bib:FischerLieb}.
    \item Build up to, statement of, and proof of the identity theorem.
    \item Consequence of the identity theorem: Holomorphic functions on $U$ are completely determined by their values on any \textbf{nondiscrete} set in $U$.
    \begin{itemize}
        \item "Properties that can be expressed via identities between holomorphic functions on [$G$] thus only need to be verified on a nondiscrete set in [$G$]" \parencite[53]{bib:FischerLieb}.
        \item Example: $\cot(\pi z)$ and $\cot(\pi z+\pi)$ coincide on the nondiscrete set $\C\setminus\Z$, and thus are equal; therefore, the periodicity of cotangent on the real numbers implies it on the complex numbers.
        \item This also means that the exponential function and those derived from it (e.g., sine and cosine) can only be extended from the real to the complex numbers in one way since $\R$ is nondiscrete in $\C$.
    \end{itemize}
    \item \textbf{Nondiscrete} (set in $U$): A set $M\subset U$ that contains an accumulation point of $M$.
    \item \textbf{Discrete} (set in $U$): A set $M\subset U$ that is not nondiscrete.
    \item Defines a \textbf{zero} (of order $L$), tapping into next week's content.
    \item Since constancy on a nondiscrete set would imply constancy everywhere (for a holomorphic function), the set $f^{-1}(w)$ of points at which a nonconstant holomorphic function takes on the same value $w\in\C$ is discrete.
    \item Characterizing holomorphicity: Let $f:U\to\C$ be a function defined on an open set $U\subset\C$. The following are equivalent:
    \begin{enumerate}[label={\roman*.}]
        \item $f$ is holomorphic.
        \item $f$ is real differentiable and satisfies the Cauchy-Riemann equations.
        \item $f$ admits a power series expansion about every point in $U$.
        \item $f$ has local primitives.
        \item $f$ is continuous, and for every closed triangle $\triangle\subset U$, $\int_{\partial\triangle}f(z)\dd{z}=0$.
    \end{enumerate}
\end{itemize}




\end{document}