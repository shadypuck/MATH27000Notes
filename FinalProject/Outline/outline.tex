\documentclass[../finalProject.tex]{subfiles}

\pagestyle{main}
\renewcommand{\leftmark}{Report and Presentation Outline}

\begin{document}




\section*{Outline}
\addcontentsline{toc}{section}{Outline}
\begin{itemize}
    \item Rough timing / spacing outline.
    \begin{itemize}
        \item Intro to quantum mechanics (5 minutes / 750 words).
        \item Using hypergeometric functions to mathematically solve the Hermite and Legendre equations (5 minutes / 750 words).
        \item The complex analysis: Applications of residues to Rodrigues expressions, contour integrals and generating functions (8 minutes / 1200 words)
        \item What the complex analysis indirectly gets you, e.g., certain physical properties like orthogonality and normalization that would be harder to compute directly (2 minutes / 300 words)
    \end{itemize}
\end{itemize}


\section*{Intro to QMech Ideas}
\begin{itemize}
    \item A preview of where the complex analysis comes in.
    \begin{itemize}
        \item This is an ordinary differential equation that physicists care about:
        \begin{equation*}
            -\frac{\hbar^2}{2m}\pdv[2]{\psi(x)}{x}+V(x)\cdot\psi(x) = E\psi(x)
        \end{equation*}
        \item What do they do with it?
        \begin{itemize}
            \item They take a potential energy function $V:\R\to\R$ of interest and use this equation to solve for a corresponding $\psi:\R\to\R$.
            \item Some potential energy functions $V$ give rise to special differential equations, such as the \textbf{Hermite equation} and \textbf{Legendre equation}.
        \end{itemize}
        \item Why do we care?
        \begin{itemize}
            \item We can use complex analysis and the hypergeometric function introduced on Problem Set 2 to solve these equations and learn about their properties.
        \end{itemize}
    \end{itemize}
    \item Quantum mechanics background.
    \begin{itemize}
        \item In the name of being concise in my background, I'm going to intentionally skip some details. You're free to ask me about these things, but I have done my best to present a cohesive, standalone introduction.
        \item Quantum mechanics is better \emph{done} than \emph{understood} at first. Understanding typically develops with experience in doing the computations, which is a strange but fairly valid pedagogical approach. However, since I don't have the time to walk you through a bunch of computations, I will do my best to offer a handwavey verbal explanation.
        \begin{itemize}
            \item Quote my physics textbook here??
        \end{itemize}
        \item Classical physics: Matter is composed of particles whose motion is governed by Newton's laws, most famously, the second-order differential equation
        \begin{equation*}
            -\dv{V}{x} = F 
            = ma
            = m\dv[2]{x}{t}
        \end{equation*}
        \begin{itemize}
            \item Analyze larger objects as collections of particles each evolving under Newton's laws.
            \item Matter has a fundamentally \emph{particle-like} nature.
        \end{itemize}
        \item New results challenge this postulate.
        \begin{itemize}
            \item Einstein (1905): The photoelectric effect equation and the mass-energy equation.
            \begin{align*}
                E &= h\nu = \frac{hc}{\lambda}&
                E &= mc^2
            \end{align*}
            \item Combining these, we find that light has mass!
            \begin{align*}
                mc^2 &= \frac{hc}{\lambda}\\
                m &= \frac{h}{\lambda c}
            \end{align*}
            \item Louis de Broglie (1924): Turns in a 4-page PhD thesis and says:
            \begin{equation*}
                \lambda = \frac{h}{mc}
            \end{equation*}
            \item Paris committee will fail him, but they write to Einstein who recognizes the importance of this work \parencite[7]{bib:CHEM26100Notes}.
            \item Takeaway: de Broglie has just postulated that fundamental particles of matter (e.g., electrons) have a wavelike nature.
            \item Davisson-Germer experiment: Update to Thomas Young's double-slit experiment. They use electrons and \emph{still} observe a diffraction pattern. Confirms de Broglie's hypothesis.
        \end{itemize}
        \item So what is matter?
        \begin{itemize}
            \item Modern physicists and chemists will say it has a \textbf{dual wave-particle nature}.
            \item What does this mean? I mean, I can picture a wave, I can picture a particle, and they don't look the same! How should I picture it?
            \item Remember, all we can do as scientists is provide a model to summarize our experimental results.
            \item Occam's razor: Simpler models are better.
            \item There are some experimental results in which light behaves like a particle and some in which it behaves like a wave. We will use each model when appropriate and leave the true nature of matter unsettled until we have more data.
        \end{itemize}
        \item For the remainder of this discussion, let us confine ourselves to one-dimensional space.
        \item So if matter is a wave, then it is spread out over all space in some sense; it does not exist locally at some point $x$, but rather at each point $x\in\R$, it has some intensity $\psi(x)$ given by a wave function $\psi:\R\to\R$.
        \item What constraints can we put on $\psi$?
        \item Schr\"{o}dinger (1925):
        \begin{equation*}
            -\frac{\hbar^2}{2m}\pdv[2]{\psi(x)}{x}+V(x)\cdot\psi(x) = E\psi(x)
        \end{equation*}
        \begin{itemize}
            \item In the Swiss Alps with his mistress.
            \begin{itemize}
                \item Wasn't just Oppenheimer.
            \end{itemize}
            \item Richard Feynman: "Where did we get that [equation] from? Nowhere. It is not possible to derive it from anything you know. It came from the mind of Schr\"{o}dinger."
            \item Feynman, true to character, was being mildly facetious, but the core of what he says is true: It was a pretty out-of-left-field result.
        \end{itemize}
        \item So say we're given some potential $V(x)$ and get a $\psi(x)$ that solves the TISE. What does $\psi(x)$ tell us?
        \begin{itemize}
            \item Nothing directly.
            \item Born (1926): $|\psi(x)|^2$ gives the probability that the wave/particle is at $x$.
            \item Examples likening densities to orbitals from Gen Chem I final review session..
        \end{itemize}
        \item The universe can still be quantized even if we can't see it.
        \begin{itemize}
            \item The Earth can still be round even if we can't see it.
            \item The pixels in a screen can still be quantized even if we can't see them.
        \end{itemize}
        \item Now, where is all of this going? Why am I talking about quantum mechanics in my complex analysis final project?
        \begin{itemize}
            \item While you or I might care about the solutions to these questions in the abstract and just for funsies, the people who will pay you to do your research might not. As such, it is important to be able to explain to a non-mathematician where your problem comes from and how a solution will benefit the average Joe.
        \end{itemize}
        \item This brings us to microwaves.
        \begin{itemize}
            \item Personally, I like microwaves. They heat up food far more quickly than a traditional oven, they're energy efficient, and they go ding when they're done.
            \item Microwaves work because of quantum mechanics.
            \item Essentially, they shoot light of just the right frequency at your food so that molecules in it --- which are already vibrating harmonically --- vibrate faster. Faster vibrations means warmer food.
            \item But how do we analyze such a vibrating molecule to know what frequency of light to shoot at it? Well, a vibrating molecule can be modeled as a quantum harmonic oscillator, that is, a quantum particle with
            \begin{equation*}
                V(x) = \frac{1}{2}kx^2
            \end{equation*}
            \item Sparing you the gory details, if we plug this into the Schr\"{o}dinger equation and do some rearranging, we end up having to solve the \textbf{Hermite equation}:
            \begin{equation*}
                \dv[2]{H}{y}-2y\dv{H}{y}+(\epsilon-1)H(y) = 0
            \end{equation*}
            \item To solve the Hermite equation, we need complex analysis and the hypergeometric function.
        \end{itemize}
        \item Alright, where else can we use such techniques?
        \begin{itemize}
            \item What if we care about chemistry, at all?
            \item Once atoms and molecules were discovered, chemistry developed as the discipline that uses atoms and molecules to do stuff, be it synthesizing a new medicine, mass-producing the ammonia fertilizer that feeds the planet, or literally anything else.
            \item "Doing stuff" with atoms and molecules, however, is greatly facilitated by a good understanding of how atoms and molecules interact, and hence how they're structured.
            \item Once again, quantum mechanics provides the answers we need.
            \item A classic example is the electronic structure of the hydrogen atom, which consists of a single electron (a quantum particle) existing in the potential
            \begin{equation*}
                V(r) = -\frac{e^2}{4\pi\varepsilon_0r}
            \end{equation*}
            \begin{itemize}
                \item FYI, that is not Euler's number in the numerator but rather the charge of an electron.
            \end{itemize}
            \item Sparing you the gory details once again, if we plug this into the Schr\"{o}dinger equation and do some rearranging, we end up having to solve the \textbf{Legendre equation}:
            \begin{equation*}
                (1-x^2)\dv[2]{P}{x}-2x\dv{P}{x}+\left[ \ell(\ell+1)-\frac{m^2}{1-x^2} \right]P(x) = 0
            \end{equation*}
        \end{itemize}
    \end{itemize}
    \item \textcite[28-31]{bib:CHEM26100Notes}: Hermite polynomials derivation.
    \begin{itemize}
        \item Address the quantum harmonic oscillator.
        \item Apply the 1D TISE.
        \item Change coordinates.
        \item Take an asymptotic solution.
        \item Discover that the general solutions are of the form $H(y)\e[-y^2/2]$.
        \item Substituting back into the TISE, we obtain the Hermite equation.
        \item Solve via a series expansion and recursion relation.
        \item Truncate the polynomial expansion to quantize.
    \end{itemize}
    \item \textcite[56-65]{bib:CHEM26100Notes}: Legendre polynomials and associated Legendre functions derivation.
    \begin{itemize}
        \item Address the hydrogen atom.
        \item Starting from the 3D TISE in spherical coordinates, use separation of variables to isolate a one-variable portion of the angular equation. When rearranged, this ODE becomes \textbf{Legendre's equation}.
        \item Solving Legendre's equation when $m=0$ gives the Legendre polynomials $P_\ell(x)$.
        \item Solving Legendre's equation when $m\neq 0$ gives the associated Legendre functions
        \begin{equation*}
            P_\ell^{|m|}(x) = (1-x^2)^{|m|/2}\dv[|m|]{x}[P_\ell(x)]
        \end{equation*}
    \end{itemize}
    \item \textcite[34-37]{bib:PHYS23410Notes}: Much more detailed asymptotic analysis and derivation of the Hermite equation.
    \begin{itemize}
        \item Here, we properly motivate the $H(y)\e[-y^2/2]$ that was just supplied last time.
        \item Hermite polynomials are eventually defined via the following formula, which is \emph{not} derived.
        \begin{equation*}
            H_n(\xi) = (-1)^n\exp(\xi^2)\dv[n]{\xi}[\exp(-\xi^2)]
        \end{equation*}
    \end{itemize}
    \item \textcite[65-66]{bib:PHYS23410Notes}: Legendre polynomials.
    \begin{itemize}
        \item \textcite{bib:CHEM26100Notes} actually does a better job of deriving Legendre's equation and motivating why we need the associated Legendre functions.
        \item The Legendre polynomials are given by Rodrigues' formula:
        \begin{equation*}
            P_\ell(u) = \frac{1}{2^\ell\ell!}\dv[\ell]{u}(u^2-1)^\ell
        \end{equation*}
        \item The associated Legendre functions are defined as in \textcite{bib:CHEM26100Notes}.
    \end{itemize}
\end{itemize}


\section*{Applying Hypergeometric Functions Ideas}
\subsection*{The Confluent Hypergeometric Equation}
\begin{itemize}
    \item In this section, \textcite{bib:Seaborn} present a purposefully handwavey derivation of the confluent hypergeometric equation (and function) from the hypergeometric equation (and function). They do this so as to emphasize the connection between the two and their solutions and not get bogged down in the algebra. Let's begin.
    \item Define $x:=bz$ in order to rewrite the hypergeometric function as follows.
    \begin{align*}
        F(a,b;c;z) &= \sum_{n=0}^\infty\frac{(a)_n(1)(b+1)\cdots(b+n-1)}{n!(c)_n}z^n\\
        &= \sum_{n=0}^\infty\frac{(a)_n(1)(1+\frac{1}{b})\cdots(1+\frac{n-1}{b})}{n!(c)_n}x^n
    \end{align*}
    \begin{itemize}
        \item Taking the limit as $b\to\infty$ of the above yields the \textbf{confluent hypergeometric function}.
    \end{itemize}
    \item \textbf{Confluent hypergeometric function}: The function defined as follows. \emph{Denoted by} $\bm{{}_1F_1}$. \emph{Given by}
    \begin{equation*}
        {}_1F_1(a;c;x) := \sum_{n=0}^\infty\frac{(a)_n}{n!(c)_n}x^n
    \end{equation*}
    \item Similarly, we may rewrite the hypergeometric equation using this substitution.
    \begin{equation*}
        x\left( 1-\frac{x}{b} \right)\dv[2]{u}{x}+\left[ c-\left( \frac{a+1}{b}+1 \right)x \right]\dv{u}{x}-au = 0
    \end{equation*}
    \begin{itemize}
        \item Note that we have to use the chain rule when replacing the derivatives; this is how all the $b$'s work out. Essentially, we substitute $z=x/b$, $u(z)=u(x)$, $\dv*{u}{z}=b\cdot\dv*{u}{x}$, and $\dv*[2]{u}{z}=b^2\cdot\dv*[2]{u}{x}$; after that, we divide through once by $b$ and simplify.
        \item Then once again, we take the limit as $b\to\infty$ to recover the \textbf{confluent hypergeometric equation}.
    \end{itemize}
    \item \textbf{Confluent hypergeometric equation}: The differential equation given as follows, where $a,c\in\C$ are constants independent of $x$. \emph{Given by}
    \begin{equation*}
        x\dv[2]{u}{x}+(c-x)\dv{u}{x}-au = 0
    \end{equation*}
    \item Let's investigate the singularities of the confluent hypergeometric equation and see how they stack up against the $0,1,\infty$ of the hypergeometric equation.
    \begin{itemize}
        \item First off, observe that the confluent hypergeometric equation has singularities at $x=0,\infty$.
        \item Rewriting the confluent hypergeometric equation in the standard form for a linear, second-order, homogeneous differential equation, we obtain
        \begin{align*}
            P(x) &= \frac{c}{x}-1&
            Q(x) &= -\frac{a}{x}
        \end{align*}
        \begin{itemize}
            \item Since $xP(x)=c-x$ and $x^2Q(x)=-ax$ are both analytic at $x=0$, the singularity at $x=0$ is regular.
        \end{itemize}
        \item How about the regularity of the singularity at $x=\infty$?
        \begin{itemize}
            \item Change the variable to $y=x^{-1}$ and consider the resultant analogous singularity at $y=0$.
            \item This yields
            \begin{equation*}
                \dv[2]{u}{y}+\frac{y+(2-c)y^2}{y^3}\dv{u}{y}-\frac{a}{y^3}u = 0
            \end{equation*}
            \item Since $yP(y)=[1+(2-c)y]/y$ and $y^2Q(y)=-a/y$ --- neither of which is analytic at $y=0$ --- the singularity at $x=\infty$ must be irregular.
            \item In particular, this is because a merging (or \textbf{confluence}) of the singularities of the hypergeometric equation at $z=1$ and $z=\infty$ has occurred.
        \end{itemize}
    \end{itemize}
    \item Finally, we will show that the confluent hypergeometric function constitutes a solution to the confluent hypergeometric equation and derive the general solution as well.
    \begin{itemize}
        \item Once again, we use the ansatz
        \begin{equation*}
            u(x) = \sum_{k=0}^\infty a_kx^{k+s}
        \end{equation*}
        \item Doing the casework and the recursion relation gets us to
        \begin{align*}
            u_1(x) &= a_0\,{}_1F_1(a;c;x)&
            u_2(x) &= a_0x^{1-c}\,{}_1F_1(1+a-c;2-c;x)
        \end{align*}
        so that if $c\notin\Z$, the general solution is
        \begin{equation*}
            u(x) = A\,{}_1F_1(a;c;x)+Bx^{1-c}\,{}_1F_1(1+a-c;2-c;x)
        \end{equation*}
    \end{itemize}
\end{itemize}


\subsection*{One-Dimensional Harmonic Oscillator}
\begin{itemize}
    \item This is a prototypical sorted example of what kinds of strategizing I will do. Here, the math is heavier so that I can see exactly how it works. In a presentation, I'll be much more handwavey and with far fewer equations.
    \item The 1D quantum harmonic oscillator will now be solved using the methods developed in the previous section.
    \item The quantum mechanics.
    \begin{itemize}
        \item Starting with the TDSE.
        \item Separation of variables.
        \item Solving the time component.
        \item Arriving at the TISE.
        \begin{equation*}
            \dv[2]{x}u(x)+\left[ \frac{2mE}{\hbar^2}-\frac{m^2\omega^2}{\hbar^2}x^2 \right]u(x) = 0
        \end{equation*}
    \end{itemize}
    \item We will now go through several changes of variable to transform the above into the confluent hypergeometric equation.
    \begin{itemize}
        \item To begin, we can clean up a lot of the constants via a change of independent variable $x=b\rho$.
        \begin{itemize}
            \item Making this substitution yields
            \begin{align*}
                0 &= \frac{1}{b^2}\dv[2]{\rho}u(\rho)+\left[ \frac{2mE}{\hbar^2}-\frac{m^2\omega^2}{\hbar^2}\cdot b^2\rho^2 \right]u(\rho)\\
                &= \dv[2]{\rho}u(\rho)+\left[ \frac{2mE}{\hbar^2}\cdot b^2-\frac{m^2\omega^2}{\hbar^2}\cdot b^4\rho^2 \right]u(\rho)
            \end{align*}
            \item Thus, if we define $b^4=\hbar^2/m^2\omega^2$ (directly, this is $b:=(\hbar/m\omega)^{1/2}$), we can entirely rid ourselves of the constants in front of the former $x^2u(x)$ term. This yields
            \begin{equation*}
                0 = \dv[2]{\rho}u(\rho)+\left[ \frac{2E}{\hbar\omega}-\rho^2 \right]u(\rho)
            \end{equation*}
            \item Defining $\mu:=2E/\hbar\omega$ further cleans up the above, yielding
            \begin{equation*}
                0 = \dv[2]{\rho}u(\rho)+(\mu-\rho^2)u(\rho)
            \end{equation*}
        \end{itemize}
        \item Continuing to push forward, try the following substitution where $h,g$ are to be determined.
        \begin{equation*}
            u(\rho) = h(\rho)\e[g(\rho)]
        \end{equation*}
        \begin{itemize}
            \item The motivation for this change is that successive differentiations keep an $\e[g(\rho)]$ factor in each term that can be cancelled out to leave a zero-order term consisting of $f(\rho)$ multiplied by an arbitrary function of $\rho$. Choosing this latter function to be equal to the constant $a$ from the confluent hypergeometric equation's zero-order term gives us a useful constraint. If this seems complicated, just watch the following computations.
            \item Making the substitution, we obtain
            \begin{align*}
                0 &= \dv[2]{\rho}[h\e[g]]+(\mu-\rho^2)h\e[g]\\
                &= \dv{\rho}[h'\e[g]+hg'\e[g]]+(\mu-\rho^2)h\e[g]\\
                &= [(h''\e[g]+h'g'\e[g])+(h'g'\e[g]+hg''\e[g]+h(g')^2\e[g])]+(\mu-\rho^2)h\e[g]\\
                &= [(h''+h'g')+(h'g'+hg''+h(g')^2)]+(\mu-\rho^2)h\\
                &= h''+2g'h'+(\mu-\rho^2+(g')^2+g'')h
            \end{align*}
            \item To make the zero-order term's factor constant, simply take $(g')^2:=\rho^2$. See how we've used the constancy constraint to define $g$! Specifically, from here we get
            \begin{align*}
                g' &= \pm\rho\\
                g &= \pm\frac{1}{2}\rho^2
            \end{align*}
            \item As to the sign question, we choose the sign that ensures $u(\rho)=h(\rho)\e[\pm\rho^2/2]$ does not blow up for large $\rho$. Naturally, this means that we choose the negative sign and obtain
            \begin{equation*}
                u(\rho) = h(\rho)\e[-\rho^2/2]
            \end{equation*}
            \item The differential equation also simplifies to the following under this definition of $g$.
            \begin{equation*}
                0 = h''-2\rho h'+(\mu-1)h
            \end{equation*}
            \begin{itemize}
                \item One may recognize this as the Hermite equation!
                \item Through this $u(\rho)$ substitution method, we've effectively avoided the handwavey asymptotic analysis that physicists and chemists frequently use to justify deriving the Hermite equation.
            \end{itemize}
        \end{itemize}
        \item Alright, so this takes care of $g$; now how about $h$?
        \item To address $h$, we will need another independent variable change.
        \begin{itemize}
            \item An independent variable change is desirable here because it can alter the first two terms without affecting the zero-order term.
            \item Begin with the general modification $s:=\alpha\rho^n$, where $\alpha,n$ are parameters to be determined.
            \item Via the chain rule, the differential operators transform under this substitution into
            \begin{align*}
                \dv{\rho} &= \dv{s}{\rho}\cdot\dv{s}\\
                &= n\alpha\rho^{n-1}\cdot\dv{s}\\
                &= n\alpha(\alpha^{-1/n}s^{1/n})^{n-1}\cdot\dv{s}\\
                % &= n\alpha(\alpha^{-(n-1)/n}s^{(n-1)/n})\cdot\dv{s}\\
                &= n\alpha^{1/n}s^{1-1/n}\cdot\dv{s}
            \end{align*}
            and, without getting into the analogous gory details,
            \begin{equation*}
                \dv[2]{\rho} = n^2\alpha^{2/n}s^{2-2/n}\dv[2]{s}+n(n-1)\alpha^{2/n}s^{1-2/n}\dv{s}
            \end{equation*}
            \item Now another thing that the confluent hypergeometric equation tells us is that the second-order term needs an $s$ in the coefficient. Thus, since $s^{2-2/n}$ is the current coefficient, we should choose $n=2$ so that $s^{2-2/2}=s^1=s$ is in the coefficient.
            \item This simplifies the operators to
            \begin{align*}
                \dv{\rho} &= 2\alpha^{1/2}s^{1/2}\cdot\dv{s}&
                \dv[2]{\rho} &= 4\alpha s\dv[2]{s}+2\alpha\dv{s}
            \end{align*}
            and hence the differential equation to
            \begin{align*}
                0 &= 4\alpha s\dv[2]{h}{s}+2\alpha\dv{h}{s}-2\cdot\alpha^{-1/2}s^{1/2}\cdot 2\alpha^{1/2}s^{1/2}\cdot\dv{h}{s}+(\mu-1)h(s)\\
                &= 4\alpha s\dv[2]{h}{s}+(2\alpha-4s)\dv{h}{s}+(\mu-1)h(s)\\
                &= \alpha s\dv[2]{h}{s}+\left( \frac{\alpha}{2}-s \right)\dv{h}{s}-\frac{1}{4}(1-\mu)h(s)
            \end{align*}
            \item Finally, to give the right coefficient in the second-order term and complete the transformation into the confluent hypergeometric equation, pick $\alpha=1$.
            \begin{equation*}
                0 = s\dv[2]{h}{s}+\left( \frac{1}{2}-s \right)\dv{h}{s}-\frac{1}{4}(1-\mu)h(s)
            \end{equation*}
        \end{itemize}
    \end{itemize}
    \item Now according to our prior general solution to the hypergeometric equation,
    \begin{equation*}
        h(s) = A\,{}_1F_1(\tfrac{1}{4}(1-\mu);\tfrac{1}{2};s)+Bs^{1/2}\,{}_1F_1(1+\tfrac{1}{4}(1-\mu)-\tfrac{1}{2};2-\tfrac{1}{2};s)
    \end{equation*}
    \begin{itemize}
        \item Under one last reverse change of variables back via $s=\rho^2$ and some simplification, we obtain
        \begin{equation*}
            h(\rho) = A\,{}_1F_1(\tfrac{1}{4}(1-\mu);\tfrac{1}{2};\rho^2)+B\rho\,{}_1F_1(\tfrac{1}{4}(3-\mu);\tfrac{3}{2};\rho^2)
        \end{equation*}
    \end{itemize}
\end{itemize}

\subsubsection*{Boundary Conditions and Energy Eigenvalues}
\begin{itemize}
    \item Come back for more detail!!
    \item Under an asymptotic analysis, the confluent hypergeometric functions are diverging at large $\rho$.
    \item To prevent this, we need the series to terminate. By our previous results about series termination, this happens when either\dots
    \begin{enumerate}
        \item $\frac{1}{4}(1-\mu)$ is a nonpositive integer and $B=0$;
        \item $\frac{1}{4}(3-\mu)$ is a nonpositive integer and $A=0$.
    \end{enumerate}
    \item The first case gives the even energy eigenvalues and Hermite polynomials, and the second case gives us the odd energy eigenvalues and Hermite polynomials.
\end{itemize}

\subsubsection*{Hermite Polynomials and the Confluent Hypergeometric Function}
\begin{itemize}
    \item Come back for more detail!!
    \item Formally defining the Hermite polynomials, and proving that they satisfy the Hermite equation.
\end{itemize}


\subsection*{Three-Dimensional Schr\"{o}dinger Equation}
\begin{itemize}
    \item Very much analogously to Chapter 3, the hypergeometric function is used to tackle Legendre's equation, Legendre polynomials, and associated Legendre functions.
    \item Finally derives where $\ell(\ell+1)$ comes from for the first time!
\end{itemize}



\section*{Complex Analysis Ideas}
\subsection*{Legendre Polynomials}
\begin{itemize}
    \item Will get to use residue and the $\Gamma$ function.
    \item Get to the Rodrigues formula.
\end{itemize}


\subsection*{Hermite Polynomials}
\begin{itemize}
    \item Rodrigues expression for the Hermite polynomials.
\end{itemize}



\section*{Tie-Back Ideas}
\begin{itemize}
    \item Chapter 10: Contour integral definitions of these two. Never seen in physics, but cool characterization!
    \item Chapter 11: Generating functions of these two. Same as above. Did Mazziotti allude to recursion relations??
    \item Mathematical applications of these to things I have seen, like normalization. How are these characterizations useful for proving certain physical properties, even if they're never discussed explicitly in intro courses?
    \item Applications to orthogonality relations: 12.4.
\end{itemize}
\newpage



\printbibliography




\end{document}