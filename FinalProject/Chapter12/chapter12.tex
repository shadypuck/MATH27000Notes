\documentclass[../finalProject.tex]{subfiles}

\pagestyle{main}
\renewcommand{\sectionmark}[1]{\markboth{Chapter \thesection\ (#1)}{}}
\setcounter{section}{11}

\begin{document}




\section{Orthogonal Functions}\label{sch:12}
\setcounter{subsection}{3}
\subsection{Orthogonality and Normalization of Special Functions}
\begin{itemize}
    \item \marginnote{5/10:}Mathematical applications of these to things I have seen, like normalization. How are these characterizations useful for proving certain physical properties, even if they're never discussed explicitly in intro courses?
    \item Applications to orthogonality relations: 12.4.
\end{itemize}




\end{document}