\documentclass[../finalProject.tex]{subfiles}

\pagestyle{main}
\renewcommand{\leftmark}{Presentation Feedback Form}
\renewcommand{\proofname}{Response}

\begin{document}




\section*{Presentation Feedback Form}
\addcontentsline{toc}{section}{Presentation Feedback Form - Eric}
\begin{itemize}
    \item \marginnote{5/13:}\textbf{Person presenting}: Eric Yin
    \item \textbf{Person giving feedback}: Steven Labalme
    \item Instructions.
    \begin{itemize}
        \item This form is a template for giving feedback on final oral presentations. The \underline{presenter} should upload this form to Gradescope together with any auxiliary documents (comments on slides, a summary of a discussion afterwards, etc.) in order for both people to earn their respective credit for this assignment.
        \item As a reminder, everyone must get feedback on their presentation from 2 people and must give feedback to at least one other person. For presenters, the first feedback form by May 13, and the second is due by the final presentation. I will check who gave feedback only once presentations are complete.
    \end{itemize}
\end{itemize}

    
\subsection*{Content}
\begin{itemize}
    \item In one sentence (and in your own words!), what was this presentation about? What was the main result?
    \begin{proof}
        Eric's presentation introduced a number of notions that generalize holomorphic functions on the complex plane to holomorphic functions first on $\C^n$ and then on manifolds. After building out this theory a bit, Eric stated and proved Chow's theorem.
    \end{proof}
    \item What did you find most interesting about this subject?
    \begin{proof}
        I had not previously considered the idea of holomorphic functions on manifolds, though in retrospect it makes sense that if it can be done for real functions, doing it for complex functions should certainly be possible, too. Eric introduced a number of notions that were either new to me or that I had not seen in a while, but did so in a clear and concise manner so that I could start building a mental picture.
    \end{proof}
    \item Were there any places in the talk that you were confused?
    \begin{proof}
        On the new notions, Eric would do well to say some more handwavey things about what they are, befitting an oral presentation format as opposed to a technical mathematical writeup. In particular, there were a number of topics that Eric assumed familiarity with that are not prerequisite knowledge for this course, so while this presentation might make perfect sense to the professor, Eric's student audience could get lost. Particular notions I flagged are: Commutative rings, integral domains, Noetherian rings, UFDs, Hausdorff spaces, and sheaves.
    \end{proof}
    \item Were there any pictures, examples, or analogies that you found especially enlightening?
    \begin{proof}
        A few more pictures and examples could actually be great. When Eric described the real projective space as "the space of all the lines," I though that was an inspired handwavey analogy! He could sprinkle in more of this. On the subject of manifolds, would it be possible to give a trivial example with the Riemann sphere to show this manifold mapping and concretize some of these definitions? I namecheck the Riemann sphere because its the nontrivial manifold $S^2$ and one that we'll all be familiar with at this point. Though perhaps that wouldn't work, but something of the sort would be great.
    \end{proof}
\end{itemize}


\subsection*{Mechanics}
\begin{itemize}
    \item How long was this presentation? Did you think that anything should be given more or less time?
    \begin{proof}
        Eric got through 17 minutes --- time will be a concern. This is very much a "down the rabbit hole" presentation; that's fine, but he will have to compensate with consistent reminders of what we've done and where we're going, as well as figuring out what can be handwaved and what needs to be rigorous. He moved quickly and clearly, though, at a good pace; it's more of a content problem than a pacing problem.
    \end{proof}
    \item The talk was well-organized.
    \begin{center}
        \fbox{Strongly agree} \hspace{1em} Agree \hspace{1em} Neutral \hspace{1em} Disagree \hspace{1em} Strongly disagree
    \end{center}
    \item The speaker demonstrated proficient board work or use of slides.
    \begin{center}
        \fbox{Strongly agree} \hspace{1em} Agree \hspace{1em} Neutral \hspace{1em} Disagree \hspace{1em} Strongly disagree
    \end{center}
    \item The speaker communicated in a clear, precise manner.
    \begin{center}
        \fbox{Strongly agree} \hspace{1em} Agree \hspace{1em} Neutral \hspace{1em} Disagree \hspace{1em} Strongly disagree
    \end{center}
\end{itemize}


\subsection*{General feedback}
\begin{itemize}
    \item The speaker captured my interest.
    \begin{center}
        \fbox{Strongly agree} \hspace{1em} Agree \hspace{1em} Neutral \hspace{1em} Disagree \hspace{1em} Strongly disagree
    \end{center}
    \item What is (at least) one thing you liked about the talk?
    \begin{proof}
        The presentation came across as very well rehearsed. Eric displays a clear mastery of the material, and his confidence in using the board shows that he has all of this stuff in his head. Eric never tripped up or seemed at a loss as to where to proceed.\par
        The content itself was also an interesting and natural extension of what we've done.
    \end{proof}
    \item What is (at least) one thing you thought could be improved? (be constructive!)
    \begin{proof}
        A high-level intro in the beginning would be awesome. Something along the lines of, "Hi, I'm Eric, and I'm going to build up just enough of the theory of complex functions on manifolds to state and prove Chow's theorem, which is cool because\dots" This would ground an unfamiliar audience and give them something to hang onto. 
    \end{proof}
    \item Raw, unedited notes on the presentation.
    \begin{proof}
        Holomorphic functions in $\C^n$. Define holomorphic as everywhere analytic on an open set. $\mO(U)$ is a commutative ring, in fact, an integral domain if $U$ is connected (justified by identity theorem).

        What is an integral domain, intuitively? This isn't an algebra course. Definition was great, but integrate some handwavey-ness for an oral presentation.

        Not a UFD. Not a great ring. Let's make a better one! This is the \textbf{germ} of holomorphic functions. Create an equivalence class of $f$ defined by equivalence on restriction to $W\ni 0$ and $W\subset U\cap V$. The ring of equivalence classes is fun.

        High level intro at the beginning so we know where we're going!

        For all $f$ in this set, $f$ has a power series converging around zero. Because of the identity theorem, this is \emph{all} of the germs we can have around zero. $O_1$ is the ring of convergent power series around the origin; $O_n$ is the ring of convergent power series around the $n$-dimensional origin. $O_n$ is commutitive, integral domain, Noetherian (??), and a UFD.

        Can we extend "holomorphism" beyond the manifold of $\C^n$.
        How do we stitch functions together between rings? We define a way to do this.

        These random properties are rooted in real, continuous, etc. classes of functions --- show this! Make it intuitive.

        Hausdorff?? This is definitely one of those "down the rabbit hole" presentations. Not a problem, just account for that!

        How do we make geometric spaces interact? Use a morphism.

        Sheafs??

        Pictures and squiggly lines could be helpful here.

        Very well rehearsed; mastery of the material, all the stuff in his head. Given on the board without hesitation and well-explained. Never tripped up.

        Distinguish between definitions, computations, results, etc. on the board.

        An interesting way to define a manifold: $X$ is a \textbf{manifold} if it's isomorphic to some subset of the complex plane via a holomorphic function on $\C^n$. How we understand holomorphic functions on a more general class of spaces! Can we give a trivial example with the Riemann sphere to show this manifold mapping and concretize some of these definitions? An example of some sort could be great!

        Real projective space as "the space of all the lines" is an inspired handwavey analogy! Sprinkle in more of this.

        Are we building up to something or just exploring a notion and its theory/consequences?
    \end{proof}
\end{itemize}




\end{document}