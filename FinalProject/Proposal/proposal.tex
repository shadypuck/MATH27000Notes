\documentclass[../finalProject.tex]{subfiles}

\pagestyle{main}
\renewcommand{\leftmark}{Final Project Proposal Form}
\renewcommand{\proofname}{Answer}

\begin{document}




\begin{enumerate}[label={\textbf{\arabic*.}}]
    \item \addcontentsline{toc}{section}{Final Project Proposal Form}
    What topic do you want to do your final project on?
    \begin{proof}
        Special differential equations in physics (Hermite, Bessel, Legendre, Laguerre, etc.) and hypergeometric functions.
    \end{proof}
    \item What will be the main reference(s) that you will base your project on?
    \begin{proof}
        \textcite{bib:Seaborn}.
    \end{proof}
    \item What is the main statement or question you want to address in your project? Be specific!
    \begin{proof}
        % I took CHEM 26100 Quantum Mechanics. It was great, but left some questions unanswered, like choosing flux coefficients and where exactly the polynomial solutions to the hydrogen atom came from. Last quarter, I took PHYS 23410 Quantum Mechanics I, which answered the former but still did not address the math behind the latter.
        % The Hermite, Legendre, and Laguerre equations are always derived, but I have never seen how they are solved for their polynomial solutions. \textcite{bib:Seaborn} goes into this detail.
        % I understand all the QMech; I'm well-positioned to focus on the math at this point.
        % Focus on Legendre and hermite.

        Where do the various formulas for the Hermite and Legendre polynomials come from? These two cases hold particular interest for me because of the time I've spent working with them in my quantum mechanics coursework without ever knowing where they come from. I'm very much a bottom-up learner, so I'm super excited to finally explore their origins from the simple to the complex, no pun intended.
    \end{proof}
    \item Everyone has to prove \emph{something} in their project (it doesn't have to be the same as the main statement/question from above). What is one statement you will explain the proof of in your writeup?
    \begin{proof}
        The Cauchy residue theorem
    \end{proof}
    \item What complex analysis topic will go into the project?
    \begin{proof}
        Applications to converting the hypergeometric definition of the Legendre polynomials into Rodrigues's formula, which I saw last quarter but which came out of nowhere.
    \end{proof}
    \item Is there other background (not in the main reference) you will need to complete the project? If you don't have it, how will you go about learning it?
    \begin{proof}
        Not particularly. I know the quantum mechanics. I'm prepared for some misconceptions regarding functional analysis (e.g., orthogonal polynomials), but I trust I can address these as they arise and that my grasp of the "big picture" is good enough that I'll be able to concentrate on the details.
    \end{proof}
\end{enumerate}
\newpage



\printbibliography




\end{document}